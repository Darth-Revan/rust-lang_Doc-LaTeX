
\begin{myquote}
Program testing can be a very effective way to show the presence of bugs, but it is hopelessly inadequate for showing their absence.

\blank

Edsger W. Dijkstra, \enquote{The Humble Programmer} (1972)
\end{myquote}

Let's talk about how to test Rust code. What we will not be talking about is the right way to test Rust code. There are many schools 
of thought regarding the right and wrong way to write tests. All of these approaches use the same basic tools, and so we'll show you 
the syntax for using them.

\subsection*{The \code{test} attribute}

At its simplest, a test in Rust is a function that's annotated with the \code{test} attribute. Let's make a new project with Cargo 
called \code{adder}:

\begin{verbatim}
$ cargo new adder
$ cd adder
\end{verbatim}

Cargo will automatically generate a simple test when you make a new project. Here's the contents of \code{src/lib.rs}:

\begin{rustc}
#[test]
fn it_works() {
}
\end{rustc}

Note the \code{\#[test]}. This attribute indicates that this is a test function. It currently has no body. That's good enough to pass! 
We can run the tests with \code{cargo test}:

\begin{verbatim}
$ cargo test
   Compiling adder v0.0.1 (file:///home/you/projects/adder)
     Running target/adder-91b3e234d4ed382a

running 1 test
test it_works ... ok

test result: ok. 1 passed; 0 failed; 0 ignored; 0 measured

   Doc-tests adder

running 0 tests

test result: ok. 0 passed; 0 failed; 0 ignored; 0 measured
\end{verbatim}

Cargo compiled and ran our tests. There are two sets of output here: one for the test we wrote, and another for documentation tests. 
We'll talk about those later. For now, see this line:

\begin{verbatim}
test it_works ... ok
\end{verbatim}

Note the \code{it\_works}. This comes from the name of our function:

\begin{rustc}
fn it_works() {
\end{rustc}

We also get a summary line:

\begin{verbatim}
test result: ok. 1 passed; 0 failed; 0 ignored; 0 measured
\end{verbatim}

So why does our do-nothing test pass? Any test which doesn't \panic\ passes, and any test that does \panic\ fails. Let's make our test fail:

\begin{rustc}
#[test]
fn it_works() {
    assert!(false);
}
\end{rustc}

\code{assert!} is a macro provided by Rust which takes one argument: if the argument is \code{true}, nothing happens. If the argument 
is \code{false}, it \panic s. Let's run our tests again:

\begin{verbatim}
$ cargo test
   Compiling adder v0.0.1 (file:///home/you/projects/adder)
     Running target/adder-91b3e234d4ed382a

running 1 test
test it_works ... FAILED

failures:

---- it_works stdout ----
        thread 'it_works' panicked at 'assertion failed: false', /home/steve/tmp/adder/src/lib.rs:3



failures:
    it_works

test result: FAILED. 0 passed; 1 failed; 0 ignored; 0 measured

thread '<main>' panicked at 'Some tests failed', /home/steve/src/rust/src/libtest/lib.rs:247
\end{verbatim}

Rust indicates that our test failed:

\begin{verbatim}
test it_works ... FAILED
\end{verbatim}

And that's reflected in the summary line:

\begin{verbatim}
test result: FAILED. 0 passed; 1 failed; 0 ignored; 0 measured
\end{verbatim}

We also get a non-zero status code. We can use \code{\$?} on OS X and Linux:

\begin{verbatim}
$ echo $?
101
\end{verbatim}

On Windows, if you’re using \code{cmd}:

\begin{verbatim}
> echo %ERRORLEVEL%
\end{verbatim}

And if you’re using PowerShell:

\begin{verbatim}
> echo $LASTEXITCODE # the code itself
> echo $? # a boolean, fail or succeed
\end{verbatim}

This is useful if you want to integrate \code{cargo test} into other tooling.

\blank

We can invert our test's failure with another attribute: \code{should\_panic}:

\begin{rustc}
#[test]
#[should_panic]
fn it_works() {
    assert!(false);
}
\end{rustc}

This test will now succeed if we \panic\ and fail if we complete. Let's try it:

\begin{verbatim}
$ cargo test
   Compiling adder v0.0.1 (file:///home/you/projects/adder)
     Running target/adder-91b3e234d4ed382a

running 1 test
test it_works ... ok

test result: ok. 1 passed; 0 failed; 0 ignored; 0 measured

   Doc-tests adder

running 0 tests

test result: ok. 0 passed; 0 failed; 0 ignored; 0 measured
\end{verbatim}

Rust provides another macro, \code{assert\_eq!}, that compares two arguments for equality:

\begin{rustc}
#[test]
#[should_panic]
fn it_works() {
    assert_eq!("Hello", "world");
}
\end{rustc}

Does this test pass or fail? Because of the \code{should\_panic} attribute, it passes:

\begin{verbatim}
$ cargo test
   Compiling adder v0.0.1 (file:///home/you/projects/adder)
     Running target/adder-91b3e234d4ed382a

running 1 test
test it_works ... ok

test result: ok. 1 passed; 0 failed; 0 ignored; 0 measured

   Doc-tests adder

running 0 tests

test result: ok. 0 passed; 0 failed; 0 ignored; 0 measured
\end{verbatim}

\code{should\_panic} tests can be fragile, as it's hard to guarantee that the test didn't fail for an unexpected reason. To help 
with this, an optional \code{expected} parameter can be added to the \code{should\_panic} attribute. The test harness will make 
sure that the failure message contains the provided text. A safer version of the example above would be:

\begin{rustc}
#[test]
#[should_panic(expected = "assertion failed")]
fn it_works() {
    assert_eq!("Hello", "world");
}
\end{rustc}

That's all there is to the basics! Let's write one 'real' test:

\begin{rustc}
pub fn add_two(a: i32) -> i32 {
    a + 2
}

#[test]
fn it_works() {
    assert_eq!(4, add_two(2));
}
\end{rustc}

This is a very common use of \code{assert\_eq!}: call some function with some known arguments and compare it to the expected output.

\subsection*{The \code{ignore} attribute}

Sometimes a few specific tests can be very time-consuming to execute. These can be disabled by default by using the \code{ignore} attribute:

\begin{rustc}
#[test]
fn it_works() {
    assert_eq!(4, add_two(2));
}

#[test]
#[ignore]
fn expensive_test() {
    // code that takes an hour to run
}
\end{rustc}

Now we run our tests and see that \code{it\_works} is run, but \code{expensive\_test} is not:

\begin{verbatim}
$ cargo test
   Compiling adder v0.0.1 (file:///home/you/projects/adder)
     Running target/adder-91b3e234d4ed382a

running 2 tests
test expensive_test ... ignored
test it_works ... ok

test result: ok. 1 passed; 0 failed; 1 ignored; 0 measured

   Doc-tests adder

running 0 tests

test result: ok. 0 passed; 0 failed; 0 ignored; 0 measured
\end{verbatim}

The expensive tests can be run explicitly using \code{cargo test -- --ignored}:

\begin{verbatim}
$ cargo test -- --ignored
     Running target/adder-91b3e234d4ed382a

running 1 test
test expensive_test ... ok

test result: ok. 1 passed; 0 failed; 0 ignored; 0 measured

   Doc-tests adder

running 0 tests

test result: ok. 0 passed; 0 failed; 0 ignored; 0 measured
\end{verbatim}

The \code{--ignored} argument is an argument to the test binary, and not to Cargo, which is why the command is 
\code{cargo test -- --ignored}.

\subsection*{The \code{tests} module}

There is one way in which our existing example is not idiomatic: it's missing the \code{tests} module. The idiomatic way of 
writing our example looks like this:

\begin{rustc}
pub fn add_two(a: i32) -> i32 {
    a + 2
}

#[cfg(test)]
mod tests {
    use super::add_two;

    #[test]
    fn it_works() {
        assert_eq!(4, add_two(2));
    }
}
\end{rustc}

There's a few changes here. The first is the introduction of a \code{mod tests} with a \code{cfg} attribute. The module allows 
us to group all of our tests together, and to also define helper functions if needed, that don't become a part of the rest of 
our crate. The \code{cfg} attribute only compiles our test code if we're currently trying to run the tests. This can save compile 
time, and also ensures that our tests are entirely left out of a normal build.

\blank

The second change is the \code{use} declaration. Because we're in an inner module, we need to bring our test function into scope. 
This can be annoying if you have a large module, and so this is a common use of globs. Let's change our \code{src/lib.rs} to make 
use of it:

\begin{rustc}
pub fn add_two(a: i32) -> i32 {
    a + 2
}

#[cfg(test)]
mod tests {
    use super::*;

    #[test]
    fn it_works() {
        assert_eq!(4, add_two(2));
    }
}
\end{rustc}

Note the different \code{use} line. Now we run our tests:

\begin{verbatim}
$ cargo test
    Updating registry `https://github.com/rust-lang/crates.io-index`
   Compiling adder v0.0.1 (file:///home/you/projects/adder)
     Running target/adder-91b3e234d4ed382a

running 1 test
test tests::it_works ... ok

test result: ok. 1 passed; 0 failed; 0 ignored; 0 measured

   Doc-tests adder

running 0 tests

test result: ok. 0 passed; 0 failed; 0 ignored; 0 measured
\end{verbatim}

It works!

\blank

The current convention is to use the \code{tests} module to hold your \enquote{unit-style} tests. Anything that tests one 
small bit of functionality makes sense to go here. But what about \enquote{integration-style} \code{tests} instead? For that, 
we have the tests directory.

\subsection*{The \code{tests} directory}

To write an integration test, let's make a \code{tests} directory, and put a \code{tests/lib.rs} file inside, with this as its contents:

\begin{rustc}
extern crate adder;

#[test]
fn it_works() {
    assert_eq!(4, adder::add_two(2));
}
\end{rustc}

This looks similar to our previous tests, but slightly different. We now have an \code{extern crate adder} at the top. This is 
because the tests in the \code{tests} directory are an entirely separate crate, and so we need to import our library. This is also 
why \code{tests} is a suitable place to write integration-style tests: they use the library like any other consumer of it would.

\blank

Let's run them:

\begin{verbatim}
$ cargo test
   Compiling adder v0.0.1 (file:///home/you/projects/adder)
     Running target/adder-91b3e234d4ed382a

running 1 test
test tests::it_works ... ok

test result: ok. 1 passed; 0 failed; 0 ignored; 0 measured

     Running target/lib-c18e7d3494509e74

running 1 test
test it_works ... ok

test result: ok. 1 passed; 0 failed; 0 ignored; 0 measured

   Doc-tests adder

running 0 tests

test result: ok. 0 passed; 0 failed; 0 ignored; 0 measured
\end{verbatim}

Now we have three sections: our previous test is also run, as well as our new one.

\blank

That's all there is to the \code{tests} directory. The \code{tests} module isn't needed here, since the whole thing is focused on tests.

\blank

Let's finally check out that third section: documentation tests.

\subsection*{Documentation tests}

Nothing is better than documentation with examples. Nothing is worse than examples that don't actually work, because the code has 
changed since the documentation has been written. To this end, Rust supports automatically running examples in your documentation 
(\textbf{note}: this only works in library crates, not binary crates). Here's a fleshed-out \code{src/lib.rs} with examples:

\begin{rustc}
//! The `adder` crate provides functions that add numbers to other numbers.
//!
//! # Examples
//!
//! ```
//! assert_eq!(4, adder::add_two(2));
//! ```

/// This function adds two to its argument.
///
/// # Examples
///
/// ```
/// use adder::add_two;
///
/// assert_eq!(4, add_two(2));
/// ```
pub fn add_two(a: i32) -> i32 {
    a + 2
}

#[cfg(test)]
mod tests {
    use super::*;

    #[test]
    fn it_works() {
        assert_eq!(4, add_two(2));
    }
}
\end{rustc}

Note the module-level documentation with \code{//!} and the function-level documentation with \code{///}. Rust's documentation 
supports Markdown in comments, and so triple graves mark code blocks. It is conventional to include the \code{\# Examples} section, 
exactly like that, with examples following.

\blank

Let's run the tests again:

\begin{verbatim}
$ cargo test
   Compiling adder v0.0.1 (file:///home/steve/tmp/adder)
     Running target/adder-91b3e234d4ed382a

running 1 test
test tests::it_works ... ok

test result: ok. 1 passed; 0 failed; 0 ignored; 0 measured

     Running target/lib-c18e7d3494509e74

running 1 test
test it_works ... ok

test result: ok. 1 passed; 0 failed; 0 ignored; 0 measured

   Doc-tests adder

running 2 tests
test add_two_0 ... ok
test _0 ... ok

test result: ok. 2 passed; 0 failed; 0 ignored; 0 measured
\end{verbatim}

Now we have all three kinds of tests running! Note the names of the documentation tests: the \code{\_0} is generated for the module 
test, and \code{add\_two\_0} for the function test. These will auto increment with names like \code{add\_two\_1} as you add more examples.

\blank

We haven’t covered all of the details with writing documentation tests (see \nameref{sec:effective_documentation}). For more, please see 
the Documentation chapter.

\blank

One final note: documentation tests \emph{cannot} be run on binary crates. To see more on file arrangement see the Crates and Modules section
(see \nameref{sec:syntax_cratesAndModules}).
