Documentation is an important part of any software project, and it's first-class in Rust. Let's talk about the tooling Rust gives 
you to document your project.

\subsection*{About \code{rustdoc}}

The Rust distribution includes a tool, \code{rustdoc}, that generates documentation. \code{rustdoc} is also used by Cargo through 
\code{cargo doc}.

\blank

Documentation can be generated in two ways: from source code, and from standalone Markdown files.

\subsection*{Documenting source code}

The primary way of documenting a Rust project is through annotating the source code. You can use documentation comments for this purpose:

\begin{rustc}
/// Constructs a new `Rc<T>`.
///
/// # Examples
///
/// ```
/// use std::rc::Rc;
///
/// let five = Rc::new(5);
/// ```
pub fn new(value: T) -> Rc<T> {
    // implementation goes here
}
\end{rustc}

This code generates documentation that looks \href{https://doc.rust-lang.org/nightly/std/rc/struct.Rc.html#method.new}{like this}. 
I've left the implementation out, with a regular comment in its place.

\blank

The first thing to notice about this annotation is that it uses \code{///} instead of \code{//}. The triple slash indicates a 
documentation comment.

\blank

Documentation comments are written in Markdown.

\blank

Rust keeps track of these comments, and uses them when generating documentation. This is important when documenting things like enums:

\begin{rustc}
/// The `Option` type. See [the module level documentation](index.html) for more.
enum Option<T> {
    /// No value
    None,
    /// Some value `T`
    Some(T),
}
\end{rustc}

The above works, but this does not:

\begin{rustc}
/// The `Option` type. See [the module level documentation](index.html) for more.
enum Option<T> {
    None, /// No value
    Some(T), /// Some value `T`
}
\end{rustc}

You'll get an error:

\begin{verbatim}
hello.rs:4:1: 4:2 error: expected ident, found `}`
hello.rs:4 }
           ^
\end{verbatim}

This \href{https://github.com/rust-lang/rust/issues/22547}{unfortunate error} is correct; documentation comments apply to the thing 
after them, and there's nothing after that last comment.

\myparagraph{Writing documentation comments}

Anyway, let's cover each part of this comment in detail:

\begin{rustc}
/// Constructs a new `Rc<T>`.
\end{rustc}

The first line of a documentation comment should be a short summary of its functionality. One sentence. Just the basics. High level.

\begin{rustc}
///
/// Other details about constructing `Rc<T>`s, maybe describing complicated
/// semantics, maybe additional options, all kinds of stuff.
///
\end{rustc}

Our original example had just a summary line, but if we had more things to say, we could have added more explanation in 
a new paragraph.

\myparagraph{Special sections}

Next, are special sections. These are indicated with a header, \code{\#}. There are four kinds of headers that are commonly used. 
They aren't special syntax, just convention, for now.

\begin{rustc}
/// # Panics
\end{rustc}

Unrecoverable misuses of a function (i.e. programming errors) in Rust are usually indicated by panics, which kill the whole 
current thread at the very least. If your function has a non-trivial contract like this, that is detected/enforced by panics, 
documenting it is very important.

\begin{rustc}
/// # Failures
\end{rustc}

If your function or method returns a \code{Result<T, E>}, then describing the conditions under which it returns \code{Err(E)} is 
a nice thing to do. This is slightly less important than \code{Panics}, because failure is encoded into the type system, but it's 
still a good thing to do.

\begin{rustc}
/// # Safety
\end{rustc}

If your function is \code{unsafe}, you should explain which invariants the caller is responsible for upholding.

\begin{rustc}
/// # Examples
///
/// ```
/// use std::rc::Rc;
///
/// let five = Rc::new(5);
/// ```
\end{rustc}

Fourth, \code{Examples}. Include one or more examples of using your function or method, and your users will love you for it. 
These examples go inside of code block annotations, which we'll talk about in a moment, and can have more than one section:

\begin{rustc}
/// # Examples
///
/// Simple `&str` patterns:
///
/// ```
/// let v: Vec<&str> = "Mary had a little lamb".split(' ').collect();
/// assert_eq!(v, vec!["Mary", "had", "a", "little", "lamb"]);
/// ```
///
/// More complex patterns with a lambda:
///
/// ```
/// let v: Vec<&str> = "abc1def2ghi".split(|c: char| c.is_numeric()).collect();
/// assert_eq!(v, vec!["abc", "def", "ghi"]);
/// ```
\end{rustc}

Let's discuss the details of these code blocks.

\myparagraph{Code block annotations}

To write some Rust code in a comment, use the triple graves:

\begin{rustc}
/// ```
/// println!("Hello, world");
/// ```
\end{rustc}

If you want something that's not Rust code, you can add an annotation:

\begin{rustc}
/// ```c
/// printf("Hello, world\n");
/// ```
\end{rustc}

This will highlight according to whatever language you're showing off. If you're only showing plain text, choose \code{text}.

\blank

It's important to choose the correct annotation here, because \code{rustdoc} uses it in an interesting way: It can be used to 
actually test your examples in a library crate, so that they don't get out of date. If you have some C code but \code{rustdoc} 
thinks it's Rust because you left off the annotation, \code{rustdoc} will complain when trying to generate the documentation.

\subsection*{Documentation as tests}

Let's discuss our sample example documentation:

\begin{rustc}
/// ```
/// println!("Hello, world");
/// ```
\end{rustc}

You'll notice that you don't need a \code{fn main()} or anything here. \code{rustdoc} will automatically add a \code{main()} 
wrapper around your code, using heuristics to attempt to put it in the right place. For example:

\begin{rustc}
/// ```
/// use std::rc::Rc;
///
/// let five = Rc::new(5);
/// ```
\end{rustc}

This will end up testing:

\begin{rustc}
fn main() {
    use std::rc::Rc;
    let five = Rc::new(5);
}
\end{rustc}

Here's the full algorithm rustdoc uses to preprocess examples:

\begin{enumerate}
  \item{Any leading \code{\#![foo]} attributes are left intact as crate attributes.}
  \item{Some common \code{allow} attributes are inserted, including \code{unused\_variables}, \code{unused\_assignments}, \code{unused\_mut}, 
      \code{unused\_attributes}, and \code{dead\_code}. Small examples often trigger these lints.}
  \item{If the example does not contain \code{extern crate}, then \code{extern crate <mycrate>}; is inserted (note the lack of 
      \code{\#[macro\_use]}).}
  \item{Finally, if the example does not contain \code{fn main}, the remainder of the text is wrapped in \code{fn main() \{ your\_code \}}.}
\end{enumerate}

This generated \code{fn main} can be a problem! If you have \code{extern crate} or a \code{mod} statements in the example code that 
are referred to by \code{use} statements, they will fail to resolve unless you include at least \code{fn main() \{\}} to inhibit step 4. 
\code{\#[macro\_use] extern crate} also does not work except at the crate root, so when testing macros an explicit \code{main} is 
always required. It doesn't have to clutter up your docs, though -- keep reading!

\blank

Sometimes this algorithm isn't enough, though. For example, all of these code samples with \code{///} we've been talking about? The raw text:

\begin{rustc}
/// Some documentation.
# fn foo() {}
\end{rustc}

looks different than the output:

\begin{rustc}
/// Some documentation.
\end{rustc}

Yes, that's right: you can add lines that start with \code{\#}, and they will be hidden from the output, but will be used when 
compiling your code. You can use this to your advantage. In this case, documentation comments need to apply to some kind of function, 
so if I want to show you just a documentation comment, I need to add a little function definition below it. At the same time, it's 
only there to satisfy the compiler, so hiding it makes the example more clear. You can use this technique to explain longer examples 
in detail, while still preserving the testability of your documentation.

\blank

For example, imagine that we wanted to document this code:

\begin{rustc}
let x = 5;
let y = 6;
println!("{}", x + y);
\end{rustc}

We might want the documentation to end up looking like this:

\begin{myquote}
First, we set \x\ to five:

\begin{rustc}
let x = 5;
\end{rustc}

Next, we set \y\ to six:

\begin{rustc}
let y = 6;
\end{rustc}

Finally, we print the sum of \x\ and \y:

\begin{rustc}
println!("{}", x + y);
\end{rustc}
\end{myquote}

To keep each code block testable, we want the whole program in each block, but we don't want the reader to see every line every time. 
Here's what we put in our source code:

\begin{verbatim}
    First, we set `x` to five:

    ```text
    let x = 5;
    # let y = 6;
    # println!("{}", x + y);
    ```

    Next, we set `y` to six:

    ```text
    # let x = 5;
    let y = 6;
    # println!("{}", x + y);
    ```

    Finally, we print the sum of `x` and `y`:

    ```text
    # let x = 5;
    # let y = 6;
    println!("{}", x + y);
    ```
\end{verbatim}

By repeating all parts of the example, you can ensure that your example still compiles, while only showing the parts that are relevant 
to that part of your explanation.

\subsection*{Documenting macros}

Here’s an example of documenting a macro:

\begin{rustc}
/// Panic with a given message unless an expression evaluates to true.
///
/// # Examples
///
/// ```
/// # #[macro_use] extern crate foo;
/// # fn main() {
/// panic_unless!(1 + 1 == 2, “Math is broken.”);
/// # }
/// ```
///
/// ```should_panic
/// # #[macro_use] extern crate foo;
/// # fn main() {
/// panic_unless!(true == false, “I’m broken.”);
/// # }
/// ```
#[macro_export]
macro_rules! panic_unless {
    ($condition:expr, $($rest:expr),+) => ({ if ! $condition { panic!($($rest),+); } });
}
\end{rustc}

You’ll note three things: we need to add our own \code{extern crate} line, so that we can add the \code{\#[macro\_use]} attribute. 
Second, we’ll need to add our own \code{main()} as well (for reasons discussed above). Finally, a judicious use of \code{\#} to 
comment out those two things, so they don’t show up in the output.

\blank

Another case where the use of \code{\#} is handy is when you want to ignore error handling. Lets say you want the following,

\begin{rustc}
/// use std::io;
/// let mut input = String::new();
/// try!(io::stdin().read_line(&mut input));
\end{rustc}

The problem is that \code{try!} returns a \code{Result<T, E>} and test functions don't return anything so this will give a 
mismatched types error.

\begin{rustc}
/// A doc test using try!
///
/// ```
/// use std::io;
/// # fn foo() -> io::Result<()> {
/// let mut input = String::new();
/// try!(io::stdin().read_line(&mut input));
/// # Ok(())
/// # }
/// ```
\end{rustc}

You can get around this by wrapping the code in a function. This catches and swallows the \code{Result<T, E>} when running tests on 
the docs. This pattern appears regularly in the standard library.

\subsection*{Running documentation tests}

To run the tests, either:

\begin{verbatim}
$ rustdoc --test path/to/my/crate/root.rs
# or
$ cargo test
\end{verbatim}

That's right, \code{cargo test} tests embedded documentation too. \textbf{However, \code{cargo test} will not test binary crates, 
only library ones.} This is due to the way \code{rustdoc} works: it links against the library to be tested, but with a binary, 
there’s nothing to link to.

\blank

There are a few more annotations that are useful to help \code{rustdoc} do the right thing when testing your code:

\begin{rustc}
/// ```ignore
/// fn foo() {
/// ```
\end{rustc}

The \code{ignore} directive tells Rust to ignore your code. This is almost never what you want, as it's the most generic. Instead, 
consider annotating it with \code{text} if it's not code, or using \code{\#}s to get a working example that only shows the part 
you care about.

\begin{rustc}
/// ```should_panic
/// assert!(false);
/// ```
\end{rustc}

\code{should\_panic} tells \code{rustdoc} that the code should compile correctly, but not actually pass as a test.

\begin{rustc}
/// ```no_run
/// loop {
///     println!("Hello, world");
/// }
/// ```
\end{rustc}

The \code{no\_run} attribute will compile your code, but not run it. This is important for examples such as \enquote{Here's how to 
start up a network service,} which you would want to make sure compile, but might run in an infinite loop!

\subsection*{Documenting modules}

Rust has another kind of doc comment, \code{//!}. This comment doesn't document the next item, but the enclosing item. In other words:

\begin{rustc}
mod foo {
    //! This is documentation for the `foo` module.
    //!
    //! # Examples

    // ...
}
\end{rustc}

This is where you'll see \code{//!} used most often: for module documentation. If you have a module in \code{foo.rs}, you'll often 
open its code and see this:

\begin{rustc}
//! A module for using `foo`s.
//!
//! The `foo` module contains a lot of useful functionality blah blah blah
\end{rustc}

\subsection*{Documentation comment style}

Check out \href{https://github.com/rust-lang/rfcs/blob/master/text/0505-api-comment-conventions.md}{RFC 505} for full conventions 
around the style and format of documentation.

\subsection*{Other documentation}

All of this behavior works in non-Rust source files too. Because comments are written in Markdown, they're often \code{.md} files.

\blank

When you write documentation in Markdown files, you don't need to prefix the documentation with comments. For example:

\begin{rustc}
/// # Examples
///
/// ```
/// use std::rc::Rc;
///
/// let five = Rc::new(5);
/// ```
\end{rustc}

is:

\begin{verbatim}
# Examples

```
use std::rc::Rc;

let five = Rc::new(5);
```
\end{verbatim}

when it's in a Markdown file. There is one wrinkle though: Markdown files need to have a title like this:

\begin{verbatim}
% The title

This is the example documentation.
\end{verbatim}

This \code{\%} line needs to be the very first line of the file.

\subsection*{\code{doc} attributes}

At a deeper level, documentation comments are syntactic sugar for documentation attributes:

\begin{rustc}
/// this

#[doc="this"]
\end{rustc}

are the same, as are these:

\begin{rustc}
//! this

#![doc="this"]
\end{rustc}

You won't often see this attribute used for writing documentation, but it can be useful when changing some options, or when writing a macro.

\subsection*{Re-exports}

\code{rustdoc} will show the documentation for a public re-export in both places:

\begin{rustc}
extern crate foo;

pub use foo::bar;
\end{rustc}

This will create documentation for \code{bar} both inside the documentation for the crate \code{foo}, as well as the documentation 
for your crate. It will use the same documentation in both places.

\blank

This behavior can be suppressed with \code{no\_inline}:

\begin{rustc}
extern crate foo;

#[doc(no_inline)]
pub use foo::bar;
\end{rustc}

\subsection*{Missing documentation}

Sometimes you want to make sure that every single public thing in your project is documented, especially when you are working on 
a library. Rust allows you to to generate warnings or errors, when an item is missing documentation. To generate warnings you use 
\code{warn}:

\begin{rustc}
#![warn(missing_docs)]
\end{rustc}

And to generate errors you use deny:

\begin{rustc}
#![deny(missing_docs)]
\end{rustc}

There are cases where you want to disable these warnings/errors to explicitly leave something undocumented. This is done by using \code{allow}:

\begin{rustc}
#[allow(missing_docs)]
struct Undocumented;
\end{rustc}

You might even want to hide items from the documentation completely:

\begin{rustc}
#[doc(hidden)]
struct Hidden;
\end{rustc}

\subsection*{Controlling HTML}

You can control a few aspects of the HTML that \code{rustdoc} generates through the \code{\#![doc]} version of the attribute:

\begin{rustc}
#![doc(html_logo_url = "https://www.rust-lang.org/logos/rust-logo-128x128-blk-v2.png",
       html_favicon_url = "https://www.rust-lang.org/favicon.ico",
       html_root_url = "https://doc.rust-lang.org/")]
\end{rustc}

This sets a few different options, with a logo, favicon, and a root URL.

\subsection*{Configuring documentation tests}

You can also configure the way that \code{rustdoc} tests your documentation examples through the \code{\#![doc(test(..))]} attribute.

\begin{rustc}
#![doc(test(attr(allow(unused_variables), deny(warnings))))]
\end{rustc}

This allows unused variables within the examples, but will fail the test for any other lint warning thrown.

\subsection*{Generation options}

\code{rustdoc} also contains a few other options on the command line, for further customization:

\begin{itemize}
  \item{\code{--html-in-header FILE}: includes the contents of FILE at the end of the \code{<head>...</head>} section.}
  \item{\code{--html-before-content FILE}: includes the contents of FILE directly after \code{<body>}, before the rendered 
      content (including the search bar).}
  \item{\code{--html-after-content FILE}: includes the contents of FILE after all the rendered content.}
\end{itemize}

\subsection*{Security note}

The Markdown in documentation comments is placed without processing into the final webpage. Be careful with literal HTML:

\begin{rustc}
/// <script>alert(document.cookie)</script>
\end{rustc}
