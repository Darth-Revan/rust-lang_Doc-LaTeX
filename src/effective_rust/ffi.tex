\subsection*{Introduction}

This guide will use the \href{https://github.com/google/snappy}{snappy} compression/decompression library as an introduction 
to writing bindings for foreign code. Rust is currently unable to call directly into a C++ library, but snappy includes a C 
interface (documented in \href{https://github.com/google/snappy/blob/master/snappy-c.h}{snappy-c.h}).

\subsubsection*{A note about libc}

Many of these examples use \href{https://crates.io/crates/libc}{the libc crate}, which provides various type definitions for 
C types, among other things. If you’re trying these examples yourself, you’ll need to add \code{libc} to your \code{Cargo.toml}:

\begin{verbatim}
[dependencies]
libc = "0.2.0"
\end{verbatim}

and add \code{extern crate libc}; to your crate root.

\subsubsection*{Calling foreign functions}

The following is a minimal example of calling a foreign function which will compile if snappy is installed:

\begin{rustc}
extern crate libc;
use libc::size_t;

#[link(name = "snappy")]
extern {
    fn snappy_max_compressed_length(source_length: size_t) -> size_t;
}

fn main() {
    let x = unsafe { snappy_max_compressed_length(100) };
    println!("max compressed length of a 100 byte buffer: {}", x);
}
\end{rustc}

The \code{extern} block is a list of function signatures in a foreign library, in this case with the platform's C ABI. 
The \code{\#[link(...)]} attribute is used to instruct the linker to link against the snappy library so the symbols are 
resolved.

\blank

Foreign functions are assumed to be unsafe so calls to them need to be wrapped with \code{unsafe \{\}} as a promise to 
the compiler that everything contained within truly is safe. C libraries often expose interfaces that aren't thread-safe, 
and almost any function that takes a pointer argument isn't valid for all possible inputs since the pointer could be dangling,
and raw pointers fall outside of Rust's safe memory model.

\blank

When declaring the argument types to a foreign function, the Rust compiler can not check if the declaration is correct, 
so specifying it correctly is part of keeping the binding correct at runtime.

\blank

The \code{extern} block can be extended to cover the entire snappy API:

\begin{rustc}
extern crate libc;
use libc::{c_int, size_t};

#[link(name = "snappy")]
extern {
    fn snappy_compress(input: *const u8,
                       input_length: size_t,
                       compressed: *mut u8,
                       compressed_length: *mut size_t) -> c_int;
    fn snappy_uncompress(compressed: *const u8,
                         compressed_length: size_t,
                         uncompressed: *mut u8,
                         uncompressed_length: *mut size_t) -> c_int;
    fn snappy_max_compressed_length(source_length: size_t) -> size_t;
    fn snappy_uncompressed_length(compressed: *const u8,
                                  compressed_length: size_t,
                                  result: *mut size_t) -> c_int;
    fn snappy_validate_compressed_buffer(compressed: *const u8,
                                         compressed_length: size_t) -> c_int;
}
\end{rustc}

\subsection*{Creating a safe interface}

The raw C API needs to be wrapped to provide memory safety and make use of higher-level concepts like vectors. A library 
can choose to expose only the safe, high-level interface and hide the unsafe internal details.

\blank

Wrapping the functions which expect buffers involves using the \code{slice::raw} module to manipulate Rust vectors as 
pointers to memory. Rust's vectors are guaranteed to be a contiguous block of memory. The length is number of elements 
currently contained, and the capacity is the total size in elements of the allocated memory. The length is less than or 
equal to the capacity.

\begin{rustc}
pub fn validate_compressed_buffer(src: &[u8]) -> bool {
    unsafe {
        snappy_validate_compressed_buffer(src.as_ptr(), src.len() as size_t) == 0
    }
}
\end{rustc}

The \code{validate\_compressed\_buffer} wrapper above makes use of an \code{unsafe} block, but it makes the guarantee 
that calling it is safe for all inputs by leaving off \code{unsafe} from the function signature.

\blank

The \code{snappy\_compress} and \code{snappy\_uncompress} functions are more complex, since a buffer has to be allocated 
to hold the output too.

\blank

The \code{snappy\_max\_compressed\_length} function can be used to allocate a vector with the maximum required capacity 
to hold the compressed output. The vector can then be passed to the \code{snappy\_compress} function as an output parameter. 
An output parameter is also passed to retrieve the true length after compression for setting the length.

\begin{rustc}
pub fn compress(src: &[u8]) -> Vec<u8> {
    unsafe {
        let srclen = src.len() as size_t;
        let psrc = src.as_ptr();

        let mut dstlen = snappy_max_compressed_length(srclen);
        let mut dst = Vec::with_capacity(dstlen as usize);
        let pdst = dst.as_mut_ptr();

        snappy_compress(psrc, srclen, pdst, &mut dstlen);
        dst.set_len(dstlen as usize);
        dst
    }
}
\end{rustc}

Decompression is similar, because snappy stores the uncompressed size as part of the compression format and 
\code{snappy\_uncompressed\_length} will retrieve the exact buffer size required.

\begin{rustc}
pub fn uncompress(src: &[u8]) -> Option<Vec<u8>> {
    unsafe {
        let srclen = src.len() as size_t;
        let psrc = src.as_ptr();

        let mut dstlen: size_t = 0;
        snappy_uncompressed_length(psrc, srclen, &mut dstlen);

        let mut dst = Vec::with_capacity(dstlen as usize);
        let pdst = dst.as_mut_ptr();

        if snappy_uncompress(psrc, srclen, pdst, &mut dstlen) == 0 {
            dst.set_len(dstlen as usize);
            Some(dst)
        } else {
            None // SNAPPY_INVALID_INPUT
        }
    }
}
\end{rustc}

For reference, the examples used here are also available as a library \href{https://github.com/thestinger/rust-snappy}{on GitHub}.

\subsection*{Destructors}

Foreign libraries often hand off ownership of resources to the calling code. When this occurs, we must use Rust's 
destructors to provide safety and guarantee the release of these resources (especially in the case of panic).

\blank

For more about destructors, see the \href{https://doc.rust-lang.org/std/ops/trait.Drop.html}{Drop} trait.

\subsection*{Callbacks from C code to Rust functions}

Some external libraries require the usage of callbacks to report back their current state or intermediate data to 
the caller. It is possible to pass functions defined in Rust to an external library. The requirement for this is that 
the callback function is marked as \code{extern} with the correct calling convention to make it callable from C code.

\blank

The callback function can then be sent through a registration call to the C library and afterwards be invoked from there.

\blank

A basic example is:

\blank

Rust code:

\begin{rustc}
extern fn callback(a: i32) {
    println!("I'm called from C with value {0}", a);
}

#[link(name = "extlib")]
extern {
   fn register_callback(cb: extern fn(i32)) -> i32;
   fn trigger_callback();
}

fn main() {
    unsafe {
        register_callback(callback);
        trigger_callback(); // Triggers the callback
    }
}
\end{rustc}

C code:

\begin{verbatim}
typedef void (*rust_callback)(int32_t);
rust_callback cb;

int32_t register_callback(rust_callback callback) {
    cb = callback;
    return 1;
}

void trigger_callback() {
  cb(7); // Will call callback(7) in Rust
}
\end{verbatim}

In this example Rust's \code{main()} will call \code{trigger\_callback()} in C, which would, in turn, call back to 
\code{callback()} in Rust.

\subsubsection*{Targeting callbacks to Rust objects}

The former example showed how a global function can be called from C code. However it is often desired that the callback 
is targeted to a special Rust object. This could be the object that represents the wrapper for the respective C object.

\blank

This can be achieved by passing an raw pointer to the object down to the C library. The C library can then include the 
pointer to the Rust object in the notification. This will allow the callback to unsafely access the referenced Rust object.

\blank

Rust code:

\begin{rustc}
#[repr(C)]
struct RustObject {
    a: i32,
    // other members
}

extern "C" fn callback(target: *mut RustObject, a: i32) {
    println!("I'm called from C with value {0}", a);
    unsafe {
        // Update the value in RustObject with the value received from the callback
        (*target).a = a;
    }
}

#[link(name = "extlib")]
extern {
   fn register_callback(target: *mut RustObject,
                        cb: extern fn(*mut RustObject, i32)) -> i32;
   fn trigger_callback();
}

fn main() {
    // Create the object that will be referenced in the callback
    let mut rust_object = Box::new(RustObject { a: 5 });

    unsafe {
        register_callback(&mut *rust_object, callback);
        trigger_callback();
    }
}
\end{rustc}

C code:

\begin{verbatim}
typedef void (*rust_callback)(void*, int32_t);
void* cb_target;
rust_callback cb;

int32_t register_callback(void* callback_target, rust_callback callback) {
    cb_target = callback_target;
    cb = callback;
    return 1;
}

void trigger_callback() {
  cb(cb_target, 7); // Will call callback(&rustObject, 7) in Rust
}
\end{verbatim}

\subsubsection*{Asynchronous callbacks}

In the previously given examples the callbacks are invoked as a direct reaction to a function call to the external C 
library. The control over the current thread is switched from Rust to C to Rust for the execution of the callback, but 
in the end the callback is executed on the same thread that called the function which triggered the callback.

\blank

Things get more complicated when the external library spawns its own threads and invokes callbacks from there. In these 
cases access to Rust data structures inside the callbacks is especially unsafe and proper synchronization mechanisms must 
be used. Besides classical synchronization mechanisms like mutexes, one possibility in Rust is to use channels (in 
\code{std::sync::mpsc}) to forward data from the C thread that invoked the callback into a Rust thread.

\blank

If an asynchronous callback targets a special object in the Rust address space it is also absolutely necessary that no more 
callbacks are performed by the C library after the respective Rust object gets destroyed. This can be achieved by unregistering 
the callback in the object's destructor and designing the library in a way that guarantees that no callback will be performed 
after deregistration.

\subsection*{Linking}

The \code{link} attribute on \code{extern} blocks provides the basic building block for instructing rustc how it will 
link to native libraries. There are two accepted forms of the link attribute today:

\begin{itemize}
  \item{\code{\#[link(name = "foo")]}}
  \item{\code{\#[link(name = "foo", kind = "bar")]}}
\end{itemize}

In both of these cases, \code{foo} is the name of the native library that we're linking to, and in the second case 
\code{bar} is the type of native library that the compiler is linking to. There are currently three known types of 
native libraries:

\begin{itemize}
  \item{Dynamic - \code{\#[link(name = \enquote{readline})]}}
  \item{Static - \code{\#[link(name = \enquote{my\_build\_dependency}, kind = \enquote{static})]}}
  \item{Frameworks - \code{\#[link(name = \enquote{CoreFoundation}, kind = \enquote{framework})]}}
\end{itemize}

Note that frameworks are only available on OSX targets.

\blank

The different \code{kind} values are meant to differentiate how the native library participates in linkage. From a 
linkage perspective, the Rust compiler creates two flavors of artifacts: partial (rlib/staticlib) and final (dylib/binary). 
Native dynamic library and framework dependencies are propagated to the final artifact boundary, while static library 
dependencies are not propagated at all, because the static libraries are integrated directly into the subsequent artifact.

\blank

A few examples of how this model can be used are:

\begin{itemize}
  \item{A native build dependency. Sometimes some C/C++ glue is needed when writing some Rust code, but distribution of 
      the C/C++ code in a library format is a burden. In this case, the code will be archived into \code{libfoo.a} and 
      then the Rust crate would declare a dependency via \code{\#[link(name = "foo", kind = "static")]}. \\ Regardless of 
      the flavor of output for the crate, the native static library will be included in the output, meaning that distribution 
      of the native static library is not necessary.}
  \item{A normal dynamic dependency. Common system libraries (like \code{readline}) are available on a large number of systems, 
      and often a static copy of these libraries cannot be found. When this dependency is included in a Rust crate, partial 
      targets (like rlibs) will not link to the library, but when the rlib is included in a final target (like a binary), the 
      native library will be linked in.}
\end{itemize}

On OSX, frameworks behave with the same semantics as a dynamic library.

\subsection*{Unsafe blocks}

Some operations, like dereferencing raw pointers or calling functions that have been marked unsafe are only allowed 
inside unsafe blocks. Unsafe blocks isolate unsafety and are a promise to the compiler that the unsafety does not leak 
out of the block.

\blank

Unsafe functions, on the other hand, advertise it to the world. An unsafe function is written like this:

\begin{rustc}
unsafe fn kaboom(ptr: *const i32) -> i32 { *ptr }
\end{rustc}

This function can only be called from an \code{unsafe} block or another \code{unsafe} function.

\subsection*{Accessing foreign globals}

Foreign APIs often export a global variable which could do something like track global state. In order to access these 
variables, you declare them in \code{extern} blocks with the \code{static} keyword:

\begin{rustc}
extern crate libc;

#[link(name = "readline")]
extern {
    static rl_readline_version: libc::c_int;
}

fn main() {
    println!("You have readline version {} installed.",
             rl_readline_version as i32);
}
\end{rustc}

Alternatively, you may need to alter global state provided by a foreign interface. To do this, statics can be declared 
with \code{mut} so we can mutate them.

\begin{rustc}
extern crate libc;

use std::ffi::CString;
use std::ptr;

#[link(name = "readline")]
extern {
    static mut rl_prompt: *const libc::c_char;
}

fn main() {
    let prompt = CString::new("[my-awesome-shell] $").unwrap();
    unsafe {
        rl_prompt = prompt.as_ptr();

        println!("{:?}", rl_prompt);

        rl_prompt = ptr::null();
    }
}
\end{rustc}

Note that all interaction with a \code{static mut} is unsafe, both reading and writing. Dealing with global mutable 
state requires a great deal of care.

\subsection*{Foreign calling conventions}

Most foreign code exposes a C ABI, and Rust uses the platform's C calling convention by default when calling foreign 
functions. Some foreign functions, most notably the Windows API, use other calling conventions. Rust provides a way to 
tell the compiler which convention to use:

\begin{rustc}
extern crate libc;

#[cfg(all(target_os = "win32", target_arch = "x86"))]
#[link(name = "kernel32")]
#[allow(non_snake_case)]
extern "stdcall" {
    fn SetEnvironmentVariableA(n: *const u8, v: *const u8) -> libc::c_int;
}
\end{rustc}

This applies to the entire \code{extern} block. The list of supported ABI constraints are:

\begin{itemize}
  \item{\code{stdcall}}
  \item{\code{aapcs}}
  \item{\code{cdecl}}
  \item{\code{fastcall}}
  \item{\code{vectorcall} This is currently hidden behind the \code{abi\_vectorcall} gate and is subject to change.}
  \item{\code{Rust}}
  \item{\code{rust-intrinsic}}
  \item{\code{system}}
  \item{\code{C}}
  \item{\code{win64}}
\end{itemize}

Most of the abis in this list are self-explanatory, but the \code{system} abi may seem a little odd. This constraint 
selects whatever the appropriate ABI is for interoperating with the target's libraries. For example, on win32 with a x86 
architecture, this means that the abi used would be \code{stdcall}. On x86\_64, however, windows uses the \code{C} calling 
convention, so \code{C} would be used. This means that in our previous example, we could have used 
\code{extern \enquote{system} \{ ... \}} to define a block for all windows systems, not only x86 ones.

\subsection*{Interoperability with foreign code}

Rust guarantees that the layout of a \struct\ is compatible with the platform's representation in C only if the 
\code{\#[repr(C)]} attribute is applied to it. \code{\#[repr(C, packed)]} can be used to lay out struct members 
without padding. \code{\#[repr(C)]} can also be applied to an enum.

\blank

Rust's owned boxes (\code{Box<T>}) use non-nullable pointers as handles which point to the contained object. However, 
they should not be manually created because they are managed by internal allocators. References can safely be assumed 
to be non-nullable pointers directly to the type. However, breaking the borrow checking or mutability rules is not 
guaranteed to be safe, so prefer using raw pointers (\code{*}) if that's needed because the compiler can't make as many 
assumptions about them.

\blank

Vectors and strings share the same basic memory layout, and utilities are available in the \code{vec} and \code{str} 
modules for working with C APIs. However, strings are not terminated with \code{\\0}. If you need a NUL-terminated string 
for interoperability with C, you should use the \code{CString} type in the \code{std::ffi} module.

\blank

The \href{https://crates.io/crates/libc}{libc crate on crates.io} includes type aliases and function definitions for 
the C standard library in the \code{libc} module, and Rust links against \code{libc} and \code{libm} by default.

\subsection*{The \enquote{nullable pointer optimization}}

Certain types are defined to not be \code{null}. This includes references (\code{\&T}, \code{\&mut T}), boxes (\code{Box<T>}), 
and function pointers (\code{extern \enquote{abi} fn()}). When interfacing with C, pointers that might be null are often used. 
As a special case, a generic \enum\ that contains exactly two variants, one of which contains no data and the other containing 
a single field, is eligible for the \enquote{nullable pointer optimization}. When such an enum is instantiated with one of 
the non-nullable types, it is represented as a single pointer, and the non-data variant is represented as the null pointer. 
So \code{Option<extern \enquote{C} fn(c\_int) -> c\_int>} is how one represents a nullable function pointer using the C ABI.

\subsection*{Calling Rust code from C}

You may wish to compile Rust code in a way so that it can be called from C. This is fairly easy, but requires a few things:

\begin{rustc}
#[no_mangle]
pub extern fn hello_rust() -> *const u8 {
    "Hello, world!\0".as_ptr()
}
\end{rustc}

The \code{extern} makes this function adhere to the C calling convention, as discussed above in \enquote{Foreign Calling 
Conventions}. The \code{no\_mangle} attribute turns off Rust's name mangling, so that it is easier to link to.

\subsection*{FFI and panics}

It’s important to be mindful of \code{panic!}s when working with FFI. A \panic\ across an FFI boundary is undefined behavior. 
If you’re writing code that may panic, you should run it in another thread, so that the panic doesn’t bubble up to C:

\begin{rustc}
use std::thread;

#[no_mangle]
pub extern fn oh_no() -> i32 {
    let h = thread::spawn(|| {
        panic!("Oops!");
    });

    match h.join() {
        Ok(_) => 1,
        Err(_) => 0,
    }
}
\end{rustc}

\subsection*{Representing opaque structs}

Sometimes, a C library wants to provide a pointer to something, but not let you know the internal details of the thing it 
wants. The simplest way is to use a \code{void *} argument:

\begin{verbatim}
void foo(void *arg);
void bar(void *arg);
\end{verbatim}

We can represent this in Rust with the \code{c\_void} type:

\begin{rustc}
extern crate libc;

extern "C" {
    pub fn foo(arg: *mut libc::c_void);
    pub fn bar(arg: *mut libc::c_void);
}
\end{rustc}

This is a perfectly valid way of handling the situation. However, we can do a bit better. To solve this, some C 
libraries will instead create a struct, where the details and memory layout of the \struct\ are private. This gives 
some amount of type safety. These structures are called ‘opaque’. Here’s an example, in C:

\begin{verbatim}
struct Foo; /* Foo is a structure, but its contents are not part of the public interface */
struct Bar;
void foo(struct Foo *arg);
void bar(struct Bar *arg);
\end{verbatim}

To do this in Rust, let’s create our own opaque types with \enum:

\begin{rustc}
pub enum Foo {}
pub enum Bar {}

extern "C" {
    pub fn foo(arg: *mut Foo);
    pub fn bar(arg: *mut Bar);
}
\end{rustc}

By using an \enum\ with no variants, we create an opaque type that we can’t instantiate, as it has no variants. But 
because our \code{Foo} and \code{Bar} types are different, we’ll get type safety between the two of them, so we cannot 
accidentally pass a pointer to \code{Foo} to \code{bar()}.
