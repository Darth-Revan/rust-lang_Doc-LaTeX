One important feature of Rust is that it lets us control the costs and guarantees of a program.

\blank

There are various \enquote{wrapper type} abstractions in the Rust standard library which embody a multitude of tradeoffs 
between cost, ergonomics, and guarantees. Many let one choose between run time and compile time enforcement. This section 
will explain a few selected abstractions in detail.

\blank

Before proceeding, it is highly recommended that one reads about ownership (see \nameref{sec:syntax_ownership}) and borrowing 
(see \nameref{sec:syntax_referencesBorrowing}) in Rust.

\subsection*{Basic pointer types}

\subsubsection*{\code{Box<T>}}

\href{https://doc.rust-lang.org/std/boxed/struct.Box.html}{Box<T>} is an \enquote{owned} pointer, or a \enquote{box}. While 
it can hand out references to the contained data, it is the only owner of the data. In particular, consider the following:

\begin{rustc}
let x = Box::new(1);
let y = x;
// x no longer accessible here
\end{rustc}

Here, the box was moved into \y. As \x\ no longer owns it, the compiler will no longer allow the programmer to use \x\ 
after this. A box can similarly be moved out of a function by returning it.

\blank

When a box (that hasn't been moved) goes out of scope, destructors are run. These destructors take care of deallocating 
the inner data.

\blank

This is a zero-cost abstraction for dynamic allocation. If you want to allocate some memory on the heap and safely pass 
around a pointer to that memory, this is ideal. Note that you will only be allowed to share references to this by the 
regular borrowing rules, checked at compile time.

\subsubsection*{\code{\&T} and \code{\&mut T}}

These are immutable and mutable references respectively. They follow the \enquote{read-write lock} pattern, such that one 
may either have only one mutable reference to some data, or any number of immutable ones, but not both. This guarantee is 
enforced at compile time, and has no visible cost at runtime. In most cases these two pointer types suffice for sharing 
cheap references between sections of code.

\blank

These pointers cannot be copied in such a way that they outlive the lifetime associated with them.

\subsubsection*{\code{*const T} and \code{*mut T}}

These are C-like raw pointers with no lifetime or ownership attached to them. They point to some location in memory 
with no other restrictions. The only guarantee that these provide is that they cannot be dereferenced except in code 
marked \code{unsafe}.

\blank

These are useful when building safe, low cost abstractions like \code{Vec<T>}, but should be avoided in safe code.

\subsubsection*{\code{Rc<T>}}

This is the first wrapper we will cover that has a runtime cost.

\blank

\href{https://doc.rust-lang.org/std/rc/struct.Rc.html}{Rc<T>} is a reference counted pointer. In other words, this lets 
us have multiple \enquote{owning} pointers to the same data, and the data will be dropped (destructors will be run) when 
all pointers are out of scope.

\blank

Internally, it contains a shared \enquote{reference count} (also called \enquote{refcount}), which is incremented each 
time the \code{Rc} is cloned, and decremented each time one of the \code{Rc}s goes out of scope. The main responsibility 
of \code{Rc<T>} is to ensure that destructors are called for shared data.

\blank

The internal data here is immutable, and if a cycle of references is created, the data will be leaked. If we want data 
that doesn't leak when there are cycles, we need a garbage collector.

\parag{Guarantees}

The main guarantee provided here is that the data will not be destroyed until all references to it are out of scope.

\blank

This should be used when we wish to dynamically allocate and share some data (read-only) between various portions of your 
program, where it is not certain which portion will finish using the pointer last. It's a viable alternative to \code{\&T} 
when \code{\&T} is either impossible to statically check for correctness, or creates extremely unergonomic code where the 
programmer does not wish to spend the development cost of working with.

\blank

This pointer is not thread safe, and Rust will not let it be sent or shared with other threads. This lets one avoid the 
cost of atomics in situations where they are unnecessary.

\blank

There is a sister smart pointer to this one, \code{Weak<T>}. This is a non-owning, but also non-borrowed, smart pointer. It 
is also similar to \code{\&T}, but it is not restricted in lifetime—a \code{Weak<T>} can be held on to forever. However, it 
is possible that an attempt to access the inner data may fail and return \none, since this can outlive the owned \code{Rc}s. 
This is useful for cyclic data structures and other things.

\parag{Cost}

As far as memory goes, \code{Rc<T>} is a single allocation, though it will allocate two extra words (i.e. two \code{usize} 
values) as compared to a regular \code{Box<T>} (for \enquote{strong} and \enquote{weak} refcounts).

\blank

\code{Rc<T>} has the computational cost of incrementing/decrementing the refcount whenever it is cloned or goes out of 
scope respectively. Note that a clone will not do a deep copy, rather it will simply increment the inner reference count 
and return a copy of the \code{Rc<T>}.

\subsection*{Cell types}

\code{Cell}s provide interior mutability. In other words, they contain data which can be manipulated even if the type 
cannot be obtained in a mutable form (for example, when it is behind an \code{\&}-ptr or \code{Rc<T>}).

\blank

\href{https://doc.rust-lang.org/std/cell/}{The documentation for the cell} module has a pretty good explanation for these.

\blank

These types are generally found in struct fields, but they may be found elsewhere too.

\subsubsection*{\code{Cell<T>}}

\href{https://doc.rust-lang.org/std/cell/struct.Cell.html}{Cell<T>} is a type that provides zero-cost interior mutability, 
but only for \code{Copy} types. Since the compiler knows that all the data owned by the contained value is on the stack, 
there's no worry of leaking any data behind references (or worse!) by simply replacing the data.

\blank

It is still possible to violate your own invariants using this wrapper, so be careful when using it. If a field is wrapped 
in \code{Cell}, it's a nice indicator that the chunk of data is mutable and may not stay the same between the time you first 
read it and when you intend to use it.

\begin{rustc}
use std::cell::Cell;

let x = Cell::new(1);
let y = &x;
let z = &x;
x.set(2);
y.set(3);
z.set(4);
println!("{}", x.get());
\end{rustc}

Note that here we were able to mutate the same value from various immutable references.

\blank

This has the same runtime cost as the following:

\begin{rustc}
let mut x = 1;
let y = &mut x;
let z = &mut x;
x = 2;
*y = 3;
*z = 4;
println!("{}", x);
\end{rustc}

but it has the added benefit of actually compiling successfully.

\parag{Guarantees}

This relaxes the \enquote{no aliasing with mutability} restriction in places where it's unnecessary. However, this also 
relaxes the guarantees that the restriction provides; so if your invariants depend on data stored within \code{Cell}, you 
should be careful.

\blank

This is useful for mutating primitives and other \code{Copy} types when there is no easy way of doing it in line with the 
static rules of \code{\&} and \code{\&mut}.

\blank

\code{Cell} does not let you obtain interior references to the data, which makes it safe to freely mutate.

\parag{Cost}

There is no runtime cost to using \code{Cell<T>,} however if you are using it to wrap larger (\code{Copy}) structs, 
it might be worthwhile to instead wrap individual fields in \code{Cell<T>} since each write is otherwise a full copy of 
the struct.

\subsubsection*{RefCell<T>}

\href{https://doc.rust-lang.org/std/cell/struct.RefCell.html}{RefCell<T>} also provides interior mutability, but isn't 
restricted to \code{Copy} types.

\blank

Instead, it has a runtime cost. \code{RefCell<T>} enforces the read-write lock pattern at runtime (it's like a single-threaded 
mutex), unlike \code{\&T}/\code{\&mut T} which do so at compile time. This is done by the \code{borrow()} and 
\code{borrow\_mut()} functions, which modify an internal reference count and return smart pointers which can be dereferenced 
immutably and mutably respectively. The refcount is restored when the smart pointers go out of scope. With this system, we 
can dynamically ensure that there are never any other borrows active when a mutable borrow is active. If the programmer attempts 
to make such a borrow, the thread will panic.

\begin{rustc}
use std::cell::RefCell;

let x = RefCell::new(vec![1,2,3,4]);
{
    println!("{:?}", *x.borrow())
}

{
    let mut my_ref = x.borrow_mut();
    my_ref.push(1);
}
\end{rustc}

Similar to \code{Cell}, this is mainly useful for situations where it's hard or impossible to satisfy the borrow checker. 
Generally we know that such mutations won't happen in a nested form, but it's good to check.

\blank

For large, complicated programs, it becomes useful to put some things in \code{RefCell}s to make things simpler. For example, 
a lot of the maps in \href{https://doc.rust-lang.org/rustc/middle/ty/struct.ctxt.html}{the ctxt struct} in the Rust compiler 
internals are inside this wrapper. These are only modified once (during creation, which is not right after initialization) or 
a couple of times in well-separated places. However, since this struct is pervasively used everywhere, juggling mutable and 
immutable pointers would be hard (perhaps impossible) and probably form a soup of \code{\&}-ptrs which would be hard to extend. 
On the other hand, the \code{RefCell} provides a cheap (not zero-cost) way of safely accessing these. In the future, if someone 
adds some code that attempts to modify the cell when it's already borrowed, it will cause a (usually deterministic) panic which 
can be traced back to the offending borrow.

\blank

Similarly, in Servo's DOM there is a lot of mutation, most of which is local to a DOM type, but some of which crisscrosses 
the DOM and modifies various things. Using \code{RefCell} and \code{Cell} to guard all mutation lets us avoid worrying 
about mutability everywhere, and it simultaneously highlights the places where mutation is \emph{actually} happening.

\blank

Note that \code{RefCell} should be avoided if a mostly simple solution is possible with \code{\&} pointers.

\parag{Guarantees}

\code{RefCell} relaxes the \emph{static} restrictions preventing aliased mutation, and replaces them with \emph{dynamic} 
ones. As such the guarantees have not changed.

\parag{Cost}

\code{RefCell} does not allocate, but it contains an additional \enquote{borrow state} indicator (one word in size) 
along with the data.

\blank

At runtime each borrow causes a modification/check of the refcount.

\subsection*{Synchronous types}

Many of the types above cannot be used in a threadsafe manner. Particularly, \code{Rc<T>} and \code{RefCell<T>}, which both 
use non-atomic reference counts (atomic reference counts are those which can be incremented from multiple threads without 
causing a data race), cannot be used this way. This makes them cheaper to use, but we need thread safe versions of these too. 
They exist, in the form of \code{Arc<T>} and \code{Mutex<T>}/\code{RwLock<T>}.

\blank

Note that the non-threadsafe types \emph{cannot} be sent between threads, and this is checked at compile time.

\blank

There are many useful wrappers for concurrent programming in the \href{https://doc.rust-lang.org/std/sync/}{sync} module, 
but only the major ones will be covered below.

\subsubsection*{Arc<T>}

\href{https://doc.rust-lang.org/std/sync/struct.Arc.html}{Arc<T>} is a version of \code{Rc<T>} that uses an atomic 
reference count (hence, \enquote{Arc}). This can be sent freely between threads.

\blank

C++'s \code{shared\_ptr} is similar to \code{Arc}, however in the case of C++ the inner data is always mutable. For 
semantics similar to that from C++, we should use \code{Arc<Mutex<T>>}, \code{Arc<RwLock<T>>}, or 
\code{Arc<UnsafeCell<T>>}\footnote{\code{Arc<UnsafeCell<T>>} actually won't compile since \code{UnsafeCell<T>} isn't 
\code{Send} or \code{Sync}, but we can wrap it in a type and implement \code{Send}/\code{Sync} for it manually to get 
\code{Arc<Wrapper<T>>} where \code{Wrapper} is \code{struct Wrapper<T>(UnsafeCell<T>}).} (\code{UnsafeCell<T>} is a cell 
type that can be used to hold any data and has no runtime cost, but accessing it requires \code{unsafe} blocks). The last 
one should only be used if we are certain that the usage won't cause any memory unsafety. Remember that writing to a struct 
is not an atomic operation, and many functions like \code{vec.push()} can reallocate internally and cause unsafe behavior, 
so even monotonicity may not be enough to justify \code{UnsafeCell}.

\parag{Guarantees}

Like \code{Rc}, this provides the (thread safe) guarantee that the destructor for the internal data will be run when the 
last \code{Arc} goes out of scope (barring any cycles).

\parag{Cost}

This has the added cost of using atomics for changing the refcount (which will happen whenever it is cloned or goes out 
of scope). When sharing data from an \code{Arc} in a single thread, it is preferable to share \code{\&} pointers whenever 
possible.

\subsubsection*{\code{Mutex<T>} and \code{RwLock<T>}}

\href{https://doc.rust-lang.org/std/sync/struct.Mutex.html}{Mutex<T>} and 
\href{https://doc.rust-lang.org/std/sync/struct.RwLock.html}{RwLock<T>} provide mutual-exclusion via RAII guards (guards 
are objects which maintain some state, like a lock, until their destructor is called). For both of these, the mutex is 
opaque until we call \code{lock()} on it, at which point the thread will block until a lock can be acquired, and then a 
guard will be returned. This guard can be used to access the inner data (mutably), and the lock will be released when the 
guard goes out of scope.

\begin{rustc}
{
    let guard = mutex.lock();
    // guard dereferences mutably to the inner type
    *guard += 1;
} // lock released when destructor runs
\end{rustc}

\code{RwLock} has the added benefit of being efficient for multiple reads. It is always safe to have multiple readers to 
shared data as long as there are no writers; and \code{RwLock} lets readers acquire a \enquote{read lock}. Such locks can 
be acquired concurrently and are kept track of via a reference count. Writers must obtain a \enquote{write lock} which can 
only be obtained when all readers have gone out of scope.

\parag{Guarantees}

Both of these provide safe shared mutability across threads, however they are prone to deadlocks. Some level of additional 
protocol safety can be obtained via the type system.

\parag{Costs}

These use internal atomic-like types to maintain the locks, which are pretty costly (they can block all memory reads across 
processors till they're done). Waiting on these locks can also be slow when there's a lot of concurrent access happening.

\subsection*{Composition}

A common gripe when reading Rust code is with types like \code{Rc<RefCell<Vec<T>>>} (or even more complicated compositions 
of such types). It's not always clear what the composition does, or why the author chose one like this (and when one should 
be using such a composition in one's own code)

\blank

Usually, it's a case of composing together the guarantees that you need, without paying for stuff that is unnecessary.

\blank

For example, \code{Rc<RefCell<T>>} is one such composition. \code{Rc<T>} itself can't be dereferenced mutably; because 
\code{Rc<T>} provides sharing and shared mutability can lead to unsafe behavior, so we put \code{RefCell<T>} inside to 
get dynamically verified shared mutability. Now we have shared mutable data, but it's shared in a way that there can only 
be one mutator (and no readers) or multiple readers.

\blank

Now, we can take this a step further, and have \code{Rc<RefCell<Vec<T>>>} or \code{Rc<Vec<RefCell<T>>>}. These are both 
shareable, mutable vectors, but they're not the same.

\blank

With the former, the \code{RefCell<T>} is wrapping the \code{Vec<T>}, so the \code{Vec<T>} in its entirety is mutable. 
At the same time, there can only be one mutable borrow of the whole \code{Vec} at a given time. This means that your code 
cannot simultaneously work on different elements of the vector from different \code{Rc} handles. However, we are able to 
push and pop from the \code{Vec<T>} at will. This is similar to a \code{\&mut Vec<T>} with the borrow checking done at runtime.

\blank

With the latter, the borrowing is of individual elements, but the overall vector is immutable. Thus, we can independently 
borrow separate elements, but we cannot push or pop from the vector. This is similar to a \code{\&mut [T]}\footnote{\code{\&[T]} 
and \code{\&mut [T]} are slices; they consist of a pointer and a length and can refer to a portion of a vector or array. 
\code{\&mut [T]} can have its elements mutated, however its length cannot be touched.}, but, again, the borrow checking is 
at runtime.

\blank

In concurrent programs, we have a similar situation with \code{Arc<Mutex<T>>}, which provides shared mutability and ownership.

\blank

When reading code that uses these, go in step by step and look at the guarantees/costs provided.

\blank

When choosing a composed type, we must do the reverse; figure out which guarantees we want, and at which point of the 
composition we need them. For example, if there is a choice between \code{Vec<RefCell<T>>} and \code{RefCell<Vec<T>>}, we 
should figure out the tradeoffs as done above and pick one.
