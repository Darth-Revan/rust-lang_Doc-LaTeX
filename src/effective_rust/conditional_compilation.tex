Rust has a special attribute, \code{\#[cfg]}, which allows you to compile code based on a flag passed to the compiler. It has two forms:

\begin{rustc}
#[cfg(foo)]

#[cfg(bar = "baz")]
\end{rustc}

They also have some helpers:

\begin{rustc}
#[cfg(any(unix, windows))]

#[cfg(all(unix, target_pointer_width = "32"))]

#[cfg(not(foo))]
\end{rustc}

These can nest arbitrarily:

\begin{rustc}
#[cfg(any(not(unix), all(target_os="macos", target_arch = "powerpc")))]
\end{rustc}

As for how to enable or disable these switches, if you’re using Cargo, they get set in the 
\href{http://doc.crates.io/manifest.html\#the-features-section}{[features] section} of your \code{Cargo.toml}:

\begin{verbatim}
[features]
# no features by default
default = []

# The “secure-password” feature depends on the bcrypt package.
secure-password = ["bcrypt"]
\end{verbatim}

When you do this, Cargo passes along a flag to \code{rustc}:

\begin{verbatim}
--cfg feature="${feature_name}"
\end{verbatim}

The sum of these \code{cfg} flags will determine which ones get activated, and therefore, which code gets compiled. Let’s take this code:

\begin{rustc}
#[cfg(feature = "foo")]
mod foo {
}
\end{rustc}

If we compile it with \code{cargo build --features "foo"}, it will send the \code{--cfg feature="foo"} flag to \code{rustc}, and the 
output will have the \code{mod foo} in it. If we compile it with a regular \code{cargo build}, no extra flags get passed on, and so, 
no \code{foo} module will exist.

\subsection*{cfg\_attr}

You can also set another attribute based on a \code{cfg} variable with \code{cfg\_attr}:

\begin{rustc}
#[cfg_attr(a, b)]
\end{rustc}

Will be the same as \code{\#[b]} if a is set by \code{cfg} attribute, and nothing otherwise.

\subsection*{cfg!}

The \code{cfg!} syntax extension (see \nameref{sec:nightly_compilerPlugins}) lets you use these kinds of flags elsewhere in your code, too:

\begin{rustc}
if cfg!(target_os = "macos") || cfg!(target_os = "ios") {
    println!("Think Different!");
}
\end{rustc}

These will be replaced by a \code{true} or \code{false} at compile-time, depending on the configuration settings.
