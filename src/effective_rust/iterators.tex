Let's talk about loops.

\blank

Remember Rust's \code{for} loop? Here's an example:

\begin{rustc}
for x in 0..10 {
    println!("{}", x);
}
\end{rustc}

Now that you know more Rust, we can talk in detail about how this works. Ranges (the \code{0..10}) are 'iterators'. An iterator is 
something that we can call the \code{.next()} method on repeatedly, and it gives us a sequence of things.

\blank

Like this:

\begin{rustc}
let mut range = 0..10;

loop {
    match range.next() {
        Some(x) => {
            println!("{}", x);
        },
        None => { break }
    }
}
\end{rustc}

We make a mutable binding to the range, which is our iterator. We then \code{loop}, with an inner \code{match}. This \code{match} is 
used on the result of \code{range.next()}, which gives us a reference to the next value of the iterator. \code{next} returns an 
\code{Option<i32>}, in this case, which will be \code{Some(i32)} when we have a value and \code{None} once we run out. If we get 
\code{Some(i32)}, we print it out, and if we get \code{None}, we \code{break} out of the loop.

\blank

This code sample is basically the same as our \code{for} loop version. The \code{for} loop is a handy way to write this 
\code{loop}/\code{match}/\code{break} construct.

\blank

\code{for} loops aren't the only thing that uses iterators, however. Writing your own iterator involves implementing the \code{Iterator} 
trait. While doing that is outside of the scope of this guide, Rust provides a number of useful iterators to accomplish various tasks. 
But first, a few notes about limitations of ranges.

\blank

Ranges are very primitive, and we often can use better alternatives. Consider the following Rust anti-pattern: using ranges to emulate a 
C-style \code{for} loop. Let’s suppose you needed to iterate over the contents of a vector. You may be tempted to write this:

\begin{rustc}
let nums = vec![1, 2, 3];

for i in 0..nums.len() {
    println!("{}", nums[i]);
}
\end{rustc}

This is strictly worse than using an actual iterator. You can iterate over vectors directly, so write this:

\begin{rustc}
let nums = vec![1, 2, 3];

for num in &nums {
    println!("{}", num);
}
\end{rustc}

There are two reasons for this. First, this more directly expresses what we mean. We iterate through the entire vector, rather than 
iterating through indexes, and then indexing the vector. Second, this version is more efficient: the first version will have extra 
bounds checking because it used indexing, \code{nums[i]}. But since we yield a reference to each element of the vector in turn with 
the iterator, there's no bounds checking in the second example. This is very common with iterators: we can ignore unnecessary bounds 
checks, but still know that we're safe.

\blank

There's another detail here that's not 100\% clear because of how \println\ works. \code{num} is actually of type \code{\&i32}. That is, 
it's a reference to an \itt, not an \itt\ itself. \println\ handles the dereferencing for us, so we don't see it. This code works fine too:

\begin{rustc}
let nums = vec![1, 2, 3];

for num in &nums {
    println!("{}", *num);
}
\end{rustc}

Now we're explicitly dereferencing \code{num}. Why does \code{\&nums} give us references? Firstly, because we explicitly asked it to 
with \code{\&}. Secondly, if it gave us the data itself, we would have to be its owner, which would involve making a copy of the data 
and giving us the copy. With references, we're only borrowing a reference to the data, and so it's only passing a reference, without 
needing to do the move.

\blank

So, now that we've established that ranges are often not what you want, let's talk about what you do want instead.

\blank

There are three broad classes of things that are relevant here: iterators, \emph{iterator adaptors}, and \emph{consumers}. Here's 
some definitions:

\begin{itemize}
  \item{\emph{iterators} give you a sequence of values.}
  \item{\emph{iterator adaptors} operate on an iterator, producing a new iterator with a different output sequence.}
  \item{\emph{consumers} operate on an iterator, producing some final set of values.}
\end{itemize}

Let's talk about consumers first, since you've already seen an iterator, ranges.

\subsection*{Consumers}

A \emph{consumer} operates on an iterator, returning some kind of value or values. The most common consumer is \code{collect()}. This 
code doesn't quite compile, but it shows the intention:

\begin{rustc}
let one_to_one_hundred = (1..101).collect();
\end{rustc}

As you can see, we call \code{collect()} on our iterator. \code{collect()} takes as many values as the iterator will give it, 
and returns a collection of the results. So why won't this compile? Rust can't determine what type of things you want to collect, 
and so you need to let it know. Here's the version that does compile:

\begin{rustc}
let one_to_one_hundred = (1..101).collect::<Vec<i32>>();
\end{rustc}

If you remember, the \code{::<>} syntax allows us to give a type hint, and so we tell it that we want a vector of integers. You 
don't always need to use the whole type, though. Using a \code{\_} will let you provide a partial hint:

\begin{rustc}
let one_to_one_hundred = (1..101).collect::<Vec<_>>();
\end{rustc}

This says \enquote{Collect into a \code{Vec<T>}, please, but infer what the \code{T} is for me.} \code{\_} is sometimes called a 
\enquote{type placeholder} for this reason.

\blank

\code{collect()} is the most common consumer, but there are others too. \code{find()} is one:

\begin{rustc}
let greater_than_forty_two = (0..100)
                             .find(|x| *x > 42);

match greater_than_forty_two {
    Some(_) => println!("Found a match!"),
    None => println!("No match found :("),
}
\end{rustc}

\code{find} takes a closure, and works on a reference to each element of an iterator. This closure returns \code{true} if the 
element is the element we're looking for, and \code{false} otherwise. \code{find} returns the first element satisfying the 
specified predicate. Because we might not find a matching element, \code{find} returns an \code{Option} rather than the element itself.

\blank

Another important consumer is \code{fold}. Here's what it looks like:

\begin{rustc}
let sum = (1..4).fold(0, |sum, x| sum + x);
\end{rustc}

\code{fold()} is a consumer that looks like this: \code{fold(base, |accumulator, element| ...)}. It takes two arguments: the first 
is an element called the \emph{base}. The second is a closure that itself takes two arguments: the first is called the \emph{accumulator}, 
and the second is an \emph{element}. Upon each iteration, the closure is called, and the result is the value of the accumulator on the 
next iteration. On the first iteration, the base is the value of the accumulator.

\blank

Okay, that's a bit confusing. Let's examine the values of all of these things in this iterator:

\begin{table}[H]
  \begin{tabular}{|l|l|l|l|}
  \hline
  \textbf{base} & \textbf{accumulator} & \textbf{element} & \textbf{closure result} \\
  \hline
  0 & 0 & 1 & 1 \\
  \hline
  0 & 1 & 2 & 3 \\
  \hline
  0 & 3 & 3 & 6 \\
  \hline
  \end{tabular}
\end{table}

We called \code{fold()} with these arguments:

\begin{rustc}
.fold(0, |sum, x| sum + x);
\end{rustc}

So, \code{0} is our base, \code{sum} is our accumulator, and \x\ is our element. On the first iteration, we set \code{sum} to \code{0}, 
and \x\ is the first element of \code{nums}, \code{1}. We then add \code{sum} and \x, which gives us \code{0 + 1 = 1}. On the second 
iteration, that value becomes our accumulator, \code{sum}, and the element is the second element of the array, \code{2}. \code{1 + 2 = 3}, 
and so that becomes the value of the accumulator for the last iteration. On that iteration, \x\ is the last element, \code{3}, and 
\code{3 + 3 = 6}, which is our final result for our \code{sum}. \code{1 + 2 + 3 = 6}, and that's the result we got.

\blank

Whew. \code{fold} can be a bit strange the first few times you see it, but once it clicks, you can use it all over the place. 
Any time you have a list of things, and you want a single result, \code{fold} is appropriate.

\blank

Consumers are important due to one additional property of iterators we haven't talked about yet: laziness. Let's talk some more 
about iterators, and you'll see why consumers matter.

\subsection*{Iterators}

As we've said before, an iterator is something that we can call the \code{.next()} method on repeatedly, and it gives us a sequence 
of things. Because you need to call the method, this means that iterators can be \emph{lazy} and not generate all of the values upfront. 
This code, for example, does not actually generate the numbers \code{1-99}, instead creating a value that merely represents the sequence:

\begin{rustc}
let nums = 1..100;
\end{rustc}

Since we didn't do anything with the range, it didn't generate the sequence. Let's add the consumer:

\begin{rustc}
let nums = (1..100).collect::<Vec<i32>>();
\end{rustc}

Now, \code{collect()} will require that the range gives it some numbers, and so it will do the work of generating the sequence.

\blank

Ranges are one of two basic iterators that you'll see. The other is \code{iter()}. \code{iter()} can turn a vector into a simple 
iterator that gives you each element in turn:

\begin{rustc}
let nums = vec![1, 2, 3];

for num in nums.iter() {
   println!("{}", num);
}
\end{rustc}

These two basic iterators should serve you well. There are some more advanced iterators, including ones that are infinite.

\blank

That's enough about iterators. Iterator adaptors are the last concept we need to talk about with regards to iterators. Let's get to it!

\subsection*{Iterator adaptors}

\emph{Iterator adaptors} take an iterator and modify it somehow, producing a new iterator. The simplest one is called \code{map}:

\begin{rustc}
(1..100).map(|x| x + 1);
\end{rustc}

\code{map} is called upon another iterator, and produces a new iterator where each element reference has the closure it's been given 
as an argument called on it. So this would give us the numbers from \code{2-100}. Well, almost! If you compile the example, you'll 
get a warning:

\begin{verbatim}
warning: unused result which must be used: iterator adaptors are lazy and
         do nothing unless consumed, #[warn(unused_must_use)] on by default
(1..100).map(|x| x + 1);
 ^~~~~~~~~~~~~~~~~~~~~~~~~~~~~~~~
\end{verbatim}

Laziness strikes again! That closure will never execute. This example doesn't print any numbers:

\begin{rustc}
(1..100).map(|x| println!("{}", x));
\end{rustc}

If you are trying to execute a closure on an iterator for its side effects, use \code{for} instead.

\blank

There are tons of interesting iterator adaptors. \code{take(n)} will return an iterator over the next \code{n} elements of the 
original iterator. Let's try it out with an infinite iterator:

\begin{rustc}
for i in (1..).take(5) {
    println!("{}", i);
}
\end{rustc}

This will print

\begin{verbatim}
1
2
3
4
5
\end{verbatim}

\code{filter()} is an adapter that takes a closure as an argument. This closure returns \code{true} or \code{false}. The new iterator 
\code{filter()} produces only the elements that the closure returns \code{true} for:

\begin{rustc}
for i in (1..100).filter(|&x| x % 2 == 0) {
    println!("{}", i);
}
\end{rustc}

This will print all of the even numbers between one and a hundred. (Note that because \code{filter} doesn't consume the 
elements that are being iterated over, it is passed a reference to each element, and thus the filter predicate uses the 
\code{\&x} pattern to extract the integer itself.)

\blank

You can chain all three things together: start with an iterator, adapt it a few times, and then consume the result. Check it out:

\begin{rustc}
(1..)
    .filter(|&x| x % 2 == 0)
    .filter(|&x| x % 3 == 0)
    .take(5)
    .collect::<Vec<i32>>();
\end{rustc}

This will give you a vector containing \code{6}, \code{12}, \code{18}, \code{24}, and \code{30}.

\blank

This is just a small taste of what iterators, iterator adaptors, and consumers can help you with. There are a number of really 
useful iterators, and you can write your own as well. Iterators provide a safe, efficient way to manipulate all kinds of lists. 
They're a little unusual at first, but if you play with them, you'll get hooked. For a full list of the different iterators and 
consumers, check out the \href{https://doc.rust-lang.org/std/iter/}{iterator module documentation}.
