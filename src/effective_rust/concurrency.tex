Concurrency and parallelism are incredibly important topics in computer science, and are also a hot topic in industry today. 
Computers are gaining more and more cores, yet many programmers aren't prepared to fully utilize them.

\blank

Rust's memory safety features also apply to its concurrency story too. Even concurrent Rust programs must be memory safe, having 
no data races. Rust's type system is up to the task, and gives you powerful ways to reason about concurrent code at compile time.

\blank

Before we talk about the concurrency features that come with Rust, it's important to understand something: Rust is low-level enough 
that the vast majority of this is provided by the standard library, not by the language. This means that if you don't like some aspect 
of the way Rust handles concurrency, you can implement an alternative way of doing things. \href{https://github.com/carllerche/mio}{mio} 
is a real-world example of this principle in action.

\subsubsection*{Background: \code{Send} and \code{Sync}}

Concurrency is difficult to reason about. In Rust, we have a strong, static type system to help us reason about our code. As such, 
Rust gives us two traits to help us make sense of code that can possibly be concurrent.

\myparagraph{\code{Send}}

The first trait we're going to talk about is \href{https://doc.rust-lang.org/std/marker/trait.Send.html}{Send}. When a type \code{T} 
implements \code{Send}, it indicates that something of this type is able to have ownership transferred safely between threads.

\blank

This is important to enforce certain restrictions. For example, if we have a channel connecting two threads, we would want to be 
able to send some data down the channel and to the other thread. Therefore, we'd ensure that \code{Send} was implemented for that type.

\blank

% TODO make FFI a hyperref
In the opposite way, if we were wrapping a library with FFI that isn't threadsafe, we wouldn't want to implement \code{Send}, 
and so the compiler will help us enforce that it can't leave the current thread.

\myparagraph{\code{Sync}}

% TODO make interior mutability a hyperref
The second of these traits is called \href{https://doc.rust-lang.org/std/marker/trait.Sync.html}{Sync}. When a type \code{T} 
implements \code{Sync}, it indicates that something of this type has no possibility of introducing memory unsafety when used from 
multiple threads concurrently through shared references. This implies that types which don't have interior mutability are inherently 
\code{Sync}, which includes simple primitive types (like \code{u8}) and aggregate types containing them.

\blank

For sharing references across threads, Rust provides a wrapper type called \code{Arc<T>}. \code{Arc<T>} implements \code{Send} and 
\code{Sync} if and only if \code{T} implements both \code{Send} and \code{Sync}. For example, an object of type \code{Arc<RefCell<U>>} 
cannot be transferred across threads because \code{RefCell} does not implement \code{Sync}, consequently \code{Arc<RefCell<U>>} would 
not implement \code{Send}.

\blank

These two traits allow you to use the type system to make strong guarantees about the properties of your code under concurrency. 
Before we demonstrate why, we need to learn how to create a concurrent Rust program in the first place!

\subsubsection*{Threads}

Rust's standard library provides a library for threads, which allow you to run Rust code in parallel. Here's a basic example of using 
\code{std::thread}:

\begin{rustc}
use std::thread;

fn main() {
    thread::spawn(|| {
        println!("Hello from a thread!");
    });
}
\end{rustc}

The \code{thread::spawn()} method accepts a closure (see \nameref{sec:syntax_closures}), which is executed in a new thread. It returns 
a handle to the thread, that can be used to wait for the child thread to finish and extract its result:

\begin{rustc}
use std::thread;

fn main() {
    let handle = thread::spawn(|| {
        "Hello from a thread!"
    });

    println!("{}", handle.join().unwrap());
}
\end{rustc}

Many languages have the ability to execute threads, but it's wildly unsafe. There are entire books about how to prevent errors that 
occur from shared mutable state. Rust helps out with its type system here as well, by preventing data races at compile time. Let's 
talk about how you actually share things between threads.

\subsubsection*{Safe Shared Mutable State}

Due to Rust's type system, we have a concept that sounds like a lie: \enquote{safe shared mutable state.} Many programmers agree 
that shared mutable state is very, very bad.

\blank

Someone once said this:

\begin{myquote}
Shared mutable state is the root of all evil. Most languages attempt to deal with this problem through the 'mutable' part, but 
Rust deals with it by solving the 'shared' part.
\end{myquote}

The same ownership system (see \nameref{sec:syntax_ownership}) that helps prevent using pointers incorrectly also helps rule out 
data races, one of the worst kinds of concurrency bugs.

\blank

As an example, here is a Rust program that would have a data race in many languages. It will not compile:

\begin{rustc}
use std::thread;
use std::time::Duration;

fn main() {
    let mut data = vec![1, 2, 3];

    for i in 0..3 {
        thread::spawn(move || {
            data[i] += 1;
        });
    }

    thread::sleep(Duration::from_millis(50));
}
\end{rustc}

This gives us an error:

\begin{verbatim}
8:17 error: capture of moved value: `data`
        data[i] += 1;
        ^~~~
\end{verbatim}

Rust knows this wouldn't be safe! If we had a reference to \code{data} in each thread, and the thread takes ownership of the reference, 
we'd have three owners!

\blank

So, we need some type that lets us have more than one reference to a value and that we can share between threads, that is it must 
implement \code{Sync}.

\blank

We'll use \code{Arc<T>}, Rust's standard atomic reference count type, which wraps a value up with some extra runtime bookkeeping 
which allows us to share the ownership of the value between multiple references at the same time.

\blank

The bookkeeping consists of a count of how many of these references exist to the value, hence the reference count part of the name.

\blank

The Atomic part means \code{Arc<T>} can safely be accessed from multiple threads. To do this the compiler guarantees that mutations 
of the internal count use indivisible operations which can't have data races.

\begin{rustc}
use std::thread;
use std::sync::Arc;
use std::time::Duration;

fn main() {
    let mut data = Arc::new(vec![1, 2, 3]);

    for i in 0..3 {
        let data = data.clone();
        thread::spawn(move || {
            data[i] += 1;
        });
    }

    thread::sleep(Duration::from_millis(50));
}
\end{rustc}

We now call \code{clone()} on our \code{Arc<T>}, which increases the internal count. This handle is then moved into the new thread.

\blank

And... still gives us an error.

\begin{verbatim}
<anon>:11:24 error: cannot borrow immutable borrowed content as mutable
<anon>:11                    data[i] += 1;
                             ^~~~
\end{verbatim}

\code{Arc<T>} assumes one more property about its contents to ensure that it is safe to share across threads: it assumes its contents 
are \code{Sync}. This is true for our value if it's immutable, but we want to be able to mutate it, so we need something else to persuade 
the borrow checker we know what we're doing.

\blank

It looks like we need some type that allows us to safely mutate a shared value, for example a type that can ensure only one thread 
at a time is able to mutate the value inside it at any one time.

\blank

For that, we can use the \code{Mutex<T>0 type! Here's the working version:

\begin{rustc}
use std::sync::{Arc, Mutex};
use std::thread;
use std::time::Duration;

fn main() {
    let data = Arc::new(Mutex::new(vec![1, 2, 3]));

    for i in 0..3 {
        let data = data.clone();
        thread::spawn(move || {
            let mut data = data.lock().unwrap();
            data[i] += 1;
        });
    }

    thread::sleep(Duration::from_millis(50));
}
\end{rustc}

Note that the value of \code{i} is bound (copied) to the closure and not shared among the threads.

\blank

Also note that \href{https://doc.rust-lang.org/std/sync/struct.Mutex.html#method.lock}{lock} method of 
\href{https://doc.rust-lang.org/std/sync/struct.Mutex.html}{Mutex} has this signature:

\begin{rustc}
fn lock(&self) -> LockResult<MutexGuard<T>>
\end{rustc}

and because \code{Send} is not implemented for \code{MutexGuard<T>}, the guard cannot cross thread boundaries, ensuring 
thread-locality of lock acquire and release.

\blank

Let's examine the body of the thread more closely:

\begin{rustc}
thread::spawn(move || {
    let mut data = data.lock().unwrap();
    data[i] += 1;
});
\end{rustc}

First, we call \code{lock()}, which acquires the mutex's lock. Because this may fail, it returns an \code{Result<T, E>}, and because 
this is just an example, we \code{unwrap()} it to get a reference to the data. Real code would have more robust error handling here. 
We're then free to mutate it, since we have the lock.

\blank

Lastly, while the threads are running, we wait on a short timer. But this is not ideal: we may have picked a reasonable amount of 
time to wait but it's more likely we'll either be waiting longer than necessary or not long enough, depending on just how much time 
the threads actually take to finish computing when the program runs.

\blank

A more precise alternative to the timer would be to use one of the mechanisms provided by the Rust standard library for synchronizing 
threads with each other. Let's talk about one of them: channels.

\subsubsection*{Channels}

Here's a version of our code that uses channels for synchronization, rather than waiting for a specific time:

\begin{rustc}
use std::sync::{Arc, Mutex};
use std::thread;
use std::sync::mpsc;

fn main() {
    let data = Arc::new(Mutex::new(0));

    let (tx, rx) = mpsc::channel();

    for _ in 0..10 {
        let (data, tx) = (data.clone(), tx.clone());

        thread::spawn(move || {
            let mut data = data.lock().unwrap();
            *data += 1;

            tx.send(()).unwrap();
        });
    }

    for _ in 0..10 {
        rx.recv().unwrap();
    }
}
\end{rustc}

We use the \code{mpsc::channel()} method to construct a new channel. We \code{send} a simple \code{()} down the channel, and then 
wait for ten of them to come back.

\blank

While this channel is sending a generic signal, we can send any data that is \code{Send} over the channel!

\begin{rustc}
use std::thread;
use std::sync::mpsc;

fn main() {
    let (tx, rx) = mpsc::channel();

    for i in 0..10 {
        let tx = tx.clone();

        thread::spawn(move || {
            let answer = i * i;

            tx.send(answer).unwrap();
        });
    }

    for _ in 0..10 {
        println!("{}", rx.recv().unwrap());
    }
}
\end{rustc}

Here we create 10 threads, asking each to calculate the square of a number (\code{i} at the time of \code{spawn()}), and then \code{send()} 
back the answer over the channel.

\subsubsection*{Panics}

A \code{panic!} will crash the currently executing thread. You can use Rust's threads as a simple isolation mechanism:

\begin{rustc}
use std::thread;

let handle = thread::spawn(move || {
    panic!("oops!");
});

let result = handle.join();

assert!(result.is_err());

\end{rustc}

\code{Thread.join()} gives us a \code{Result} back, which allows us to check if the thread has panicked or not.
