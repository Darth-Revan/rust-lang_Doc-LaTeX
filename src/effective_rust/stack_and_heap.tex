As a systems language, Rust operates at a low level. If you're coming from a high-level language, there are some aspects of systems
programming that you may not be familiar with. The most important one is how memory works, with a stack and a heap. If you're familiar
with how C-like languages use stack allocation, this chapter will be a refresher. If you're not, you'll learn about this more general
concept, but with a Rust-y focus.

\blank

As with most things, when learning about them, we'll use a simplified model to start. This lets you get a handle on the basics, without
getting bogged down with details which are, for now, irrelevant. The examples we'll use aren't 100\% accurate, but are representative for
the level we're trying to learn at right now. Once you have the basics down, learning more about how allocators are implemented, virtual
memory, and other advanced topics will reveal the leaks in this particular abstraction.

\subsubsection*{Memory management}

These two terms are about memory management. The stack and the heap are abstractions that help you determine when to allocate and
deallocate memory.

\blank

Here's a high-level comparison:

\blank

The stack is very fast, and is where memory is allocated in Rust by default. But the allocation is local to a function call, and
is limited in size. The heap, on the other hand, is slower, and is explicitly allocated by your program. But it's effectively
unlimited in size, and is globally accessible.

\subsubsection*{The Stack}

Let's talk about this Rust program:

\begin{rustc}
fn main() {
    let x = 42;
}
\end{rustc}

This program has one variable binding, \x. This memory needs to be allocated from somewhere. Rust 'stack allocates' by default, which means that basic values 'go on the stack'. What does that mean?

\blank

Well, when a function gets called, some memory gets allocated for all of its local variables and some other information. This is called a 
'stack frame', and for the purpose of this tutorial, we're going to ignore the extra information and only consider the local variables 
we're allocating. So in this case, when \code{main()} is run, we'll allocate a single 32-bit integer for our stack frame. This is 
automatically handled for you, as you can see; we didn't have to write any special Rust code or anything.

\blank

When the function exits, its stack frame gets deallocated. This happens automatically as well.

\blank

That's all there is for this simple program. The key thing to understand here is that stack allocation is very, very fast. Since we know 
all the local variables we have ahead of time, we can grab the memory all at once. And since we'll throw them all away at the same time as 
well, we can get rid of it very fast too.

\blank

The downside is that we can't keep values around if we need them for longer than a single function. We also haven't talked about what 
the word, 'stack', means. To do that, we need a slightly more complicated example:

\begin{rustc}
fn foo() {
    let y = 5;
    let z = 100;
}

fn main() {
    let x = 42;

    foo();
}
\end{rustc}

This program has three variables total: two in \code{foo()}, one in \code{main()}. Just as before, when \code{main()} is called, a single 
integer is allocated for its stack frame. But before we can show what happens when \code{foo()} is called, we need to visualize what's 
going on with memory. Your operating system presents a view of memory to your program that's pretty simple: a huge list of addresses, from 
0 to a large number, representing how much RAM your computer has. For example, if you have a gigabyte of RAM, your addresses go from \code{0} 
to \code{1,073,741,823}. That number comes from $2^{30}$, the number of bytes in a gigabyte.\footnote{'Gigabyte' can mean two things: 
$10^{9}$, or $2^{30}$. The SI standard resolved this by stating that 'gigabyte' is $10^{9}$, and 'gibibyte' is $2^{30}$. However, very 
few people use this terminology, and rely on context to differentiate. We follow in that tradition here.}

\blank

This memory is kind of like a giant array: addresses start at zero and go up to the final number. So here's a diagram of our first 
stack frame:

\begin{table}[H]
  \begin{tabular}{|l|l|l|}
    \hline
    \textbf{Address} & \textbf{Name} & \textbf{Value} \\
    \hline
    0 & x & 42 \\
    \hline
  \end{tabular}
\end{table}

We've got \x\ located at address \code{0}, with the value \code{42}.

When \code{foo()} is called, a new stack frame is allocated:

\begin{table}[H]
  \begin{tabular}{|l|l|l|}
    \hline
    \textbf{Address} & \textbf{Name} & \textbf{Value} \\
    \hline
    2 & z & 100 \\
    \hline
    1 & y & 5 \\
    \hline
    0 & x & 42 \\
    \hline
  \end{tabular}
\end{table}

Because \code{0} was taken by the first frame, \code{1} and \code{2} are used for \code{foo()}'s stack frame. It grows upward, the more 
functions we call.

\blank

There are some important things we have to take note of here. The numbers 0, 1, and 2 are all solely for illustrative purposes, and 
bear no relationship to the address values the computer will use in reality. In particular, the series of addresses are in reality 
going to be separated by some number of bytes that separate each address, and that separation may even exceed the size of the value 
being stored.

\blank

After \code{foo()} is over, its frame is deallocated:

\begin{table}[H]
  \begin{tabular}{|l|l|l|}
    \hline
    \textbf{Address} & \textbf{Name} & \textbf{Value} \\
    \hline
    0 & x & 42 \\
    \hline
  \end{tabular}
\end{table}

And then, after \code{main()}, even this last value goes away. Easy!

\blank

It's called a 'stack' because it works like a stack of dinner plates: the first plate you put down is the last plate to pick back up. 
Stacks are sometimes called 'last in, first out queues' for this reason, as the last value you put on the stack is the first one you 
retrieve from it.

\blank

Let's try a three-deep example:

\begin{rustc}
fn italic() {
    let i = 6;
}

fn bold() {
    let a = 5;
    let b = 100;
    let c = 1;

    italic();
}

fn main() {
    let x = 42;

    bold();
}
\end{rustc}

We have some kooky function names to make the diagrams clearer.

\blank

Okay, first, we call \code{main()}:

\begin{table}[H]
  \begin{tabular}{|l|l|l|}
    \hline
    \textbf{Address} & \textbf{Name} & \textbf{Value} \\
    \hline
    0 & x & 42 \\
    \hline
  \end{tabular}
\end{table}

Next up, \code{main()} calls \code{bold()}:

\begin{table}[H]
  \begin{tabular}{|l|l|l|}
    \hline
    \textbf{Address} & \textbf{Name} & \textbf{Value} \\
    \hline
    \textbf{3} & \textbf{c} & \textbf{1} \\
    \hline
    \textbf{2} & \textbf{b} & \textbf{100} \\
    \hline
    \textbf{1} & \textbf{a} & \textbf{5} \\
    \hline
    0 & x & 42 \\
    \hline
  \end{tabular}
\end{table}

And then \code{bold()} calls \code{italic()}:

\begin{table}[H]
  \begin{tabular}{|l|l|l|}
    \hline
    \textbf{Address} & \textbf{Name} & \textbf{Value} \\
    \hline
    \textit{4} & \textit{i} & \textit{6} \\
    \hline
    \textbf{3} & \textbf{c} & \textbf{1} \\
    \hline
    \textbf{2} & \textbf{b} & \textbf{100} \\
    \hline
    \textbf{1} & \textbf{a} & \textbf{5} \\
    \hline
    0 & x & 42 \\
    \hline
  \end{tabular}
\end{table}

Whew! Our stack is growing tall.

\blank

After \code{italic()} is over, its frame is deallocated, leaving only \code{bold()} and \code{main()}:

\begin{table}[H]
  \begin{tabular}{|l|l|l|}
    \hline
    \textbf{Address} & \textbf{Name} & \textbf{Value} \\
    \hline
    \textbf{3} & \textbf{c} & \textbf{1} \\
    \hline
    \textbf{2} & \textbf{b} & \textbf{100} \\
    \hline
    \textbf{1} & \textbf{a} & \textbf{5} \\
    \hline
    0 & x & 42 \\
    \hline
  \end{tabular}
\end{table}

And then \code{bold()} ends, leaving only \code{main()}:

\begin{table}[H]
  \begin{tabular}{|l|l|l|}
    \hline
    \textbf{Address} & \textbf{Name} & \textbf{Value} \\
    \hline
    0 & x & 42 \\
    \hline
  \end{tabular}
\end{table}

And then we're done. Getting the hang of it? It's like piling up dishes: you add to the top, you take away from the top.


\subsubsection*{The Heap}

Now, this works pretty well, but not everything can work like this. Sometimes, you need to pass some memory between different 
functions, or keep it alive for longer than a single function's execution. For this, we can use the heap.

\blank

In Rust, you can allocate memory on the heap with the \href{https://doc.rust-lang.org/std/boxed/}{Box<T> type}. Here's an example:

\begin{rustc}
fn main() {
    let x = Box::new(5);
    let y = 42;
}
\end{rustc}

Here's what happens in memory when \code{main()} is called:

\begin{table}[H]
  \begin{tabular}{|l|l|l|}
    \hline
    \textbf{Address} & \textbf{Name} & \textbf{Value} \\
    \hline
    1 & y & 42 \\
    \hline
    0 & x & ?????? \\
    \hline
  \end{tabular}
\end{table}

We allocate space for two variables on the stack. \y\ is \code{42}, as it always has been, but what about \x? Well, \x\ is a 
\code{Box<i32>}, and boxes allocate memory on the heap. The actual value of the box is a structure which has a pointer to 'the heap'. 
When we start executing the function, and \code{Box::new()} is called, it allocates some memory for the heap, and puts \code{5} there. 
The memory now looks like this:

\begin{table}[H]
  \begin{tabular}{|l|l|l|}
    \hline
    \textbf{Address} & \textbf{Name} & \textbf{Value} \\
    \hline
    ($2^{30}$) - 1 & & 5 \\
    \hline
    \ldots & \ldots & \ldots \\
    \hline
    1 & y & 42 \\
    \hline
    0 & x & 42 \\
    \hline
  \end{tabular}
\end{table}

% TODO make raw pointer a hyperref
We have ($2^{30}$) - 1 addresses in our hypothetical computer with 1GB of RAM. And since our stack grows from zero, the easiest place 
to allocate memory is from the other end. So our first value is at the highest place in memory. And the value of the struct at \x\ has a 
raw pointer to the place we've allocated on the heap, so the value of \x\ is ($2^{30}$) - 1, the memory location we've asked for.

\blank

We haven't really talked too much about what it actually means to allocate and deallocate memory in these contexts. Getting into very 
deep detail is out of the scope of this tutorial, but what's important to point out here is that the heap isn't a stack that grows from 
the opposite end. We'll have an example of this later in the book, but because the heap can be allocated and freed in any order, it can 
end up with 'holes'. Here's a diagram of the memory layout of a program which has been running for a while now:

\begin{table}[H]
  \begin{tabular}{|l|l|l|}
    \hline
    \textbf{Address} & \textbf{Name} & \textbf{Value} \\
    \hline
    ($2^{30}$) - 1 & & 5 \\
    \hline
    ($2^{30}$) - 2 & & \\
    \hline
    ($2^{30}$) - 3 & & \\
    \hline
    ($2^{30}$) - 4 & & 40 \\
    \hline
    \ldots & \ldots & \ldots \\
    \hline
    3 & y & $\rightarrow(2^{30})$-4 \\
    \hline
    2 & y & 42 \\
    \hline
    1 & y & 42 \\
    \hline
    0 & x & $\rightarrow(2^{30})$-1 \\
    \hline
  \end{tabular}
\end{table}

In this case, we've allocated four things on the heap, but deallocated two of them. There's a gap between ($2^{30}$) - 1 and ($2^{30}$) - 4 
which isn't currently being used. The specific details of how and why this happens depends on what kind of strategy you use to manage the 
heap. Different programs can use different 'memory allocators', which are libraries that manage this for you. Rust programs use 
\href{http://www.canonware.com/jemalloc/}{jemalloc} for this purpose.

\blank

Anyway, back to our example. Since this memory is on the heap, it can stay alive longer than the function which allocates the box. In 
this case, however, it doesn't.\footnote{We can make the memory live longer by transferring ownership, sometimes called 'moving out of 
the box'. More complex examples will be covered later.} When the function is over, we need to free the stack frame for \code{main()}. 
\code{Box<T>}, though, has a trick up its sleeve: Drop (see \nameref{sec:syntax_drop}). The implementation of \code{Drop} for \code{Box}
deallocates the memory that was allocated when it was created. Great! So when \x\ goes away, it first frees the memory allocated on the 
heap:

\begin{table}[H]
  \begin{tabular}{|l|l|l|}
    \hline
    \textbf{Address} & \textbf{Name} & \textbf{Value} \\
    \hline
    1 & y & 42 \\
    \hline
    0 & x & ?????? \\
    \hline
  \end{tabular}
\end{table}

And then the stack frame goes away, freeing all of our memory.

\subsubsection*{Arguments and borrowing}

We've got some basic examples with the stack and the heap going, but what about function arguments and borrowing? Here's a small 
Rust program:

\begin{rustc}
fn foo(i: &i32) {
    let z = 42;
}

fn main() {
    let x = 5;
    let y = &x;

    foo(y);
}
\end{rustc}

When we enter \code{main()}, memory looks like this:

\begin{table}[H]
  \begin{tabular}{|l|l|l|}
    \hline
    \textbf{Address} & \textbf{Name} & \textbf{Value} \\
    \hline
    1 & y & $\rightarrow$ 0 \\
    \hline
    0 & x & 5 \\
    \hline
  \end{tabular}
\end{table}

\x\ is a plain old \code{5}, and \y\ is a reference to \x. So its value is the memory location that \x\ lives at, which in this 
case is \code{0}.

\blank

What about when we call \code{foo()}, passing \y\ as an argument?

\begin{table}[H]
  \begin{tabular}{|l|l|l|}
    \hline
    \textbf{Address} & \textbf{Name} & \textbf{Value} \\
    \hline
    3 & z & 42 \\
    \hline
    2 & i & $\rightarrow$ 0 \\
    \hline
    1 & y & $\rightarrow$ 0 \\
    \hline
    0 & x & 5 \\
    \hline
  \end{tabular}
\end{table}

Stack frames aren't only for local bindings, they're for arguments too. So in this case, we need to have both \code{i}, our argument, 
and \z, our local variable binding. \code{i} is a copy of the argument, \y. Since \code{y}'s value is \code{0}, so is \code{i}'s.

\blank

This is one reason why borrowing a variable doesn't deallocate any memory: the value of a reference is a pointer to a memory location. 
If we got rid of the underlying memory, things wouldn't work very well.

\subsubsection*{A complex example}

Okay, let's go through this complex program step-by-step:

\begin{rustc}
fn foo(x: &i32) {
    let y = 10;
    let z = &y;

    baz(z);
    bar(x, z);
}

fn bar(a: &i32, b: &i32) {
    let c = 5;
    let d = Box::new(5);
    let e = &d;

    baz(e);
}

fn baz(f: &i32) {
    let g = 100;
}

fn main() {
    let h = 3;
    let i = Box::new(20);
    let j = &h;

    foo(j);
}
\end{rustc}

First, we call \code{main()}:

\begin{table}[H]
  \begin{tabular}{|l|l|l|}
    \hline
    \textbf{Address} & \textbf{Name} & \textbf{Value} \\
    \hline
    ($2^{30}$) - 1 & & 20 \\
    \hline
    \ldots & \ldots & \ldots \\
    \hline
    2 & j & $\rightarrow$ 0 \\
    \hline
    1 & i & $\rightarrow$ ($2^{30}$)-1 \\
    \hline
    0 & h & 3 \\
    \hline
  \end{tabular}
\end{table}

We allocate memory for \code{j}, \code{i}, and \code{h}. \code{i} is on the heap, and so has a value pointing there.

\blank

Next, at the end of \code{main()}, \code{foo()} gets called:

\begin{table}[H]
  \begin{tabular}{|l|l|l|}
    \hline
    \textbf{Address} & \textbf{Name} & \textbf{Value} \\
    \hline
    ($2^{30}$) - 1 & & 20 \\
    \hline
    \ldots & \ldots & \ldots \\
    \hline
    5 & z & $\rightarrow$ 4 \\
    \hline
    4 & y & 10 \\
    \hline
    3 & x & $\rightarrow$ 0 \\
    \hline
    2 & j & $\rightarrow$ 0 \\
    \hline
    1 & i & $\rightarrow$ ($2^{30}$)-1 \\
    \hline
    0 & h & 3 \\
    \hline
  \end{tabular}
\end{table}

Space gets allocated for \x, \y, and \z. The argument \x\ has the same value as \code{j}, since that's what we passed it in. It's 
a pointer to the \code{0} address, since \code{j} points at \code{h}.

\blank

Next, \code{foo()} calls \code{baz()}, passing \z:

\begin{table}[H]
  \begin{tabular}{|l|l|l|}
    \hline
    \textbf{Address} & \textbf{Name} & \textbf{Value} \\
    \hline
    ($2^{30}$) - 1 & & 20 \\
    \hline
    \ldots & \ldots & \ldots \\
    \hline
    7 & g & 100 \\
    \hline
    6 & f & $\rightarrow$ 4 \\
    \hline
    5 & z & $\rightarrow$ 4 \\
    \hline
    4 & y & 10 \\
    \hline
    3 & x & $\rightarrow$ 0 \\
    \hline
    2 & j & $\rightarrow$ 0 \\
    \hline
    1 & i & $\rightarrow$ ($2^{30}$)-1 \\
    \hline
    0 & h & 3 \\
    \hline
  \end{tabular}
\end{table}

We've allocated memory for \code{f} and \code{g}. \code{baz()} is very short, so when it's over, we get rid of its stack frame:

\begin{table}[H]
  \begin{tabular}{|l|l|l|}
    \hline
    \textbf{Address} & \textbf{Name} & \textbf{Value} \\
    \hline
    ($2^{30}$) - 1 & & 20 \\
    \hline
    \ldots & \ldots & \ldots \\
    \hline
    5 & z & $\rightarrow$ 4 \\
    \hline
    4 & y & 10 \\
    \hline
    3 & x & $\rightarrow$ 0 \\
    \hline
    2 & j & $\rightarrow$ 0 \\
    \hline
    1 & i & $\rightarrow$ ($2^{30}$)-1 \\
    \hline
    0 & h & 3 \\
    \hline
  \end{tabular}
\end{table}

Next, \code{foo()} calls \code{bar()} with \x\ and \z:

\begin{table}[H]
  \begin{tabular}{|l|l|l|}
    \hline
    \textbf{Address} & \textbf{Name} & \textbf{Value} \\
    \hline
    ($2^{30}$) - 1 & & 20 \\
    \hline
    ($2^{30}$) - 2 & & 5 \\
    \hline
    \hline
    \ldots & \ldots & \ldots \\
    \hline
    10 & e & $\rightarrow$ 9 \\
    \hline
    9 & d & $\rightarrow$ ($2^{30}$)-2 \\
    \hline
    8 & c & 5 \\
    \hline
    7 & b & $\rightarrow$ 4 \\
    \hline
    6 & a & $\rightarrow$ 0 \\
    \hline
    5 & z & $\rightarrow$ 4 \\
    \hline
    4 & y & 10 \\
    \hline
    3 & x & $\rightarrow$ 0 \\
    \hline
    2 & j & $\rightarrow$ 0 \\
    \hline
    1 & i & $\rightarrow$ ($2^{30}$)-1 \\
    \hline
    0 & h & 3 \\
    \hline
  \end{tabular}
\end{table}

We end up allocating another value on the heap, and so we have to subtract one from ($2^{30}$) - 1. It's easier to write that than 
\code{1,073,741,822}. In any case, we set up the variables as usual.

At the end of \code{bar()}, it calls \code{baz()}:

\begin{table}[H]
  \begin{tabular}{|l|l|l|}
    \hline
    \textbf{Address} & \textbf{Name} & \textbf{Value} \\
    \hline
    ($2^{30}$) - 1 & & 20 \\
    \hline
    ($2^{30}$) - 2 & & 5 \\
    \hline
    \ldots & \ldots & \ldots \\
    12 & g & 100 \\
    \hline
    11 & f & $\rightarrow$ ($2^{30}$)-2\\
    \hline
    10 & e & $\rightarrow$ 9 \\
    \hline
    9 & d & $\rightarrow$ ($2^{30}$)-2 \\
    \hline
    8 & c & 5 \\
    \hline
    7 & b & $\rightarrow$ 4 \\
    \hline
    6 & a & $\rightarrow$ 0 \\
    \hline
    5 & z & $\rightarrow$ 4 \\
    \hline
    4 & y & 10 \\
    \hline
    3 & x & $\rightarrow$ 0 \\
    \hline
    2 & j & $\rightarrow$ 0 \\
    \hline
    1 & i & $\rightarrow$ ($2^{30}$)-1 \\
    \hline
    0 & h & 3 \\
    \hline
  \end{tabular}
\end{table}

With this, we're at our deepest point! Whew! Congrats for following along this far.

\blank

After \code{baz()} is over, we get rid of \code{f} and \code{g}:

\begin{table}[H]
  \begin{tabular}{|l|l|l|}
    \hline
    \textbf{Address} & \textbf{Name} & \textbf{Value} \\
    \hline
    ($2^{30}$) - 1 & & 20 \\
    \hline
    ($2^{30}$) - 2 & & 5 \\
    \hline
    \ldots & \ldots & \ldots \\
    \hline
    10 & e & $\rightarrow$ 9 \\
    \hline
    9 & d & $\rightarrow$ ($2^{30}$)-2 \\
    \hline
    8 & c & 5 \\
    \hline
    7 & b & $\rightarrow$ 4 \\
    \hline
    6 & a & $\rightarrow$ 0 \\
    \hline
    5 & z & $\rightarrow$ 4 \\
    \hline
    4 & y & 10 \\
    \hline
    3 & x & $\rightarrow$ 0 \\
    \hline
    2 & j & $\rightarrow$ 0 \\
    \hline
    1 & i & $\rightarrow$ ($2^{30}$)-1 \\
    \hline
    0 & h & 3 \\
    \hline
  \end{tabular}
\end{table}

Next, we return from \code{bar()}. \code{d} in this case is a \code{Box<T>}, so it also frees what it points to: ($2^{30}$) - 2.

\begin{table}[H]
  \begin{tabular}{|l|l|l|}
    \hline
    \textbf{Address} & \textbf{Name} & \textbf{Value} \\
    \hline
    ($2^{30}$) - 1 & & 20 \\
    \hline
    \ldots & \ldots & \ldots \\
    \hline
    5 & z & $\rightarrow$ 4 \\
    \hline
    4 & y & 10 \\
    \hline
    3 & x & $\rightarrow$ 0 \\
    \hline
    2 & j & $\rightarrow$ 0 \\
    \hline
    1 & i & $\rightarrow$ ($2^{30}$)-1 \\
    \hline
    0 & h & 3 \\
    \hline
  \end{tabular}
\end{table}

And after that, \code{foo()} returns:

\begin{table}[H]
  \begin{tabular}{|l|l|l|}
    \hline
    \textbf{Address} & \textbf{Name} & \textbf{Value} \\
    \hline
    ($2^{30}$) - 1 & & 20 \\
    \hline
    \ldots & \ldots & \ldots \\
    \hline
    2 & j & $\rightarrow$ 0 \\
    \hline
    1 & i & $\rightarrow$ ($2^{30}$)-1 \\
    \hline
    0 & h & 3 \\
    \hline
  \end{tabular}
\end{table}

And then, finally, \code{main()}, which cleans the rest up. When \code{i} is \code{Drop}ped, it will clean up the last of the heap too.

\subsubsection*{What do other languages do?}

Most languages with a garbage collector heap-allocate by default. This means that every value is boxed. There are a number of reasons 
why this is done, but they're out of scope for this tutorial. There are some possible optimizations that don't make it true 100\% of 
the time, too. Rather than relying on the stack and \code{Drop} to clean up memory, the garbage collector deals with the heap instead.

\subsubsection*{Which to use?}

So if the stack is faster and easier to manage, why do we need the heap? A big reason is that Stack-allocation alone means you only 
have 'Last In First Out (LIFO)' semantics for reclaiming storage. Heap-allocation is strictly more general, allowing storage to be 
taken from and returned to the pool in arbitrary order, but at a complexity cost.


Generally, you should prefer stack allocation, and so, Rust stack-allocates by default. The LIFO model of the stack is simpler, 
at a fundamental level. This has two big impacts: runtime efficiency and semantic impact.

\myparagraph{Runtime Efficiency}

Managing the memory for the stack is trivial: The machine increments or decrements a single value, the so-called \enquote{stack pointer}.
Managing memory for the heap is non-trivial: heap-allocated memory is freed at arbitrary points, and each block of heap-allocated memory 
can be of arbitrary size, so the memory manager must generally work much harder to identify memory for reuse.

\blank

If you'd like to dive into this topic in greater detail, \href{http://citeseerx.ist.psu.edu/viewdoc/summary?doi=10.1.1.143.4688}{this paper}
is a great introduction.

\myparagraph{Semantic impact}

Stack-allocation impacts the Rust language itself, and thus the developer's mental model. The LIFO semantics is what drives how the 
Rust language handles automatic memory management. Even the deallocation of a uniquely-owned heap-allocated box can be driven by the 
stack-based LIFO semantics, as discussed throughout this chapter. The flexibility (i.e. expressiveness) of non LIFO-semantics means that 
in general the compiler cannot automatically infer at compile-time where memory should be freed; it has to rely on dynamic protocols, 
potentially from outside the language itself, to drive deallocation (reference counting, as used by \code{Rc<T>} and \code{Arc<T>}, is 
one example of this).

\blank

When taken to the extreme, the increased expressive power of heap allocation comes at the cost of either significant runtime support 
(e.g. in the form of a garbage collector) or significant programmer effort (in the form of explicit memory management calls that 
require verification not provided by the Rust compiler).
