The Rust project uses a concept called 'release channels' to manage releases. It's important to understand this process 
to choose which version of Rust your project should use.

\subsection*{Overview}

There are three channels for Rust releases:

\begin{itemize}
  \item{Nightly}
  \item{Beta}
  \item{Stable}
\end{itemize}

New nightly releases are created once a day. Every six weeks, the latest nightly release is promoted to 'Beta'. At that 
point, it will only receive patches to fix serious errors. Six weeks later, the beta is promoted to 'Stable', and becomes 
the next release of \code{1.x}.

\blank

This process happens in parallel. So every six weeks, on the same day, nightly goes to beta, beta goes to stable. When 
\code{1.x} is released, at the same time, \code{1.(x + 1)-beta} is released, and the nightly becomes the first version of 
\code{1.(x + 2)-nightly}.

\subsection*{Choosing a version}

Generally speaking, unless you have a specific reason, you should be using the stable release channel. These releases 
are intended for a general audience.

\blank

However, depending on your interest in Rust, you may choose to use nightly instead. The basic tradeoff is this: in the 
nightly channel, you can use unstable, new Rust features. However, unstable features are subject to change, and so any 
new nightly release may break your code. If you use the stable release, you cannot use experimental features, but the next 
release of Rust will not cause significant issues through breaking changes.

\subsection*{Helping the ecosystem through CI}

What about beta? We encourage all Rust users who use the stable release channel to also test against the beta channel 
in their continuous integration systems. This will help alert the team in case there's an accidental regression.

\blank

Additionally, testing against nightly can catch regressions even sooner, and so if you don't mind a third build, we'd 
appreciate testing against all channels.

\blank

As an example, many Rust programmers use \href{https://travis-ci.org/}{Travis} to test their crates, which is free for 
open source projects. Travis \href{http://docs.travis-ci.com/user/languages/rust/}{supports Rust directly}, and you can 
use a \code{.travis.yml} file like this to test on all channels:

\begin{verbatim}
language: rust
rust:
  - nightly
  - beta
  - stable

matrix:
  allow_failures:
    - rust: nightly

\end{verbatim}

With this configuration, Travis will test all three channels, but if something breaks on nightly, it won't fail your 
build. A similar configuration is recommended for any CI system, check the documentation of the one you're using for 
more details.
