The \href{https://doc.rust-lang.org/std/borrow/trait.Borrow.html}{Borrow} and 
\href{https://doc.rust-lang.org/std/convert/trait.AsRef.html}{AsRef} traits are very similar, but different. Here's a 
quick refresher on what these two traits mean.

\subsection*{Borrow}

The \code{Borrow} trait is used when you're writing a datastructure, and you want to use either an owned or borrowed 
type as synonymous for some purpose.

\blank

For example, \href{https://doc.rust-lang.org/std/collections/struct.HashMap.html}{HashMap} has a 
\href{https://doc.rust-lang.org/std/collections/struct.HashMap.html\#method.get}{get method} which uses \code{Borrow}:

\begin{rustc}
fn get<Q: ?Sized>(&self, k: &Q) -> Option<&V>
    where K: Borrow<Q>,
          Q: Hash + Eq
\end{rustc}

This signature is pretty complicated. The \code{K} parameter is what we're interested in here. It refers to a parameter 
of the \code{HashMap} itself:

\begin{rustc}
struct HashMap<K, V, S = RandomState> {
\end{rustc}

The \code{K} parameter is the type of key the \code{HashMap} uses. So, looking at the signature of \code{get()} again, 
we can use \code{get()} when the key implements \code{Borrow<Q>}. That way, we can make a \code{HashMap} which uses \String\ 
keys, but use \code{\&str}s when we're searching:

\begin{rustc}
use std::collections::HashMap;

let mut map = HashMap::new();
map.insert("Foo".to_string(), 42);

assert_eq!(map.get("Foo"), Some(&42));
\end{rustc}

This is because the standard library has \code{impl Borrow<str> for String}.

\blank

For most types, when you want to take an owned or borrowed type, a \code{\&T} is enough. But one area where \code{Borrow} 
is effective is when there's more than one kind of borrowed value. This is especially true of references and slices: you can 
have both an \code{\&T} or a \code{\&mut T}. If we wanted to accept both of these types, \code{Borrow} is up for it:

\begin{rustc}
use std::borrow::Borrow;
use std::fmt::Display;

fn foo<T: Borrow<i32> + Display>(a: T) {
    println!("a is borrowed: {}", a);
}

let mut i = 5;

foo(&i);
foo(&mut i);
\end{rustc}

This will print out \code{a is borrowed: 5} twice.

\subsection*{AsRef}

The \code{AsRef} trait is a conversion trait. It's used for converting some value to a reference in generic code. Like this:

\begin{rustc}
let s = "Hello".to_string();

fn foo<T: AsRef<str>>(s: T) {
    let slice = s.as_ref();
}
\end{rustc}

\subsection*{Which should I use?}

We can see how they're kind of the same: they both deal with owned and borrowed versions of some type. However, they're a 
bit different.

\blank

Choose \code{Borrow} when you want to abstract over different kinds of borrowing, or when you're building a datastructure 
that treats owned and borrowed values in equivalent ways, such as hashing and comparison.

\blank

Choose \code{AsRef} when you want to convert something to a reference directly, and you're writing generic code.
