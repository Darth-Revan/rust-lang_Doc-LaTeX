% Like most programming languages, Rust encourages the programmer to handle errors in a particular way. Generally speaking, 
% error handling is divided into two broad categories: exceptions and return values. Rust opts for return values.
%
% \blank
%
% In this section, we intend to provide a comprehensive treatment of how to deal with errors in Rust. More than that, we will 
% attempt to introduce error handling one piece at a time so that you'll come away with a solid working knowledge of how everything 
% fits together.
%
% \blank
%
% When done naively, error handling in Rust can be verbose and annoying. This section will explore those stumbling blocks and 
% demonstrate how to use the standard library to make error handling concise and ergonomic.
%
% \subsubsection*{Table of Contents}
%
% This section is very long, mostly because we start at the very beginning with sum types and combinators, and try to motivate the way Rust does error handling incrementally. As such, programmers with experience in other expressive type systems may want to jump around.
%
%     The Basics
%         Unwrapping explained
%         The Option type
%             Composing Option<T> values
%         The Result type
%             Parsing integers
%             The Result type alias idiom
%         A brief interlude: unwrapping isn't evil
%     Working with multiple error types
%         Composing Option and Result
%         The limits of combinators
%         Early returns
%         The try! macro
%         Defining your own error type
%     Standard library traits used for error handling
%         The Error trait
%         The From trait
%         The real try! macro
%         Composing custom error types
%         Advice for library writers
%     Case study: A program to read population data
%         Initial setup
%         Argument parsing
%         Writing the logic
%         Error handling with Box<Error>
%         Reading from stdin
%         Error handling with a custom type
%         Adding functionality
%     The short story
%
% The Basics
%
% You can think of error handling as using case analysis to determine whether a computation was successful or not. As you will see, the key to ergonomic error handling is reducing the amount of explicit case analysis the programmer has to do while keeping code composable.
%
% Keeping code composable is important, because without that requirement, we could panic whenever we come across something unexpected. (panic causes the current task to unwind, and in most cases, the entire program aborts.) Here's an example:
%
% // Guess a number between 1 and 10.
% // If it matches the number we had in mind, return true. Else, return false.
% fn guess(n: i32) -> bool {
%     if n < 1 || n > 10 {
%         panic!("Invalid number: {}", n);
%     }
%     n == 5
% }
%
% fn main() {
%     guess(11);
% }
%
% If you try running this code, the program will crash with a message like this:
%
% thread '<main>' panicked at 'Invalid number: 11', src/bin/panic-simple.rs:5
%
% Here's another example that is slightly less contrived. A program that accepts an integer as an argument, doubles it and prints it.
%
% use std::env;
%
% fn main() {
%     let mut argv = env::args();
%     let arg: String = argv.nth(1).unwrap(); // error 1
%     let n: i32 = arg.parse().unwrap(); // error 2
%     println!("{}", 2 * n);
% }
%
% If you give this program zero arguments (error 1) or if the first argument isn't an integer (error 2), the program will panic just like in the first example.
%
% You can think of this style of error handling as similar to a bull running through a china shop. The bull will get to where it wants to go, but it will trample everything in the process.
% Unwrapping explained
%
% In the previous example, we claimed that the program would simply panic if it reached one of the two error conditions, yet, the program does not include an explicit call to panic like the first example. This is because the panic is embedded in the calls to unwrap.
%
% To “unwrap” something in Rust is to say, “Give me the result of the computation, and if there was an error, panic and stop the program.” It would be better if we showed the code for unwrapping because it is so simple, but to do that, we will first need to explore the Option and Result types. Both of these types have a method called unwrap defined on them.
% The Option type
%
% The Option type is defined in the standard library:
%
% enum Option<T> {
%     None,
%     Some(T),
% }
%
% The Option type is a way to use Rust's type system to express the possibility of absence. Encoding the possibility of absence into the type system is an important concept because it will cause the compiler to force the programmer to handle that absence. Let's take a look at an example that tries to find a character in a string:
%
% // Searches `haystack` for the Unicode character `needle`. If one is found, the
% // byte offset of the character is returned. Otherwise, `None` is returned.
% fn find(haystack: &str, needle: char) -> Option<usize> {
%     for (offset, c) in haystack.char_indices() {
%         if c == needle {
%             return Some(offset);
%         }
%     }
%     None
% }
%
% Notice that when this function finds a matching character, it doesn't only return the offset. Instead, it returns Some(offset). Some is a variant or a value constructor for the Option type. You can think of it as a function with the type fn<T>(value: T) -> Option<T>. Correspondingly, None is also a value constructor, except it has no arguments. You can think of None as a function with the type fn<T>() -> Option<T>.
%
% This might seem like much ado about nothing, but this is only half of the story. The other half is using the find function we've written. Let's try to use it to find the extension in a file name.
%
% fn main() {
%     let file_name = "foobar.rs";
%     match find(file_name, '.') {
%         None => println!("No file extension found."),
%         Some(i) => println!("File extension: {}", &file_name[i+1..]),
%     }
% }
%
% This code uses pattern matching to do case analysis on the Option<usize> returned by the find function. In fact, case analysis is the only way to get at the value stored inside an Option<T>. This means that you, as the programmer, must handle the case when an Option<T> is None instead of Some(t).
%
% But wait, what about unwrap, which we used previously? There was no case analysis there! Instead, the case analysis was put inside the unwrap method for you. You could define it yourself if you want:
%
% enum Option<T> {
%     None,
%     Some(T),
% }
%
% impl<T> Option<T> {
%     fn unwrap(self) -> T {
%         match self {
%             Option::Some(val) => val,
%             Option::None =>
%               panic!("called `Option::unwrap()` on a `None` value"),
%         }
%     }
% }
%
% The unwrap method abstracts away the case analysis. This is precisely the thing that makes unwrap ergonomic to use. Unfortunately, that panic! means that unwrap is not composable: it is the bull in the china shop.
% Composing Option<T> values
%
% In an example from before, we saw how to use find to discover the extension in a file name. Of course, not all file names have a . in them, so it's possible that the file name has no extension. This possibility of absence is encoded into the types using Option<T>. In other words, the compiler will force us to address the possibility that an extension does not exist. In our case, we only print out a message saying as such.
%
% Getting the extension of a file name is a pretty common operation, so it makes sense to put it into a function:
%
% // Returns the extension of the given file name, where the extension is defined
% // as all characters proceeding the first `.`.
% // If `file_name` has no `.`, then `None` is returned.
% fn extension_explicit(file_name: &str) -> Option<&str> {
%     match find(file_name, '.') {
%         None => None,
%         Some(i) => Some(&file_name[i+1..]),
%     }
% }
%
% (Pro-tip: don't use this code. Use the extension method in the standard library instead.)
%
% The code stays simple, but the important thing to notice is that the type of find forces us to consider the possibility of absence. This is a good thing because it means the compiler won't let us accidentally forget about the case where a file name doesn't have an extension. On the other hand, doing explicit case analysis like we've done in extension_explicit every time can get a bit tiresome.
%
% In fact, the case analysis in extension_explicit follows a very common pattern: map a function on to the value inside of an Option<T>, unless the option is None, in which case, return None.
%
% Rust has parametric polymorphism, so it is very easy to define a combinator that abstracts this pattern:
%
% fn map<F, T, A>(option: Option<T>, f: F) -> Option<A> where F: FnOnce(T) -> A {
%     match option {
%         None => None,
%         Some(value) => Some(f(value)),
%     }
% }
%
% Indeed, map is defined as a method on Option<T> in the standard library.
%
% Armed with our new combinator, we can rewrite our extension_explicit method to get rid of the case analysis:
%
% // Returns the extension of the given file name, where the extension is defined
% // as all characters proceeding the first `.`.
% // If `file_name` has no `.`, then `None` is returned.
% fn extension(file_name: &str) -> Option<&str> {
%     find(file_name, '.').map(|i| &file_name[i+1..])
% }
%
% One other pattern we commonly find is assigning a default value to the case when an Option value is None. For example, maybe your program assumes that the extension of a file is rs even if none is present. As you might imagine, the case analysis for this is not specific to file extensions - it can work with any Option<T>:
%
% fn unwrap_or<T>(option: Option<T>, default: T) -> T {
%     match option {
%         None => default,
%         Some(value) => value,
%     }
% }
%
% The trick here is that the default value must have the same type as the value that might be inside the Option<T>. Using it is dead simple in our case:
%
% fn main() {
%     assert_eq!(extension("foobar.csv").unwrap_or("rs"), "csv");
%     assert_eq!(extension("foobar").unwrap_or("rs"), "rs");
% }
%
% (Note that unwrap_or is defined as a method on Option<T> in the standard library, so we use that here instead of the free-standing function we defined above. Don't forget to check out the more general unwrap_or_else method.)
%
% There is one more combinator that we think is worth paying special attention to: and_then. It makes it easy to compose distinct computations that admit the possibility of absence. For example, much of the code in this section is about finding an extension given a file name. In order to do this, you first need the file name which is typically extracted from a file path. While most file paths have a file name, not all of them do. For example, ., .. or /.
%
% So, we are tasked with the challenge of finding an extension given a file path. Let's start with explicit case analysis:
%
% fn file_path_ext_explicit(file_path: &str) -> Option<&str> {
%     match file_name(file_path) {
%         None => None,
%         Some(name) => match extension(name) {
%             None => None,
%             Some(ext) => Some(ext),
%         }
%     }
% }
%
% fn file_name(file_path: &str) -> Option<&str> {
%   // implementation elided
%   unimplemented!()
% }
%
% You might think that we could use the map combinator to reduce the case analysis, but its type doesn't quite fit. Namely, map takes a function that does something only with the inner value. The result of that function is then always rewrapped with Some. Instead, we need something like map, but which allows the caller to return another Option. Its generic implementation is even simpler than map:
%
% fn and_then<F, T, A>(option: Option<T>, f: F) -> Option<A>
%         where F: FnOnce(T) -> Option<A> {
%     match option {
%         None => None,
%         Some(value) => f(value),
%     }
% }
%
% Now we can rewrite our file_path_ext function without explicit case analysis:
%
% fn file_path_ext(file_path: &str) -> Option<&str> {
%     file_name(file_path).and_then(extension)
% }
%
% The Option type has many other combinators defined in the standard library. It is a good idea to skim this list and familiarize yourself with what's available—they can often reduce case analysis for you. Familiarizing yourself with these combinators will pay dividends because many of them are also defined (with similar semantics) for Result, which we will talk about next.
%
% Combinators make using types like Option ergonomic because they reduce explicit case analysis. They are also composable because they permit the caller to handle the possibility of absence in their own way. Methods like unwrap remove choices because they will panic if Option<T> is None.
% The Result type
%
% The Result type is also defined in the standard library:
%
% enum Result<T, E> {
%     Ok(T),
%     Err(E),
% }
%
% The Result type is a richer version of Option. Instead of expressing the possibility of absence like Option does, Result expresses the possibility of error. Usually, the error is used to explain why the execution of some computation failed. This is a strictly more general form of Option. Consider the following type alias, which is semantically equivalent to the real Option<T> in every way:
%
% type Option<T> = Result<T, ()>;
%
% This fixes the second type parameter of Result to always be () (pronounced “unit” or “empty tuple”). Exactly one value inhabits the () type: (). (Yup, the type and value level terms have the same notation!)
%
% The Result type is a way of representing one of two possible outcomes in a computation. By convention, one outcome is meant to be expected or “Ok” while the other outcome is meant to be unexpected or “Err”.
%
% Just like Option, the Result type also has an unwrap method defined in the standard library. Let's define it:
%
% impl<T, E: ::std::fmt::Debug> Result<T, E> {
%     fn unwrap(self) -> T {
%         match self {
%             Result::Ok(val) => val,
%             Result::Err(err) =>
%               panic!("called `Result::unwrap()` on an `Err` value: {:?}", err),
%         }
%     }
% }
%
% This is effectively the same as our definition for Option::unwrap, except it includes the error value in the panic! message. This makes debugging easier, but it also requires us to add a Debug constraint on the E type parameter (which represents our error type). Since the vast majority of types should satisfy the Debug constraint, this tends to work out in practice. (Debug on a type simply means that there's a reasonable way to print a human readable description of values with that type.)
%
% OK, let's move on to an example.
% Parsing integers
%
% The Rust standard library makes converting strings to integers dead simple. It's so easy in fact, that it is very tempting to write something like the following:
%
% fn double_number(number_str: &str) -> i32 {
%     2 * number_str.parse::<i32>().unwrap()
% }
%
% fn main() {
%     let n: i32 = double_number("10");
%     assert_eq!(n, 20);
% }
%
% At this point, you should be skeptical of calling unwrap. For example, if the string doesn't parse as a number, you'll get a panic:
%
% thread '<main>' panicked at 'called `Result::unwrap()` on an `Err` value: ParseIntError { kind: InvalidDigit }', /home/rustbuild/src/rust-buildbot/slave/beta-dist-rustc-linux/build/src/libcore/result.rs:729
%
% This is rather unsightly, and if this happened inside a library you're using, you might be understandably annoyed. Instead, we should try to handle the error in our function and let the caller decide what to do. This means changing the return type of double_number. But to what? Well, that requires looking at the signature of the parse method in the standard library:
%
% impl str {
%     fn parse<F: FromStr>(&self) -> Result<F, F::Err>;
% }
%
% Hmm. So we at least know that we need to use a Result. Certainly, it's possible that this could have returned an Option. After all, a string either parses as a number or it doesn't, right? That's certainly a reasonable way to go, but the implementation internally distinguishes why the string didn't parse as an integer. (Whether it's an empty string, an invalid digit, too big or too small.) Therefore, using a Result makes sense because we want to provide more information than simply “absence.” We want to say why the parsing failed. You should try to emulate this line of reasoning when faced with a choice between Option and Result. If you can provide detailed error information, then you probably should. (We'll see more on this later.)
%
% OK, but how do we write our return type? The parse method as defined above is generic over all the different number types defined in the standard library. We could (and probably should) also make our function generic, but let's favor explicitness for the moment. We only care about i32, so we need to find its implementation of FromStr (do a CTRL-F in your browser for “FromStr”) and look at its associated type Err. We did this so we can find the concrete error type. In this case, it's std::num::ParseIntError. Finally, we can rewrite our function:
%
% use std::num::ParseIntError;
%
% fn double_number(number_str: &str) -> Result<i32, ParseIntError> {
%     match number_str.parse::<i32>() {
%         Ok(n) => Ok(2 * n),
%         Err(err) => Err(err),
%     }
% }
%
% fn main() {
%     match double_number("10") {
%         Ok(n) => assert_eq!(n, 20),
%         Err(err) => println!("Error: {:?}", err),
%     }
% }
%
% This is a little better, but now we've written a lot more code! The case analysis has once again bitten us.
%
% Combinators to the rescue! Just like Option, Result has lots of combinators defined as methods. There is a large intersection of common combinators between Result and Option. In particular, map is part of that intersection:
%
% use std::num::ParseIntError;
%
% fn double_number(number_str: &str) -> Result<i32, ParseIntError> {
%     number_str.parse::<i32>().map(|n| 2 * n)
% }
%
% fn main() {
%     match double_number("10") {
%         Ok(n) => assert_eq!(n, 20),
%         Err(err) => println!("Error: {:?}", err),
%     }
% }
%
% The usual suspects are all there for Result, including unwrap_or and and_then. Additionally, since Result has a second type parameter, there are combinators that affect only the error type, such as map_err (instead of map) and or_else (instead of and_then).
% The Result type alias idiom
%
% In the standard library, you may frequently see types like Result<i32>. But wait, we defined Result to have two type parameters. How can we get away with only specifying one? The key is to define a Result type alias that fixes one of the type parameters to a particular type. Usually the fixed type is the error type. For example, our previous example parsing integers could be rewritten like this:
%
% use std::num::ParseIntError;
% use std::result;
%
% type Result<T> = result::Result<T, ParseIntError>;
%
% fn double_number(number_str: &str) -> Result<i32> {
%     unimplemented!();
% }
%
% Why would we do this? Well, if we have a lot of functions that could return ParseIntError, then it's much more convenient to define an alias that always uses ParseIntError so that we don't have to write it out all the time.
%
% The most prominent place this idiom is used in the standard library is with io::Result. Typically, one writes io::Result<T>, which makes it clear that you're using the io module's type alias instead of the plain definition from std::result. (This idiom is also used for fmt::Result.)
% A brief interlude: unwrapping isn't evil
%
% If you've been following along, you might have noticed that I've taken a pretty hard line against calling methods like unwrap that could panic and abort your program. Generally speaking, this is good advice.
%
% However, unwrap can still be used judiciously. What exactly justifies use of unwrap is somewhat of a grey area and reasonable people can disagree. I'll summarize some of my opinions on the matter.
%
%     In examples and quick 'n' dirty code. Sometimes you're writing examples or a quick program, and error handling simply isn't important. Beating the convenience of unwrap can be hard in such scenarios, so it is very appealing.
%     When panicking indicates a bug in the program. When the invariants of your code should prevent a certain case from happening (like, say, popping from an empty stack), then panicking can be permissible. This is because it exposes a bug in your program. This can be explicit, like from an assert! failing, or it could be because your index into an array was out of bounds.
%
% This is probably not an exhaustive list. Moreover, when using an Option, it is often better to use its expect method. expect does exactly the same thing as unwrap, except it prints a message you give to expect. This makes the resulting panic a bit nicer to deal with, since it will show your message instead of “called unwrap on a None value.”
%
% My advice boils down to this: use good judgment. There's a reason why the words “never do X” or “Y is considered harmful” don't appear in my writing. There are trade offs to all things, and it is up to you as the programmer to determine what is acceptable for your use cases. My goal is only to help you evaluate trade offs as accurately as possible.
%
% Now that we've covered the basics of error handling in Rust, and explained unwrapping, let's start exploring more of the standard library.
% Working with multiple error types
%
% Thus far, we've looked at error handling where everything was either an Option<T> or a Result<T, SomeError>. But what happens when you have both an Option and a Result? Or what if you have a Result<T, Error1> and a Result<T, Error2>? Handling composition of distinct error types is the next challenge in front of us, and it will be the major theme throughout the rest of this section.
% Composing Option and Result
%
% So far, I've talked about combinators defined for Option and combinators defined for Result. We can use these combinators to compose results of different computations without doing explicit case analysis.
%
% Of course, in real code, things aren't always as clean. Sometimes you have a mix of Option and Result types. Must we resort to explicit case analysis, or can we continue using combinators?
%
% For now, let's revisit one of the first examples in this section:
%
% use std::env;
%
% fn main() {
%     let mut argv = env::args();
%     let arg: String = argv.nth(1).unwrap(); // error 1
%     let n: i32 = arg.parse().unwrap(); // error 2
%     println!("{}", 2 * n);
% }
%
% Given our new found knowledge of Option, Result and their various combinators, we should try to rewrite this so that errors are handled properly and the program doesn't panic if there's an error.
%
% The tricky aspect here is that argv.nth(1) produces an Option while arg.parse() produces a Result. These aren't directly composable. When faced with both an Option and a Result, the solution is usually to convert the Option to a Result. In our case, the absence of a command line parameter (from env::args()) means the user didn't invoke the program correctly. We could use a String to describe the error. Let's try:
%
% use std::env;
%
% fn double_arg(mut argv: env::Args) -> Result<i32, String> {
%     argv.nth(1)
%         .ok_or("Please give at least one argument".to_owned())
%         .and_then(|arg| arg.parse::<i32>().map_err(|err| err.to_string()))
%         .map(|n| 2 * n)
% }
%
% fn main() {
%     match double_arg(env::args()) {
%         Ok(n) => println!("{}", n),
%         Err(err) => println!("Error: {}", err),
%     }
% }
%
% There are a couple new things in this example. The first is the use of the Option::ok_or combinator. This is one way to convert an Option into a Result. The conversion requires you to specify what error to use if Option is None. Like the other combinators we've seen, its definition is very simple:
%
% fn ok_or<T, E>(option: Option<T>, err: E) -> Result<T, E> {
%     match option {
%         Some(val) => Ok(val),
%         None => Err(err),
%     }
% }
%
% The other new combinator used here is Result::map_err. This is like Result::map, except it maps a function on to the error portion of a Result value. If the Result is an Ok(...) value, then it is returned unmodified.
%
% We use map_err here because it is necessary for the error types to remain the same (because of our use of and_then). Since we chose to convert the Option<String> (from argv.nth(1)) to a Result<String, String>, we must also convert the ParseIntError from arg.parse() to a String.
% The limits of combinators
%
% Doing IO and parsing input is a very common task, and it's one that I personally have done a lot of in Rust. Therefore, we will use (and continue to use) IO and various parsing routines to exemplify error handling.
%
% Let's start simple. We are tasked with opening a file, reading all of its contents and converting its contents to a number. Then we multiply it by 2 and print the output.
%
% Although I've tried to convince you not to use unwrap, it can be useful to first write your code using unwrap. It allows you to focus on your problem instead of the error handling, and it exposes the points where proper error handling need to occur. Let's start there so we can get a handle on the code, and then refactor it to use better error handling.
%
% use std::fs::File;
% use std::io::Read;
% use std::path::Path;
%
% fn file_double<P: AsRef<Path>>(file_path: P) -> i32 {
%     let mut file = File::open(file_path).unwrap(); // error 1
%     let mut contents = String::new();
%     file.read_to_string(&mut contents).unwrap(); // error 2
%     let n: i32 = contents.trim().parse().unwrap(); // error 3
%     2 * n
% }
%
% fn main() {
%     let doubled = file_double("foobar");
%     println!("{}", doubled);
% }
%
% (N.B. The AsRef<Path> is used because those are the same bounds used on std::fs::File::open. This makes it ergonomic to use any kind of string as a file path.)
%
% There are three different errors that can occur here:
%
%     A problem opening the file.
%     A problem reading data from the file.
%     A problem parsing the data as a number.
%
% The first two problems are described via the std::io::Error type. We know this because of the return types of std::fs::File::open and std::io::Read::read_to_string. (Note that they both use the Result type alias idiom described previously. If you click on the Result type, you'll see the type alias, and consequently, the underlying io::Error type.) The third problem is described by the std::num::ParseIntError type. The io::Error type in particular is pervasive throughout the standard library. You will see it again and again.
%
% Let's start the process of refactoring the file_double function. To make this function composable with other components of the program, it should not panic if any of the above error conditions are met. Effectively, this means that the function should return an error if any of its operations fail. Our problem is that the return type of file_double is i32, which does not give us any useful way of reporting an error. Thus, we must start by changing the return type from i32 to something else.
%
% The first thing we need to decide: should we use Option or Result? We certainly could use Option very easily. If any of the three errors occur, we could simply return None. This will work and it is better than panicking, but we can do a lot better. Instead, we should pass some detail about the error that occurred. Since we want to express the possibility of error, we should use Result<i32, E>. But what should E be? Since two different types of errors can occur, we need to convert them to a common type. One such type is String. Let's see how that impacts our code:
%
% use std::fs::File;
% use std::io::Read;
% use std::path::Path;
%
% fn file_double<P: AsRef<Path>>(file_path: P) -> Result<i32, String> {
%     File::open(file_path)
%          .map_err(|err| err.to_string())
%          .and_then(|mut file| {
%               let mut contents = String::new();
%               file.read_to_string(&mut contents)
%                   .map_err(|err| err.to_string())
%                   .map(|_| contents)
%          })
%          .and_then(|contents| {
%               contents.trim().parse::<i32>()
%                       .map_err(|err| err.to_string())
%          })
%          .map(|n| 2 * n)
% }
%
% fn main() {
%     match file_double("foobar") {
%         Ok(n) => println!("{}", n),
%         Err(err) => println!("Error: {}", err),
%     }
% }
%
% This code looks a bit hairy. It can take quite a bit of practice before code like this becomes easy to write. The way we write it is by following the types. As soon as we changed the return type of file_double to Result<i32, String>, we had to start looking for the right combinators. In this case, we only used three different combinators: and_then, map and map_err.
%
% and_then is used to chain multiple computations where each computation could return an error. After opening the file, there are two more computations that could fail: reading from the file and parsing the contents as a number. Correspondingly, there are two calls to and_then.
%
% map is used to apply a function to the Ok(...) value of a Result. For example, the very last call to map multiplies the Ok(...) value (which is an i32) by 2. If an error had occurred before that point, this operation would have been skipped because of how map is defined.
%
% map_err is the trick that makes all of this work. map_err is like map, except it applies a function to the Err(...) value of a Result. In this case, we want to convert all of our errors to one type: String. Since both io::Error and num::ParseIntError implement ToString, we can call the to_string() method to convert them.
%
% With all of that said, the code is still hairy. Mastering use of combinators is important, but they have their limits. Let's try a different approach: early returns.
% Early returns
%
% I'd like to take the code from the previous section and rewrite it using early returns. Early returns let you exit the function early. We can't return early in file_double from inside another closure, so we'll need to revert back to explicit case analysis.
%
% use std::fs::File;
% use std::io::Read;
% use std::path::Path;
%
% fn file_double<P: AsRef<Path>>(file_path: P) -> Result<i32, String> {
%     let mut file = match File::open(file_path) {
%         Ok(file) => file,
%         Err(err) => return Err(err.to_string()),
%     };
%     let mut contents = String::new();
%     if let Err(err) = file.read_to_string(&mut contents) {
%         return Err(err.to_string());
%     }
%     let n: i32 = match contents.trim().parse() {
%         Ok(n) => n,
%         Err(err) => return Err(err.to_string()),
%     };
%     Ok(2 * n)
% }
%
% fn main() {
%     match file_double("foobar") {
%         Ok(n) => println!("{}", n),
%         Err(err) => println!("Error: {}", err),
%     }
% }
%
% Reasonable people can disagree over whether this code is better than the code that uses combinators, but if you aren't familiar with the combinator approach, this code looks simpler to read to me. It uses explicit case analysis with match and if let. If an error occurs, it simply stops executing the function and returns the error (by converting it to a string).
%
% Isn't this a step backwards though? Previously, we said that the key to ergonomic error handling is reducing explicit case analysis, yet we've reverted back to explicit case analysis here. It turns out, there are multiple ways to reduce explicit case analysis. Combinators aren't the only way.
% The try! macro
%
% A cornerstone of error handling in Rust is the try! macro. The try! macro abstracts case analysis like combinators, but unlike combinators, it also abstracts control flow. Namely, it can abstract the early return pattern seen above.
%
% Here is a simplified definition of a try! macro:
%
% macro_rules! try {
%     ($e:expr) => (match $e {
%         Ok(val) => val,
%         Err(err) => return Err(err),
%     });
% }
%
% (The real definition is a bit more sophisticated. We will address that later.)
%
% Using the try! macro makes it very easy to simplify our last example. Since it does the case analysis and the early return for us, we get tighter code that is easier to read:
%
% use std::fs::File;
% use std::io::Read;
% use std::path::Path;
%
% fn file_double<P: AsRef<Path>>(file_path: P) -> Result<i32, String> {
%     let mut file = try!(File::open(file_path).map_err(|e| e.to_string()));
%     let mut contents = String::new();
%     try!(file.read_to_string(&mut contents).map_err(|e| e.to_string()));
%     let n = try!(contents.trim().parse::<i32>().map_err(|e| e.to_string()));
%     Ok(2 * n)
% }
%
% fn main() {
%     match file_double("foobar") {
%         Ok(n) => println!("{}", n),
%         Err(err) => println!("Error: {}", err),
%     }
% }
%
% The map_err calls are still necessary given our definition of try!. This is because the error types still need to be converted to String. The good news is that we will soon learn how to remove those map_err calls! The bad news is that we will need to learn a bit more about a couple important traits in the standard library before we can remove the map_err calls.
% Defining your own error type
%
% Before we dive into some of the standard library error traits, I'd like to wrap up this section by removing the use of String as our error type in the previous examples.
%
% Using String as we did in our previous examples is convenient because it's easy to convert errors to strings, or even make up your own errors as strings on the spot. However, using String for your errors has some downsides.
%
% The first downside is that the error messages tend to clutter your code. It's possible to define the error messages elsewhere, but unless you're unusually disciplined, it is very tempting to embed the error message into your code. Indeed, we did exactly this in a previous example.
%
% The second and more important downside is that Strings are lossy. That is, if all errors are converted to strings, then the errors we pass to the caller become completely opaque. The only reasonable thing the caller can do with a String error is show it to the user. Certainly, inspecting the string to determine the type of error is not robust. (Admittedly, this downside is far more important inside of a library as opposed to, say, an application.)
%
% For example, the io::Error type embeds an io::ErrorKind, which is structured data that represents what went wrong during an IO operation. This is important because you might want to react differently depending on the error. (e.g., A BrokenPipe error might mean quitting your program gracefully while a NotFound error might mean exiting with an error code and showing an error to the user.) With io::ErrorKind, the caller can examine the type of an error with case analysis, which is strictly superior to trying to tease out the details of an error inside of a String.
%
% Instead of using a String as an error type in our previous example of reading an integer from a file, we can define our own error type that represents errors with structured data. We endeavor to not drop information from underlying errors in case the caller wants to inspect the details.
%
% The ideal way to represent one of many possibilities is to define our own sum type using enum. In our case, an error is either an io::Error or a num::ParseIntError, so a natural definition arises:
%
% use std::io;
% use std::num;
%
% // We derive `Debug` because all types should probably derive `Debug`.
% // This gives us a reasonable human readable description of `CliError` values.
% #[derive(Debug)]
% enum CliError {
%     Io(io::Error),
%     Parse(num::ParseIntError),
% }
%
% Tweaking our code is very easy. Instead of converting errors to strings, we simply convert them to our CliError type using the corresponding value constructor:
%
% use std::fs::File;
% use std::io::Read;
% use std::path::Path;
%
% fn file_double<P: AsRef<Path>>(file_path: P) -> Result<i32, CliError> {
%     let mut file = try!(File::open(file_path).map_err(CliError::Io));
%     let mut contents = String::new();
%     try!(file.read_to_string(&mut contents).map_err(CliError::Io));
%     let n: i32 = try!(contents.trim().parse().map_err(CliError::Parse));
%     Ok(2 * n)
% }
%
% fn main() {
%     match file_double("foobar") {
%         Ok(n) => println!("{}", n),
%         Err(err) => println!("Error: {:?}", err),
%     }
% }
%
% The only change here is switching map_err(|e| e.to_string()) (which converts errors to strings) to map_err(CliError::Io) or map_err(CliError::Parse). The caller gets to decide the level of detail to report to the user. In effect, using a String as an error type removes choices from the caller while using a custom enum error type like CliError gives the caller all of the conveniences as before in addition to structured data describing the error.
%
% A rule of thumb is to define your own error type, but a String error type will do in a pinch, particularly if you're writing an application. If you're writing a library, defining your own error type should be strongly preferred so that you don't remove choices from the caller unnecessarily.
% Standard library traits used for error handling
%
% The standard library defines two integral traits for error handling: std::error::Error and std::convert::From. While Error is designed specifically for generically describing errors, the From trait serves a more general role for converting values between two distinct types.
% The Error trait
%
% The Error trait is defined in the standard library:
%
% use std::fmt::{Debug, Display};
%
% trait Error: Debug + Display {
%   /// A short description of the error.
%   fn description(&self) -> &str;
%
%   /// The lower level cause of this error, if any.
%   fn cause(&self) -> Option<&Error> { None }
% }
%
% This trait is super generic because it is meant to be implemented for all types that represent errors. This will prove useful for writing composable code as we'll see later. Otherwise, the trait allows you to do at least the following things:
%
%     Obtain a Debug representation of the error.
%     Obtain a user-facing Display representation of the error.
%     Obtain a short description of the error (via the description method).
%     Inspect the causal chain of an error, if one exists (via the cause method).
%
% The first two are a result of Error requiring impls for both Debug and Display. The latter two are from the two methods defined on Error. The power of Error comes from the fact that all error types impl Error, which means errors can be existentially quantified as a trait object. This manifests as either Box<Error> or &Error. Indeed, the cause method returns an &Error, which is itself a trait object. We'll revisit the Error trait's utility as a trait object later.
%
% For now, it suffices to show an example implementing the Error trait. Let's use the error type we defined in the previous section:
%
% use std::io;
% use std::num;
%
% // We derive `Debug` because all types should probably derive `Debug`.
% // This gives us a reasonable human readable description of `CliError` values.
% #[derive(Debug)]
% enum CliError {
%     Io(io::Error),
%     Parse(num::ParseIntError),
% }
%
% This particular error type represents the possibility of two types of errors occurring: an error dealing with I/O or an error converting a string to a number. The error could represent as many error types as you want by adding new variants to the enum definition.
%
% Implementing Error is pretty straight-forward. It's mostly going to be a lot explicit case analysis.
%
% use std::error;
% use std::fmt;
%
% impl fmt::Display for CliError {
%     fn fmt(&self, f: &mut fmt::Formatter) -> fmt::Result {
%         match *self {
%             // Both underlying errors already impl `Display`, so we defer to
%             // their implementations.
%             CliError::Io(ref err) => write!(f, "IO error: {}", err),
%             CliError::Parse(ref err) => write!(f, "Parse error: {}", err),
%         }
%     }
% }
%
% impl error::Error for CliError {
%     fn description(&self) -> &str {
%         // Both underlying errors already impl `Error`, so we defer to their
%         // implementations.
%         match *self {
%             CliError::Io(ref err) => err.description(),
%             CliError::Parse(ref err) => err.description(),
%         }
%     }
%
%     fn cause(&self) -> Option<&error::Error> {
%         match *self {
%             // N.B. Both of these implicitly cast `err` from their concrete
%             // types (either `&io::Error` or `&num::ParseIntError`)
%             // to a trait object `&Error`. This works because both error types
%             // implement `Error`.
%             CliError::Io(ref err) => Some(err),
%             CliError::Parse(ref err) => Some(err),
%         }
%     }
% }
%
% We note that this is a very typical implementation of Error: match on your different error types and satisfy the contracts defined for description and cause.
% The From trait
%
% The std::convert::From trait is defined in the standard library:
%
% trait From<T> {
%     fn from(T) -> Self;
% }
%
% Deliciously simple, yes? From is very useful because it gives us a generic way to talk about conversion from a particular type T to some other type (in this case, “some other type” is the subject of the impl, or Self). The crux of From is the set of implementations provided by the standard library.
%
% Here are a few simple examples demonstrating how From works:
%
% let string: String = From::from("foo");
% let bytes: Vec<u8> = From::from("foo");
% let cow: ::std::borrow::Cow<str> = From::from("foo");
%
% OK, so From is useful for converting between strings. But what about errors? It turns out, there is one critical impl:
%
% impl<'a, E: Error + 'a> From<E> for Box<Error + 'a>
%
% This impl says that for any type that impls Error, we can convert it to a trait object Box<Error>. This may not seem terribly surprising, but it is useful in a generic context.
%
% Remember the two errors we were dealing with previously? Specifically, io::Error and num::ParseIntError. Since both impl Error, they work with From:
%
% use std::error::Error;
% use std::fs;
% use std::io;
% use std::num;
%
% // We have to jump through some hoops to actually get error values.
% let io_err: io::Error = io::Error::last_os_error();
% let parse_err: num::ParseIntError = "not a number".parse::<i32>().unwrap_err();
%
% // OK, here are the conversions.
% let err1: Box<Error> = From::from(io_err);
% let err2: Box<Error> = From::from(parse_err);
%
% There is a really important pattern to recognize here. Both err1 and err2 have the same type. This is because they are existentially quantified types, or trait objects. In particular, their underlying type is erased from the compiler's knowledge, so it truly sees err1 and err2 as exactly the same. Additionally, we constructed err1 and err2 using precisely the same function call: From::from. This is because From::from is overloaded on both its argument and its return type.
%
% This pattern is important because it solves a problem we had earlier: it gives us a way to reliably convert errors to the same type using the same function.
%
% Time to revisit an old friend; the try! macro.
% The real try! macro
%
% Previously, we presented this definition of try!:
%
% macro_rules! try {
%     ($e:expr) => (match $e {
%         Ok(val) => val,
%         Err(err) => return Err(err),
%     });
% }
%
% This is not its real definition. Its real definition is in the standard library:
%
% macro_rules! try {
%     ($e:expr) => (match $e {
%         Ok(val) => val,
%         Err(err) => return Err(::std::convert::From::from(err)),
%     });
% }
%
% There's one tiny but powerful change: the error value is passed through From::from. This makes the try! macro a lot more powerful because it gives you automatic type conversion for free.
%
% Armed with our more powerful try! macro, let's take a look at code we wrote previously to read a file and convert its contents to an integer:
%
% use std::fs::File;
% use std::io::Read;
% use std::path::Path;
%
% fn file_double<P: AsRef<Path>>(file_path: P) -> Result<i32, String> {
%     let mut file = try!(File::open(file_path).map_err(|e| e.to_string()));
%     let mut contents = String::new();
%     try!(file.read_to_string(&mut contents).map_err(|e| e.to_string()));
%     let n = try!(contents.trim().parse::<i32>().map_err(|e| e.to_string()));
%     Ok(2 * n)
% }
%
% Earlier, we promised that we could get rid of the map_err calls. Indeed, all we have to do is pick a type that From works with. As we saw in the previous section, From has an impl that lets it convert any error type into a Box<Error>:
%
% use std::error::Error;
% use std::fs::File;
% use std::io::Read;
% use std::path::Path;
%
% fn file_double<P: AsRef<Path>>(file_path: P) -> Result<i32, Box<Error>> {
%     let mut file = try!(File::open(file_path));
%     let mut contents = String::new();
%     try!(file.read_to_string(&mut contents));
%     let n = try!(contents.trim().parse::<i32>());
%     Ok(2 * n)
% }
%
% We are getting very close to ideal error handling. Our code has very little overhead as a result from error handling because the try! macro encapsulates three things simultaneously:
%
%     Case analysis.
%     Control flow.
%     Error type conversion.
%
% When all three things are combined, we get code that is unencumbered by combinators, calls to unwrap or case analysis.
%
% There's one little nit left: the Box<Error> type is opaque. If we return a Box<Error> to the caller, the caller can't (easily) inspect underlying error type. The situation is certainly better than String because the caller can call methods like description and cause, but the limitation remains: Box<Error> is opaque. (N.B. This isn't entirely true because Rust does have runtime reflection, which is useful in some scenarios that are beyond the scope of this section.)
%
% It's time to revisit our custom CliError type and tie everything together.
% Composing custom error types
%
% In the last section, we looked at the real try! macro and how it does automatic type conversion for us by calling From::from on the error value. In particular, we converted errors to Box<Error>, which works, but the type is opaque to callers.
%
% To fix this, we use the same remedy that we're already familiar with: a custom error type. Once again, here is the code that reads the contents of a file and converts it to an integer:
%
% use std::fs::File;
% use std::io::{self, Read};
% use std::num;
% use std::path::Path;
%
% // We derive `Debug` because all types should probably derive `Debug`.
% // This gives us a reasonable human readable description of `CliError` values.
% #[derive(Debug)]
% enum CliError {
%     Io(io::Error),
%     Parse(num::ParseIntError),
% }
%
% fn file_double_verbose<P: AsRef<Path>>(file_path: P) -> Result<i32, CliError> {
%     let mut file = try!(File::open(file_path).map_err(CliError::Io));
%     let mut contents = String::new();
%     try!(file.read_to_string(&mut contents).map_err(CliError::Io));
%     let n: i32 = try!(contents.trim().parse().map_err(CliError::Parse));
%     Ok(2 * n)
% }
%
% Notice that we still have the calls to map_err. Why? Well, recall the definitions of try! and From. The problem is that there is no From impl that allows us to convert from error types like io::Error and num::ParseIntError to our own custom CliError. Of course, it is easy to fix this! Since we defined CliError, we can impl From with it:
%
% use std::io;
% use std::num;
%
% impl From<io::Error> for CliError {
%     fn from(err: io::Error) -> CliError {
%         CliError::Io(err)
%     }
% }
%
% impl From<num::ParseIntError> for CliError {
%     fn from(err: num::ParseIntError) -> CliError {
%         CliError::Parse(err)
%     }
% }
%
% All these impls are doing is teaching From how to create a CliError from other error types. In our case, construction is as simple as invoking the corresponding value constructor. Indeed, it is typically this easy.
%
% We can finally rewrite file_double:
%
%
% use std::fs::File;
% use std::io::Read;
% use std::path::Path;
%
% fn file_double<P: AsRef<Path>>(file_path: P) -> Result<i32, CliError> {
%     let mut file = try!(File::open(file_path));
%     let mut contents = String::new();
%     try!(file.read_to_string(&mut contents));
%     let n: i32 = try!(contents.trim().parse());
%     Ok(2 * n)
% }
%
% The only thing we did here was remove the calls to map_err. They are no longer needed because the try! macro invokes From::from on the error value. This works because we've provided From impls for all the error types that could appear.
%
% If we modified our file_double function to perform some other operation, say, convert a string to a float, then we'd need to add a new variant to our error type:
%
% use std::io;
% use std::num;
%
% enum CliError {
%     Io(io::Error),
%     ParseInt(num::ParseIntError),
%     ParseFloat(num::ParseFloatError),
% }
%
% And add a new From impl:
%
%
% use std::num;
%
% impl From<num::ParseFloatError> for CliError {
%     fn from(err: num::ParseFloatError) -> CliError {
%         CliError::ParseFloat(err)
%     }
% }
%
% And that's it!
% Advice for library writers
%
% If your library needs to report custom errors, then you should probably define your own error type. It's up to you whether or not to expose its representation (like ErrorKind) or keep it hidden (like ParseIntError). Regardless of how you do it, it's usually good practice to at least provide some information about the error beyond its String representation. But certainly, this will vary depending on use cases.
%
% At a minimum, you should probably implement the Error trait. This will give users of your library some minimum flexibility for composing errors. Implementing the Error trait also means that users are guaranteed the ability to obtain a string representation of an error (because it requires impls for both fmt::Debug and fmt::Display).
%
% Beyond that, it can also be useful to provide implementations of From on your error types. This allows you (the library author) and your users to compose more detailed errors. For example, csv::Error provides From impls for both io::Error and byteorder::Error.
%
% Finally, depending on your tastes, you may also want to define a Result type alias, particularly if your library defines a single error type. This is used in the standard library for io::Result and fmt::Result.
% Case study: A program to read population data
%
% This section was long, and depending on your background, it might be rather dense. While there is plenty of example code to go along with the prose, most of it was specifically designed to be pedagogical. So, we're going to do something new: a case study.
%
% For this, we're going to build up a command line program that lets you query world population data. The objective is simple: you give it a location and it will tell you the population. Despite the simplicity, there is a lot that can go wrong!
%
% The data we'll be using comes from the Data Science Toolkit. I've prepared some data from it for this exercise. You can either grab the world population data (41MB gzip compressed, 145MB uncompressed) or only the US population data (2.2MB gzip compressed, 7.2MB uncompressed).
%
% Up until now, we've kept the code limited to Rust's standard library. For a real task like this though, we'll want to at least use something to parse CSV data, parse the program arguments and decode that stuff into Rust types automatically. For that, we'll use the csv, and rustc-serialize crates.
% Initial setup
%
% We're not going to spend a lot of time on setting up a project with Cargo because it is already covered well in the Cargo section and Cargo's documentation.
%
% To get started from scratch, run cargo new --bin city-pop and make sure your Cargo.toml looks something like this:
%
% [package]
% name = "city-pop"
% version = "0.1.0"
% authors = ["Andrew Gallant <jamslam@gmail.com>"]
%
% [[bin]]
% name = "city-pop"
%
% [dependencies]
% csv = "0.*"
% rustc-serialize = "0.*"
% getopts = "0.*"
%
% You should already be able to run:
%
% cargo build --release
% ./target/release/city-pop
% # Outputs: Hello, world!
%
% Argument parsing
%
% Let's get argument parsing out of the way. We won't go into too much detail on Getopts, but there is some good documentation describing it. The short story is that Getopts generates an argument parser and a help message from a vector of options (The fact that it is a vector is hidden behind a struct and a set of methods). Once the parsing is done, we can decode the program arguments into a Rust struct. From there, we can get information about the flags, for instance, whether they were passed in, and what arguments they had. Here's our program with the appropriate extern crate statements, and the basic argument setup for Getopts:
%
% extern crate getopts;
% extern crate rustc_serialize;
%
% use getopts::Options;
% use std::env;
%
% fn print_usage(program: &str, opts: Options) {
%     println!("{}", opts.usage(&format!("Usage: {} [options] <data-path> <city>", program)));
% }
%
% fn main() {
%     let args: Vec<String> = env::args().collect();
%     let program = args[0].clone();
%
%     let mut opts = Options::new();
%     opts.optflag("h", "help", "Show this usage message.");
%
%     let matches = match opts.parse(&args[1..]) {
%         Ok(m)  => { m }
%         Err(e) => { panic!(e.to_string()) }
%     };
%     if matches.opt_present("h") {
%         print_usage(&program, opts);
%         return;
%     }
%     let data_path = args[1].clone();
%     let city = args[2].clone();
%
%     // Do stuff with information
% }
%
% First, we get a vector of the arguments passed into our program. We then store the first one, knowing that it is our program's name. Once that's done, we set up our argument flags, in this case a simplistic help message flag. Once we have the argument flags set up, we use Options.parse to parse the argument vector (starting from index one, because index 0 is the program name). If this was successful, we assign matches to the parsed object, if not, we panic. Once past that, we test if the user passed in the help flag, and if so print the usage message. The option help messages are constructed by Getopts, so all we have to do to print the usage message is tell it what we want it to print for the program name and template. If the user has not passed in the help flag, we assign the proper variables to their corresponding arguments.
% Writing the logic
%
% We all write code differently, but error handling is usually the last thing we want to think about. This isn't great for the overall design of a program, but it can be useful for rapid prototyping. Because Rust forces us to be explicit about error handling (by making us call unwrap), it is easy to see which parts of our program can cause errors.
%
% In this case study, the logic is really simple. All we need to do is parse the CSV data given to us and print out a field in matching rows. Let's do it. (Make sure to add extern crate csv; to the top of your file.)
%
% use std::fs::File;
% use std::path::Path;
%
% // This struct represents the data in each row of the CSV file.
% // Type based decoding absolves us of a lot of the nitty gritty error
% // handling, like parsing strings as integers or floats.
% #[derive(Debug, RustcDecodable)]
% struct Row {
%     country: String,
%     city: String,
%     accent_city: String,
%     region: String,
%
%     // Not every row has data for the population, latitude or longitude!
%     // So we express them as `Option` types, which admits the possibility of
%     // absence. The CSV parser will fill in the correct value for us.
%     population: Option<u64>,
%     latitude: Option<f64>,
%     longitude: Option<f64>,
% }
%
% fn print_usage(program: &str, opts: Options) {
%     println!("{}", opts.usage(&format!("Usage: {} [options] <data-path> <city>", program)));
% }
%
% fn main() {
%     let args: Vec<String> = env::args().collect();
%     let program = args[0].clone();
%
%     let mut opts = Options::new();
%     opts.optflag("h", "help", "Show this usage message.");
%
%     let matches = match opts.parse(&args[1..]) {
%         Ok(m)  => { m }
%         Err(e) => { panic!(e.to_string()) }
%     };
%
%     if matches.opt_present("h") {
%         print_usage(&program, opts);
%         return;
%     }
%
%     let data_file = args[1].clone();
%     let data_path = Path::new(&data_file);
%     let city = args[2].clone();
%
%     let file = File::open(data_path).unwrap();
%     let mut rdr = csv::Reader::from_reader(file);
%
%     for row in rdr.decode::<Row>() {
%         let row = row.unwrap();
%
%         if row.city == city {
%             println!("{}, {}: {:?}",
%                 row.city, row.country,
%                 row.population.expect("population count"));
%         }
%     }
% }
%
% Let's outline the errors. We can start with the obvious: the three places that unwrap is called:
%
%     File::open can return an io::Error.
%     csv::Reader::decode decodes one record at a time, and decoding a record (look at the Item associated type on the Iterator impl) can produce a csv::Error.
%     If row.population is None, then calling expect will panic.
%
% Are there any others? What if we can't find a matching city? Tools like grep will return an error code, so we probably should too. So we have logic errors specific to our problem, IO errors and CSV parsing errors. We're going to explore two different ways to approach handling these errors.
%
% I'd like to start with Box<Error>. Later, we'll see how defining our own error type can be useful.
% Error handling with Box<Error>
%
% Box<Error> is nice because it just works. You don't need to define your own error types and you don't need any From implementations. The downside is that since Box<Error> is a trait object, it erases the type, which means the compiler can no longer reason about its underlying type.
%
% Previously we started refactoring our code by changing the type of our function from T to Result<T, OurErrorType>. In this case, OurErrorType is only Box<Error>. But what's T? And can we add a return type to main?
%
% The answer to the second question is no, we can't. That means we'll need to write a new function. But what is T? The simplest thing we can do is to return a list of matching Row values as a Vec<Row>. (Better code would return an iterator, but that is left as an exercise to the reader.)
%
% Let's refactor our code into its own function, but keep the calls to unwrap. Note that we opt to handle the possibility of a missing population count by simply ignoring that row.
%
% struct Row {
%     // unchanged
% }
%
% struct PopulationCount {
%     city: String,
%     country: String,
%     // This is no longer an `Option` because values of this type are only
%     // constructed if they have a population count.
%     count: u64,
% }
%
% fn print_usage(program: &str, opts: Options) {
%     println!("{}", opts.usage(&format!("Usage: {} [options] <data-path> <city>", program)));
% }
%
% fn search<P: AsRef<Path>>(file_path: P, city: &str) -> Vec<PopulationCount> {
%     let mut found = vec![];
%     let file = File::open(file_path).unwrap();
%     let mut rdr = csv::Reader::from_reader(file);
%     for row in rdr.decode::<Row>() {
%         let row = row.unwrap();
%         match row.population {
%             None => { } // skip it
%             Some(count) => if row.city == city {
%                 found.push(PopulationCount {
%                     city: row.city,
%                     country: row.country,
%                     count: count,
%                 });
%             },
%         }
%     }
%     found
% }
%
% fn main() {
%     let args: Vec<String> = env::args().collect();
%     let program = args[0].clone();
%
%     let mut opts = Options::new();
%     opts.optflag("h", "help", "Show this usage message.");
%
%     let matches = match opts.parse(&args[1..]) {
%         Ok(m)  => { m }
%         Err(e) => { panic!(e.to_string()) }
%     };
%     if matches.opt_present("h") {
%         print_usage(&program, opts);
%         return;
%     }
%
%     let data_file = args[1].clone();
%     let data_path = Path::new(&data_file);
%     let city = args[2].clone();
%     for pop in search(&data_path, &city) {
%         println!("{}, {}: {:?}", pop.city, pop.country, pop.count);
%     }
% }
%
% While we got rid of one use of expect (which is a nicer variant of unwrap), we still should handle the absence of any search results.
%
% To convert this to proper error handling, we need to do the following:
%
%     Change the return type of search to be Result<Vec<PopulationCount>, Box<Error>>.
%     Use the try! macro so that errors are returned to the caller instead of panicking the program.
%     Handle the error in main.
%
% Let's try it:
%
% use std::error::Error;
%
% // The rest of the code before this is unchanged
%
% fn search<P: AsRef<Path>>
%          (file_path: P, city: &str)
%          -> Result<Vec<PopulationCount>, Box<Error+Send+Sync>> {
%     let mut found = vec![];
%     let file = try!(File::open(file_path));
%     let mut rdr = csv::Reader::from_reader(file);
%     for row in rdr.decode::<Row>() {
%         let row = try!(row);
%         match row.population {
%             None => { } // skip it
%             Some(count) => if row.city == city {
%                 found.push(PopulationCount {
%                     city: row.city,
%                     country: row.country,
%                     count: count,
%                 });
%             },
%         }
%     }
%     if found.is_empty() {
%         Err(From::from("No matching cities with a population were found."))
%     } else {
%         Ok(found)
%     }
% }
%
% Instead of x.unwrap(), we now have try!(x). Since our function returns a Result<T, E>, the try! macro will return early from the function if an error occurs.
%
% There is one big gotcha in this code: we used Box<Error + Send + Sync> instead of Box<Error>. We did this so we could convert a plain string to an error type. We need these extra bounds so that we can use the corresponding From impls:
%
% // We are making use of this impl in the code above, since we call `From::from`
% // on a `&'static str`.
% impl<'a, 'b> From<&'b str> for Box<Error + Send + Sync + 'a>
%
% // But this is also useful when you need to allocate a new string for an
% // error message, usually with `format!`.
% impl From<String> for Box<Error + Send + Sync>
%
% Since search now returns a Result<T, E>, main should use case analysis when calling search:
%
% ...
% match search(&data_file, &city) {
%     Ok(pops) => {
%         for pop in pops {
%             println!("{}, {}: {:?}", pop.city, pop.country, pop.count);
%         }
%     }
%     Err(err) => println!("{}", err)
% }
% ...
%
% Now that we've seen how to do proper error handling with Box<Error>, let's try a different approach with our own custom error type. But first, let's take a quick break from error handling and add support for reading from stdin.
% Reading from stdin
%
% In our program, we accept a single file for input and do one pass over the data. This means we probably should be able to accept input on stdin. But maybe we like the current format too—so let's have both!
%
% Adding support for stdin is actually quite easy. There are only three things we have to do:
%
%     Tweak the program arguments so that a single parameter—the city—can be accepted while the population data is read from stdin.
%     Modify the program so that an option -f can take the file, if it is not passed into stdin.
%     Modify the search function to take an optional file path. When None, it should know to read from stdin.
%
% First, here's the new usage:
%
% fn print_usage(program: &str, opts: Options) {
%     println!("{}", opts.usage(&format!("Usage: {} [options] <city>", program)));
% }
%
% The next part is going to be only a little harder:
%
% ...
% let mut opts = Options::new();
% opts.optopt("f", "file", "Choose an input file, instead of using STDIN.", "NAME");
% opts.optflag("h", "help", "Show this usage message.");
% ...
% let file = matches.opt_str("f");
% let data_file = file.as_ref().map(Path::new);
%
% let city = if !matches.free.is_empty() {
%     matches.free[0].clone()
% } else {
%     print_usage(&program, opts);
%     return;
% };
%
% match search(&data_file, &city) {
%     Ok(pops) => {
%         for pop in pops {
%             println!("{}, {}: {:?}", pop.city, pop.country, pop.count);
%         }
%     }
%     Err(err) => println!("{}", err)
% }
% ...
%
% In this piece of code, we take file (which has the type Option<String>), and convert it to a type that search can use, in this case, &Option<AsRef<Path>>. To do this, we take a reference of file, and map Path::new onto it. In this case, as_ref() converts the Option<String> into an Option<&str>, and from there, we can execute Path::new to the content of the optional, and return the optional of the new value. Once we have that, it is a simple matter of getting the city argument and executing search.
%
% Modifying search is slightly trickier. The csv crate can build a parser out of any type that implements io::Read. But how can we use the same code over both types? There's actually a couple ways we could go about this. One way is to write search such that it is generic on some type parameter R that satisfies io::Read. Another way is to use trait objects:
%
% use std::io;
%
% // The rest of the code before this is unchanged
%
% fn search<P: AsRef<Path>>
%          (file_path: &Option<P>, city: &str)
%          -> Result<Vec<PopulationCount>, Box<Error+Send+Sync>> {
%     let mut found = vec![];
%     let input: Box<io::Read> = match *file_path {
%         None => Box::new(io::stdin()),
%         Some(ref file_path) => Box::new(try!(File::open(file_path))),
%     };
%     let mut rdr = csv::Reader::from_reader(input);
%     // The rest remains unchanged!
% }
%
% Error handling with a custom type
%
% Previously, we learned how to compose errors using a custom error type. We did this by defining our error type as an enum and implementing Error and From.
%
% Since we have three distinct errors (IO, CSV parsing and not found), let's define an enum with three variants:
%
% #[derive(Debug)]
% enum CliError {
%     Io(io::Error),
%     Csv(csv::Error),
%     NotFound,
% }
%
% And now for impls on Display and Error:
%
% impl fmt::Display for CliError {
%     fn fmt(&self, f: &mut fmt::Formatter) -> fmt::Result {
%         match *self {
%             CliError::Io(ref err) => err.fmt(f),
%             CliError::Csv(ref err) => err.fmt(f),
%             CliError::NotFound => write!(f, "No matching cities with a \
%                                              population were found."),
%         }
%     }
% }
%
% impl Error for CliError {
%     fn description(&self) -> &str {
%         match *self {
%             CliError::Io(ref err) => err.description(),
%             CliError::Csv(ref err) => err.description(),
%             CliError::NotFound => "not found",
%         }
%     }
% }
%
% Before we can use our CliError type in our search function, we need to provide a couple From impls. How do we know which impls to provide? Well, we'll need to convert from both io::Error and csv::Error to CliError. Those are the only external errors, so we'll only need two From impls for now:
%
% impl From<io::Error> for CliError {
%     fn from(err: io::Error) -> CliError {
%         CliError::Io(err)
%     }
% }
%
% impl From<csv::Error> for CliError {
%     fn from(err: csv::Error) -> CliError {
%         CliError::Csv(err)
%     }
% }
%
% The From impls are important because of how try! is defined. In particular, if an error occurs, From::from is called on the error, which in this case, will convert it to our own error type CliError.
%
% With the From impls done, we only need to make two small tweaks to our search function: the return type and the “not found” error. Here it is in full:
%
% fn search<P: AsRef<Path>>
%          (file_path: &Option<P>, city: &str)
%          -> Result<Vec<PopulationCount>, CliError> {
%     let mut found = vec![];
%     let input: Box<io::Read> = match *file_path {
%         None => Box::new(io::stdin()),
%         Some(ref file_path) => Box::new(try!(File::open(file_path))),
%     };
%     let mut rdr = csv::Reader::from_reader(input);
%     for row in rdr.decode::<Row>() {
%         let row = try!(row);
%         match row.population {
%             None => { } // skip it
%             Some(count) => if row.city == city {
%                 found.push(PopulationCount {
%                     city: row.city,
%                     country: row.country,
%                     count: count,
%                 });
%             },
%         }
%     }
%     if found.is_empty() {
%         Err(CliError::NotFound)
%     } else {
%         Ok(found)
%     }
% }
%
% No other changes are necessary.
% Adding functionality
%
% Writing generic code is great, because generalizing stuff is cool, and it can then be useful later. But sometimes, the juice isn't worth the squeeze. Look at what we just did in the previous step:
%
%     Defined a new error type.
%     Added impls for Error, Display and two for From.
%
% The big downside here is that our program didn't improve a whole lot. There is quite a bit of overhead to representing errors with enums, especially in short programs like this.
%
% One useful aspect of using a custom error type like we've done here is that the main function can now choose to handle errors differently. Previously, with Box<Error>, it didn't have much of a choice: just print the message. We're still doing that here, but what if we wanted to, say, add a --quiet flag? The --quiet flag should silence any verbose output.
%
% Right now, if the program doesn't find a match, it will output a message saying so. This can be a little clumsy, especially if you intend for the program to be used in shell scripts.
%
% So let's start by adding the flags. Like before, we need to tweak the usage string and add a flag to the Option variable. Once we've done that, Getopts does the rest:
%
% ...
% let mut opts = Options::new();
% opts.optopt("f", "file", "Choose an input file, instead of using STDIN.", "NAME");
% opts.optflag("h", "help", "Show this usage message.");
% opts.optflag("q", "quiet", "Silences errors and warnings.");
% ...
%
% Now we only need to implement our “quiet” functionality. This requires us to tweak the case analysis in main:
%
% match search(&args.arg_data_path, &args.arg_city) {
%     Err(CliError::NotFound) if args.flag_quiet => process::exit(1),
%     Err(err) => panic!("{}", err),
%     Ok(pops) => for pop in pops {
%         println!("{}, {}: {:?}", pop.city, pop.country, pop.count);
%     }
% }
%
% Certainly, we don't want to be quiet if there was an IO error or if the data failed to parse. Therefore, we use case analysis to check if the error type is NotFound and if --quiet has been enabled. If the search failed, we still quit with an exit code (following grep's convention).
%
% If we had stuck with Box<Error>, then it would be pretty tricky to implement the --quiet functionality.
%
% This pretty much sums up our case study. From here, you should be ready to go out into the world and write your own programs and libraries with proper error handling.
% The Short Story
%
% Since this section is long, it is useful to have a quick summary for error handling in Rust. These are some good “rules of thumb." They are emphatically not commandments. There are probably good reasons to break every one of these heuristics!
%
%     If you're writing short example code that would be overburdened by error handling, it's probably fine to use unwrap (whether that's Result::unwrap, Option::unwrap or preferably Option::expect). Consumers of your code should know to use proper error handling. (If they don't, send them here!)
%     If you're writing a quick 'n' dirty program, don't feel ashamed if you use unwrap. Be warned: if it winds up in someone else's hands, don't be surprised if they are agitated by poor error messages!
%     If you're writing a quick 'n' dirty program and feel ashamed about panicking anyway, then use either a String or a Box<Error + Send + Sync> for your error type (the Box<Error + Send + Sync> type is because of the available From impls).
%     Otherwise, in a program, define your own error types with appropriate From and Error impls to make the try! macro more ergonomic.
%     If you're writing a library and your code can produce errors, define your own error type and implement the std::error::Error trait. Where appropriate, implement From to make both your library code and the caller's code easier to write. (Because of Rust's coherence rules, callers will not be able to impl From on your error type, so your library should do it.)
%     Learn the combinators defined on Option and Result. Using them exclusively can be a bit tiring at times, but I've personally found a healthy mix of try! and combinators to be quite appealing. and_then, map and unwrap_or are my favorites.
%
