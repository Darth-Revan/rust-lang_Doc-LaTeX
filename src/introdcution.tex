Welcome! This book will teach you about the \href{https://www.rust-lang.org/}{Rust Programming Language}.
Rust is a systems programming language focused on three goals: safety, speed, and concurrency. It maintains 
these goals without having a garbage collector, making it a useful language for a number of use cases other 
languages aren't good at: embedding in other languages, programs with specific space and time requirements, 
and writing low-level code, like device drivers and operating systems. It improves on current languages targeting 
this space by having a number of compile-time safety checks that produce no runtime overhead, while eliminating 
all data races. Rust also aims to achieve 'zero-cost abstractions' even though some of these abstractions feel 
like those of a high-level language. Even then, Rust still allows precise control like a low-level language would.

\blank
\enquote{The Rust Programming Language} is split into chapters. This introduction is the first. After this:

\begin{itemize}
    \item{\nameref{sec:gettingstarted} - Set up your computer for Rust development.}
    \item{\nameref{sec:tutorial} - Learn some Rust with a small project.}
    \item{\nameref{sec:syntax} - Each bit of Rust, broken down into small chunks.}
    \item{\nameref{sec:effective_rust} - Higher-level concepts for writing excellent Rust code.}
    \item{\nameref{sec:nightly_rust} - Cutting-edge features that aren't in stable builds yet.}
    \item{\nameref{sec:glossary} - A reference of terms used in the book.}
    \item{\nameref{sec:bib} - Background on Rust's influences, papers about Rust.}
\end{itemize}

\subsection*{Contributing}

The source files from which this book is generated can be found on
\href{https://github.com/rust-lang/rust/tree/master/src/doc/book}{GitHub}.