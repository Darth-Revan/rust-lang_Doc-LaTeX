This chapter breaks Rust down into small chunks, one for each concept.

\blank

If you'd like to learn Rust from the bottom up, reading this in order is a great way to do that.

\blank

These sections also form a reference for each concept, so if you're reading another tutorial and find something confusing, 
you can find it explained somewhere in here.

\subsection{Variable Bindings}
\label{sec:syntax_variableBindings}
Virtually every non-'Hello World' Rust program uses \emph{variable bindings}. They bind some value to a name, so it can be used 
later. \keylet\ is used to introduce a binding, like this:

\begin{rustc}
fn main() {
    let x = 5;
}
\end{rustc}

Putting \code{fn main() \{} in each example is a bit tedious, so we'll leave that out in the future. If you're following along, 
make sure to edit your \code{main()} function, rather than leaving it off. Otherwise, you'll get an error.

\subsection*{Patterns}

% TODO make patern a hyperref
In many languages, a variable binding would be called a \emph{variable}, but Rust's variable bindings have a few tricks up 
their sleeves. For example the left-hand side of a \keylet\ expression is a 'pattern', not a variable name. This means we can 
do things like:

\begin{rustc}
let (x, y) = (1, 2);
\end{rustc}

% TODO make their own section a hyperref
After this expression is evaluated, \x\ will be one, and \y\ will be two. Patterns are really powerful, and have their 
own section in the book. We don't need those features for now, so we'll keep this in the back of our minds as we go forward.

\subsection*{Type annotations}

Rust is a statically typed language, which means that we specify our types up front, and they're checked at compile time. So why 
does our first example compile? Well, Rust has this thing called 'type inference'. If it can figure out what the type of something
is, Rust doesn't require you to actually type it out.

\blank

We can add the type if we want to, though. Types come after a colon (\code{:}):

\begin{rustc}
let x: i32 = 5;
\end{rustc}

If I asked you to read this out loud to the rest of the class, you'd say “\x\ is a binding with the type \itt\ and the
value \code{five}.”

\blank

In this case we chose to represent \x\ as a 32-bit signed integer. Rust has many different primitive integer types. They 
begin with \code{i} for signed integers and u for unsigned integers. The possible integer sizes are 8, 16, 32, and 64 bits.

\blank

In future examples, we may annotate the type in a comment. The examples will look like this:

\begin{rustc}
fn main() {
    let x = 5; // x: i32
}
\end{rustc}

Note the similarities between this annotation and the syntax you use with \keylet. Including these kinds of comments is not
idiomatic Rust, but we'll occasionally include them to help you understand what the types that Rust infers are.

\subsection*{Mutability}

By default, bindings are \emph{immutable}. This code will not compile:

\begin{rustc}
let x = 5;
x = 10;
\end{rustc}

It will give you this error:

\begin{verbatim}
error: re-assignment of immutable variable `x`
     x = 10;
     ^~~~~~~
\end{verbatim}

If you want a binding to be mutable, you can use \mut:

\begin{rustc}
let mut x = 5; // mut x: i32
x = 10;
\end{rustc}

There is no single reason that bindings are immutable by default, but we can think about it through one of Rust's primary focuses:
safety. If you forget to say \mut, the compiler will catch it, and let you know that you have mutated something you may not
have intended to mutate. If bindings were mutable by default, the compiler would not be able to tell you this. If you \emph{did}
intend mutation, then the solution is quite easy: add \mut.

\blank

There are other good reasons to avoid mutable state when possible, but they're out of the scope of this guide. In general, you can
often avoid explicit mutation, and so it is preferable in Rust. That said, sometimes, mutation is what you need, so it's not 
verboten.

\subsection*{Initializing bindings}

Rust variable bindings have one more aspect that differs from other languages: bindings are required to be initialized with a 
value before you're allowed to use them.

\blank

Let's try it out. Change your \code{src/main.rs} file to look like this:

\begin{rustc}
fn main() {
    let x: i32;

    println!("Hello world!");
}
\end{rustc}

You can use \code{cargo build} on the command line to build it. You'll get a warning, but it will still print "Hello, world!":

\begin{verbatim}
   Compiling hello_world v0.0.1 (file:///home/you/projects/hello_world)
src/main.rs:2:9: 2:10 warning: unused variable: `x`, #[warn(unused_variable)]
   on by default
src/main.rs:2     let x: i32;
                      ^
\end{verbatim}

Rust warns us that we never use the variable binding, but since we never use it, no harm, no foul. Things change if we try to
actually use this \x, however. Let's do that. Change your program to look like this:

\begin{rustc}
fn main() {
    let x: i32;

    println!("The value of x is: {}", x);
}
\end{rustc}

And try to build it. You'll get an error:

\begin{verbatim}
$ cargo build
   Compiling hello_world v0.0.1 (file:///home/you/projects/hello_world)
src/main.rs:4:39: 4:40 error: use of possibly uninitialized variable: `x`
src/main.rs:4     println!("The value of x is: {}", x);
                                                    ^
note: in expansion of format_args!
<std macros>:2:23: 2:77 note: expansion site
<std macros>:1:1: 3:2 note: in expansion of println!
src/main.rs:4:5: 4:42 note: expansion site
error: aborting due to previous error
Could not compile `hello_world`.
\end{verbatim}

Rust will not let us use a value that has not been initialized. Next, let's talk about this stuff we've added to \println.

\blank

If you include two curly braces (\code{\{\}}, some call them moustaches...) in your string to print, Rust will interpret this as 
a request to interpolate some sort of value. \emph{String interpolation} is a computer science term that means "stick in the middle
of a string." We add a comma, and then \x, to indicate that we want \x\ to be the value we're interpolating. The comma 
is used to separate arguments we pass to functions and macros, if you're passing more than one.

\blank

When you use the curly braces, Rust will attempt to display the value in a meaningful way by checking out its type. If you want 
to specify the format in a more detailed manner, there are a \href{https://doc.rust-lang.org/std/fmt/}{wide number of options
available}. For now, we'll stick to the default: integers aren't very complicated to print.

\subsection*{Scope and shadowing}

Let's get back to bindings. Variable bindings have a scope - they are constrained to live in a block they were defined in. A block 
is a collection of statements enclosed by \code{\{} and \code{\}}. Function definitions are also blocks! In the following example 
we define two variable bindings, \x\ and \y, which live in different blocks. \x\ can be accessed from inside the 
\code{fn main() \{\}} block, while \y\ can be accessed only from inside the inner block:

\begin{rustc}
fn main() {
    let x: i32 = 17;
    {
        let y: i32 = 3;
        println!("The value of x is {} and value of y is {}", x, y);
    }
    println!("The value of x is {} and value of y is {}", x, y); // This won't work
}
\end{rustc}

The first \println\ would print "The value of x is 17 and the value of y is 3", but this example cannot be compiled
successfully, because the second \println\ cannot access the value of \y, since it is not in scope anymore. Instead 
we get this error:

\begin{verbatim}
$ cargo build
   Compiling hello v0.1.0 (file:///home/you/projects/hello_world)
main.rs:7:62: 7:63 error: unresolved name `y`. Did you mean `x`? [E0425]
main.rs:7     println!("The value of x is {} and value of y is {}", x, y); // This won't work
                                                                       ^
note: in expansion of format_args!
<std macros>:2:25: 2:56 note: expansion site
<std macros>:1:1: 2:62 note: in expansion of print!
<std macros>:3:1: 3:54 note: expansion site
<std macros>:1:1: 3:58 note: in expansion of println!
main.rs:7:5: 7:65 note: expansion site
main.rs:7:62: 7:63 help: run `rustc --explain E0425` to see a detailed explanation
error: aborting due to previous error
Could not compile `hello`.

To learn more, run the command again with --verbose.
\end{verbatim}

Additionally, variable bindings can be shadowed. This means that a later variable binding with the same name as another 
binding, that's currently in scope, will override the previous binding.

\begin{rustc}
let x: i32 = 8;
{
    println!("{}", x); // Prints "8"
    let x = 12;
    println!("{}", x); // Prints "12"
}
println!("{}", x); // Prints "8"
let x =  42;
println!("{}", x); // Prints "42"
\end{rustc}

Shadowing and mutable bindings may appear as two sides of the same coin, but they are two distinct concepts that can't always be 
used interchangeably. For one, shadowing enables us to rebind a name to a value of a different type. It is also possible to 
change the mutability of a binding.

\begin{rustc}
let mut x: i32 = 1;
x = 7;
let x = x; // x is now immutable and is bound to 7

let y = 4;
let y = "I can also be bound to text!"; // y is now of a different type
\end{rustc}


\subsection{Functions}
\label{sec:syntax_functions}
Every Rust program has at least one function, the \code{main} function:

\begin{rustc}
fn main() {
}
\end{rustc}

This is the simplest possible function declaration. As we mentioned before, \code{fn} says 'this is a function', followed by 
the name, some parentheses because this function takes no arguments, and then some curly braces to indicate the body. Here's 
a function named \code{foo}:

\begin{rustc}
fn foo() {
}
\end{rustc}

So, what about taking arguments? Here's a function that prints a number:

\begin{rustc}
fn print_number(x: i32) {
  println!("x is: {}", x);
}
\end{rustc}

Here's a complete program that uses \code{print\_number}:

\begin{rustc}
fn main() {
    print_number(5);
}

fn print_number(x: i32) {
    println!("x is: {}", x);
}
\end{rustc}

As you can see, function arguments work very similar to \keylet\ declarations: you add a type to the argument name, after a 
colon.

\blank

Here's a complete program that adds two numbers together and prints them:

\begin{rustc}
fn main() {
    print_sum(5, 6);
}

fn print_sum(x: i32, y: i32) {
    println!("sum is: {}", x + y);
}
\end{rustc}

You separate arguments with a comma, both when you call the function, as well as when you declare it.

\blank

Unlike \keylet, you must declare the types of function arguments. This does not work:

\begin{rustc}
fn print_sum(x, y) {
    println!("sum is: {}", x + y);
}
\end{rustc}

You get this error:

\begin{verbatim}
expected one of `!`, `:`, or `@`, found `)`
fn print_number(x, y) {
\end{verbatim}

This is a deliberate design decision. While full-program inference is possible, languages which have it, like Haskell, often 
suggest that documenting your types explicitly is a best-practice. We agree that forcing functions to declare types while 
allowing for inference inside of function bodies is a wonderful sweet spot between full inference and no inference.

\blank

What about returning a value? Here's a function that adds one to an integer:

\begin{rustc}
fn add_one(x: i32) -> i32 {
    x + 1
}
\end{rustc}

Rust functions return exactly one value, and you declare the type after an 'arrow', which is a dash (\code{-}) followed by a 
greater-than sign (\code{>}). The last line of a function determines what it returns. You'll note the lack of a semicolon here. 
If we added it in:

\begin{rustc}
fn add_one(x: i32) -> i32 {
    x + 1;
}
\end{rustc}

We would get an error:

\begin{verbatim}
error: not all control paths return a value
fn add_one(x: i32) -> i32 {
     x + 1;
}

help: consider removing this semicolon:
     x + 1;
          ^
\end{verbatim}

This reveals two interesting things about Rust: it is an expression-based language, and semicolons are different from semicolons 
in other 'curly brace and semicolon'-based languages. These two things are related.

\subsection*{Expressions vs. Statements}

Rust is primarily an expression-based language. There are only two kinds of statements, and everything else is an expression.

\blank

So what's the difference? Expressions return a value, and statements do not. That's why we end up with 'not all control paths 
return a value' here: the statement \code{x + 1;} doesn't return a value. There are two kinds of statements in Rust: 'declaration
statements' and 'expression statements'. Everything else is an expression. Let's talk about declaration statements first.

\blank

In some languages, variable bindings can be written as expressions, not statements. Like Ruby:

\begin{verbatim}
x = y = 5
\end{verbatim}

In Rust, however, using \keylet\ to introduce a binding is \emph{not} an expression. The following will produce a compile-time
error:

\begin{rustc}
let x = (let y = 5); // expected identifier, found keyword `let`
\end{rustc}

The compiler is telling us here that it was expecting to see the beginning of an expression, and a \keylet\ can only begin a
statement, not an expression.

\blank

Note that assigning to an already-bound variable (e.g. \code{y = 5}) is still an expression, although its value is not 
particularly useful. Unlike other languages where an assignment evaluates to the assigned value (e.g. \code{5} in the previous
example), in Rust the value of an assignment is an empty tuple \code{()} because the assigned value can have only one owner 
(see \nameref{sec:syntax_ownership}), and any other returned value would be too surprising:

\begin{rustc}
let mut y = 5;

let x = (y = 6);  // x has the value `()`, not `6`
\end{rustc}

The second kind of statement in Rust is the \emph{expression statement}. Its purpose is to turn any expression into a statement. 
In practical terms, Rust's grammar expects statements to follow other statements. This means that you use semicolons to separate
expressions from each other. This means that Rust looks a lot like most other languages that require you to use semicolons at the 
end of every line, and you will see semicolons at the end of almost every line of Rust code you see.

\blank

What is this exception that makes us say "almost"? You saw it already, in this code:

\begin{rustc}
fn add_one(x: i32) -> i32 {
    x + 1
}
\end{rustc}

Our function claims to return an \itt, but with a semicolon, it would return \code{()} instead. Rust realizes this 
probably isn't what we want, and suggests removing the semicolon in the error we saw before.

\subsection*{Early returns}

But what about early returns? Rust does have a keyword for that, \code{return}:

\begin{rustc}
fn foo(x: i32) -> i32 {
    return x;

    // we never run this code!
    x + 1
}
\end{rustc}

Using a \code{return} as the last line of a function works, but is considered poor style:

\begin{rustc}
fn foo(x: i32) -> i32 {
    return x + 1;
}
\end{rustc}

The previous definition without \code{return} may look a bit strange if you haven't worked in an expression-based language 
before, but it becomes intuitive over time.

\subsection*{Diverging functions}

Rust has some special syntax for 'diverging functions', which are functions that do not return:

\begin{rustc}
fn diverges() -> ! {
    panic!("This function never returns!");
}
\end{rustc}

\panic\ is a macro, similar to \code{println!()} that we've already seen. Unlike \code{println!()}, \code{panic!()} causes 
the current thread of execution to crash with the given message. Because this function will cause a crash, it will never return, 
and so it has the type '\code{!}', which is read 'diverges'.

\blank

If you add a main function that calls \code{diverges()} and run it, you'll get some output that looks like this:

\begin{verbatim}
thread '<main>' panicked at 'This function never returns!', hello.rs:2
\end{verbatim}

If you want more information, you can get a backtrace by setting the \code{RUST\_BACKTRACE} environment variable:

\begin{verbatim}
$ RUST_BACKTRACE=1 ./diverges
thread '<main>' panicked at 'This function never returns!', hello.rs:2
stack backtrace:
   1:     0x7f402773a829 - sys::backtrace::write::h0942de78b6c02817K8r
   2:     0x7f402773d7fc - panicking::on_panic::h3f23f9d0b5f4c91bu9w
   3:     0x7f402773960e - rt::unwind::begin_unwind_inner::h2844b8c5e81e79558Bw
   4:     0x7f4027738893 - rt::unwind::begin_unwind::h4375279447423903650
   5:     0x7f4027738809 - diverges::h2266b4c4b850236beaa
   6:     0x7f40277389e5 - main::h19bb1149c2f00ecfBaa
   7:     0x7f402773f514 - rt::unwind::try::try_fn::h13186883479104382231
   8:     0x7f402773d1d8 - __rust_try
   9:     0x7f402773f201 - rt::lang_start::ha172a3ce74bb453aK5w
  10:     0x7f4027738a19 - main
  11:     0x7f402694ab44 - __libc_start_main
  12:     0x7f40277386c8 - <unknown>
  13:                0x0 - <unknown>
\end{verbatim}

\code{RUST\_BACKTRACE} also works with Cargo's \code{run} command:

\begin{verbatim}
$ RUST_BACKTRACE=1 cargo run
     Running `target/debug/diverges`
thread '<main>' panicked at 'This function never returns!', hello.rs:2
stack backtrace:
   1:     0x7f402773a829 - sys::backtrace::write::h0942de78b6c02817K8r
   2:     0x7f402773d7fc - panicking::on_panic::h3f23f9d0b5f4c91bu9w
   3:     0x7f402773960e - rt::unwind::begin_unwind_inner::h2844b8c5e81e79558Bw
   4:     0x7f4027738893 - rt::unwind::begin_unwind::h4375279447423903650
   5:     0x7f4027738809 - diverges::h2266b4c4b850236beaa
   6:     0x7f40277389e5 - main::h19bb1149c2f00ecfBaa
   7:     0x7f402773f514 - rt::unwind::try::try_fn::h13186883479104382231
   8:     0x7f402773d1d8 - __rust_try
   9:     0x7f402773f201 - rt::lang_start::ha172a3ce74bb453aK5w
  10:     0x7f4027738a19 - main
  11:     0x7f402694ab44 - __libc_start_main
  12:     0x7f40277386c8 - <unknown>
  13:                0x0 - <unknown>
\end{verbatim}

A diverging function can be used as any type:

\begin{rustc}
let x: i32 = diverges();
let x: String = diverges();
\end{rustc}

\subsection*{Function pointers}

We can also create variable bindings which point to functions:

\begin{rustc}
let f: fn(i32) -> i32;
\end{rustc}

\code{f} is a variable binding which points to a function that takes an \itt\ as an argument and returns an \itt. 
For example:

\begin{rustc}
fn plus_one(i: i32) -> i32 {
    i + 1
}

// without type inference
let f: fn(i32) -> i32 = plus_one;

// with type inference
let f = plus_one;
\end{rustc}

We can then use \code{f} to call the function:

\begin{rustc}
let six = f(5);
\end{rustc}


\subsection{Primitive Types}
\label{sec:syntax_primitives}
The Rust language has a number of types that are considered 'primitive'. This means that they're built-in to the language. Rust is
structured in such a way that the standard library also provides a number of useful types built on top of these ones, as well, but 
these are the most primitive.

\subsection*{Booleans}

Rust has a built in boolean type, named \code{bool}. It has two values, \code{true} and \code{false}:

\begin{rustc}
let x = true;

let y: bool = false;
\end{rustc}

% TODO make if conditionals a hyperref
A common use of booleans is in if conditionals.

\blank

You can find more documentation for \code{bool}s \href{https://doc.rust-lang.org/std/primitive.bool.html}{in the standard library
documentation}.

\subsection*{char}

The \varchar\ type represents a single Unicode scalar value. You can create \varchar s with a single tick: (\code{'})

\begin{rustc}
let x = 'x';
let two_hearts = '💕';
\end{rustc}

Unlike some other languages, this means that Rust's \varchar\ is not a single byte, but four.

\blank

You can find more documentation for \varchar s \href{https://doc.rust-lang.org/std/primitive.char.html}{in the standard library
documentation}.

\subsection*{Numeric types}

Rust has a variety of numeric types in a few categories: signed and unsigned, fixed and variable, floating-point and integer.

\blank

These types consist of two parts: the category, and the size. For example, \code{u16} is an unsigned type with sixteen bits of size. 
More bits lets you have bigger numbers.

\blank

If a number literal has nothing to cause its type to be inferred, it defaults:

\begin{rustc}
let x = 42; // x has type i32

let y = 1.0; // y has type f64
\end{rustc}

Here's a list of the different numeric types, with links to their documentation in the standard library:

\begin{itemize}
  \item{\href{https://doc.rust-lang.org/std/primitive.i8.html}{i8}}
  \item{\href{https://doc.rust-lang.org/std/primitive.i16.html}{i16}}
  \item{\href{https://doc.rust-lang.org/std/primitive.i32.html}{i32}}
  \item{\href{https://doc.rust-lang.org/std/primitive.i64.html}{i64}}
  \item{\href{https://doc.rust-lang.org/std/primitive.u8.html}{u8}}
  \item{\href{https://doc.rust-lang.org/std/primitive.u16.html}{u16}}
  \item{\href{https://doc.rust-lang.org/std/primitive.u32.html}{u32}}
  \item{\href{https://doc.rust-lang.org/std/primitive.u64.html}{u64}}
  \item{\href{https://doc.rust-lang.org/std/primitive.isize.html}{isize}}
  \item{\href{https://doc.rust-lang.org/std/primitive.usize.html}{usize}}
  \item{\href{https://doc.rust-lang.org/std/primitive.f32.html}{f32}}
  \item{\href{https://doc.rust-lang.org/std/primitive.f64.html}{f64}}
\end{itemize}

Let's go over them by category:

\myparagraph{Signed and Unsigned}

Integer types come in two varieties: signed and unsigned. To understand the difference, let's consider a number with four bits of size. 
A signed, four-bit number would let you store numbers from \code{-8} to \code{+7}. Signed numbers use “two's complement representation”. 
An unsigned four bit number, since it does not need to store negatives, can store values from \code{0} to \code{+15}.

\blank

Unsigned types use a \code{u} for their category, and signed types use \code{i}. The \code{i} is for 'integer'. So \code{u8} is an 
eight-bit unsigned number, and \code{i8} is an eight-bit signed number.

\myparagraph{Fixed size types}

Fixed size types have a specific number of bits in their representation. Valid bit sizes are \code{8}, \code{16}, \code{3}2, and \code{64}.
So, \code{u32} is an unsigned, 32-bit integer, and \code{i64} is a signed, 64-bit integer.

\myparagraph{Variable sized types}

Rust also provides types whose size depends on the size of a pointer of the underlying machine. These types have 'size' as the category,
and come in signed and unsigned varieties. This makes for two types: \code{isize} and \code{usize}.

\myparagraph{Floating-point types}

Rust also has two floating point types: \code{f32} and \code{f64}. These correspond to IEEE-754 single and double precision numbers.

\subsection*{Arrays}

Like many programming languages, Rust has list types to represent a sequence of things. The most basic is the \emph{array}, a 
fixed-size list of elements of the same type. By default, arrays are immutable.

\begin{rustc}
let a = [1, 2, 3]; // a: [i32; 3]
let mut m = [1, 2, 3]; // m: [i32; 3]
\end{rustc}

% TODO make generics section a hyperref
Arrays have type \code{[T; N]}. We'll talk about this \code{T} notation in the generics section. The \code{N} is a compile-time 
constant, for the length of the array.

\blank

There's a shorthand for initializing each element of an array to the same value. In this example, each element of a will be initialized 
to \code{0}:

\begin{rustc}
let a = [0; 20]; // a: [i32; 20]
\end{rustc}

You can get the number of elements in an array \code{a} with \code{a.len()}:

\begin{rustc}
let a = [1, 2, 3];

println!("a has {} elements", a.len());
\end{rustc}

You can access a particular element of an array with \emph{subscript notation}:

\begin{rustc}
let names = ["Graydon", "Brian", "Niko"]; // names: [&str; 3]

println!("The second name is: {}", names[1]);
\end{rustc}

Subscripts start at zero, like in most programming languages, so the first name is \code{names[0]} and the second name is \code{names[1]}.
The above example prints \code{The second name is: Brian}. If you try to use a subscript that is not in the array, you will get an error:
array access is bounds-checked at run-time. Such errant access is the source of many bugs in other systems programming languages.

\blank

You can find more documentation for \code{array}s \href{https://doc.rust-lang.org/std/primitive.array.html}{in the standard library
documentation}.

\subsection*{Slices}

A 'slice' is a reference to (or “view” into) another data structure. They are useful for allowing safe, efficient access to a portion 
of an array without copying. For example, you might want to reference only one line of a file read into memory. By nature, a slice is 
not created directly, but from an existing variable binding. Slices have a defined length, can be mutable or immutable.

\myparagraph{Slicing syntax}

% TODO make references a hyperref
You can use a combo of \code{\&} and \code{[]} to create a slice from various things. The \code{\&} indicates that slices are similar 
to references, which we will cover in detail later in this section. The \code{[]}s, with a range, let you define the length of the slice:

\begin{rustc}
let a = [0, 1, 2, 3, 4];
let complete = &a[..]; // A slice containing all of the elements in a
let middle = &a[1..4]; // A slice of a: only the elements 1, 2, and 3
\end{rustc}

% TODO make generics a hyperref
Slices have type \code{\&[T]}. We'll talk about that \code{T} when we cover generics.

\blank

You can find more documentation for slices \href{https://doc.rust-lang.org/std/primitive.slice.html}{in the standard library documentation}.

\subsection*{str}

% TODO make unsized type a hyperref, make strings a hyperref and references a hyperref
Rust's \code{str} type is the most primitive string type. As an unsized type, it's not very useful by itself, but becomes useful 
when placed behind a reference, like \code{\&str}. We'll elaborate further when we cover Strings and references.

\blank

You can find more documentation for \code{str} \href{https://doc.rust-lang.org/std/primitive.str.html}{in the standard library
documentation}.

\subsection*{Tuples}

A tuple is an ordered list of fixed size. Like this:

\begin{rustc}
let x = (1, "hello");
\end{rustc}

The parentheses and commas form this two-length tuple. Here's the same code, but with the type annotated:

\begin{rustc}
let x: (i32, &str) = (1, "hello");
\end{rustc}

As you can see, the type of a tuple looks like the tuple, but with each position having a type name rather than the value. Careful 
readers will also note that tuples are heterogeneous: we have an \itt\ and a \code{\&str} in this tuple. In systems programming
languages, strings are a bit more complex than in other languages. For now, read \code{\&str} as a \emph{string slice}, and we'll 
learn more soon.

\blank

You can assign one tuple into another, if they have the same contained types and \nameref{sec:gloss_arity}. Tuples have the same arity 
when they have the same length.

\begin{rustc}
let mut x = (1, 2); // x: (i32, i32)
let y = (2, 3); // y: (i32, i32)

x = y;
\end{rustc}

You can access the fields in a tuple through a \emph{destructuring let}. Here's an example:

\begin{rustc}
let (x, y, z) = (1, 2, 3);

println!("x is {}", x);
\end{rustc}

Remember before when I said the left-hand side of a \keylet\ statement was more powerful than assigning a binding? Here we are. 
We can put a pattern on the left-hand side of the \keylet, and if it matches up to the right-hand side, we can assign multiple 
bindings at once. In this case, \keylet\ “destructures” or “breaks up” the tuple, and assigns the bits to three bindings.

\blank

This pattern is very powerful, and we'll see it repeated more later.

\blank

You can disambiguate a single-element tuple from a value in parentheses with a comma:

\begin{rustc}
(0,); // single-element tuple
(0); // zero in parentheses
\end{rustc}

\myparagraph{Tuple Indexing}

You can also access fields of a tuple with indexing syntax:

\begin{rustc}
let tuple = (1, 2, 3);

let x = tuple.0;
let y = tuple.1;
let z = tuple.2;

println!("x is {}", x);
\end{rustc}

Like array indexing, it starts at zero, but unlike array indexing, it uses a \code{.}, rather than \code{[]}s.

\blank

You can find more documentation for tuples \href{https://doc.rust-lang.org/std/primitive.tuple.html}{in the standard library 
documentation}.

\subsection*{Functions}

Functions also have a type! They look like this:

\begin{rustc}
fn foo(x: i32) -> i32 { x }

let x: fn(i32) -> i32 = foo;
\end{rustc}

In this case, \x\ is a 'function pointer' to a function that takes an \itt\ and returns an \itt.


\subsection{Comments}
\label{sec:syntax_comments}
Now that we have some functions, it's a good idea to learn about comments. Comments are notes that you leave to other programmers to 
help explain things about your code. The compiler mostly ignores them.

\blank

Rust has two kinds of comments that you should care about: \emph{line comments} and \emph{doc comments}.

\begin{rustc}
// Line comments are anything after '//' and extend to the end of the line.

let x = 5; // this is also a line comment.

// If you have a long explanation for something, you can put line comments next
// to each other. Put a space between the // and your comment so that it's
// more readable.
\end{rustc}

The other kind of comment is a doc comment. Doc comments use \code{///} instead of \code{//}, and support Markdown notation inside:

\begin{rustc}
/// Adds one to the number given.
///
/// # Examples
///
/// '''
/// let five = 5;
///
/// assert_eq!(6, add_one(5));
/// # fn add_one(x: i32) -> i32 {
/// #     x + 1
/// # }
/// '''
fn add_one(x: i32) -> i32 {
    x + 1
}
\end{rustc}

There is another style of doc comment, \code{//!}, to comment containing items (e.g. crates, modules or functions), instead of the 
items following it. Commonly used inside crates root (lib.rs) or modules root (mod.rs):

\begin{rustc}
//! # The Rust Standard Library
//!
//! The Rust Standard Library provides the essential runtime
//! functionality for building portable Rust software.
\end{rustc}

When writing doc comments, providing some examples of usage is very, very helpful. You'll notice we've used a new macro here: 
\code{assert\_eq!}. This compares two values, and \panic s if they're not equal to each other. It's very helpful in documentation.
There's another macro, \code{assert!}, which \panic s if the value passed to it is \code{false}.

\blank

% TODO Make rustdoc a hyperref
You can use the rustdoc tool to generate HTML documentation from these doc comments, and also to run the code examples as tests!


\subsection{If}
\label{sec:syntax_if}
Rust's take on \keyif\ is not particularly complex, but it's much more like the \keyif\ you'll find in a dynamically typed language 
than in a more traditional systems language. So let's talk about it, to make sure you grasp the nuances.

\blank

\keyif\ is a specific form of a more general concept, the 'branch'. The name comes from a branch in a tree: a decision point, where
depending on a choice, multiple paths can be taken.

\blank

In the case of \keyif, there is one choice that leads down two paths:

\begin{rustc}
let x = 5;

if x == 5 {
    println!("x is five!");
}
\end{rustc}

If we changed the value of \x\ to something else, this line would not print. More specifically, if the expression after the 
\keyif\ evaluates to \code{true}, then the block is executed. If it's \code{false}, then it is not.

\blank

If you want something to happen in the \code{false} case, use an \code{else}:

\begin{rustc}
let x = 5;

if x == 5 {
    println!("x is five!");
} else {
    println!("x is not five :(");
}
\end{rustc}

If there is more than one case, use an \code{else if}:

\begin{rustc}
let x = 5;

if x == 5 {
    println!("x is five!");
} else if x == 6 {
    println!("x is six!");
} else {
    println!("x is not five or six :(");
}
\end{rustc}

This is all pretty standard. However, you can also do this:

\begin{rustc}
let x = 5;

let y = if x == 5 {
    10
} else {
    15
}; // y: i32
\end{rustc}

Which we can (and probably should) write like this:

\begin{rustc}
let x = 5;

let y = if x == 5 { 10 } else { 15 }; // y: i32
\end{rustc}

This works because \keyif\ is an expression. The value of the expression is the value of the last expression in whichever branch 
was chosen. An \keyif\ without an \code{else} always results in \code{()} as the value.


\subsection{Loops}
\label{sec:syntax_loops}
Rust currently provides three approaches to performing some kind of iterative activity. They are: \code{loop}, \code{while} and 
\code{for}. Each approach has its own set of uses.

\subsection*{loop}

The infinite \code{loop} is the simplest form of loop available in Rust. Using the keyword \code{loop}, Rust provides a way to 
loop indefinitely until some terminating statement is reached. Rust's infinite \code{loop}s look like this:

\begin{rustc}
loop {
    println!("Loop forever!");
}
\end{rustc}

\subsection*{while}

Rust also has a \code{while} loop. It looks like this:

\begin{rustc}
let mut x = 5; // mut x: i32
let mut done = false; // mut done: bool

while !done {
    x += x - 3;

    println!("{}", x);

    if x % 5 == 0 {
        done = true;
    }
}
\end{rustc}

\code{while} loops are the correct choice when you're not sure how many times you need to loop.

\blank

If you need an infinite loop, you may be tempted to write this:

\begin{rustc}
while true {
\end{rustc}

However, \code{loop} is far better suited to handle this case:

\begin{rustc}
loop {
\end{rustc}

Rust's control-flow analysis treats this construct differently than a \code{while true}, since we know that it will always loop. 
In general, the more information we can give to the compiler, the better it can do with safety and code generation, so you should 
always prefer \code{loop} when you plan to loop infinitely.


\subsection*{for}

The \code{for} loop is used to loop a particular number of times. Rust's \code{for} loops work a bit differently than in other 
systems languages, however. Rust's \code{for} loop doesn't look like this “C-style” \code{for} loop:

\begin{verbatim}
for (x = 0; x < 10; x++) {
    printf( "%d\n", x );
}
\end{verbatim}

Instead, it looks like this:

\begin{rustc}
for x in 0..10 {
    println!("{}", x); // x: i32
}
\end{rustc}

In slightly more abstract terms,

\begin{rustc}
for var in expression {
    code
}
\end{rustc}

% TODO make iterator a hyperref
The expression is an item that can be converted into an iterator using 
\href{https://doc.rust-lang.org/std/iter/trait.IntoIterator.html\{IntoIterator}. The iterator gives back a series of elements. Each 
element is one iteration of the loop. That value is then bound to the name \code{var}, which is valid for the loop body. Once the 
body is over, the next value is fetched from the iterator, and we loop another time. When there are no more values, the for loop is 
over.

\blank

In our example, \code{0..10} is an expression that takes a start and an end position, and gives an iterator over those values. 
The upper bound is exclusive, though, so our loop will print \code{0} through \code{9}, not \code{10}.

\blank

Rust does not have the “C-style” \code{for} loop on purpose. Manually controlling each element of the loop is complicated and 
error prone, even for experienced C developers.

\myparagraph{Enumerate}

When you need to keep track of how many times you already looped, you can use the \code{.enumerate()} function.

\myparagraph{On ranges:}

\begin{rustc}
for (i,j) in (5..10).enumerate() {
    println!("i = {} and j = {}", i, j);
}
\end{rustc}

Outputs:

\begin{verbatim}
i = 0 and j = 5
i = 1 and j = 6
i = 2 and j = 7
i = 3 and j = 8
i = 4 and j = 9
\end{verbatim}

Don't forget to add the parentheses around the range.

\myparagraph{On iterators:}

\begin{rustc}
for (linenumber, line) in lines.enumerate() {
    println!("{}: {}", linenumber, line);
}
\end{rustc}

Outputs:

\begin{verbatim}
0: Content of line one
1: Content of line two
2: Content of line three
3: Content of line four
\end{verbatim}

\subsection*{Ending iteration early}

Let's take a look at that \code{while} loop we had earlier:

\begin{rustc}
let mut x = 5;
let mut done = false;

while !done {
    x += x - 3;

    println!("{}", x);

    if x % 5 == 0 {
        done = true;
    }
}
\end{rustc}

We had to keep a dedicated \mut\ boolean variable binding, \code{done}, to know when we should exit out of the loop. Rust 
has two keywords to help us with modifying iteration: \code{break} and \code{continue}.

\blank

In this case, we can write the loop in a better way with \code{break}:

\begin{rustc}
let mut x = 5;

loop {
    x += x - 3;

    println!("{}", x);

    if x % 5 == 0 { break; }
}
\end{rustc}

We now loop forever with \code{loop} and use \code{break} to break out early. Issuing an explicit \code{return} statement will also 
serve to terminate the loop early.

\blank

\code{continue} is similar, but instead of ending the loop, goes to the next iteration. This will only print the odd numbers:

\begin{rustc}
for x in 0..10 {
    if x % 2 == 0 { continue; }

    println!("{}", x);
}
\end{rustc}

\subsection*{Loop labels}

You may also encounter situations where you have nested loops and need to specify which one your \code{break} or \code{continue} 
statement is for. Like most other languages, by default a \code{break} or \code{continue} will apply to innermost loop. In a situation where you would like to a \code{break} or \code{continue} for one of the outer loops, you can use labels to specify which loop the 
\code{break} or \code{continue} statement applies to. This will only print when both \x\ and \y\ are odd:

\begin{rustc}
'outer: for x in 0..10 {
    'inner: for y in 0..10 {
        if x % 2 == 0 { continue 'outer; } // continues the loop over x 
        if y % 2 == 0 { continue 'inner; } // continues the loop over y
        println!("x: {}, y: {}", x, y);
    }
}
\end{rustc}


\subsection{Ownership}
\label{sec:syntax_ownership}
This guide is one of three presenting Rust's ownership system. This is one of Rust's most unique and compelling features, with 
which Rust developers should become quite acquainted. Ownership is how Rust achieves its largest goal, memory safety. There are 
a few distinct concepts, each with its own chapter:

\begin{itemize}
  \item{ownership, which you're reading now}
  \item{\nameref{sec:syntax_referencesBorrowing}, and their associated feature 'references'}
  \item{\nameref{sec:syntax_lifetimes}, an advanced concept of borrowing}
\end{itemize}

These three chapters are related, and in order. You'll need all three to fully understand the ownership system.

\subsection*{Meta}

Before we get to the details, two important notes about the ownership system.

\blank

Rust has a focus on safety and speed. It accomplishes these goals through many 'zero-cost abstractions', which means that in 
Rust, abstractions cost as little as possible in order to make them work. The ownership system is a prime example of a zero-cost
abstraction. All of the analysis we'll talk about in this guide is \emph{done at compile time}. You do not pay any run-time cost 
for any of these features.

\blank

However, this system does have a certain cost: learning curve. Many new users to Rust experience something we like to call 
'fighting with the borrow checker', where the Rust compiler refuses to compile a program that the author thinks is valid. 
This often happens because the programmer's mental model of how ownership should work doesn't match the actual rules that Rust 
implements. You probably will experience similar things at first. There is good news, however: more experienced Rust developers 
report that once they work with the rules of the ownership system for a period of time, they fight the borrow checker less and less.

\blank

With that in mind, let's learn about ownership.

\subsection*{Ownership}

\nameref{sec:syntax_variableBindings} have a property in Rust: they 'have ownership' of what they're bound to. This means that when 
a binding goes out of scope, Rust will free the bound resources. For example:

\begin{rustc}
fn foo() {
    let v = vec![1, 2, 3];
}
\end{rustc}

When \code{v} comes into scope, a new [vector] is created, and it allocates space on the heap for each of its elements 
(see \nameref{sec:effective_stackAndHeap}). When \code{v} goes out of scope at the end of \code{foo()}, Rust will clean up 
everything related to the vector, even the heap-allocated memory. This happens deterministically, at the end of the scope.

\blank

We'll cover vectors in detail later in this chapter (see \nameref{sec:syntax_vectors}); we only use them here as an example of a 
type that allocates space on the heap at runtime. They behave like arrays, except their size may change by \code{push()}ing more 
elements onto them.

\blank

Vectors have a generic type \code{Vec<T>} , so in this example \code{v} will have type 
\code{Vec<i32>}. We'll cover generics in detail later in this chapter (see \nameref{sec:syntax_generics}).

\subsection*{Move semantics}

There's some more subtlety here, though: Rust ensures that there is \emph{exactly one} binding to any given resource. For 
example, if we have a vector, we can assign it to another binding:

\begin{rustc}
let v = vec![1, 2, 3];

let v2 = v;
\end{rustc}

But, if we try to use \code{v} afterwards, we get an error:

\begin{rustc}
let v = vec![1, 2, 3];

let v2 = v;

println!("v[0] is: {}", v[0]);
\end{rustc}

It looks like this:

\begin{verbatim}
error: use of moved value: `v`
println!("v[0] is: {}", v[0]);
                        ^
\end{verbatim}

A similar thing happens if we define a function which takes ownership, and try to use something after we've passed it as an 
argument:

\begin{rustc}
fn take(v: Vec<i32>) {
    // what happens here isn't important.
}

let v = vec![1, 2, 3];

take(v);

println!("v[0] is: {}", v[0]);
\end{rustc}

Same error: 'use of moved value'. When we transfer ownership to something else, we say that we've 'moved' the thing we refer to. 
You don't need some sort of special annotation here, it's the default thing that Rust does.

\myparagraph{The details}

The reason that we cannot use a binding after we've moved it is subtle, but important. When we write code like this:

\begin{rustc}
let v = vec![1, 2, 3];

let v2 = v;
\end{rustc}

The first line allocates memory for the vector object, \code{v}, and for the data it contains. The vector object is stored on the 
stack and contains a pointer to the content (\code{[1, 2, 3]}) stored on the heap. When we move \code{v} to \code{v2}, it creates 
a copy of that pointer, for \code{v2}. Which means that there would be two pointers to the content of the vector on the heap. It 
would violate Rust's safety guarantees by introducing a data race. Therefore, Rust forbids using \code{v} after we've done the move.

\blank

It's also important to note that optimizations may remove the actual copy of the bytes on the stack, depending on circumstances. 
So it may not be as inefficient as it initially seems.

\myparagraph{\code{Copy} types}

We've established that when ownership is transferred to another binding, you cannot use the original binding. However, there's a 
trait that changes this behavior, and it's called \code{Copy}. We haven't discussed traits yet, but for now, you can think of them 
as an annotation to a particular type that adds extra behavior (see \nameref{sec:syntax_generics}). For example:

\begin{rustc}
let v = 1;

let v2 = v;

println!("v is: {}", v);
\end{rustc}

In this case, \code{v} is an \itt, which implements the \code{Copy} trait. This means that, just like a move, when we assign 
\code{v} to \code{v2}, a copy of the data is made. But, unlike a move, we can still use \code{v} afterward. This is because an 
\itt\ has no pointers to data somewhere else, copying it is a full copy.

\blank

All primitive types implement the \code{Copy} trait and their ownership is therefore not moved like one would assume, following the
'ownership rules'. To give an example, the two following snippets of code only compile because the \itt\ and \code{bool} types
implement the \code{Copy} trait.

\begin{rustc}
fn main() {
    let a = 5;

    let _y = double(a);
    println!("{}", a);
}

fn double(x: i32) -> i32 {
    x * 2
}


fn main() {
    let a = true;

    let _y = change_truth(a);
    println!("{}", a);
}

fn change_truth(x: bool) -> bool {
    !x
}
\end{rustc}

If we had used types that do not implement the \code{Copy} trait, we would have gotten a compile error because we tried to use a 
moved value.

\begin{verbatim}
error: use of moved value: `a`
println!("{}", a);
               ^
\end{verbatim}

We will discuss how to make your own types \code{Copy} in the traits section (see \nameref{sec:syntax_traits}).

\subsection*{More than ownership}

Of course, if we had to hand ownership back with every function we wrote:

\begin{rustc}
fn foo(v: Vec<i32>) -> Vec<i32> {
    // do stuff with v

    // hand back ownership
    v
}
\end{rustc}

This would get very tedious. It gets worse the more things we want to take ownership of:

\begin{rustc}
fn foo(v1: Vec<i32>, v2: Vec<i32>) -> (Vec<i32>, Vec<i32>, i32) {
    // do stuff with v1 and v2

    // hand back ownership, and the result of our function
    (v1, v2, 42)
}

let v1 = vec![1, 2, 3];
let v2 = vec![1, 2, 3];

let (v1, v2, answer) = foo(v1, v2);
\end{rustc}

Ugh! The return type, return line, and calling the function gets way more complicated.

\blank

Luckily, Rust offers a feature, borrowing, which helps us solve this problem. It's the topic of the next section!


\subsection{References and Borrowing}
\label{sec:syntax_referencesBorrowing}
This guide is one of three presenting Rust's ownership system. This is one of Rust's most unique and compelling features, with 
which Rust developers should become quite acquainted. Ownership is how Rust achieves its largest goal, memory safety. There are 
a few distinct concepts, each with its own chapter:

% TODO make ownership and lifetimes hyperrefs
\begin{itemize}
  \item{ownership, which you're reading now}
  \item{borrowing, and their associated feature 'references'}
  \item{lifetimes, an advanced concept of borrowing}
\end{itemize}

These three chapters are related, and in order. You'll need all three to fully understand the ownership system.

\subsection*{Meta}

Before we get to the details, two important notes about the ownership system.

\blank

Rust has a focus on safety and speed. It accomplishes these goals through many 'zero-cost abstractions', which means that in 
Rust, abstractions cost as little as possible in order to make them work. The ownership system is a prime example of a zero-cost
abstraction. All of the analysis we'll talk about in this guide is \emph{done at compile time}. You do not pay any run-time cost 
for any of these features.

\blank

However, this system does have a certain cost: learning curve. Many new users to Rust experience something we like to call 
'fighting with the borrow checker', where the Rust compiler refuses to compile a program that the author thinks is valid. 
This often happens because the programmer's mental model of how ownership should work doesn't match the actual rules that Rust 
implements. You probably will experience similar things at first. There is good news, however: more experienced Rust developers 
report that once they work with the rules of the ownership system for a period of time, they fight the borrow checker less and less.

\blank

With that in mind, let's learn about borrowing.

\subsection*{Borrowing}

At the end of the \nameref{sec:syntax_ownership} section, we had a nasty function that looked like this:

\begin{rustc}
fn foo(v1: Vec<i32>, v2: Vec<i32>) -> (Vec<i32>, Vec<i32>, i32) {
    // do stuff with v1 and v2

    // hand back ownership, and the result of our function
    (v1, v2, 42)
}

let v1 = vec![1, 2, 3];
let v2 = vec![1, 2, 3];

let (v1, v2, answer) = foo(v1, v2);
\end{rustc}

This is not idiomatic Rust, however, as it doesn't take advantage of borrowing. Here's the first step:

\begin{rustc}
fn foo(v1: &Vec<i32>, v2: &Vec<i32>) -> i32 {
    // do stuff with v1 and v2

    // return the answer
    42
}

let v1 = vec![1, 2, 3];
let v2 = vec![1, 2, 3];

let answer = foo(&v1, &v2);

// we can use v1 and v2 here!

\end{rustc}

Instead of taking \code{Vec<i32>}s as our arguments, we take a reference: \code{\&Vec<i32>}. And instead of passing \code{v1} and 
\code{v2} directly, we pass \code{\&v1} and \code{\&v2}. We call the \code{\&T} type a 'reference', and rather than owning the 
resource, it borrows ownership. A binding that borrows something does not deallocate the resource when it goes out of scope. This 
means that after the call to \code{foo()}, we can use our original bindings again.

\blank

References are immutable, like bindings. This means that inside of \code{foo()}, the vectors can't be changed at all:

\begin{rustc}
fn foo(v: &Vec<i32>) {
     v.push(5);
}

let v = vec![];

foo(&v);
\end{rustc}

errors with:

\begin{verbatim}
error: cannot borrow immutable borrowed content `*v` as mutable
v.push(5);
^
\end{verbatim}

Pushing a value mutates the vector, and so we aren't allowed to do it.

\subsection*{\&mut references}

There's a second kind of reference: \code{\&mut T}. A 'mutable reference' allows you to mutate the resource you're borrowing. 
For example:

\begin{rustc}
let mut x = 5;
{
    let y = &mut x;
    *y += 1;
}
println!("{}", x);
\end{rustc}

This will print \code{6}. We make \y\ a mutable reference to \x\, then add one to the thing \y\ points at. You'll 
notice that \x\ had to be marked \mut\ as well. If it wasn't, we couldn't take a mutable borrow to an immutable value.

\blank

You'll also notice we added an asterisk (\code{*}) in front of \y, making it \code{*y}, this is because \y\ is a 
\code{\&mut} reference. You'll also need to use them for accessing the contents of a reference as well.

\blank

Otherwise, \code{\&mut} references are like references. There is a large difference between the two, and how they interact, though. 
You can tell something is fishy in the above example, because we need that extra scope, with the \code{\{} and \code{\}}. If we 
remove them, we get an error:

\begin{verbatim}
error: cannot borrow `x` as immutable because it is also borrowed as mutable
    println!("{}", x);
                   ^
note: previous borrow of `x` occurs here; the mutable borrow prevents
subsequent moves, borrows, or modification of `x` until the borrow ends
        let y = &mut x;
                     ^
note: previous borrow ends here
fn main() {

}
^
\end{verbatim}

As it turns out, there are rules.

\subsection*{The Rules}

Here's the rules about borrowing in Rust:

First, any borrow must last for a scope no greater than that of the owner. Second, you may have one or the other of these two 
kinds of borrows, but not both at the same time:

\begin{itemize}
  \item{one or more references (\code{\&T}) to a resource,}
  \item{exactly one mutable reference (\code{\&mut T}).}
\end{itemize}

You may notice that this is very similar, though not exactly the same as, to the definition of a data race:

\begin{myquote}
There is a 'data race' when two or more pointers access the same memory location at the same time, where at least one of 
them is writing, and the operations are not synchronized.
\end{myquote}

With references, you may have as many as you'd like, since none of them are writing. However, as we can only have one \code{\&mut}
at a time, it is impossible to have a data race. This is how Rust prevents data races at compile time: we'll get errors if we break 
the rules.

\blank

With this in mind, let's consider our example again.

\myparagraph{Thinking in scopes}

Here's the code:

\begin{rustc}
let mut x = 5;
let y = &mut x;

*y += 1;

println!("{}", x);
\end{rustc}

This code gives us this error:

\begin{verbatim}
error: cannot borrow `x` as immutable because it is also borrowed as mutable
    println!("{}", x);
                   ^
\end{verbatim}

This is because we've violated the rules: we have a \code{\&mut T} pointing to \x, and so we aren't allowed to create any 
\code{\&T}s. One or the other. The note hints at how to think about this problem:

\begin{verbatim}
note: previous borrow ends here
fn main() {

}
^
\end{verbatim}

In other words, the mutable borrow is held through the rest of our example. What we want is for the mutable borrow to end before 
we try to call \println\ and make an immutable borrow. In Rust, borrowing is tied to the scope that the borrow is valid for. 
And our scopes look like this:

\begin{rustc}
let mut x = 5;

let y = &mut x;    // -+ &mut borrow of x starts here
                   //  |
*y += 1;           //  |
                   //  |
println!("{}", x); // -+ - try to borrow x here
                   // -+ &mut borrow of x ends here
\end{rustc}

The scopes conflict: we can't make an \code{\&x} while \y\ is in scope.

\blank

So when we add the curly braces:

\begin{rustc}
let mut x = 5;

{
    let y = &mut x; // -+ &mut borrow starts here
    *y += 1;        //  |
}                   // -+ ... and ends here

println!("{}", x);  // <- try to borrow x here
\end{rustc}

There's no problem. Our mutable borrow goes out of scope before we create an immutable one. But scope is the key to seeing how 
long a borrow lasts for.

\myparagraph{Issues borrowing prevents}

Why have these restrictive rules? Well, as we noted, these rules prevent data races. What kinds of issues do data races cause? 
Here's a few.

\myparagraph{Iterator invalidation}

One example is 'iterator invalidation', which happens when you try to mutate a collection that you're iterating over. Rust's 
borrow checker prevents this from happening:

\begin{rustc}
let mut v = vec![1, 2, 3];

for i in &v {
    println!("{}", i);
}
\end{rustc}

This prints out one through three. As we iterate through the vector, we're only given references to the elements. And \code{v} is 
itself borrowed as immutable, which means we can't change it while we're iterating:

\begin{rustc}
let mut v = vec![1, 2, 3];

for i in &v {
    println!("{}", i);
    v.push(34);
}
\end{rustc}

Here's the error:

\begin{verbatim}
error: cannot borrow `v` as mutable because it is also borrowed as immutable
    v.push(34);
    ^
note: previous borrow of `v` occurs here; the immutable borrow prevents
subsequent moves or mutable borrows of `v` until the borrow ends
for i in &v {
          ^
note: previous borrow ends here
for i in &v {
    println!(“{}”, i);
    v.push(34);
}
^
\end{verbatim}

We can't modify \code{v} because it's borrowed by the loop.

\myparagraph{use after free}

References must not live longer than the resource they refer to. Rust will check the scopes of your references to ensure that 
this is true.

\blank

If Rust didn't check this property, we could accidentally use a reference which was invalid. For example:

\begin{rustc}
let y: &i32;
{
    let x = 5;
    y = &x;
}

println!("{}", y);
\end{rustc}

We get this error:

\begin{verbatim}
error: `x` does not live long enough
    y = &x;
         ^
note: reference must be valid for the block suffix following statement 0 at
2:16...
let y: &i32;
{
    let x = 5;
    y = &x;
}

note: ...but borrowed value is only valid for the block suffix following
statement 0 at 4:18
    let x = 5;
    y = &x;
}
\end{verbatim}

In other words, \y\ is only valid for the scope where \x\ exists. As soon as \x\ goes away, it becomes invalid to 
refer to it. As such, the error says that the borrow 'doesn't live long enough' because it's not valid for the right amount of time.

\blank

The same problem occurs when the reference is declared before the variable it refers to. This is because resources within the same 
scope are freed in the opposite order they were declared:

\begin{rustc}
let y: &i32;
let x = 5;
y = &x;

println!("{}", y);
\end{rustc}

We get this error:

\begin{verbatim}
error: `x` does not live long enough
y = &x;
     ^
note: reference must be valid for the block suffix following statement 0 at
2:16...
    let y: &i32;
    let x = 5;
    y = &x;

    println!("{}", y);
}

note: ...but borrowed value is only valid for the block suffix following
statement 1 at 3:14
    let x = 5;
    y = &x;

    println!("{}", y);
}
\end{verbatim}

In the above example, \y\ is declared before \x, meaning that \y\ lives longer than \x, which is not allowed.


\subsection{Lifetimes}
\label{sec:syntax_lifetimes}
This guide is one of three presenting Rust's ownership system. This is one of Rust's most unique and compelling features, with 
which Rust developers should become quite acquainted. Ownership is how Rust achieves its largest goal, memory safety. There are 
a few distinct concepts, each with its own chapter:

% TODO make borrowing and ownership hyperrefs
\begin{itemize}
  \item{ownership, which you're reading now}
  \item{borrowing, and their associated feature 'references'}
  \item{lifetimes, an advanced concept of borrowing}
\end{itemize}

These three chapters are related, and in order. You'll need all three to fully understand the ownership system.

\subsection*{Meta}

Before we get to the details, two important notes about the ownership system.

\blank

Rust has a focus on safety and speed. It accomplishes these goals through many 'zero-cost abstractions', which means that in 
Rust, abstractions cost as little as possible in order to make them work. The ownership system is a prime example of a zero-cost
abstraction. All of the analysis we'll talk about in this guide is \emph{done at compile time}. You do not pay any run-time cost 
for any of these features.

\blank

However, this system does have a certain cost: learning curve. Many new users to Rust experience something we like to call 
'fighting with the borrow checker', where the Rust compiler refuses to compile a program that the author thinks is valid. 
This often happens because the programmer's mental model of how ownership should work doesn't match the actual rules that Rust 
implements. You probably will experience similar things at first. There is good news, however: more experienced Rust developers 
report that once they work with the rules of the ownership system for a period of time, they fight the borrow checker less and less.

\blank

With that in mind, let's learn about lifetimes.

\subsection*{Lifetimes}

Lending out a reference to a resource that someone else owns can be complicated. For example, imagine this set of operations:

\begin{enumerate}
  \item{I acquire a handle to some kind of resource.}
  \item{I lend you a reference to the resource.}
  \item{I decide I'm done with the resource, and deallocate it, while you still have your reference.}
  \item{You decide to use the resource.}
\end{enumerate}

Uh oh! Your reference is pointing to an invalid resource. This is called a dangling pointer or 'use after free', when the resource 
is memory.

\blank

To fix this, we have to make sure that step four never happens after step three. The ownership system in Rust does this through a 
concept called lifetimes, which describe the scope that a reference is valid for.

\blank

When we have a function that takes a reference by argument, we can be implicit or explicit about the lifetime of the reference:

\begin{rustc}
// implicit
fn foo(x: &i32) {
}

// explicit
fn bar<'a>(x: &'a i32) {
}
\end{rustc}

The \code{'a} reads 'the lifetime a'. Technically, every reference has some lifetime associated with it, but the compiler lets 
you elide (i.e. omit, see \enquote{\nameref{paragraph:lifetime_elision}} below) them in common cases. Before we get to that, though, let's break the explicit 
example down:

\begin{rustc}
fn bar<'a>(...)
\end{rustc}

% TODO make later in the book a hyperref
We previously talked a little about function syntax (\nameref{sec:syntax_functions}), but we didn't discuss the \code{<>}s after 
a function's name. A function can have 'generic parameters' between the \code{<>}s, of which lifetimes are one kind. We'll discuss 
other kinds of generics later in the book, but for now, let's focus on the lifetimes aspect.

\blank

We use \code{<>} to declare our lifetimes. This says that bar has one lifetime, \code{'a}. If we had two reference parameters, it 
would look like this:

\begin{rustc}
fn bar<'a, 'b>(...)
\end{rustc}

Then in our parameter list, we use the lifetimes we've named:

\begin{rustc}
...(x: &'a i32)
\end{rustc}

If we wanted a \code{\&mut} reference, we'd do this:

\begin{rustc}
...(x: &'a mut i32)
\end{rustc}

If you compare \code{\&mut i32} to \code{\&'a mut i32}, they're the same, it's that the lifetime \code{'a} has snuck in between the 
\code{\&} and the \code{mut i32}. We read \code{\&mut i32} as 'a mutable reference to an \itt' and \code{\&'a mut i32} as 
'a mutable reference to an \itt\ with the lifetime \code{'a}'.

\subsection*{In \code{struct}s}

% TODO make structs a hyperref
You'll also need explicit lifetimes when working with structs that contain references:

\begin{rustc}
struct Foo<'a> {
    x: &'a i32,
}

fn main() {
    let y = &5; // this is the same as `let _y = 5; let y = &_y;`
    let f = Foo { x: y };

    println!("{}", f.x);
}
\end{rustc}

As you can see, \struct s can also have lifetimes. In a similar way to functions,

\begin{rustc}
struct Foo<'a> {
\end{rustc}

declares a lifetime, and

\begin{rustc}
x: &'a i32,
\end{rustc}

uses it. So why do we need a lifetime here? We need to ensure that any reference to a \code{Foo} cannot outlive the reference to an 
\itt\ it contains.

\myparagraph{\code{impl} blocks}

Let's implement a method on \code{Foo}:

\begin{rustc}
struct Foo<'a> {
    x: &'a i32,
}

impl<'a> Foo<'a> {
    fn x(&self) -> &'a i32 { self.x }
}

fn main() {
    let y = &5; // this is the same as `let _y = 5; let y = &_y;`
    let f = Foo { x: y };

    println!("x is: {}", f.x());
}
\end{rustc}

As you can see, we need to declare a lifetime for \code{Foo} in the \code{impl} line. We repeat \code{'a} twice, like on functions:
\code{impl<'a>} defines a lifetime \code{'a}, and \code{Foo<'a>} uses it.

\myparagraph{Multiple lifetimes}

If you have multiple references, you can use the same lifetime multiple times:

\begin{rustc}
fn x_or_y<'a>(x: &'a str, y: &'a str) -> &'a str {
\end{rustc}

This says that \x\ and \y\ both are alive for the same scope, and that the return value is also alive for that scope. 
If you wanted \x\ and \y\ to have different lifetimes, you can use multiple lifetime parameters:

\begin{rustc}
fn x_or_y<'a, 'b>(x: &'a str, y: &'b str) -> &'a str {
\end{rustc}

In this example, \x\ and \y\ have different valid scopes, but the return value has the same lifetime as \x.

\myparagraph{Thinking in scopes}

A way to think about lifetimes is to visualize the scope that a reference is valid for. For example:

\begin{rustc}
fn main() {
    let y = &5;     // -+ y goes into scope
                    //  |
    // stuff        //  |
                    //  |
}                   // -+ y goes out of scope
\end{rustc}

Adding in our \code{Foo}:

\begin{rustc}
struct Foo<'a> {
    x: &'a i32,
}

fn main() {
    let y = &5;           // -+ y goes into scope
    let f = Foo { x: y }; // -+ f goes into scope
    // stuff              //  |
                          //  |
}                         // -+ f and y go out of scope
\end{rustc}

Our \code{f} lives within the scope of \y, so everything works. What if it didn't? This code won't work:

\begin{rustc}
struct Foo<'a> {
    x: &'a i32,
}

fn main() {
    let x;                    // -+ x goes into scope
                              //  |
    {                         //  |
        let y = &5;           // ---+ y goes into scope
        let f = Foo { x: y }; // ---+ f goes into scope
        x = &f.x;             //  | | error here
    }                         // ---+ f and y go out of scope
                              //  |
    println!("{}", x);        //  |
}                             // -+ x goes out of scope
\end{rustc}

Whew! As you can see here, the scopes of \code{f} and \y\ are smaller than the scope of \x. But when we do 
\code{x = \&f.x}, we make \x\ a reference to something that's about to go out of scope.

\blank

Named lifetimes are a way of giving these scopes a name. Giving something a name is the first step towards being able to talk 
about it.

\myparagraph{'static}

The lifetime named 'static' is a special lifetime. It signals that something has the lifetime of the entire program. Most 
Rust programmers first come across \code{'static} when dealing with strings:

\begin{rustc}
let x: &'static str = "Hello, world.";
\end{rustc}

String literals have the type \code{\&'static str} because the reference is always alive: they are baked into the data segment of 
the final binary. Another example are globals:

\begin{rustc}
static FOO: i32 = 5;
let x: &'static i32 = &FOO;
\end{rustc}

This adds an \itt\ to the data segment of the binary, and \x\ is a reference to it.

\myparagraph{Lifetime Elision}
\label{paragraph:lifetime_elision}

Rust supports powerful local type inference in function bodies, but it's forbidden in item signatures to allow reasoning about the 
types based on the item signature alone. However, for ergonomic reasons a very restricted secondary inference algorithm called 
\enquote{lifetime elision} applies in function signatures. It infers only based on the signature components themselves and not based 
on the body of the function, only infers lifetime parameters, and does this with only three easily memorizable and unambiguous rules. 
This makes lifetime elision a shorthand for writing an item signature, while not hiding away the actual types involved as full local
inference would if applied to it.

\blank

When talking about lifetime elision, we use the term \emph{input lifetime} and \emph{output lifetime}. An \emph{input lifetime} is 
a lifetime associated with a parameter of a function, and an \emph{output lifetime} is a lifetime associated with the return value 
of a function. For example, this function has an input lifetime:

\begin{rustc}
fn foo<'a>(bar: &'a str)
\end{rustc}

This one has an output lifetime:

\begin{rustc}
fn foo<'a>() -> &'a str
\end{rustc}

This one has a lifetime in both positions:

\begin{rustc}
fn foo<'a>(bar: &'a str) -> &'a str
\end{rustc}

Here are the three rules:

\begin{itemize}
  \item{Each elided lifetime in a function's arguments becomes a distinct lifetime parameter.}
  \item{If there is exactly one input lifetime, elided or not, that lifetime is assigned to all elided lifetimes in the return 
      values of that function.}
  \item{If there are multiple input lifetimes, but one of them is \code{\&self} or \code{\&mut self}, the lifetime of self is 
      assigned to all elided output lifetimes.}
\end{itemize}

Otherwise, it is an error to elide an output lifetime.

\myparagraph{Examples}

Here are some examples of functions with elided lifetimes. We've paired each example of an elided lifetime with its expanded form.

\begin{rustc}
fn print(s: &str); // elided
fn print<'a>(s: &'a str); // expanded

fn debug(lvl: u32, s: &str); // elided
fn debug<'a>(lvl: u32, s: &'a str); // expanded

// In the preceding example, `lvl` doesn't need a lifetime because it's not a
// reference (`&`). Only things relating to references (such as a `struct`
// which contains a reference) need lifetimes.

fn substr(s: &str, until: u32) -> &str; // elided
fn substr<'a>(s: &'a str, until: u32) -> &'a str; // expanded

fn get_str() -> &str; // ILLEGAL, no inputs

fn frob(s: &str, t: &str) -> &str; // ILLEGAL, two inputs
fn frob<'a, 'b>(s: &'a str, t: &'b str) -> &str; // Expanded: Output lifetime is ambiguous

fn get_mut(&mut self) -> &mut T; // elided
fn get_mut<'a>(&'a mut self) -> &'a mut T; // expanded

fn args<T: ToCStr>(&mut self, args: &[T]) -> &mut Command; // elided
fn args<'a, 'b, T: ToCStr>(&'a mut self, args: &'b [T]) -> &'a mut Command; // expanded

fn new(buf: &mut [u8]) -> BufWriter; // elided
fn new<'a>(buf: &'a mut [u8]) -> BufWriter<'a>; // expanded
\end{rustc}


\subsection{Mutability}
\label{sec:syntax_mutability}
Mutability, the ability to change something, works a bit differently in Rust than in other languages. The first aspect of 
mutability is its non-default status:

\begin{rustc}
let x = 5;
x = 6; // error!
\end{rustc}

We can introduce mutability with the \mut\ keyword:

\begin{rustc}
let mut x = 5;

x = 6; // no problem!
\end{rustc}

This is a mutable \nameref{sec:syntax_variableBindings}. When a binding is mutable, it means you're allowed to change what the 
binding points to. So in the above example, it's not so much that the value at \x\ is changing, but that the binding changed
from one \itt\ to another.

\blank

If you want to change what the binding points to, you'll need a mutable reference (see \nameref{sec:syntax_referencesBorrowing}):

\begin{rustc}
let mut x = 5;
let y = &mut x;
\end{rustc}

\y\ is an immutable binding to a mutable reference, which means that you can't bind \y\ to something else (
\code{y = \&mut z}), but you can mutate the thing that's bound to \y\ (\code{*y = 5}). A subtle distinction.

\blank

Of course, if you need both:

\begin{rustc}
let mut x = 5;
let mut y = &mut x;
\end{rustc}

Now \y\ can be bound to another value, and the value it's referencing can be changed.

\blank

% TODO make pattern a hyperref
It's important to note that \mut\ is part of a pattern, so you can do things like this:

\begin{rustc}
let (mut x, y) = (5, 6);

fn foo(mut x: i32) {
\end{rustc}

\subsection*{Interior vs. Exterior Mutability}

However, when we say something is 'immutable' in Rust, that doesn't mean that it's not able to be changed: we mean something has 
'exterior mutability'. Consider, for example, \href{https://doc.rust-lang.org/std/sync/struct.Arc.html}{Arc<T>}:

\begin{rustc}
use std::sync::Arc;

let x = Arc::new(5);
let y = x.clone();
\end{rustc}

When we call \code{clone()}, the \code{Arc<T>} needs to update the reference count. Yet we've not used any muts here, \x\ is 
an immutable binding, and we didn't take \code{\&mut 5} or anything. So what gives?

\blank

To understand this, we have to go back to the core of Rust's guiding philosophy, memory safety, and the mechanism by which Rust 
guarantees it, the ownership system (see \nameref{sec:syntax_ownership}), and more specifically, borrowing (see 
\nameref{sec:syntax_referencesBorrowing}):

\begin{myquote}
You may have one or the other of these two kinds of borrows, but not both at the same time:
\begin{itemize}
  \item{one or more references (\code{\&T}) to a resource,}
  \item{exactly one mutable reference (\code{\&mut T}).}
\end{itemize}
\end{myquote}

So, that's the real definition of 'immutability': is this safe to have two pointers to? In \code{Arc<T>}'s case, yes: the mutation 
is entirely contained inside the structure itself. It's not user facing. For this reason, it hands out \code{\&T} with \code{clone()}. 
If it handed out \code{\&mut T}s, though, that would be a problem.

\blank

Other types, like the ones in the \href{https://doc.rust-lang.org/std/cell/}{std::cell} module, have the opposite: interior 
mutability. For example:

\begin{rustc}
use std::cell::RefCell;

let x = RefCell::new(42);

let y = x.borrow_mut();
\end{rustc}

RefCell hands out \code{\&mut} references to what's inside of it with the \code{borrow\_mut()} method. Isn't that dangerous? What if 
we do:

\begin{rustc}
use std::cell::RefCell;

let x = RefCell::new(42);

let y = x.borrow_mut();
let z = x.borrow_mut();
\end{rustc}

This will in fact panic, at runtime. This is what \code{RefCell} does: it enforces Rust's borrowing rules at runtime, and 
\panic s if they're violated. This allows us to get around another aspect of Rust's mutability rules. Let's talk about it 
first.

\myparagraph{Field-level mutability}

% TODO make struct a hyperref
Mutability is a property of either a borrow (\code{\&mut}) or a binding (\code{let mut}). This means that, for example, you cannot 
have a struct with some fields mutable and some immutable:

\begin{rustc}
struct Point {
    x: i32,
    mut y: i32, // nope
}
\end{rustc}

The mutability of a struct is in its binding:

\begin{rustc}
struct Point {
    x: i32,
    y: i32,
}

let mut a = Point { x: 5, y: 6 };

a.x = 10;

let b = Point { x: 5, y: 6};

b.x = 10; // error: cannot assign to immutable field `b.x`
\end{rustc}

However, by using \href{https://doc.rust-lang.org/std/cell/struct.Cell.html}{Cell<T>}, you can emulate field-level mutability:

\begin{rustc}
use std::cell::Cell;

struct Point {
    x: i32,
    y: Cell<i32>,
}

let point = Point { x: 5, y: Cell::new(6) };

point.y.set(7);

println!("y: {:?}", point.y);
\end{rustc}

This will print \code{y: Cell { value: 7 }}. We've successfully updated \y.


\subsection{Structs}
\label{sec:syntax_structs}
\struct s are a way of creating more complex data types. For example, if we were doing calculations involving coordinates in 
2D space, we would need both an \x\ and a \y\ value:

\begin{rustc}
let origin_x = 0;
let origin_y = 0;
\end{rustc}

A \struct\ lets us combine these two into a single, unified datatype with \x\ and \y\ as field labels:

\begin{rustc}
struct Point {
    x: i32,
    y: i32,
}

fn main() {
    let origin = Point { x: 0, y: 0 }; // origin: Point

    println!("The origin is at ({}, {})", origin.x, origin.y);
}
\end{rustc}

There's a lot going on here, so let's break it down. We declare a \struct\ with the \struct\ keyword, and then with a name. By 
convention, \struct s begin with a capital letter and are camel cased: \code{PointInSpace}, not \code{Point\_In\_Space}.

\blank

We can create an instance of our \struct\ via \keylet, as usual, but we use a \code{key: value} style syntax to set each field. The 
order doesn't need to be the same as in the original declaration.

\blank

Finally, because fields have names, we can access them through dot notation: \code{origin.x}.

\blank

The values in \struct s are immutable by default, like other bindings in Rust. Use \mut\ to make them mutable:

\begin{rustc}
struct Point {
    x: i32,
    y: i32,
}

fn main() {
    let mut point = Point { x: 0, y: 0 };

    point.x = 5;

    println!("The point is at ({}, {})", point.x, point.y);
}
\end{rustc}

This will print \code{The point is at (5, 0)}.

\blank

Rust does not support field mutability at the language level, so you cannot write something like this:

\begin{rustc}
struct Point {
    mut x: i32,
    y: i32,
}
\end{rustc}

Mutability is a property of the binding, not of the structure itself. If you're used to field-level mutability, this may seem 
strange at first, but it significantly simplifies things. It even lets you make things mutable on a temporary basis:

\begin{rustc}
struct Point {
    x: i32,
    y: i32,
}

fn main() {
    let mut point = Point { x: 0, y: 0 };

    point.x = 5;

    let point = point; // now immutable

    point.y = 6; // this causes an error
}
\end{rustc}

Your structure can still contain \code{\&mut} pointers, which will let you do some kinds of mutation:

\begin{rustc}
struct Point {
    x: i32,
    y: i32,
}

struct PointRef<'a> {
    x: &'a mut i32,
    y: &'a mut i32,
}

fn main() {
    let mut point = Point { x: 0, y: 0 };

    {
        let r = PointRef { x: &mut point.x, y: &mut point.y };

        *r.x = 5;
        *r.y = 6;
    }

    assert_eq!(5, point.x);
    assert_eq!(6, point.y);
}
\end{rustc}

\subsection*{Update syntax}

A \struct\ can include \code{..} to indicate that you want to use a copy of some other \struct\ for some of the values. For example:

\begin{rustc}
struct Point3d {
    x: i32,
    y: i32,
    z: i32,
}

let mut point = Point3d { x: 0, y: 0, z: 0 };
point = Point3d { y: 1, .. point };
\end{rustc}

This gives \code{point} a new \y, but keeps the old \x\ and \z\ values. It doesn't have to be the same \struct\ either, you can use 
this syntax when making new ones, and it will copy the values you don't specify:

\begin{rustc}
let origin = Point3d { x: 0, y: 0, z: 0 };
let point = Point3d { z: 1, x: 2, .. origin };
\end{rustc}

\subsection*{Tuple structs}
\label{paragraph:tuple_structs}

Rust has another data type that's like a hybrid between a tuple and a \struct, called a 'tuple struct'. Tuple structs have a name, 
but their fields don't. They are declared with the \struct\ keyword, and then with a name followed by a tuple:

\begin{rustc}
struct Color(i32, i32, i32);
struct Point(i32, i32, i32);

let black = Color(0, 0, 0);
let origin = Point(0, 0, 0);
\end{rustc}

Here, \code{black} and \code{origin} are not equal, even though they contain the same values.

\blank

It is almost always better to use a \struct\ than a tuple struct. We would write \code{Color} and \code{Point} like this instead:

\begin{rustc}
struct Color {
    red: i32,
    blue: i32,
    green: i32,
}

struct Point {
    x: i32,
    y: i32,
    z: i32,
}
\end{rustc}

Good names are important, and while values in a tuple struct can be referenced with dot notation as well, a \struct\ gives us 
actual names, rather than positions.

\blank

There is one case when a tuple struct is very useful, though, and that is when it has only one element. We call this the 'newtype' 
pattern, because it allows you to create a new type that is distinct from its contained value and also expresses its own semantic 
meaning:

\begin{rustc}
struct Inches(i32);

let length = Inches(10);

let Inches(integer_length) = length;
println!("length is {} inches", integer_length);
\end{rustc}

As you can see here, you can extract the inner integer type through a destructuring \keylet, as with regular tuples. In this case, the 
\code{let Inches(integer\_length)} assigns \code{10} to \code{integer\_length}.

\subsection*{Unit-like structs}

You can define a \struct\ with no members at all:

\begin{rustc}
struct Electron;

let x = Electron;
\end{rustc}

Such a \struct\ is called 'unit-like' because it resembles the empty tuple, \code{()}, sometimes called 'unit'. Like a tuple struct, 
it defines a new type.

\blank

This is rarely useful on its own (although sometimes it can serve as a marker type), but in combination with other features, it can 
become useful. For instance, a library may ask you to create a structure that implements a certain trait to handle events 
(see \nameref{sec:syntax_traits}). If you don't have any data you need to store in the structure, you can create a unit-like \struct.


\subsection{Enums}
\label{sec:syntax_enums}
\input{src/syntax/enums.tex}

\subsection{Match}
\label{sec:syntax_match}
Often, a simple \code{if/else} (see \nameref{sec:syntax_if}) isn't enough, because you have more than two possible options. Also, 
conditions can get quite complex. Rust has a keyword, \match, that allows you to replace complicated \code{if/else} groupings with 
something more powerful. Check it out:

\begin{rustc}
let x = 5;

match x {
    1 => println!("one"),
    2 => println!("two"),
    3 => println!("three"),
    4 => println!("four"),
    5 => println!("five"),
    _ => println!("something else"),
}
\end{rustc}

% TODO make sections a hyperref
\match\ takes an expression and then branches based on its value. Each 'arm' of the branch is of the form \code{val => expression}. When 
the value matches, that arm's expression will be evaluated. It's called \match\ because of the term 'pattern matching', which \match\ is 
an implementation of. There's a separate section on patterns that covers all the patterns that are possible here.

\blank

One of the many advantages of \match\ is it enforces 'exhaustiveness checking'. For example if we remove the last arm with the 
underscore \code{\_}, the compiler will give us an error:

\begin{verbatim}
error: non-exhaustive patterns: `_` not covered
\end{verbatim}

Rust is telling us that we forgot a value. The compiler infers from \x\ that it can have any positive 32bit value; for example 1 
to 2,147,483,647. The \code{\_} acts as a 'catch-all', and will catch all possible values that aren't specified in an arm of \match. 
As you can see with the previous example, we provide \match\ arms for integers 1-5, if \x\ is 6 or any other value, then it is caught 
by \code{\_}.

\blank

\match\ is also an expression, which means we can use it on the right-hand side of a \keylet\ binding or directly where an expression 
is used:

\begin{rustc}
let x = 5;

let number = match x {
    1 => "one",
    2 => "two",
    3 => "three",
    4 => "four",
    5 => "five",
    _ => "something else",
};
\end{rustc}

Sometimes it's a nice way of converting something from one type to another; in this example the integers are converted to \String.

\subsubsection*{Matching on enums}

Another important use of the \match\ keyword is to process the possible variants of an \enum:

\begin{rustc}
enum Message {
    Quit,
    ChangeColor(i32, i32, i32),
    Move { x: i32, y: i32 },
    Write(String),
}

fn quit() { /* ... */ }
fn change_color(r: i32, g: i32, b: i32) { /* ... */ }
fn move_cursor(x: i32, y: i32) { /* ... */ }

fn process_message(msg: Message) {
    match msg {
        Message::Quit => quit(),
        Message::ChangeColor(r, g, b) => change_color(r, g, b),
        Message::Move { x: x, y: y } => move_cursor(x, y),
        Message::Write(s) => println!("{}", s),
    };
}
\end{rustc}

Again, the Rust compiler checks exhaustiveness, so it demands that you have a match arm for every variant of the enum. If you leave 
one off, it will give you a compile-time error unless you use \code{\_} or provide all possible arms.

\blank

% TODO make if let a hyperref
Unlike the previous uses of \match, you can't use the normal \keyif\ statement to do this. You can use the \code{if let} (see) 
statement, which can be seen as an abbreviated form of \match.


\subsection{Patterns}
\label{sec:syntax_patterns}
Patterns are quite common in Rust. We use them in \nameref{sec:syntax_variableBindings}, match statements (see \nameref{sec:syntax_match}),
and other places, too. Let's go on a whirlwind tour of all of the things patterns can do!

\blank

A quick refresher: you can match against literals directly, and \code{\_} acts as an 'any' case:

\begin{rustc}
let x = 1;

match x {
    1 => println!("one"),
    2 => println!("two"),
    3 => println!("three"),
    _ => println!("anything"),
}
\end{rustc}

This prints \code{one}.

There's one pitfall with patterns: like anything that introduces a new binding, they introduce shadowing. For example:

\begin{rustc}
let x = 1;
let c = 'c';

match c {
    x => println!("x: {} c: {}", x, c),
}

println!("x: {}", x)
\end{rustc}

This prints:

\begin{verbatim}
x: c c: c
x: 1
\end{verbatim}

In other words, \code{x =>} matches the pattern and introduces a new binding named \x. This new binding is in scope for the match arm 
and takes on the value of \code{c}. Notice that the value of \x\ outside the scope of the match has no bearing on the value of \x\ 
within it. Because we already have a binding named \x\, this new \x\ shadows it.

\subsubsection*{Multiple patterns}

You can match multiple patterns with \code{|}:

\begin{rustc}
let x = 1;

match x {
    1 | 2 => println!("one or two"),
    3 => println!("three"),
    _ => println!("anything"),
}
\end{rustc}

This prints \code{one or two}.

\subsection*{Destructuring}

If you have a compound data type, like a \struct\ (see \nameref{sec:syntax_structs}), you can destructure it inside of a pattern:

\begin{rustc}
struct Point {
    x: i32,
    y: i32,
}

let origin = Point { x: 0, y: 0 };

match origin {
    Point { x, y } => println!("({},{})", x, y),
}
\end{rustc}

We can use \code{:} to give a value a different name.

\begin{rustc}
struct Point {
    x: i32,
    y: i32,
}

let origin = Point { x: 0, y: 0 };

match origin {
    Point { x: x1, y: y1 } => println!("({},{})", x1, y1),
}
\end{rustc}

If we only care about some of the values, we don't have to give them all names:

\begin{rustc}
struct Point {
    x: i32,
    y: i32,
}

let origin = Point { x: 0, y: 0 };

match origin {
    Point { x, .. } => println!("x is {}", x),
}
\end{rustc}

This prints \code{x is 0}.

You can do this kind of match on any member, not only the first:

\begin{rustc}
struct Point {
    x: i32,
    y: i32,
}

let origin = Point { x: 0, y: 0 };

match origin {
    Point { y, .. } => println!("y is {}", y),
}
\end{rustc}

This prints \code{y is 0}.

\blank

% TODO make tuples and enums hyperrefs
This 'destructuring' behavior works on any compound data type, like tuples or enums.

\subsubsection*{Ignoring bindings}

You can use \code{\_} in a pattern to disregard the type and value. For example, here's a \match\ against a \code{Result<T, E>}:

\begin{rustc}
match some_value {
    Ok(value) => println!("got a value: {}", value),
    Err(_) => println!("an error occurred"),
}
\end{rustc}

In the first arm, we bind the value inside the \code{Ok} variant to value. But in the \code{Err} arm, we use \code{\_} to disregard 
the specific error, and print a general error message.

\blank

\code{\_} is valid in any pattern that creates a binding. This can be useful to ignore parts of a larger structure:

\begin{rustc}
fn coordinate() -> (i32, i32, i32) {
    // generate and return some sort of triple tuple
}

let (x, _, z) = coordinate();
\end{rustc}

Here, we bind the first and last element of the tuple to \x\ and \z, but ignore the middle element.

\blank

Similarly, you can use \code{..} in a pattern to disregard multiple values.

\begin{rustc}
enum OptionalTuple {
    Value(i32, i32, i32),
    Missing,
}

let x = OptionalTuple::Value(5, -2, 3);

match x {
    OptionalTuple::Value(..) => println!("Got a tuple!"),
    OptionalTuple::Missing => println!("No such luck."),
}
\end{rustc}

This prints \code{Got a tuple!}.

\subsubsection*{ref and ref mut}

If you want to get a reference (see \nameref{sec:syntax_referencesBorrowing}), use the \keyref\ keyword:

\begin{rustc}
let x = 5;

match x {
    ref r => println!("Got a reference to {}", r),
}
\end{rustc}

This prints \code{Got a reference to 5}.

\blank

Here, the \code{r} inside the \match\ has the type \code{\&i32}. In other words, the \keyref\ keyword creates a reference, for 
use in the pattern. If you need a mutable reference, \keyref\code{ mut} will work in the same way:

\begin{rustc}
let mut x = 5;

match x {
    ref mut mr => println!("Got a mutable reference to {}", mr),
}
\end{rustc}

\subsubsection*{Ranges}

You can match a range of values with \code{...}:

\begin{rustc}
let x = 1;

match x {
    1 ... 5 => println!("one through five"),
    _ => println!("anything"),
}
\end{rustc}

This prints \code{one through five}.

\blank

Ranges are mostly used with integers and \varchar s:

\begin{rustc}
let x = 'ä';

match x {
    'a' ... 'j' => println!("early letter"),
    'k' ... 'z' => println!("late letter"),
    _ => println!("something else"),
}
\end{rustc}

This prints \code{something else}.

\subsubsection*{Bindings}

You can bind values to names with \code{@}:

\begin{rustc}
let x = 1;

match x {
    e @ 1 ... 5 => println!("got a range element {}", e),
    _ => println!("anything"),
}
\end{rustc}

This prints \code{got a range element 1}. This is useful when you want to do a complicated match of part of a data structure:

\begin{rustc}
#[derive(Debug)]
struct Person {
    name: Option<String>,
}

let name = "Steve".to_string();
let mut x: Option<Person> = Some(Person { name: Some(name) });
match x {
    Some(Person { name: ref a @ Some(_), .. }) => println!("{:?}", a),
    _ => {}
}
\end{rustc}

This prints \code{Some("Steve")}: we've bound the inner \code{name} to \code{a}.

\blank

If you use \code{@} with \code{|}, you need to make sure the name is bound in each part of the pattern:

\begin{rustc}
let x = 5;

match x {
    e @ 1 ... 5 | e @ 8 ... 10 => println!("got a range element {}", e),
    _ => println!("anything"),
}
\end{rustc}

\subsubsection*{Guards}

You can introduce 'match guards' with \keyif:

\begin{rustc}
enum OptionalInt {
    Value(i32),
    Missing,
}

let x = OptionalInt::Value(5);

match x {
    OptionalInt::Value(i) if i > 5 => println!("Got an int bigger than five!"),
    OptionalInt::Value(..) => println!("Got an int!"),
    OptionalInt::Missing => println!("No such luck."),
}
\end{rustc}

This prints \code{Got an int!}.

\blank

If you're using \keyif\ with multiple patterns, the \keyif\ applies to both sides:

\begin{rustc}
let x = 4;
let y = false;

match x {
    4 | 5 if y => println!("yes"),
    _ => println!("no"),
}
\end{rustc}

This prints \code{no}, because the \keyif\ applies to the whole of \code{4 | 5}, and not to only the \code{5}. In other words, 
the precedence of \keyif\ behaves like this:

\begin{verbatim}
(4 | 5) if y => ...
\end{verbatim}

not this:

\begin{verbatim}
4 | (5 if y) => ...
\end{verbatim}

\subsubsection*{Mix and Match}

Whew! That's a lot of different ways to match things, and they can all be mixed and matched, depending on what you're doing:

\begin{rustc}
match x {
    Foo { x: Some(ref name), y: None } => ...
}
\end{rustc}

Patterns are very powerful. Make good use of them.


\subsection{Method Syntax}
\label{sec:syntax_methodSyntax}
Functions are great, but if you want to call a bunch of them on some data, it can be awkward. Consider this code:

\begin{rustc}
baz(bar(foo));
\end{rustc}

We would read this left-to-right, and so we see 'baz bar foo'. But this isn't the order that the functions would get called in, that's 
inside-out: 'foo bar baz'. Wouldn't it be nice if we could do this instead?

\begin{rustc}
foo.bar().baz();
\end{rustc}

Luckily, as you may have guessed with the leading question, you can! Rust provides the ability to use this 'method call syntax' via the 
\code{impl} keyword.

\subsection*{Method calls}

Here's how it works:

\begin{rustc}
struct Circle {
    x: f64,
    y: f64,
    radius: f64,
}

impl Circle {
    fn area(&self) -> f64 {
        std::f64::consts::PI * (self.radius * self.radius)
    }
}

fn main() {
    let c = Circle { x: 0.0, y: 0.0, radius: 2.0 };
    println!("{}", c.area());
}
\end{rustc}

This will print \code{12.566371}.

We've made a \struct\ that represents a circle. We then write an \code{impl} block, and inside it, define a method, \code{area}.

\blank

Methods take a special first parameter, of which there are three variants: \code{self}, \code{\&self}, and \code{\&mut self}. You can think 
of this first parameter as being the \code{foo} in \code{foo.bar()}. The three variants correspond to the three kinds of things \code{foo} 
could be: \code{self} if it's a value on the stack, \code{\&self} if it's a reference, and \code{\&mut self} if it's a mutable reference. 
Because we took the \code{\&self} parameter to \code{area}, we can use it like any other parameter. Because we know it's a \code{Circle}, 
we can access the \code{radius} like we would with any other \struct.

\blank

We should default to using \code{\&self}, as you should prefer borrowing over taking ownership, as well as taking immutable references over 
mutable ones. Here's an example of all three variants:

\begin{rustc}
struct Circle {
    x: f64,
    y: f64,
    radius: f64,
}

impl Circle {
    fn reference(&self) {
       println!("taking self by reference!");
    }

    fn mutable_reference(&mut self) {
       println!("taking self by mutable reference!");
    }

    fn takes_ownership(self) {
       println!("taking ownership of self!");
    }
}
\end{rustc}

You can use as many \code{impl} blocks as you'd like. The previous example could have also been written like this:

\begin{rustc}
struct Circle {
    x: f64,
    y: f64,
    radius: f64,
}

impl Circle {
    fn reference(&self) {
       println!("taking self by reference!");
    }
}

impl Circle {
    fn mutable_reference(&mut self) {
       println!("taking self by mutable reference!");
    }
}

impl Circle {
    fn takes_ownership(self) {
       println!("taking ownership of self!");
    }
}
\end{rustc}

\subsection*{Chaining method calls}

So, now we know how to call a method, such as \code{foo.bar()}. But what about our original example, \code{foo.bar().baz()}? This is called 
'method chaining'. Let's look at an example:

\begin{rustc}
struct Circle {
    x: f64,
    y: f64,
    radius: f64,
}

impl Circle {
    fn area(&self) -> f64 {
        std::f64::consts::PI * (self.radius * self.radius)
    }

    fn grow(&self, increment: f64) -> Circle {
        Circle { x: self.x, y: self.y, radius: self.radius + increment }
    }
}

fn main() {
    let c = Circle { x: 0.0, y: 0.0, radius: 2.0 };
    println!("{}", c.area());

    let d = c.grow(2.0).area();
    println!("{}", d);
}
\end{rustc}

Check the return type:

\begin{rustc}
fn grow(&self, increment: f64) -> Circle {
\end{rustc}

We say we're returning a \code{Circle}. With this method, we can grow a new \code{Circle} to any arbitrary size.

\subsection*{Associated functions}

You can also define associated functions that do not take a \code{self} parameter. Here's a pattern that's very common in Rust code:

\begin{rustc}
struct Circle {
    x: f64,
    y: f64,
    radius: f64,
}

impl Circle {
    fn new(x: f64, y: f64, radius: f64) -> Circle {
        Circle {
            x: x,
            y: y,
            radius: radius,
        }
    }
}

fn main() {
    let c = Circle::new(0.0, 0.0, 2.0);
}
\end{rustc}

This 'associated function' builds a new \code{Circle} for us. Note that associated functions are called with the \code{Struct::function()} 
syntax, rather than the \code{ref.method()} syntax. Some other languages call associated functions 'static methods'.

\subsection*{Builder Pattern}

Let's say that we want our users to be able to create \code{Circles}, but we will allow them to only set the properties they care about. 
Otherwise, the \x\ and \y\ attributes will be \code{0.0}, and the \code{radius} will be \code{1.0}. Rust doesn't have method overloading, 
named arguments, or variable arguments. We employ the builder pattern instead. It looks like this:

\begin{rustc}
struct Circle {
    x: f64,
    y: f64,
    radius: f64,
}

impl Circle {
    fn area(&self) -> f64 {
        std::f64::consts::PI * (self.radius * self.radius)
    }
}

struct CircleBuilder {
    x: f64,
    y: f64,
    radius: f64,
}

impl CircleBuilder {
    fn new() -> CircleBuilder {
        CircleBuilder { x: 0.0, y: 0.0, radius: 1.0, }
    }

    fn x(&mut self, coordinate: f64) -> &mut CircleBuilder {
        self.x = coordinate;
        self
    }

    fn y(&mut self, coordinate: f64) -> &mut CircleBuilder {
        self.y = coordinate;
        self
    }

    fn radius(&mut self, radius: f64) -> &mut CircleBuilder {
        self.radius = radius;
        self
    }

    fn finalize(&self) -> Circle {
        Circle { x: self.x, y: self.y, radius: self.radius }
    }
}

fn main() {
    let c = CircleBuilder::new()
                .x(1.0)
                .y(2.0)
                .radius(2.0)
                .finalize();

    println!("area: {}", c.area());
    println!("x: {}", c.x);
    println!("y: {}", c.y);
}
\end{rustc}

What we've done here is make another \struct, \code{CircleBuilder}. We've defined our builder methods on it. We've also defined our \code{area()}
method on \code{Circle}. We also made one more method on \code{CircleBuilder}: \code{finalize()}. This method creates our final \code{Circle} 
from the builder. Now, we've used the type system to enforce our concerns: we can use the methods on \code{CircleBuilder} to constrain making
\code{Circles} in any way we choose.


\subsection{Vectors}
\label{sec:syntax_vectors}
A 'vector' is a dynamic or 'growable' array, implemented as the standard library type \href{https://doc.rust-lang.org/std/vec/}{Vec<T>}. The 
\code{T} means that we can have vectors of any type (see the chapter on \nameref{sec:syntax_generics} for more). Vectors always allocate 
their data on the heap. You can create them with the \code{vec!} macro:

\begin{rustc}
let v = vec![1, 2, 3, 4, 5]; // v: Vec<i32>
\end{rustc}

(Notice that unlike the \println\ macro we've used in the past, we use square brackets \code{[]} with \code{vec!} macro. Rust allows you to 
use either in either situation, this is just convention.)

\blank

There's an alternate form of \code{vec!} for repeating an initial value:

\begin{rustc}
let v = vec![0; 10]; // ten zeroes
\end{rustc}

\subsection*{Accessing elements}

To get the value at a particular index in the vector, we use \code{[]}s:

\begin{rustc}
let v = vec![1, 2, 3, 4, 5];

println!("The third element of v is {}", v[2]);
\end{rustc}

The indices count from \code{0}, so the third element is \code{v[2]}.

\blank

It's also important to note that you must index with the \code{usize} type:

\begin{rustc}
let v = vec![1, 2, 3, 4, 5];

let i: usize = 0;
let j: i32 = 0;

// works
v[i];

// doesn't
v[j];
\end{rustc}

Indexing with a non-\code{usize} type gives an error that looks like this:

\begin{verbatim}
error: the trait `core::ops::Index<i32>` is not implemented for the type
`collections::vec::Vec<_>` [E0277]
v[j];
^~~~
note: the type `collections::vec::Vec<_>` cannot be indexed by `i32`
error: aborting due to previous error
\end{verbatim}

There's a lot of punctuation in that message, but the core of it makes sense: you cannot index with an \itt.

\subsection*{Out-of-bounds Access}

If you try to access an index that doesn't exist:

\begin{rustc}
let v = vec![1, 2, 3];
println!("Item 7 is {}", v[7]);
\end{rustc}

then the current thread will panic (see \nameref{sec:effective_concurrency}) with a message like this:

\begin{verbatim}
thread '<main>' panicked at 'index out of bounds: the len is 3 but the index is 7'
\end{verbatim}

If you want to handle out-of-bounds errors without panicking, you can use methods like 
\href{http://doc.rust-lang.org/std/vec/struct.Vec.html#method.get}{get} or 
\href{http://doc.rust-lang.org/std/vec/struct.Vec.html#method.get_mut}{get\_mut} that return \code{None} when given an invalid index:

\begin{rustc}
let v = vec![1, 2, 3];
match v.get(7) {
    Some(x) => println!("Item 7 is {}", x),
    None => println!("Sorry, this vector is too short.")
}
\end{rustc}

\subsection*{Iterating}

Once you have a vector, you can iterate through its elements with \code{for}. There are three versions:

\begin{rustc}
let mut v = vec![1, 2, 3, 4, 5];

for i in &v {
    println!("A reference to {}", i);
}

for i in &mut v {
    println!("A mutable reference to {}", i);
}

for i in v {
    println!("Take ownership of the vector and its element {}", i);
}
\end{rustc}

Vectors have many more useful methods, which you can read about in \href{https://doc.rust-lang.org/std/vec/}{their API documentation}.


\subsection{Strings}
\label{sec:syntax_strings}
Strings are an important concept for any programmer to master. Rust's string handling system is a bit different from other languages, due to 
its systems focus. Any time you have a data structure of variable size, things can get tricky, and strings are a re-sizable data structure. 
That being said, Rust's strings also work differently than in some other systems languages, such as C.

\blank

Let's dig into the details. A 'string' is a sequence of Unicode scalar values encoded as a stream of UTF-8 bytes. All strings are guaranteed 
to be a valid encoding of UTF-8 sequences. Additionally, unlike some systems languages, strings are not null-terminated and can contain null 
bytes.

\blank

Rust has two main types of strings: \code{\&str} and \String. Let's talk about \code{\&str} first. These are called 'string slices'. A string 
slice has a fixed size, and cannot be mutated. It is a reference to a sequence of UTF-8 bytes.

\begin{rustc}
let greeting = "Hello there."; // greeting: &'static str
\end{rustc}

\code{"Hello there."} is a string literal and its type is \code{\&'static str}. A string literal is a string slice that is statically allocated,
meaning that it's saved inside our compiled program, and exists for the entire duration it runs. The \code{greeting} binding is a reference to 
this statically allocated string. Any function expecting a string slice will also accept a string literal.

\blank

String literals can span multiple lines. There are two forms. The first will include the newline and the leading spaces:

\begin{rustc}
let s = "foo
    bar";

assert_eq!("foo\n        bar", s);
\end{rustc}

The second, with a \code{\\}, trims the spaces and the newline:

\begin{rustc}
let s = "foo\
    bar"; 

assert_eq!("foobar", s);
\end{rustc}

Rust has more than only \code{\&str}s though. A \String, is a heap-allocated string. This string is growable, and is also guaranteed to be 
UTF-8. \String s are commonly created by converting from a string slice using the \code{to\_string} method.

\begin{rustc}
let mut s = "Hello".to_string(); // mut s: String
println!("{}", s);

s.push_str(", world.");
println!("{}", s);
\end{rustc}

\String s will coerce into \code{\&str} with an \&:

\begin{rustc}
fn takes_slice(slice: &str) {
    println!("Got: {}", slice);
}

fn main() {
    let s = "Hello".to_string();
    takes_slice(&s);
}
\end{rustc}

This coercion does not happen for functions that accept one of \code{\&str}'s traits instead of \code{\&str}. For example, 
\href{https://doc.rust-lang.org/std/net/struct.TcpStream.html#method.connect}{TcpStream::connect} has a parameter of type \code{ToSocketAddrs}. 
A \code{\&str} is okay but a \String\ must be explicitly converted using \code{\&*}.

\begin{rustc}
use std::net::TcpStream;

TcpStream::connect("192.168.0.1:3000"); // &str parameter

let addr_string = "192.168.0.1:3000".to_string();
TcpStream::connect(&*addr_string); // convert addr_string to &str
\end{rustc}

Viewing a \String\ as a \code{\&str} is cheap, but converting the \code{\&str} to a \String\ involves allocating memory. No reason to do 
that unless you have to!

\subsubsection*{Indexing}

Because strings are valid UTF-8, strings do not support indexing:

\begin{rustc}
let s = "hello";

println!("The first letter of s is {}", s[0]); // ERROR!!!
\end{rustc}

Usually, access to a vector with \code{[]} is very fast. But, because each character in a UTF-8 encoded string can be multiple bytes, 
you have to walk over the string to find the $n_{th}$ letter of a string. This is a significantly more expensive operation, and we don't 
want to be misleading. Furthermore, 'letter' isn't something defined in Unicode, exactly. We can choose to look at a string as individual 
bytes, or as codepoints:

\begin{rustc}
let hachiko = "忠犬ハチ公";

for b in hachiko.as_bytes() {
    print!("{}, ", b);
}

println!("");

for c in hachiko.chars() {
    print!("{}, ", c);
}

println!("");
\end{rustc}

\begin{rustc}
let hachiko = "忠犬ハチ公";

for b in hachiko.as_bytes() {
    print!("{}, ", b);
}

println!("");

for c in hachiko.chars() {
    print!("{}, ", c);
}

println!("");
\end{rustc}

This prints:

\begin{verbatim}
229, 191, 160, 231, 138, 172, 227, 131, 143, 227, 131, 129, 229, 133, 172,
忠, 犬, ハ, チ, 公,
\end{verbatim}

As you can see, there are more bytes than \varchar s.

\blank

You can get something similar to an index like this:

\begin{rustc}
let dog = hachiko.chars().nth(1); // kinda like hachiko[1]
\end{rustc}

This emphasizes that we have to walk from the beginning of the list of \varchar s.

\subsubsection*{Slicing}

You can get a slice of a string with slicing syntax:

\begin{rustc}
let dog = "hachiko";
let hachi = &dog[0..5];
\end{rustc}

But note that these are \emph{byte offsets}, not \emph{character offsets}. So this will fail at runtime:

\begin{rustc}
let dog = "忠犬ハチ公";
let hachi = &dog[0..2];
\end{rustc}

with this error:

\begin{verbatim}
thread '<main>' panicked at 'index 0 and/or 2 in `忠犬ハチ公` do not lie on
character boundary'
\end{verbatim}

\subsubsection*{Concatenation}

If you have a \String, you can concatenate a \code{\&str} to the end of it:

\begin{rustc}
let hello = "Hello ".to_string();
let world = "world!";

let hello_world = hello + world;
\end{rustc}

But if you have two \String s, you need an \code{\&}:

\begin{rustc}
let hello = "Hello ".to_string();
let world = "world!".to_string();

let hello_world = hello + &world;
\end{rustc}

% TODO make deref coercion a hyperref
This is because \code{\&String} can automatically coerce to a \code{\&str}. This is a feature called 'Deref coercions'.


\subsection{Generics}
\label{sec:syntax_generics}
Sometimes, when writing a function or data type, we may want it to work for multiple types of arguments. In Rust, we can do this with 
generics. Generics are called 'parametric polymorphism' in type theory, which means that they are types or functions that have multiple 
forms ('poly' is multiple, 'morph' is form) over a given parameter ('parametric').

\blank

Anyway, enough type theory, let's check out some generic code. Rust's standard library provides a type, \code{Option<T>}, that's generic:

\begin{rustc}
enum Option<T> {
    Some(T),
    None,
}
\end{rustc}

The \code{<T>} part, which you've seen a few times before, indicates that this is a generic data type. Inside the declaration of our \enum, 
wherever we see a \code{T}, we substitute that type for the same type used in the generic. Here's an example of using \code{Option<T>}, with 
some extra type annotations:

\begin{rustc}
let x: Option<i32> = Some(5);
\end{rustc}

In the type declaration, we say \code{Option<i32>}. Note how similar this looks to \code{Option<T>}. So, in this particular Option, \code{T} has 
the value of \itt. On the right-hand side of the binding, we make a \code{Some(T)}, where \code{T} is \code{5}. Since that's an \itt, the two 
sides match, and Rust is happy. If they didn't match, we'd get an error:

\begin{rustc}
let x: Option<f64> = Some(5);
// error: mismatched types: expected `core::option::Option<f64>`,
// found `core::option::Option<_>` (expected f64 but found integral variable)
\end{rustc}

That doesn't mean we can't make \code{Option<T>}s that hold an \code{f64}! They have to match up:

\begin{rustc}
let x: Option<i32> = Some(5);
let y: Option<f64> = Some(5.0f64);
\end{rustc}

This is just fine. One definition, multiple uses.

\blank

Generics don't have to only be generic over one type. Consider another type from Rust's standard library that's similar, \code{Result<T, E>}:

\begin{rustc}
enum Result<T, E> {
    Ok(T),
    Err(E),
}
\end{rustc}

This type is generic over two types: \code{T} and \code{E}. By the way, the capital letters can be any letter you'd like. We could define 
\code{Result<T, E>} as:

\begin{rustc}
enum Result<A, Z> {
    Ok(A),
    Err(Z),
}
\end{rustc}

if we wanted to. Convention says that the first generic parameter should be \code{T}, for 'type', and that we use \code{E} for 'error'. Rust 
doesn't care, however.

\blank

The \code{Result<T, E>} type is intended to be used to return the result of a computation, and to have the ability to return an error if it 
didn't work out.

\subsection*{Generic functions}

We can write functions that take generic types with a similar syntax:

\begin{rustc}
fn takes_anything<T>(x: T) {
    // do something with x
}
\end{rustc}

The syntax has two parts: the \code{<T>} says \enquote{this function is generic over one type, \code{T}}, and the \code{x: T} says 
\enquote{\x\ has the type \code{T}.}

\blank

Multiple arguments can have the same generic type:

\begin{rustc}
fn takes_two_of_the_same_things<T>(x: T, y: T) {
    // ...
}
\end{rustc}

We could write a version that takes multiple types:

\begin{rustc}
fn takes_two_things<T, U>(x: T, y: U) {
    // ...
}
\end{rustc}

\subsection*{Generic structs}

You can store a generic type in a \struct\ as well:

\begin{rustc}
struct Point<T> {
    x: T,
    y: T,
}

let int_origin = Point { x: 0, y: 0 };
let float_origin = Point { x: 0.0, y: 0.0 };
\end{rustc}

Similar to functions, the \code{<T>} is where we declare the generic parameters, and we then use \code{x: T} in the type declaration, too.

\blank

When you want to add an implementation for the generic \struct, you declare the type parameter after the \code{impl}:

\begin{rustc}
impl<T> Point<T> {
    fn swap(&mut self) {
        std::mem::swap(&mut self.x, &mut self.y);
    }
}
\end{rustc}

So far you've seen generics that take absolutely any type. These are useful in many cases: you've already seen \code{Option<T>}, and later 
you'll meet universal container types like \href{https://doc.rust-lang.org/std/vec/struct.Vec.html}{Vec<T>}. On the other hand, often you 
want to trade that flexibility for increased expressive power. Read about trait bounds to see why and how (see \nameref{sec:syntax_traits}).


\subsection{Traits}
\label{sec:syntax_traits}
A trait is a language feature that tells the Rust compiler about functionality a type must provide.

\blank

Recall the \code{impl} keyword, used to call a function with method syntax:

\begin{rustc}
struct Circle {
    x: f64,
    y: f64,
    radius: f64,
}

impl Circle {
    fn area(&self) -> f64 {
        std::f64::consts::PI * (self.radius * self.radius)
    }
}
\end{rustc}

Traits are similar, except that we first define a trait with a method signature, then implement the trait for a type. In this example, we 
implement the trait \code{HasArea} for \code{Circle}:

\begin{rustc}
struct Circle {
    x: f64,
    y: f64,
    radius: f64,
}

trait HasArea {
    fn area(&self) -> f64;
}

impl HasArea for Circle {
    fn area(&self) -> f64 {
        std::f64::consts::PI * (self.radius * self.radius)
    }
}
\end{rustc}

As you can see, the \code{trait} block looks very similar to the \code{impl} block, but we don't define a body, only a type signature. When 
we \code{impl} a trait, we use \code{impl Trait for Item}, rather than only \code{impl Item}.

\subsection*{Trait bounds on generic functions}

Traits are useful because they allow a type to make certain promises about its behavior. Generic functions can exploit this to constrain, 
or \nameref{sec:gloss_bounds}, the types they accept. Consider this function, which does not compile:

\begin{rustc}
fn print_area<T>(shape: T) {
    println!("This shape has an area of {}", shape.area());
}
\end{rustc}

Rust complains:

\begin{verbatim}
error: no method named `area` found for type `T` in the current scope
\end{verbatim}

Because \code{T} can be any type, we can't be sure that it implements the \code{area} method. But we can add a trait bound to our generic 
\code{T}, ensuring that it does:

\begin{rustc}
fn print_area<T: HasArea>(shape: T) {
    println!("This shape has an area of {}", shape.area());
}
\end{rustc}

The syntax \code{<T: HasArea>} means \enquote{any type that implements the \code{HasArea} trait.} Because traits define function type signatures, 
we can be sure that any type which implements \code{HasArea} will have an \code{.area()} method.

\blank

Here's an extended example of how this works:

\begin{rustc}
trait HasArea {
    fn area(&self) -> f64;
}

struct Circle {
    x: f64,
    y: f64,
    radius: f64,
}

impl HasArea for Circle {
    fn area(&self) -> f64 {
        std::f64::consts::PI * (self.radius * self.radius)
    }
}

struct Square {
    x: f64,
    y: f64,
    side: f64,
}

impl HasArea for Square {
    fn area(&self) -> f64 {
        self.side * self.side
    }
}

fn print_area<T: HasArea>(shape: T) {
    println!("This shape has an area of {}", shape.area());
}

fn main() {
    let c = Circle {
        x: 0.0f64,
        y: 0.0f64,
        radius: 1.0f64,
    };

    let s = Square {
        x: 0.0f64,
        y: 0.0f64,
        side: 1.0f64,
    };

    print_area(c);
    print_area(s);
}
\end{rustc}

This program outputs:

\begin{verbatim}
This shape has an area of 3.141593
This shape has an area of 1
\end{verbatim}

As you can see, \code{print\_area} is now generic, but also ensures that we have passed in the correct types. If we pass in an incorrect type:

\begin{rustc}
print_area(5);
\end{rustc}

We get a compile-time error:

\begin{verbatim}
error: the trait `HasArea` is not implemented for the type `_` [E0277]
\end{verbatim}

\subsection*{Trait bounds on generic structs}

Your generic structs can also benefit from trait bounds. All you need to do is append the bound when you declare type parameters. Here is a 
new type \code{Rectangle<T>} and its operation \code{is\_square()}:

\begin{rustc}
struct Rectangle<T> {
    x: T,
    y: T,
    width: T,
    height: T,
}

impl<T: PartialEq> Rectangle<T> {
    fn is_square(&self) -> bool {
        self.width == self.height
    }
}

fn main() {
    let mut r = Rectangle {
        x: 0,
        y: 0,
        width: 47,
        height: 47,
    };

    assert!(r.is_square());

    r.height = 42;
    assert!(!r.is_square());
}
\end{rustc}

\code{is\_square()} needs to check that the sides are equal, so the sides must be of a type that implements the 
\href{https://doc.rust-lang.org/core/cmp/trait.PartialEq.html}{core::cmp::PartialEq} trait:

\begin{rustc}
impl<T: PartialEq> Rectangle<T> { ... }
\end{rustc}

Now, a rectangle can be defined in terms of any type that can be compared for equality.

\blank

% TODO make operators traits a hyperref
Here we defined a new struct \code{Rectangle} that accepts numbers of any precision—really, objects of pretty much any type—as long as they 
can be compared for equality. Could we do the same for our \code{HasArea} structs, \code{Square} and \code{Circle}? Yes, but they need
multiplication, and to work with that we need to know more about operator traits.

\subsection*{Rules for implementing traits}

So far, we've only added trait implementations to structs, but you can implement a trait for any type. So technically, we \emph{could} 
implement \code{HasArea} for \itt:

\begin{rustc}
trait HasArea {
    fn area(&self) -> f64;
}

impl HasArea for i32 {
    fn area(&self) -> f64 {
        println!("this is silly");

        *self as f64
    }
}

5.area();
\end{rustc}

It is considered poor style to implement methods on such primitive types, even though it is possible.

\blank

This may seem like the Wild West, but there are two restrictions around implementing traits that prevent this from getting out of hand. 
The first is that if the trait isn't defined in your scope, it doesn't apply. Here's an example: the standard library provides a 
\href{https://doc.rust-lang.org/std/io/trait.Write.html}{Write} trait which adds extra functionality to \code{File}s, for doing file I/O. 
By default, a \code{File} won't have its methods:

\begin{rustc}
let mut f = std::fs::File::open("foo.txt").expect("Couldn't open foo.txt");
let buf = b"whatever"; // byte string literal. buf: &[u8; 8]
let result = f.write(buf);
\end{rustc}

Here's the error:

\begin{verbatim}
error: type `std::fs::File` does not implement any method in scope named `write`
let result = f.write(buf);
               ^~~~~~~~~~
\end{verbatim}

We need to \code{use} the \code{Write} trait first:

\begin{rustc}
use std::io::Write;

let mut f = std::fs::File::open("foo.txt").expect("Couldn't open foo.txt");
let buf = b"whatever";
let result = f.write(buf);
\end{rustc}

This will compile without error.

\blank

This means that even if someone does something bad like add methods to \itt, it won't affect you, unless you \code{use} that trait.

\blank

There's one more restriction on implementing traits: either the trait, or the type you're writing the \code{impl} for, must be defined 
by you. So, we could implement the \code{HasArea} type for \itt, because \code{HasArea} is in our code. But if we tried to implement 
\code{ToString}, a trait provided by Rust, for \itt, we could not, because neither the trait nor the type are in our code.

\blank

% TODO make trait objects a hyperref
One last thing about traits: generic functions with a trait bound use 'monomorphization' (mono: one, morph: form), so they are 
statically dispatched. What's that mean? Check out the chapter on trait objects for more details.

\subsection*{Multiple trait bounds}

You've seen that you can bound a generic type parameter with a trait:

\begin{rustc}
fn foo<T: Clone>(x: T) {
    x.clone();
}
\end{rustc}

If you need more than one bound, you can use \code{+}:

\begin{rustc}
use std::fmt::Debug;

fn foo<T: Clone + Debug>(x: T) {
    x.clone();
    println!("{:?}", x);
}
\end{rustc}

\code{T} now needs to be both \code{Clone} as well as \code{Debug}.

\subsection*{Where clause}

Writing functions with only a few generic types and a small number of trait bounds isn't too bad, but as the number increases, the 
syntax gets increasingly awkward:

\begin{rustc}
use std::fmt::Debug;

fn foo<T: Clone, K: Clone + Debug>(x: T, y: K) {
    x.clone();
    y.clone();
    println!("{:?}", y);
}
\end{rustc}

The name of the function is on the far left, and the parameter list is on the far right. The bounds are getting in the way.

\blank

Rust has a solution, and it's called a '\code{where} clause':

\begin{rustc}
use std::fmt::Debug;

fn foo<T: Clone, K: Clone + Debug>(x: T, y: K) {
    x.clone();
    y.clone();
    println!("{:?}", y);
}

fn bar<T, K>(x: T, y: K) where T: Clone, K: Clone + Debug {
    x.clone();
    y.clone();
    println!("{:?}", y);
}

fn main() {
    foo("Hello", "world");
    bar("Hello", "world");
}
\end{rustc}

\code{foo()} uses the syntax we showed earlier, and \code{bar()} uses a \code{where} clause. All you need to do is leave off the bounds 
when defining your type parameters, and then add \code{where} after the parameter list. For longer lists, whitespace can be added:

\begin{rustc}
use std::fmt::Debug;

fn bar<T, K>(x: T, y: K)
    where T: Clone,
          K: Clone + Debug {

    x.clone();
    y.clone();
    println!("{:?}", y);
}
\end{rustc}

This flexibility can add clarity in complex situations.

\blank

\code{where} is also more powerful than the simpler syntax. For example:

\begin{rustc}
trait ConvertTo<Output> {
    fn convert(&self) -> Output;
}

impl ConvertTo<i64> for i32 {
    fn convert(&self) -> i64 { *self as i64 }
}

// can be called with T == i32
fn normal<T: ConvertTo<i64>>(x: &T) -> i64 {
    x.convert()
}

// can be called with T == i64
fn inverse<T>() -> T
        // this is using ConvertTo as if it were "ConvertTo<i64>"
        where i32: ConvertTo<T> {
    42.convert()
}
\end{rustc}

This shows off the additional feature of \code{where} clauses: they allow bounds on the left-hand side not only of type parameters \code{T}, 
but also of types (\itt\ in this case). In this example, \itt\ must implement \code{ConvertTo<T>}. Rather than defining what \itt\ is (since 
that's obvious), the \code{where} clause here constrains \code{T}.

\subsection*{Default methods}

A default method can be added to a trait definition if it is already known how a typical implementor will define a method. For example, 
\code{is\_invalid()} is defined as the opposite of \code{is\_valid()}:

\begin{rustc}
trait Foo {
    fn is_valid(&self) -> bool;

    fn is_invalid(&self) -> bool { !self.is_valid() }
}
\end{rustc}

Implementors of the \code{Foo} trait need to implement \code{is\_valid()} but not \code{is\_invalid()} due to the added default behavior. 
This default behavior can still be overridden as in:

\begin{rustc}
struct UseDefault;

impl Foo for UseDefault {
    fn is_valid(&self) -> bool {
        println!("Called UseDefault.is_valid.");
        true
    }
}

struct OverrideDefault;

impl Foo for OverrideDefault {
    fn is_valid(&self) -> bool {
        println!("Called OverrideDefault.is_valid.");
        true
    }

    fn is_invalid(&self) -> bool {
        println!("Called OverrideDefault.is_invalid!");
        true // overrides the expected value of is_invalid()
    }
}

let default = UseDefault;
assert!(!default.is_invalid()); // prints "Called UseDefault.is_valid."

let over = OverrideDefault;
assert!(over.is_invalid()); // prints "Called OverrideDefault.is_invalid!"
\end{rustc}

\subsection*{Inheritance}

Sometimes, implementing a trait requires implementing another trait:

\begin{rustc}
trait Foo {
    fn foo(&self);
}

trait FooBar : Foo {
    fn foobar(&self);
}
\end{rustc}

Implementors of \code{FooBar} must also implement \code{Foo}, like this:

\begin{rustc}
struct Baz;

impl Foo for Baz {
    fn foo(&self) { println!("foo"); }
}

impl FooBar for Baz {
    fn foobar(&self) { println!("foobar"); }
}
\end{rustc}

If we forget to implement \code{Foo}, Rust will tell us:

\begin{verbatim}
error: the trait `main::Foo` is not implemented for the type `main::Baz` [E0277]
\end{verbatim}

\subsection*{Deriving}

% TODO make atttribute a hyperref
Implementing traits like \code{Debug} and \code{Default} repeatedly can become quite tedious. For that reason, Rust provides an attribute 
that allows you to let Rust automatically implement traits for you:

\begin{rustc}
#[derive(Debug)]
struct Foo;

fn main() {
    println!("{:?}", Foo);
}
\end{rustc}

However, deriving is limited to a certain set of traits:

\begin{itemize}
  \item{\href{https://doc.rust-lang.org/core/clone/trait.Clone.html}{Clone}}
  \item{\href{https://doc.rust-lang.org/core/marker/trait.Copy.html}{Copy}}
  \item{\href{https://doc.rust-lang.org/core/fmt/trait.Debug.html}{Debug}}
  \item{\href{https://doc.rust-lang.org/core/default/trait.Default.html}{Default}}
  \item{\href{https://doc.rust-lang.org/core/cmp/trait.Eq.html}{Eq}}
  \item{\href{https://doc.rust-lang.org/core/hash/trait.Hash.html}{Hash}}
  \item{\href{https://doc.rust-lang.org/core/cmp/trait.Ord.html}{Ord}}
  \item{\href{https://doc.rust-lang.org/core/cmp/trait.PartialEq.html}{PartialEq}}
  \item{\href{https://doc.rust-lang.org/core/cmp/trait.PartialOrd.html}{PartialOrd}}
\end{itemize}


\subsection{Drop}
\label{sec:syntax_drop}
Now that we've discussed traits, let's talk about a particular trait provided by the Rust standard library, 
\href{https://doc.rust-lang.org/std/ops/trait.Drop.html}{Drop}. The \code{Drop} trait provides a way to run some code when a value goes out of 
scope. For example:

\begin{rustc}
struct HasDrop;

impl Drop for HasDrop {
    fn drop(&mut self) {
        println!("Dropping!");
    }
}

fn main() {
    let x = HasDrop;

    // do stuff

} // x goes out of scope here
\end{rustc}

When \x\ goes out of scope at the end of \code{main()}, the code for \code{Drop} will run. \code{Drop} has one method, which is also called 
\code{drop()}. It takes a mutable reference to \code{self}.

\blank

That's it! The mechanics of \code{Drop} are very simple, but there are some subtleties. For example, values are dropped in the opposite 
order they are declared. Here's another example:

\begin{rustc}
struct Firework {
    strength: i32,
}

impl Drop for Firework {
    fn drop(&mut self) {
        println!("BOOM times {}!!!", self.strength);
    }
}

fn main() {
    let firecracker = Firework { strength: 1 };
    let tnt = Firework { strength: 100 };
}
\end{rustc}

This will output:

\begin{verbatim}
BOOM times 100!!!
BOOM times 1!!!
\end{verbatim}

The TNT goes off before the firecracker does, because it was declared afterwards. Last in, first out.

\blank

So what is \code{Drop} good for? Generally, \code{Drop} is used to clean up any resources associated with a \struct. For example, the 
\href{https://doc.rust-lang.org/std/sync/struct.Arc.html}{Arc<T>} type is a reference-counted type. When \code{Drop} is called, it will 
decrement the reference count, and if the total number of references is zero, will clean up the underlying value.


\subsection{if let}
\label{sec:syntax_iflet}
\code{if let} allows you to combine \keyif\ and \keylet\ together to reduce the overhead of certain kinds of pattern matches.

\blank

For example, let's say we have some sort of \code{Option<T>}. We want to call a function on it if it's \code{Some<T>}, but do nothing if it's 
\code{None}. That looks like this:

\begin{rustc}
match option {
    Some(x) => { foo(x) },
    None => {},
}
\end{rustc}

We don't have to use \match\ here, for example, we could use \keyif:

\begin{rustc}
if option.is_some() {
    let x = option.unwrap();
    foo(x);
}
\end{rustc}

Neither of these options is particularly appealing. We can use \code{if let} to do the same thing in a nicer way:

\begin{rustc}
if let Some(x) = option {
    foo(x);
}
\end{rustc}

If a pattern (see \nameref{sec:syntax_patterns}) matches successfully, it binds any appropriate parts of the value to the identifiers in 
the pattern, then evaluates the expression. If the pattern doesn't match, nothing happens.

\blank

If you want to do something else when the pattern does not match, you can use \code{else}:

\begin{rustc}
if let Some(x) = option {
    foo(x);
} else {
    bar();
}
\end{rustc}

\subsubsection*{\code{while let}}

In a similar fashion, \code{while let} can be used when you want to conditionally loop as long as a value matches a certain pattern. It 
turns code like this:

\begin{rustc}
let mut v = vec![1, 3, 5, 7, 11];
loop {
    match v.pop() {
        Some(x) =>  println!("{}", x),
        None => break,
    }
}
\end{rustc}

Into code like this:

\begin{rustc}
let mut v = vec![1, 3, 5, 7, 11];
while let Some(x) = v.pop() {
    println!("{}", x);
}
\end{rustc}


\subsection{Trait Objects}
\label{sec:syntax_traitObjects}
When code involves polymorphism, there needs to be a mechanism to determine which specific version is actually run. This is called 
'dispatch'. There are two major forms of dispatch: static dispatch and dynamic dispatch. While Rust favors static dispatch, it also 
supports dynamic dispatch through a mechanism called 'trait objects'.

\subsubsection*{Background}

For the rest of this chapter, we'll need a trait and some implementations. Let's make a simple one, \code{Foo}. It has one method that 
is expected to return a \String.

\begin{rustc}
trait Foo {
    fn method(&self) -> String;
}
\end{rustc}

We'll also implement this trait for \code{u8} and \String:

\begin{rustc}
impl Foo for u8 {
    fn method(&self) -> String { format!("u8: {}", *self) }
}

impl Foo for String {
    fn method(&self) -> String { format!("string: {}", *self) }
}
\end{rustc}

\subsubsection*{Static dispatch}

We can use this trait to perform static dispatch with trait bounds:

\begin{rustc}
fn do_something<T: Foo>(x: T) {
    x.method();
}

fn main() {
    let x = 5u8;
    let y = "Hello".to_string();

    do_something(x);
    do_something(y);
}
\end{rustc}

Rust uses 'monomorphization' to perform static dispatch here. This means that Rust will create a special version of \code{do\_something()} for 
both \code{u8} and \String, and then replace the call sites with calls to these specialized functions. In other words, Rust generates something 
like this:

\begin{rustc}
fn do_something_u8(x: u8) {
    x.method();
}

fn do_something_string(x: String) {
    x.method();
}

fn main() {
    let x = 5u8;
    let y = "Hello".to_string();

    do_something_u8(x);
    do_something_string(y);
}
\end{rustc}

This has a great upside: static dispatch allows function calls to be inlined because the callee is known at compile time, and inlining is the 
key to good optimization. Static dispatch is fast, but it comes at a tradeoff: 'code bloat', due to many copies of the same function existing 
in the binary, one for each type.

\blank

Furthermore, compilers aren't perfect and may \enquote{optimize} code to become slower. For example, functions inlined too eagerly will bloat 
the instruction cache (cache rules everything around us). This is part of the reason that \code{\#[inline]} and \code{\#[inline(always)]} 
should be used carefully, and one reason why using a dynamic dispatch is sometimes more efficient.

\blank

However, the common case is that it is more efficient to use static dispatch, and one can always have a thin statically-dispatched wrapper 
function that does a dynamic dispatch, but not vice versa, meaning static calls are more flexible. The standard library tries to be 
statically dispatched where possible for this reason.

\subsubsection*{Dynamic dispatch}

Rust provides dynamic dispatch through a feature called 'trait objects'. Trait objects, like \code{\&Foo} or \code{Box<Foo>}, are normal 
values that store a value of \emph{any type} that implements the given trait, where the precise type can only be known at runtime.

\blank

A trait object can be obtained from a pointer to a concrete type that implements the trait by casting it (e.g. \code{\&x} as \code{\&Foo}) 
or coercing it (e.g. using \code{\&x} as an argument to a function that takes \code{\&Foo}).

\blank

These trait object coercions and casts also work for pointers like \code{\&mut T} to \code{\&mut Foo} and \code{Box<T>} to \code{Box<Foo>}, 
but that's all at the moment. Coercions and casts are identical.

\blank

This operation can be seen as 'erasing' the compiler's knowledge about the specific type of the pointer, and hence trait objects are 
sometimes referred to as 'type erasure'.

\blank

Coming back to the example above, we can use the same trait to perform dynamic dispatch with trait objects by casting:

\begin{rustc}
fn do_something(x: &Foo) {
    x.method();
}

fn main() {
    let x = 5u8;
    do_something(&x as &Foo);
}
\end{rustc}

or by coercing:

\begin{rustc}
fn do_something(x: &Foo) {
    x.method();
}

fn main() {
    let x = "Hello".to_string();
    do_something(&x);
}
\end{rustc}

A function that takes a trait object is not specialized to each of the types that implements \code{Foo}: only one copy is generated, often 
(but not always) resulting in less code bloat. However, this comes at the cost of requiring slower virtual function calls, and effectively
inhibiting any chance of inlining and related optimizations from occurring.

\subsubsection*{Why pointers?}

Rust does not put things behind a pointer by default, unlike many managed languages, so types can have different sizes. Knowing the size of 
the value at compile time is important for things like passing it as an argument to a function, moving it about on the stack and allocating 
(and deallocating) space on the heap to store it.

\blank

For \code{Foo}, we would need to have a value that could be at least either a \String\ (24 bytes) or a \code{u8} (1 byte), as well as any 
other type for which dependent crates may implement \code{Foo} (any number of bytes at all). There's no way to guarantee that this last 
point can work if the values are stored without a pointer, because those other types can be arbitrarily large.

\blank

Putting the value behind a pointer means the size of the value is not relevant when we are tossing a trait object around, only the size 
of the pointer itself.

\subsubsection*{Representation}

The methods of the trait can be called on a trait object via a special record of function pointers traditionally called a 'vtable' 
(created and managed by the compiler).

\blank

Trait objects are both simple and complicated: their core representation and layout is quite straight-forward, but there are some curly 
error messages and surprising behaviors to discover.

\blank

Let's start simple, with the runtime representation of a trait object. The \code{std::raw} module contains structs with layouts that are 
the same as the complicated built-in types, \href{https://doc.rust-lang.org/std/raw/struct.TraitObject.html}{including trait objects}:

\begin{rustc}
pub struct TraitObject {
    pub data: *mut (),
    pub vtable: *mut (),
}
\end{rustc}

That is, a trait object like \code{\&Foo} consists of a 'data' pointer and a 'vtable' pointer.

\blank

The data pointer addresses the data (of some unknown type \code{T}) that the trait object is storing, and the vtable pointer points to the 
vtable ('virtual method table') corresponding to the implementation of \code{Foo} for \code{T}.

\blank

A vtable is essentially a struct of function pointers, pointing to the concrete piece of machine code for each method in the implementation. 
A method call like \code{trait\_object.method()} will retrieve the correct pointer out of the vtable and then do a dynamic call of it. 
For example:

\begin{rustc}
struct FooVtable {
    destructor: fn(*mut ()),
    size: usize,
    align: usize,
    method: fn(*const ()) -> String,
}

// u8:

fn call_method_on_u8(x: *const ()) -> String {
    // the compiler guarantees that this function is only called
    // with `x` pointing to a u8
    let byte: &u8 = unsafe { &*(x as *const u8) };

    byte.method()
}

static Foo_for_u8_vtable: FooVtable = FooVtable {
    destructor: /* compiler magic */,
    size: 1,
    align: 1,

    // cast to a function pointer
    method: call_method_on_u8 as fn(*const ()) -> String,
};


// String:

fn call_method_on_String(x: *const ()) -> String {
    // the compiler guarantees that this function is only called
    // with `x` pointing to a String
    let string: &String = unsafe { &*(x as *const String) };

    string.method()
}

static Foo_for_String_vtable: FooVtable = FooVtable {
    destructor: /* compiler magic */,
    // values for a 64-bit computer, halve them for 32-bit ones
    size: 24,
    align: 8,

    method: call_method_on_String as fn(*const ()) -> String,
};
\end{rustc}

The \code{destructor} field in each vtable points to a function that will clean up any resources of the vtable's type: for \code{u8} it is 
trivial, but for \String\ it will free the memory. This is necessary for owning trait objects like \code{Box<Foo>}, which need to clean-up 
both the \code{Box} allocation as well as the internal type when they go out of scope. The \code{size} and \code{align} fields store the 
size of the erased type, and its alignment requirements; these are essentially unused at the moment since the information is embedded in 
the destructor, but will be used in the future, as trait objects are progressively made more flexible.

\blank

Suppose we've got some values that implement \code{Foo}. The explicit form of construction and use of \code{Foo} trait objects might look 
a bit like (ignoring the type mismatches: they're all pointers anyway):

\begin{rustc}
let a: String = "foo".to_string();
let x: u8 = 1;

// let b: &Foo = &a;
let b = TraitObject {
    // store the data
    data: &a,
    // store the methods
    vtable: &Foo_for_String_vtable
};

// let y: &Foo = x;
let y = TraitObject {
    // store the data
    data: &x,
    // store the methods
    vtable: &Foo_for_u8_vtable
};

// b.method();
(b.vtable.method)(b.data);

// y.method();
(y.vtable.method)(y.data);
\end{rustc}

\subsubsection*{Object Safety}

Not every trait can be used to make a trait object. For example, vectors implement \code{Clone}, but if we try to make a trait object:

\begin{rustc}
let v = vec![1, 2, 3];
let o = &v as &Clone;
\end{rustc}

We get an error:

\begin{verbatim}
error: cannot convert to a trait object because trait `core::clone::Clone` is not object-safe [E0038]
let o = &v as &Clone;
        ^~
note: the trait cannot require that `Self : Sized`
let o = &v as &Clone;
        ^~
\end{verbatim}

The error says that \code{Clone} is not 'object-safe'. Only traits that are object-safe can be made into trait objects. A trait is 
object-safe if both of these are true:

\begin{itemize}
  \item{the trait does not require that \code{Self: Sized}}
  \item{all of its methods are object-safe}
\end{itemize}

So what makes a method object-safe? Each method must require that \code{Self: Sized} or all of the following:

\begin{itemize}
  \item{must not have any type parameters}
  \item{must not use \code{Self}}
\end{itemize}

Whew! As we can see, almost all of these rules talk about \code{Self}. A good intuition is \enquote{except in special circumstances, if your 
trait's method uses \code{Self}, it is not object-safe.}


\subsection{Closures}
\label{sec:syntax_closures}
Sometimes it is useful to wrap up a function and \emph{free variables} for better clarity and reuse. The free variables that can be used 
come from the enclosing scope and are 'closed over' when used in the function. From this, we get the name 'closures' and Rust provides a 
really great implementation of them, as we'll see.

\subsubsection*{Syntax}

Closures look like this:

\begin{rustc}
let plus_one = |x: i32| x + 1;

assert_eq!(2, plus_one(1));
\end{rustc}

We create a binding, \code{plus\_one}, and assign it to a closure. The closure's arguments go between the pipes (\code{|}), and the body 
is an expression, in this case, \code{x + 1}. Remember that \code{\{ \}} is an expression, so we can have multi-line closures too:

\begin{rustc}
let plus_two = |x| {
    let mut result: i32 = x;

    result += 1;
    result += 1;

    result
};

assert_eq!(4, plus_two(2));
\end{rustc}

You'll notice a few things about closures that are a bit different from regular named functions defined with \code{fn}. The first is that 
we did not need to annotate the types of arguments the closure takes or the values it returns. We can:

\begin{rustc}
let plus_one = |x: i32| -> i32 { x + 1 };

assert_eq!(2, plus_one(1));
\end{rustc}

But we don't have to. Why is this? Basically, it was chosen for ergonomic reasons. While specifying the full type for named functions 
is helpful with things like documentation and type inference, the full type signatures of closures are rarely documented since they're 
anonymous, and they don't cause the kinds of error-at-a-distance problems that inferring named function types can.

\blank

The second is that the syntax is similar, but a bit different. I've added spaces here for easier comparison:

\begin{rustc}
fn  plus_one_v1   (x: i32) -> i32 { x + 1 }
let plus_one_v2 = |x: i32| -> i32 { x + 1 };
let plus_one_v3 = |x: i32|          x + 1  ;
\end{rustc}

Small differences, but they're similar.

\subsubsection*{Closures and their environment}

The environment for a closure can include bindings from its enclosing scope in addition to parameters and local bindings. It looks like this:

\begin{rustc}
let num = 5;
let plus_num = |x: i32| x + num;

assert_eq!(10, plus_num(5));
\end{rustc}

This closure, \code{plus\_num}, refers to a \keylet\ binding in its scope: \code{num}. More specifically, it borrows the binding. If we 
do something that would conflict with that binding, we get an error. Like this one:

\begin{rustc}
let mut num = 5;
let plus_num = |x: i32| x + num;

let y = &mut num;
\end{rustc}

Which errors with:

\begin{verbatim}
error: cannot borrow `num` as mutable because it is also borrowed as immutable
    let y = &mut num;
                 ^~~
note: previous borrow of `num` occurs here due to use in closure; the immutable
  borrow prevents subsequent moves or mutable borrows of `num` until the borrow
  ends
    let plus_num = |x| x + num;
                   ^~~~~~~~~~~
note: previous borrow ends here
fn main() {
    let mut num = 5;
    let plus_num = |x| x + num;

    let y = &mut num;
}
^
\end{verbatim}

A verbose yet helpful error message! As it says, we can't take a mutable borrow on \code{num} because the closure is already borrowing 
it. If we let the closure go out of scope, we can:

\begin{rustc}
let mut num = 5;
{
    let plus_num = |x: i32| x + num;

} // plus_num goes out of scope, borrow of num ends

let y = &mut num;
\end{rustc}

If your closure requires it, however, Rust will take ownership and move the environment instead. This doesn't work:

\begin{rustc}
let nums = vec![1, 2, 3];

let takes_nums = || nums;

println!("{:?}", nums);
\end{rustc}

We get this error:

\begin{verbatim}
note: `nums` moved into closure environment here because it has type
  `[closure(()) -> collections::vec::Vec<i32>]`, which is non-copyable
let takes_nums = || nums;
                 ^~~~~~~
\end{verbatim}

\code{Vec<T>} has ownership over its contents, and therefore, when we refer to it in our closure, we have to take ownership of \code{nums}. 
It's the same as if we'd passed \code{nums} to a function that took ownership of it.

\subsubsection*{\code{move} closures}

We can force our closure to take ownership of its environment with the \code{move} keyword:

\begin{rustc}
let num = 5;

let owns_num = move |x: i32| x + num;
\end{rustc}

Now, even though the keyword is \code{move}, the variables follow normal move semantics. In this case, \code{5} implements \code{Copy}, 
and so \code{owns\_num} takes ownership of a copy of \code{num}. So what's the difference?

\begin{rustc}
let mut num = 5;

{
    let mut add_num = |x: i32| num += x;

    add_num(5);
}

assert_eq!(10, num);
\end{rustc}

So in this case, our closure took a mutable reference to \code{num}, and then when we called \code{add\_num}, it mutated the underlying 
value, as we'd expect. We also needed to declare \code{add\_num} as \mut\ too, because we're mutating its environment.

\blank

If we change to a \code{move} closure, it's different:

\begin{rustc}
let mut num = 5;

{
    let mut add_num = move |x: i32| num += x;

    add_num(5);
}

assert_eq!(5, num);
\end{rustc}

We only get \code{5}. Rather than taking a mutable borrow out on our \code{num}, we took ownership of a copy.

\blank

Another way to think about \code{move} closures: they give a closure its own stack frame. Without \code{move}, a closure may be tied 
to the stack frame that created it, while a \code{move} closure is self-contained. This means that you cannot generally return a non-\code{move} 
closure from a function, for example.

\blank

But before we talk about taking and returning closures, we should talk some more about the way that closures are implemented. As a systems 
language, Rust gives you tons of control over what your code does, and closures are no different.

\subsubsection*{Closure implementation}

Rust's implementation of closures is a bit different than other languages. They are effectively syntax sugar for traits. You'll want to 
make sure to have read the traits section (see \nameref{sec:syntax_traits}) before this one, as well as the section on 
trait objects (see \nameref{sec:syntax_traitObjects}).

\blank

Got all that? Good.

\blank

The key to understanding how closures work under the hood is something a bit strange: Using \code{()} to call a function, like \code{foo()}, 
is an overloadable operator. From this, everything else clicks into place. In Rust, we use the trait system to overload operators. Calling 
functions is no different. We have three separate traits to overload with:

\begin{rustc}
pub trait Fn<Args> : FnMut<Args> {
    extern "rust-call" fn call(&self, args: Args) -> Self::Output;
}

pub trait FnMut<Args> : FnOnce<Args> {
    extern "rust-call" fn call_mut(&mut self, args: Args) -> Self::Output;
}

pub trait FnOnce<Args> {
    type Output;

    extern "rust-call" fn call_once(self, args: Args) -> Self::Output;
}
\end{rustc}

You'll notice a few differences between these traits, but a big one is \code{self}: \code{Fn} takes \code{\&self}, \code{FnMut} takes 
\code{\&mut self}, and \code{FnOnce} takes \code{self}. This covers all three kinds of self via the usual method call syntax. But we've 
split them up into three traits, rather than having a single one. This gives us a large amount of control over what kind of closures we 
can take.

\blank

The \code{|| \{\}} syntax for closures is sugar for these three traits. Rust will generate a \struct\ for the environment, \code{impl} 
the appropriate trait, and then use it.

\subsubsection*{Taking closures as arguments}

Now that we know that closures are traits, we already know how to accept and return closures: the same as any other trait!

\blank

This also means that we can choose static vs dynamic dispatch as well. First, let's write a function which takes something callable, 
calls it, and returns the result:

\begin{rustc}
fn call_with_one<F>(some_closure: F) -> i32
    where F : Fn(i32) -> i32 {

    some_closure(1)
}

let answer = call_with_one(|x| x + 2);

assert_eq!(3, answer);
\end{rustc}

We pass our closure, \code{|x| x + 2}, to \code{call\_with\_one}. It does what it suggests: it calls the closure, giving it \code{1} as 
an argument.

\blank

Let's examine the signature of \code{call\_with\_one} in more depth:

\begin{rustc}
fn call_with_one<F>(some_closure: F) -> i32
\end{rustc}

We take one parameter, and it has the type \code{F}. We also return a \itt. This part isn't interesting. The next part is:

\begin{rustc}
    where F : Fn(i32) -> i32 {
\end{rustc}

Because \code{Fn} is a trait, we can bound our generic with it. In this case, our closure takes a \itt\ as an argument and returns an 
\itt, and so the generic bound we use is \code{Fn(i32) -> i32}.

\blank

There's one other key point here: because we're bounding a generic with a trait, this will get monomorphized, and therefore, we'll 
be doing static dispatch into the closure. That's pretty neat. In many languages, closures are inherently heap allocated, and will 
always involve dynamic dispatch. In Rust, we can stack allocate our closure environment, and statically dispatch the call. This happens 
quite often with iterators and their adapters, which often take closures as arguments.

\blank

Of course, if we want dynamic dispatch, we can get that too. A trait object handles this case, as usual:

\begin{rustc}
fn call_with_one(some_closure: &Fn(i32) -> i32) -> i32 {
    some_closure(1)
}

let answer = call_with_one(&|x| x + 2);

assert_eq!(3, answer);
\end{rustc}

Now we take a trait object, a \code{\&Fn}. And we have to make a reference to our closure when we pass it to \code{call\_with\_one}, 
so we use \code{\&||}.

\subsubsection*{Function pointers and closures}

A function pointer is kind of like a closure that has no environment. As such, you can pass a function pointer to any function 
expecting a closure argument, and it will work:

\begin{rustc}
fn call_with_one(some_closure: &Fn(i32) -> i32) -> i32 {
    some_closure(1)
}

fn add_one(i: i32) -> i32 {
    i + 1
}

let f = add_one;

let answer = call_with_one(&f);

assert_eq!(2, answer);
\end{rustc}

In this example, we don't strictly need the intermediate variable \code{f}, the name of the function works just fine too:

\begin{rustc}
let answer = call_with_one(&add_one);
\end{rustc}

\subsubsection*{Returning closures}

It's very common for functional-style code to return closures in various situations. If you try to return a closure, you may run 
into an error. At first, it may seem strange, but we'll figure it out. Here's how you'd probably try to return a closure from a 
function:

\begin{rustc}
fn factory() -> (Fn(i32) -> i32) {
    let num = 5;

    |x| x + num
}

let f = factory();

let answer = f(1);
assert_eq!(6, answer);
\end{rustc}

This gives us these long, related errors:

\begin{verbatim}
error: the trait `core::marker::Sized` is not implemented for the type
`core::ops::Fn(i32) -> i32` [E0277]
fn factory() -> (Fn(i32) -> i32) {
                ^~~~~~~~~~~~~~~~
note: `core::ops::Fn(i32) -> i32` does not have a constant size known at compile-time
fn factory() -> (Fn(i32) -> i32) {
                ^~~~~~~~~~~~~~~~
error: the trait `core::marker::Sized` is not implemented for the type `core::ops::Fn(i32) -> i32` [E0277]
let f = factory();
    ^
note: `core::ops::Fn(i32) -> i32` does not have a constant size known at compile-time
let f = factory();
    ^
\end{verbatim}

In order to return something from a function, Rust needs to know what size the return type is. But since \code{Fn} is a trait, it 
could be various things of various sizes: many different types can implement \code{Fn}. An easy way to give something a size is to 
take a reference to it, as references have a known size. So we'd write this:

\begin{rustc}
fn factory() -> &(Fn(i32) -> i32) {
    let num = 5;

    |x| x + num
}

let f = factory();

let answer = f(1);
assert_eq!(6, answer);
\end{rustc}

But we get another error:

\begin{verbatim}
error: missing lifetime specifier [E0106]
fn factory() -> &(Fn(i32) -> i32) {
                ^~~~~~~~~~~~~~~~~
\end{verbatim}

Right. Because we have a reference, we need to give it a lifetime. But our \code{factory()} function takes no arguments, so 
elision (see \nameref{paragraph:lifetime_elision}) doesn't kick in here. Then what choices do we have? Try \code{'static}:

\begin{rustc}
fn factory() -> &'static (Fn(i32) -> i32) {
    let num = 5;

    |x| x + num
}

let f = factory();

let answer = f(1);
assert_eq!(6, answer);
\end{rustc}

But we get another error:

\begin{verbatim}
error: mismatched types:
 expected `&'static core::ops::Fn(i32) -> i32`,
    found `[closure@<anon>:7:9: 7:20]`
(expected &-ptr,
    found closure) [E0308]
         |x| x + num
         ^~~~~~~~~~~
\end{verbatim}

This error is letting us know that we don't have a \code{\&'static Fn(i32) -> i32}, we have a \code{[closure@<anon>:7:9: 7:20]}. Wait, what?

\blank

Because each closure generates its own environment \struct\ and implementation of \code{Fn} and friends, these types are anonymous. 
They exist solely for this closure. So Rust shows them as \code{closure@<anon>}, rather than some autogenerated name.

\blank

The error also points out that the return type is expected to be a reference, but what we are trying to return is not. Further, we 
cannot directly assign a \code{'static} lifetime to an object. So we'll take a different approach and return a 'trait object' by 
\code{Box}ing up the \code{Fn}. This almost works:

\begin{rustc}
fn factory() -> Box<Fn(i32) -> i32> {
    let num = 5;

    Box::new(|x| x + num)
}
let f = factory();

let answer = f(1);
assert_eq!(6, answer);
\end{rustc}

There's just one last problem:

\begin{verbatim}
error: closure may outlive the current function, but it borrows `num`,
which is owned by the current function [E0373]
Box::new(|x| x + num)
         ^~~~~~~~~~~
\end{verbatim}

Well, as we discussed before, closures borrow their environment. And in this case, our environment is based on a stack-allocated \code{5}, 
the \code{num} variable binding. So the borrow has a lifetime of the stack frame. So if we returned this closure, the function call 
would be over, the stack frame would go away, and our closure is capturing an environment of garbage memory! With one last fix, we can 
make this work:

\begin{rustc}
fn factory() -> Box<Fn(i32) -> i32> {
    let num = 5;

    Box::new(move |x| x + num)
}
let f = factory();

let answer = f(1);
assert_eq!(6, answer);
\end{rustc}

By making the inner closure a \code{move Fn}, we create a new stack frame for our closure. By \code{Box}ing it up, we've given it a known 
size, and allowing it to escape our stack frame.


\subsection{Universal Function Call Syntax}
\label{sec:syntax_universalFunctionCallSyntax}
Sometimes, functions can have the same names. Consider this code:

\begin{rustc}
trait Foo {
    fn f(&self);
}

trait Bar {
    fn f(&self);
}

struct Baz;

impl Foo for Baz {
    fn f(&self) { println!("Baz's impl of Foo"); }
}

impl Bar for Baz {
    fn f(&self) { println!("Baz's impl of Bar"); }
}

let b = Baz;
\end{rustc}

If we were to try to call \code{b.f()}, we'd get an error:

\begin{verbatim}
error: multiple applicable methods in scope [E0034]
b.f();
  ^~~
note: candidate #1 is defined in an impl of the trait `main::Foo` for the type
`main::Baz`
    fn f(&self) { println!("Baz's impl of Foo"); }
    ^~~~~~~~~~~~~~~~~~~~~~~~~~~~~~~~~~~~~~~~~~~~~~
note: candidate #2 is defined in an impl of the trait `main::Bar` for the type
`main::Baz`
    fn f(&self) { println!("Baz's impl of Bar"); }
    ^~~~~~~~~~~~~~~~~~~~~~~~~~~~~~~~~~~~~~~~~~~~~~
\end{verbatim}

We need a way to disambiguate which method we need. This feature is called ‘universal function call syntax', and it looks like this:

\begin{rustc}
Foo::f(&b);
Bar::f(&b);
\end{rustc}

Let's break it down.

\begin{rustc}
Foo::
Bar::
\end{rustc}

These halves of the invocation are the types of the two traits: \code{Foo} and \code{Bar}. This is what ends up actually doing the 
disambiguation between the two: Rust calls the one from the trait name you use.

\begin{rustc}
f(&b)
\end{rustc}

When we call a method like \code{b.f()} using method syntax (see \nameref{sec:syntax_methodSyntax}), Rust will automatically borrow 
\code{b} if \code{f()} takes \code{\&self}. In this case, Rust will not, and so we need to pass an explicit \code{\&b}.

\subsection*{Angle-bracket Form}

The form of UFCS we just talked about:

\begin{rustc}
Trait::method(args);
\end{rustc}

Is a short-hand. There's an expanded form of this that's needed in some situations:

\begin{rustc}
<Type as Trait>::method(args);
\end{rustc}

The \code{<>::} syntax is a means of providing a type hint. The type goes inside the \code{<>}s. In this case, the type is 
\code{Type as Trait}, indicating that we want \code{Trait}'s version of \code{method} to be called here. The \code{as Trait} part 
is optional if it's not ambiguous. Same with the angle brackets, hence the shorter form.

\blank

Here's an example of using the longer form.

\begin{rustc}
trait Foo {
    fn foo() -> i32;
}

struct Bar;

impl Bar {
    fn foo() -> i32 {
        20
    }
}

impl Foo for Bar {
    fn foo() -> i32 {
        10
    }
}

fn main() {
    assert_eq!(10, <Bar as Foo>::foo());
    assert_eq!(20, Bar::foo());
}
\end{rustc}

Using the angle bracket syntax lets you call the trait method instead of the inherent one.


\subsection{Crates and Modules}
\label{sec:syntax_cratesAndModules}
When a project starts getting large, it's considered good software engineering practice to split it up into a bunch of smaller 
pieces, and then fit them together. It is also important to have a well-defined interface, so that some of your functionality is 
private, and some is public. To facilitate these kinds of things, Rust has a module system.

\subsubsection*{Basic terminology: Crates and Modules}

Rust has two distinct terms that relate to the module system: 'crate' and 'module'. A crate is synonymous with a 'library' or 
'package' in other languages. Hence “Cargo” as the name of Rust's package management tool: you ship your crates to others with 
Cargo. Crates can produce an executable or a library, depending on the project.

\blank

Each crate has an implicit root module that contains the code for that crate. You can then define a tree of sub-modules under 
that root module. Modules allow you to partition your code within the crate itself.

\blank

As an example, let's make a phrases crate, which will give us various phrases in different languages. To keep things simple, we'll 
stick to 'greetings' and 'farewells' as two kinds of phrases, and use English and Japanese (日本語) as two languages for those phrases 
to be in. We'll use this module layout:

\begin{verbatim}
                                    +-----------+
                                +---| greetings |
                                |   +-----------+
                  +---------+   |
              +---| english |---+
              |   +---------+   |   +-----------+
              |                 +---| farewells |
+---------+   |                     +-----------+
| phrases |---+
+---------+   |                     +-----------+
              |                 +---| greetings |
              |   +----------+  |   +-----------+
              +---| japanese |--+
                  +----------+  |
                                |   +-----------+
                                +---| farewells |
                                    +-----------+
\end{verbatim}

In this example, \code{phrases} is the name of our crate. All of the rest are modules. You can see that they form a tree, branching out 
from the crate \emph{root}, which is the root of the tree: \code{phrases} itself.

\blank

Now that we have a plan, let's define these modules in code. To start, generate a new crate with Cargo:

\begin{verbatim}
$ cargo new phrases
$ cd phrases
\end{verbatim}

If you remember, this generates a simple project for us:

\begin{verbatim}
$ tree .
.
├── Cargo.toml
└── src
    └── lib.rs

1 directory, 2 files
\end{verbatim}

\code{src/lib.rs} is our crate root, corresponding to the \code{phrases} in our diagram above.

\subsubsection*{Defining Modules}

To define each of our modules, we use the \code{mod} keyword. Let's make our \code{src/lib.rs} look like this:

\begin{rustc}
mod english {
    mod greetings {
    }

    mod farewells {
    }
}

mod japanese {
    mod greetings {
    }

    mod farewells {
    }
}
\end{rustc}

After the \code{mod} keyword, you give the name of the module. Module names follow the conventions for other Rust identifiers: 
\code{lower\_snake\_case}. The contents of each module are within curly braces (\code{\{\}}).

\blank

Within a given mod, you can declare sub-\code{mod}s. We can refer to sub-modules with double-colon (\code{::}) notation: our four 
nested modules are \code{english::greetings}, \code{english::farewells}, \code{japanese::greetings}, and \code{japanese::farewells}. 
Because these sub-modules are namespaced under their parent module, the names don't conflict: \code{english::greetings} and 
\code{japanese::greetings} are distinct, even though their names are both \code{greetings}.

\blank

Because this crate does not have a \code{main()} function, and is called \code{lib.rs}, Cargo will build this crate as a library:

\begin{verbatim}
$ cargo build
   Compiling phrases v0.0.1 (file:///home/you/projects/phrases)
$ ls target/debug
build  deps  examples  libphrases-a7448e02a0468eaa.rlib  native
\end{verbatim}

\code{libphrases-hash.rlib} is the compiled crate. Before we see how to use this crate from another crate, let's break it up into 
multiple files.

\subsubsection*{Multiple file crates}

If each crate were just one file, these files would get very large. It's often easier to split up crates into multiple files, 
and Rust supports this in two ways.

\blank

Instead of declaring a module like this:

\begin{rustc}
mod english {
    // contents of our module go here
}
\end{rustc}

We can instead declare our module like this:

\begin{rustc}
mod english;
\end{rustc}

If we do that, Rust will expect to find either a \code{english.rs} file, or a \code{english/mod.rs} file with the contents of our module.

\blank

Note that in these files, you don't need to re-declare the module: that's already been done with the initial \code{mod} declaration.

\blank

Using these two techniques, we can break up our crate into two directories and seven files:

\begin{verbatim}
$ tree .
.
├── Cargo.lock
├── Cargo.toml
├── src
│   ├── english
│   │   ├── farewells.rs
│   │   ├── greetings.rs
│   │   └── mod.rs
│   ├── japanese
│   │   ├── farewells.rs
│   │   ├── greetings.rs
│   │   └── mod.rs
│   └── lib.rs
└── target
    └── debug
        ├── build
        ├── deps
        ├── examples
        ├── libphrases-a7448e02a0468eaa.rlib
        └── native
\end{verbatim}

\code{src/lib.rs} is our crate root, and looks like this:

\begin{rustc}
mod english;
mod japanese;
\end{rustc}

These two declarations tell Rust to look for either \code{src/english.rs} and \code{src/japanese.rs}, or \code{src/english/mod.rs} 
and \code{src/japanese/mod.rs}, depending on our preference. In this case, because our modules have sub-modules, we've chosen the 
second. Both \code{src/english/mod.rs} and \code{src/japanese/mod.rs} look like this:

\begin{rustc}
mod greetings;
mod farewells;
\end{rustc}

Again, these declarations tell Rust to look for either \code{src/english/greetings.rs} and \code{src/japanese/greetings.rs} or 
\code{src/english/farewells/mod.rs} and \code{src/japanese/farewells/mod.rs}. Because these sub-modules don't have their own sub-modules, 
we've chosen to make them \code{src/english/greetings.rs} and \code{src/japanese/farewells.rs}. Whew!

\blank

The contents of \code{src/english/greetings.rs} and \code{src/japanese/farewells.rs} are both empty at the moment. Let's add some functions.

\blank

Put this in \code{src/english/greetings.rs}:

\begin{rustc}
fn hello() -> String {
    "Hello!".to_string()
}
\end{rustc}

Put this in \code{src/english/farewells.rs}:

\begin{rustc}
fn goodbye() -> String {
    "Goodbye.".to_string()
}
\end{rustc}

Put this in \code{src/japanese/greetings.rs}:

\begin{rustc}
fn hello() -> String {
    "こんにちは".to_string()
}
\end{rustc}

Of course, you can copy and paste this from this web page, or type something else. It's not important that you actually put 
'konnichiwa' to learn about the module system.

\blank

Put this in \code{src/japanese/farewells.rs}:

\begin{rustc}
fn goodbye() -> String {
    "さようなら".to_string()
}
\end{rustc}

(This is 'Sayōnara', if you're curious.)

\blank

Now that we have some functionality in our crate, let's try to use it from another crate.

\subsubsection*{Importing External Crates}

We have a library crate. Let's make an executable crate that imports and uses our library.

\blank

Make a \code{src/main.rs} and put this in it (it won't quite compile yet):

\begin{rustc}
extern crate phrases;

fn main() {
    println!("Hello in English: {}", phrases::english::greetings::hello());
    println!("Goodbye in English: {}", phrases::english::farewells::goodbye());

    println!("Hello in Japanese: {}", phrases::japanese::greetings::hello());
    println!("Goodbye in Japanese: {}", phrases::japanese::farewells::goodbye());
}
\end{rustc}

The \code{extern crate} declaration tells Rust that we need to compile and link to the \code{phrases} crate. We can then use \code{phrases}' 
modules in this one. As we mentioned earlier, you can use double colons to refer to sub-modules and the functions inside of them.

\blank

(Note: when importing a crate that has dashes in its name \enquote{like-this}, which is not a valid Rust identifier, it will be 
converted by changing the dashes to underscores, so you would write \code{extern crate like\_this;}.)

\blank

Also, Cargo assumes that \code{src/main.rs} is the crate root of a binary crate, rather than a library crate. Our package now has two 
crates: \code{src/lib.rs} and \code{src/main.rs}. This pattern is quite common for executable crates: most functionality is in a library 
crate, and the executable crate uses that library. This way, other programs can also use the library crate, and it's also a nice separation 
of concerns.

\blank

This doesn't quite work yet, though. We get four errors that look similar to this:

\begin{verbatim}
$ cargo build
   Compiling phrases v0.0.1 (file:///home/you/projects/phrases)
src/main.rs:4:38: 4:72 error: function `hello` is private
src/main.rs:4     println!("Hello in English: {}", phrases::english::greetings::hello());
                                                   ^~~~~~~~~~~~~~~~~~~~~~~~~~~~~~~~~~
note: in expansion of format_args!
<std macros>:2:25: 2:58 note: expansion site
<std macros>:1:1: 2:62 note: in expansion of print!
<std macros>:3:1: 3:54 note: expansion site
<std macros>:1:1: 3:58 note: in expansion of println!
phrases/src/main.rs:4:5: 4:76 note: expansion site
\end{verbatim}

By default, everything is private in Rust. Let's talk about this in some more depth.

\subsubsection*{Exporting a Public Interface}

Rust allows you to precisely control which aspects of your interface are public, and so private is the default. To make things public, 
you use the \code{pub} keyword. Let's focus on the english module first, so let's reduce our \code{src/main.rs} to only this:

\begin{rustc}
extern crate phrases;

fn main() {
    println!("Hello in English: {}", phrases::english::greetings::hello());
    println!("Goodbye in English: {}", phrases::english::farewells::goodbye());
}
\end{rustc}

In our \code{src/lib.rs}, let's add \code{pub} to the \code{english} module declaration:

\begin{rustc}
pub mod english;
mod japanese;
\end{rustc}

And in our \code{src/english/mod.rs}, let's make both \code{pub}:

\begin{rustc}
pub mod greetings;
pub mod farewells;
\end{rustc}

In our \code{src/english/greetings.rs}, let's add \code{pub} to our \code{fn} declaration:

\begin{rustc}
pub fn hello() -> String {
    "Hello!".to_string()
}
\end{rustc}

And also in \code{src/english/farewells.rs}:

\begin{rustc}
pub fn goodbye() -> String {
    "Goodbye.".to_string()
}
\end{rustc}

Now, our crate compiles, albeit with warnings about not using the \code{japanese} functions:

\begin{verbatim}
$ cargo run
   Compiling phrases v0.0.1 (file:///home/you/projects/phrases)
src/japanese/greetings.rs:1:1: 3:2 warning: function is never used: `hello`, #[warn(dead_code)] on by default
src/japanese/greetings.rs:1 fn hello() -> String {
src/japanese/greetings.rs:2     "こんにちは".to_string()
src/japanese/greetings.rs:3 }
src/japanese/farewells.rs:1:1: 3:2 warning: function is never used: `goodbye`, #[warn(dead_code)] on by default
src/japanese/farewells.rs:1 fn goodbye() -> String {
src/japanese/farewells.rs:2     "さようなら".to_string()
src/japanese/farewells.rs:3 }
     Running `target/debug/phrases`
Hello in English: Hello!
Goodbye in English: Goodbye.
\end{verbatim}

\code{pub} also applies to \struct s and their member fields. In keeping with Rust's tendency toward safety, simply making a \struct\ 
public won't automatically make its members public: you must mark the fields individually with \code{pub}.

\blank

Now that our functions are public, we can use them. Great! However, typing out \code{phrases::english::greetings::hello()} is very long 
and repetitive. Rust has another keyword for importing names into the current scope, so that you can refer to them with shorter names. 
Let's talk about \code{use}.

\subsubsection*{Importing Modules with \code{use}}

Rust has a \code{use} keyword, which allows us to import names into our local scope. Let's change our \code{src/main.rs} to look like this:

\begin{rustc}
extern crate phrases;

use phrases::english::greetings;
use phrases::english::farewells;

fn main() {
    println!("Hello in English: {}", greetings::hello());
    println!("Goodbye in English: {}", farewells::goodbye());
}
\end{rustc}

The two \code{use} lines import each module into the local scope, so we can refer to the functions by a much shorter name. By convention, 
when importing functions, it's considered best practice to import the module, rather than the function directly. In other words, you can 
do this:

\begin{rustc}
extern crate phrases;

use phrases::english::greetings::hello;
use phrases::english::farewells::goodbye;

fn main() {
    println!("Hello in English: {}", hello());
    println!("Goodbye in English: {}", goodbye());
}
\end{rustc}

But it is not idiomatic. This is significantly more likely to introduce a naming conflict. In our short program, it's not a big deal, but 
as it grows, it becomes a problem. If we have conflicting names, Rust will give a compilation error. For example, if we made the 
\code{japanese} functions public, and tried to do this:

\begin{rustc}
extern crate phrases;

use phrases::english::greetings::hello;
use phrases::japanese::greetings::hello;

fn main() {
    println!("Hello in English: {}", hello());
    println!("Hello in Japanese: {}", hello());
}
\end{rustc}

Rust will give us a compile-time error:

\begin{verbatim}
   Compiling phrases v0.0.1 (file:///home/you/projects/phrases)
src/main.rs:4:5: 4:40 error: a value named `hello` has already been imported in this module [E0252]
src/main.rs:4 use phrases::japanese::greetings::hello;
                  ^~~~~~~~~~~~~~~~~~~~~~~~~~~~~~~~~~~
error: aborting due to previous error
Could not compile `phrases`.
\end{verbatim}

If we're importing multiple names from the same module, we don't have to type it out twice. Instead of this:

\begin{rustc}
use phrases::english::greetings;
use phrases::english::farewells;
\end{rustc}

We can use this shortcut:

\begin{rustc}
use phrases::english::{greetings, farewells};
\end{rustc}

\subsubsection*{Re-exporting with \code{pub use}}

You don't only use \code{use} to shorten identifiers. You can also use it inside of your crate to re-export a function inside another 
module. This allows you to present an external interface that may not directly map to your internal code organization.

\blank

Let's look at an example. Modify your \code{src/main.rs} to read like this:

\begin{rustc}
extern crate phrases;

use phrases::english::{greetings,farewells};
use phrases::japanese;

fn main() {
    println!("Hello in English: {}", greetings::hello());
    println!("Goodbye in English: {}", farewells::goodbye());

    println!("Hello in Japanese: {}", japanese::hello());
    println!("Goodbye in Japanese: {}", japanese::goodbye());
}
\end{rustc}

Then, modify your \code{src/lib.rs} to make the \code{japanese} mod public:

\begin{rustc}
pub mod english;
pub mod japanese;
\end{rustc}

Next, make the two functions public, first in \code{src/japanese/greetings.rs}:

\begin{rustc}
pub fn hello() -> String {
    "こんにちは".to_string()
}
\end{rustc}

And then in \code{src/japanese/farewells.rs}:

\begin{rustc}
pub fn goodbye() -> String {
    "さようなら".to_string()
}
\end{rustc}

Finally, modify your \code{src/japanese/mod.rs} to read like this:

\begin{rustc}
pub use self::greetings::hello;
pub use self::farewells::goodbye;

mod greetings;
mod farewells;
\end{rustc}

The \code{pub use} declaration brings the function into scope at this part of our module hierarchy. Because we've \code{pub use}d this 
inside of our \code{japanese} module, we now have a \code{phrases::japanese::hello()} function and a \code{phrases::japanese::goodbye()}
function, even though the code for them lives in \code{phrases::japanese::greetings::hello()} and \code{phrases::japanese::farewells::goodbye()}.
Our internal organization doesn't define our external interface.

\blank

Here we have a \code{pub use} for each function we want to bring into the \code{japanese} scope. We could alternatively use the wildcard 
syntax to include everything from \code{greetings} into the current scope: \code{pub use self::greetings::*}.

\blank

What about the \code{self}? Well, by default, use declarations are absolute paths, starting from your crate root. \code{self} makes that 
path relative to your current place in the hierarchy instead. There's one more special form of \code{use}: you can \code{use super::} to 
reach one level up the tree from your current location. Some people like to think of \code{self} as \code{.} and \code{super} as \code{..}, 
from many shells' display for the current directory and the parent directory.

\blank

Outside of \code{use}, paths are relative: \code{foo::bar()} refers to a function inside of \code{foo} relative to where we are. If 
that's prefixed with \code{::}, as in \code{::foo::bar()}, it refers to a different \code{foo}, an absolute path from your crate root.

\blank

This will build and run:

\begin{verbatim}
$ cargo run
   Compiling phrases v0.0.1 (file:///home/you/projects/phrases)
     Running `target/debug/phrases`
Hello in English: Hello!
Goodbye in English: Goodbye.
Hello in Japanese: こんにちは
Goodbye in Japanese: さようなら
\end{verbatim}

\subsubsection*{Complex imports}

Rust offers several advanced options that can add compactness and convenience to your \code{extern crate} and \code{use} statements. 
Here is an example:

\begin{rustc}
extern crate phrases as sayings;

use sayings::japanese::greetings as ja_greetings;
use sayings::japanese::farewells::*;
use sayings::english::{self, greetings as en_greetings, farewells as en_farewells};

fn main() {
    println!("Hello in English; {}", en_greetings::hello());
    println!("And in Japanese: {}", ja_greetings::hello());
    println!("Goodbye in English: {}", english::farewells::goodbye());
    println!("Again: {}", en_farewells::goodbye());
    println!("And in Japanese: {}", goodbye());
}
\end{rustc}

What's going on here?

\blank

First, both \code{extern crate} and \code{use} allow renaming the thing that is being imported. So the crate is still called \enquote{phrases}, 
but here we will refer to it as \enquote{sayings}. Similarly, the first \code{use} statement pulls in the \code{japanese::greetings} module 
from the crate, but makes it available as \code{ja\_greetings} as opposed to simply \code{greetings}. This can help to avoid ambiguity 
when importing similarly-named items from different places.

\blank

The second \code{use} statement uses a star glob to bring in \emph{all} symbols from the \code{sayings::japanese::farewells} module. As 
you can see we can later refer to the Japanese \code{goodbye} function with no module qualifiers. This kind of glob should be used sparingly.

\blank

The third \code{use} statement bears more explanation. It's using \enquote{brace expansion} globbing to compress three \code{use} statements 
into one (this sort of syntax may be familiar if you've written Linux shell scripts before). The uncompressed form of this statement would be:

\begin{rustc}
use sayings::english;
use sayings::english::greetings as en_greetings;
use sayings::english::farewells as en_farewells;
\end{rustc}

As you can see, the curly brackets compress \code{use} statements for several items under the same path, and in this context \code{self} 
refers back to that path. Note: The curly brackets cannot be nested or mixed with star globbing.


\subsection{'const' and 'static'}
\label{sec:syntax_constAndStatic}
Rust has a way of defining constants with the \code{const} keyword:

\begin{rustc}
const N: i32 = 5;
\end{rustc}

Unlike \keylet\ bindings (see \nameref{sec:syntax_variableBindings}), you must annotate the type of a \code{const}.

\blank

Constants live for the entire lifetime of a program. More specifically, constants in Rust have no fixed address in memory. This 
is because they're effectively inlined to each place that they're used. References to the same constant are not necessarily guaranteed 
to refer to the same memory address for this reason.

\subsection*{\code{static}}
\label{paragraph:static}

Rust provides a 'global variable' sort of facility in static items. They're similar to constants, but static items aren't inlined upon 
use. This means that there is only one instance for each value, and it's at a fixed location in memory.

\blank

Here's an example:

\begin{rustc}
static N: i32 = 5;
\end{rustc}

Unlike \keylet\ bindings, you must annotate the type of a \code{static}.

\blank

Statics live for the entire lifetime of a program, and therefore any reference stored in a constant has a \code{'static} lifetime
(see \nameref{sec:syntax_lifetimes}):

\begin{rustc}
static NAME: &'static str = "Steve";
\end{rustc}

\subsection*{Mutability}

You can introduce mutability with the \mut\ keyword:

\begin{rustc}
static mut N: i32 = 5;
\end{rustc}

Because this is mutable, one thread could be updating \code{N} while another is reading it, causing memory unsafety. As such both 
accessing and mutating a \code{static mut} is unsafe, and so must be done in an \code{unsafe} block (see \nameref{sec:syntax_unsafe}):

\begin{rustc}
unsafe {
    N += 1;

    println!("N: {}", N);
}
\end{rustc}

Furthermore, any type stored in a \code{static} must be \code{Sync}, and may not have a \code{Drop} implementation (see \nameref{sec:syntax_drop}).

\subsection*{Initializing}

Both \code{const} and \code{static} have requirements for giving them a value. They may only be given a value that's a constant 
expression. In other words, you cannot use the result of a function call or anything similarly complex or at runtime.

\subsection*{Which construct should I use?}

Almost always, if you can choose between the two, choose \code{const}. It's pretty rare that you actually want a memory location 
associated with your constant, and using a const allows for optimizations like constant propagation not only in your crate but 
downstream crates.


\subsection{Attributes}
\label{sec:syntax_attributes}
Declarations can be annotated with 'attributes' in Rust. They look like this:

\begin{rustc}
#[test]
\end{rustc}

or like this:

\begin{rustc}
#![test]
\end{rustc}

The difference between the two is the \code{!}, which changes what the attribute applies to:

\begin{rustc}
#[foo]
struct Foo;

mod bar {
    #![bar]
}
\end{rustc}

The \code{\#[foo]} attribute applies to the next item, which is the \struct\ declaration. The \code{\#![bar]} attribute applies to the 
item enclosing it, which is the \code{mod} declaration. Otherwise, they're the same. Both change the meaning of the item they're 
attached to somehow.

\blank

For example, consider a function like this:

\begin{rustc}
#[test]
fn check() {
    assert_eq!(2, 1 + 1);
}
\end{rustc}

It is marked with \code{\#[test]}. This means it's special: when you run tests, this function will execute. When you compile as usual, 
it won't even be included. This function is now a test function (see \nameref{sec:effective_testing}).

\blank

Attributes may also have additional data:

\begin{rustc}
#[inline(always)]
fn super_fast_fn() {
\end{rustc}

Or even keys and values:

\begin{rustc}
#[cfg(target_os = "macos")]
mod macos_only {
\end{rustc}

Rust attributes are used for a number of different things. There is a full list of attributes 
\href{https://doc.rust-lang.org/reference.html#attributes}{in the reference}. Currently, you are not allowed to create your own 
attributes, the Rust compiler defines them.


\subsection{'type' Aliases}
\label{sec:syntax_typeAliases}
The \code{type} keyword lets you declare an alias of another type:

\begin{rustc}
type Name = String;
\end{rustc}

You can then use this type as if it were a real type:

\begin{rustc}
type Name = String;

let x: Name = "Hello".to_string();
\end{rustc}

Note, however, that this is an \emph{alias}, not a new type entirely. In other words, because Rust is strongly typed, you'd expect 
a comparison between two different types to fail:

\begin{rustc}
let x: i32 = 5;
let y: i64 = 5;

if x == y {
   // ...
}
\end{rustc}

this gives

\begin{verbatim}
error: mismatched types:
 expected `i32`,
    found `i64`
(expected i32,
    found i64) [E0308]
     if x == y {
             ^
\end{verbatim}

But, if we had an alias:

\begin{rustc}
type Num = i32;

let x: i32 = 5;
let y: Num = 5;

if x == y {
   // ...
}
\end{rustc}

This compiles without error. Values of a \code{Num} type are the same as a value of type \itt, in every way. You can use 
tuple struct (see \nameref{paragraph:tuple_structs}) to really get a new type.

\blank

You can also use type aliases with generics:

\begin{rustc}
use std::result;

enum ConcreteError {
    Foo,
    Bar,
}

type Result<T> = result::Result<T, ConcreteError>;
\end{rustc}

This creates a specialized version of the \code{Result} type, which always has a \code{ConcreteError} for the \code{E} part of 
\code{Result<T, E>}. This is commonly used in the standard library to create custom errors for each subsection. For example, 
\href{https://doc.rust-lang.org/std/io/type.Result.html}{io::Result}.


\subsection{Casting Between Types}
\label{sec:syntax_casting}
Rust, with its focus on safety, provides two different ways of casting different types between each other. The first, \code{as}, is 
for safe casts. In contrast, \code{transmute} allows for arbitrary casting, and is one of the most dangerous features of Rust!

\subsection*{Coercion}

Coercion between types is implicit and has no syntax of its own, but can be spelled out with \code{as}.

\blank

Coercion occurs in \keylet, \code{const}, and \code{static} statements; in function call arguments; in field values in struct 
initialization; and in a function result.

\blank

The most common case of coercion is removing mutability from a reference:

\begin{itemize}
  \item{\code{\&mut T} to \code{\&T}}
\end{itemize}

An analogous conversion is to remove mutability from a raw pointer (see \nameref{sec:syntax_rawPointers}):

\begin{itemize}
  \item{\code{*mut T} to \code{*const T}}
\end{itemize}

References can also be coerced to raw pointers:

\begin{itemize}
  \item{\code{\&T} to \code{*const T}}
  \item{\code{\&mut T} to \code{*mut T}}
\end{itemize}

Custom coercions may be defined using Deref (see \nameref{sec:syntax_derefCoercions}).

\blank

Coercion is transitive.

\subsection*{\code{as}}

The \code{as} keyword does safe casting:

\begin{rustc}
let x: i32 = 5;

let y = x as i64;
\end{rustc}

There are three major categories of safe cast: explicit coercions, casts between numeric types, and pointer casts.

\blank

Casting is not transitive: even if \code{e as U1 as U2} is a valid expression, \code{e as U2} is not necessarily so (in fact it will 
only be valid if \code{U1} coerces to \code{U2}).

\myparagraph{Explicit coercions}

A cast \code{e as U} is valid if \code{e} has type \code{T} and \code{T} \emph{coerces} to \code{U}.

\myparagraph{Numeric casts}

A cast \code{e as U} is also valid in any of the following cases:

\begin{itemize}
  \item{\code{e} has type \code{T} and \code{T} and \code{U} are any numeric types; \emph{numeric-cast}}
  \item{\code{e} is a C-like enum (with no data attached to the variants), and \code{U} is an integer type; \emph{enum-cast}}
  \item{\code{e} has type \code{bool} or \varchar\ and \code{U} is an integer type; \emph{prim-int-cast}}
  \item{\code{e} has type \code{u8} and \code{U} is \varchar; \emph{u8-char-cast}}
\end{itemize}

For example

\begin{rustc}
let one = true as u8;
let at_sign = 64 as char;
let two_hundred = -56i8 as u8;
\end{rustc}

The semantics of numeric casts are:

\begin{itemize}
  \item{Casting between two integers of the same size (e.g. i32 -> u32) is a no-op}
  \item{Casting from a larger integer to a smaller integer (e.g. u32 -> u8) will truncate}
  \item{Casting from a smaller integer to a larger integer (e.g. u8 -> u32) will}
  \begin{itemize}
    \item{zero-extend if the source is unsigned}
    \item{sign-extend if the source is signed}
  \end{itemize}
  \item{Casting from a float to an integer will round the float towards zero}
  \begin{itemize}
    \item{\href{https://github.com/rust-lang/rust/issues/10184}{NOTE: currently this will cause Undefined Behavior if the rounded 
        value cannot be represented by the target integer type.} This includes Inf and NaN. This is a bug and will be fixed.}
  \end{itemize}
  \item{Casting from an integer to float will produce the floating point representation of the integer, rounded if necessary 
      (rounding strategy unspecified)}
  \item{Casting from an f32 to an f64 is perfect and lossless}
  \item{Casting from an f64 to an f32 will produce the closest possible value (rounding strategy unspecified)}
  \begin{itemize}
    \item{\href{https://github.com/rust-lang/rust/issues/15536}{NOTE: currently this will cause Undefined Behavior if the value 
        is finite but larger or smaller than the largest or smallest finite value representable by f32.} This is a bug and will be fixed.}
  \end{itemize}
\end{itemize}

\myparagraph{Pointer casts}

Perhaps surprisingly, it is safe to cast raw pointers to and from integers, and to cast between pointers to different types subject 
to some constraints (see \nameref{sec:syntax_rawPointers}). It is only unsafe to dereference the pointer:

\begin{rustc}
let a = 300 as *const char; // a pointer to location 300
let b = a as u32;
\end{rustc}

\code{e as U} is a valid pointer cast in any of the following cases:

\begin{itemize}
  \item{\code{e} has type \code{*T}, \code{U} has type \code{*U\_0}, and either \code{U\_0: Sized} or \code{unsize\_kind(T) == unsize\_kind(U\_0)}; 
      \emph{a ptr-ptr-cast}}
  \item{\code{e} has type \code{*T} and \code{U} is a numeric type, while \code{T: Sized}; \emph{ptr-addr-cast}}
  \item{\code{e} is an integer and \code{U} is \code{*U\_0}, while \code{U\_0: Sized}; \emph{addr-ptr-cast}}
  \item{\code{e} has type \code{\&[T; n]} and \code{U} is \code{*const T}; \emph{array-ptr-cast}}
  \item{\code{e} is a function pointer type and \code{U} has type \code{*T}, while \code{T: Sized}; \emph{fptr-ptr-cast}}
  \item{\code{e} is a function pointer type and \code{U} is an integer; \emph{fptr-addr-cast}}
\end{itemize}

\subsection*{\code{transmute}}

\code{as} only allows safe casting, and will for example reject an attempt to cast four bytes into a \code{u32}:

\begin{rustc}
let a = [0u8, 0u8, 0u8, 0u8];

let b = a as u32; // four eights makes 32
\end{rustc}

This errors with:

\begin{verbatim}
error: non-scalar cast: `[u8; 4]` as `u32`
let b = a as u32; // four eights makes 32
        ^~~~~~~~
\end{verbatim}

This is a 'non-scalar cast' because we have multiple values here: the four elements of the array. These kinds of casts are very 
dangerous, because they make assumptions about the way that multiple underlying structures are implemented. For this, we need 
something more dangerous.

\blank

The \code{transmute} function is provided by a compiler intrinsic (see \nameref{sec:nightly_intrinsics}), and what it does is very 
simple, but very scary. It tells Rust to treat a value of one type as though it were another type. It does this regardless of the 
typechecking system, and completely trusts you.

\blank

In our previous example, we know that an array of four \code{u8}s represents a \code{u32} properly, and so we want to do the cast. 
Using \code{transmute} instead of \code{as}, Rust lets us:

\begin{rustc}
use std::mem;

unsafe {
    let a = [0u8, 0u8, 0u8, 0u8];

    let b = mem::transmute::<[u8; 4], u32>(a);
}
\end{rustc}

We have to wrap the operation in an \code{unsafe} block for this to compile successfully. Technically, only the \code{mem::transmute} call 
itself needs to be in the block, but it's nice in this case to enclose everything related, so you know where to look. In this case, the 
details about \code{a} are also important, and so they're in the block. You'll see code in either style, sometimes the context is too far away, 
and wrapping all of the code in \code{unsafe} isn't a great idea.

\blank

While \code{transmute} does very little checking, it will at least make sure that the types are the same size. This errors:

\begin{rustc}
use std::mem;

unsafe {
    let a = [0u8, 0u8, 0u8, 0u8];

    let b = mem::transmute::<[u8; 4], u64>(a);
}
\end{rustc}

with:

\begin{verbatim}
error: transmute called with differently sized types: [u8; 4] (32 bits) to u64
(64 bits)
\end{verbatim}

Other than that, you're on your own!


\subsection{Associated Types}
\label{sec:syntax_associatedTypes}
Associated types are a powerful part of Rust's type system. They're related to the idea of a 'type family', in other words, grouping 
multiple types together. That description is a bit abstract, so let's dive right into an example. If you want to write a \code{Graph} 
trait, you have two types to be generic over: the node type and the edge type. So you might write a trait, \code{Graph<N, E>}, that
looks like this:

\begin{rustc}
trait Graph<N, E> {
    fn has_edge(&self, &N, &N) -> bool;
    fn edges(&self, &N) -> Vec<E>;
    // etc
}
\end{rustc}

While this sort of works, it ends up being awkward. For example, any function that wants to take a \code{Graph} as a parameter now also 
needs to be generic over the \code{N}ode and \code{E}dge types too:

\begin{rustc}
fn distance<N, E, G: Graph<N, E>>(graph: &G, start: &N, end: &N) -> u32 { ... }
\end{rustc}

Our distance calculation works regardless of our \code{Edge} type, so the \code{E} stuff in this signature is a distraction.

\blank

What we really want to say is that a certain \code{E}dge and \code{N}ode type come together to form each kind of \code{Graph}. We can 
do that with associated types:

\begin{rustc}
trait Graph {
    type N;
    type E;

    fn has_edge(&self, &Self::N, &Self::N) -> bool;
    fn edges(&self, &Self::N) -> Vec<Self::E>;
    // etc
}
\end{rustc}

Now, our clients can be abstract over a given \code{Graph}:

\begin{rustc}
fn distance<G: Graph>(graph: &G, start: &G::N, end: &G::N) -> u32 { ... }
\end{rustc}

No need to deal with the \code{E}dge type here!

\blank

Let's go over all this in more detail.

\subsection*{Defining associated types}

Let's build that \code{Graph} trait. Here's the definition:

\begin{rustc}
trait Graph {
    type N;
    type E;

    fn has_edge(&self, &Self::N, &Self::N) -> bool;
    fn edges(&self, &Self::N) -> Vec<Self::E>;
}
\end{rustc}

Simple enough. Associated types use the \code{type} keyword, and go inside the body of the trait, with the functions.

\blank

These \code{type} declarations can have all the same thing as functions do. For example, if we wanted our \code{N} type to implement 
\code{Display}, so we can print the nodes out, we could do this:

\begin{rustc}
use std::fmt;

trait Graph {
    type N: fmt::Display;
    type E;

    fn has_edge(&self, &Self::N, &Self::N) -> bool;
    fn edges(&self, &Self::N) -> Vec<Self::E>;
}
\end{rustc}

\subsection*{Implementing associated types}

Just like any trait, traits that use associated types use the \code{impl} keyword to provide implementations. Here's a simple 
implementation of Graph:

\begin{rustc}
struct Node;

struct Edge;

struct MyGraph;

impl Graph for MyGraph {
    type N = Node;
    type E = Edge;

    fn has_edge(&self, n1: &Node, n2: &Node) -> bool {
        true
    }

    fn edges(&self, n: &Node) -> Vec<Edge> {
        Vec::new()
    }
}
\end{rustc}

This silly implementation always returns \code{true} and an empty \code{Vec<Edge>}, but it gives you an idea of how to implement 
this kind of thing. We first need three \struct s, one for the graph, one for the node, and one for the edge. If it made more sense 
to use a different type, that would work as well, we're going to use \struct s for all three here.

\blank

Next is the \code{impl} line, which is an implementation like any other trait.

\blank

From here, we use \code{=} to define our associated types. The name the trait uses goes on the left of the \code{=}, and the concrete 
type we're \code{impl}ementing this for goes on the right. Finally, we use the concrete types in our function declarations.

\subsection*{Trait objects with associated types}

There's one more bit of syntax we should talk about: trait objects. If you try to create a trait object from an associated type, like this:

\begin{rustc}
let graph = MyGraph;
let obj = Box::new(graph) as Box<Graph>;
\end{rustc}

You'll get two errors:

\begin{verbatim}
error: the value of the associated type `E` (from the trait `main::Graph`) must
be specified [E0191]
let obj = Box::new(graph) as Box<Graph>;
          ^~~~~~~~~~~~~~~~~~~~~~~~~~~~~
24:44 error: the value of the associated type `N` (from the trait
`main::Graph`) must be specified [E0191]
let obj = Box::new(graph) as Box<Graph>;
          ^~~~~~~~~~~~~~~~~~~~~~~~~~~~~
\end{verbatim}

We can't create a trait object like this, because we don't know the associated types. Instead, we can write this:

\begin{rustc}
let graph = MyGraph;
let obj = Box::new(graph) as Box<Graph<N=Node, E=Edge>>;
\end{rustc}

The \code{N=Node} syntax allows us to provide a concrete type, \code{Node}, for the \code{N} type parameter. Same with \code{E=Edge}. 
If we didn't provide this constraint, we couldn't be sure which \code{impl} to match this trait object to.


\subsection{Unsized Types}
\label{sec:syntax_unsizedTypes}
Most types have a particular size, in bytes, that is knowable at compile time. For example, an \itt\ is thirty-two bits big, or four bytes. 
However, there are some types which are useful to express, but do not have a defined size. These are called 'unsized' or 'dynamically sized' 
types. One example is \code{[T]}. This type represents a certain number of \code{T} in sequence. But we don't know how many there are, so 
the size is not known.

\blank

Rust understands a few of these types, but they have some restrictions. There are three:

\begin{enumerate}
  \item{We can only manipulate an instance of an unsized type via a pointer. An \code{\&[T]} works fine, but a \code{[T]} does not.}
  \item{Variables and arguments cannot have dynamically sized types.}
  \item{Only the last field in a \struct\ may have a dynamically sized type; the other fields must not. Enum variants must not have 
      dynamically sized types as data.}
\end{enumerate}

So why bother? Well, because \code{[T]} can only be used behind a pointer, if we didn't have language support for unsized types, 
it would be impossible to write this:

\begin{rustc}
impl Foo for str {
\end{rustc}

or

\begin{rustc}
impl<T> Foo for [T] {
\end{rustc}

Instead, you would have to write:

\begin{rustc}
impl Foo for &str {
\end{rustc}

Meaning, this implementation would only work for references (see \nameref{sec:syntax_referencesBorrowing}), and not other types of pointers. 
With the \code{impl for str}, all pointers, including (at some point, there are some bugs to fix first) user-defined custom smart pointers, 
can use this \code{impl}.

\subsection*{?Sized}

If you want to write a function that accepts a dynamically sized type, you can use the special bound, \code{?Sized}:

\begin{rustc}
struct Foo<T: ?Sized> {
    f: T,
}
\end{rustc}

This \code{?}, read as \enquote{T may be \code{Sized}}, means that this bound is special: it lets us match more kinds, not less. It's almost 
like every \code{T} implicitly has \code{T: Sized}, and the \code{?} undoes this default.


\subsection{Operators and Overloading}
\label{sec:syntax_operatorsAndOverloading}
Rust allows for a limited form of operator overloading. There are certain operators that are able to be overloaded. To support a particular 
operator between types, there's a specific trait that you can implement, which then overloads the operator.

\blank

For example, the \code{+} operator can be overloaded with the \code{Add} trait:

\begin{rustc}
use std::ops::Add;

#[derive(Debug)]
struct Point {
    x: i32,
    y: i32,
}

impl Add for Point {
    type Output = Point;

    fn add(self, other: Point) -> Point {
        Point { x: self.x + other.x, y: self.y + other.y }
    }
}

fn main() {
    let p1 = Point { x: 1, y: 0 };
    let p2 = Point { x: 2, y: 3 };

    let p3 = p1 + p2;

    println!("{:?}", p3);
}
\end{rustc}

In \code{main}, we can use \code{+} on our two \code{Point}s, since we've implemented \code{Add<Output=Point>} for \code{Point}.

\blank

There are a number of operators that can be overloaded this way, and all of their associated traits live in the 
\href{https://doc.rust-lang.org/std/ops/}{std::ops} module. Check out its documentation for the full list.

\blank

Implementing these traits follows a pattern. Let's look at \href{https://doc.rust-lang.org/std/ops/trait.Add.html}{Add} in more detail:

\begin{rustc}
pub trait Add<RHS = Self> {
    type Output;

    fn add(self, rhs: RHS) -> Self::Output;
}
\end{rustc}

There's three types in total involved here: the type you \code{impl Add} for, \code{RHS}, which defaults to \code{Self}, and \code{Output}. 
For an expression \code{let z = x + y}, \x\ is the \code{Self} type, \y\ is the \code{RHS}, and \z\ is the \code{Self::Output} type.

\begin{rustc}
impl Add<i32> for Point {
    type Output = f64;

    fn add(self, rhs: i32) -> f64 {
        // add an i32 to a Point and get an f64
    }
}
\end{rustc}

will let you do this:

\begin{rustc}
let p: Point = // ...
let x: f64 = p + 2i32;
\end{rustc}

\subsubsection*{Using operator traits in generic structs}

Now that we know how operator traits are defined, we can define our \code{HasArea} trait and \code{Square} struct from the 
traits chapter (see \nameref{sec:syntax_traits}) more generically:

\begin{rustc}
use std::ops::Mul;

trait HasArea<T> {
    fn area(&self) -> T;
}

struct Square<T> {
    x: T,
    y: T,
    side: T,
}

impl<T> HasArea<T> for Square<T>
        where T: Mul<Output=T> + Copy {
    fn area(&self) -> T {
        self.side * self.side
    }
}

fn main() {
    let s = Square {
        x: 0.0f64,
        y: 0.0f64,
        side: 12.0f64,
    };

    println!("Area of s: {}", s.area());
}
\end{rustc}

For \code{HasArea} and \code{Square}, we declare a type parameter \code{T} and replace \code{f64} with it. The \code{impl} needs more 
involved modifications:

\begin{rustc}
impl<T> HasArea<T> for Square<T>
        where T: Mul<Output=T> + Copy { ... }
\end{rustc}

The \code{area} method requires that we can multiply the sides, so we declare that type \code{T} must implement \code{std::ops::Mul}. 
Like \code{Add}, mentioned above, \code{Mul} itself takes an \code{Output} parameter: since we know that numbers don't change type when 
multiplied, we also set it to \code{T}. \code{T} must also support copying, so Rust doesn't try to move \code{self.side} into the return value.


\subsection{'Deref' coercions}
\label{sec:syntax_derefCoercions}
The standard library provides a special trait, \href{https://doc.rust-lang.org/std/ops/trait.Deref.html}{Deref}. It's normally used to 
overload \code{*}, the dereference operator:

\begin{rustc}
use std::ops::Deref;

struct DerefExample<T> {
    value: T,
}

impl<T> Deref for DerefExample<T> {
    type Target = T;

    fn deref(&self) -> &T {
        &self.value
    }
}

fn main() {
    let x = DerefExample { value: 'a' };
    assert_eq!('a', *x);
}
\end{rustc}

This is useful for writing custom pointer types. However, there's a language feature related to \code{Deref}: 'deref coercions'. Here's the 
rule: If you have a type \code{U}, and it implements \code{Deref<Target=T>}, values of \code{\&U} will automatically coerce to a \code{\&T}. 
Here's an example:

\begin{rustc}
fn foo(s: &str) {
    // borrow a string for a second
}

// String implements Deref<Target=str>
let owned = "Hello".to_string();

// therefore, this works:
foo(&owned);
\end{rustc}

Using an ampersand in front of a value takes a reference to it. So \code{owned} is a \String, \code{\&owned} is an \code{\&String}, and since 
\code{impl Deref<Target=str> for String}, \code{\&String} will deref to \code{\&str}, which \code{foo()} takes.

\blank

That's it. This rule is one of the only places in which Rust does an automatic conversion for you, but it adds a lot of flexibility. For 
example, the \code{Rc<T>} type implements \code{Deref<Target=T>}, so this works:

\begin{rustc}
use std::rc::Rc;

fn foo(s: &str) {
    // borrow a string for a second
}

// String implements Deref<Target=str>
let owned = "Hello".to_string();
let counted = Rc::new(owned);

// therefore, this works:
foo(&counted);
\end{rustc}

All we've done is wrap our \String\ in an \code{Rc<T>}. But we can now pass the \code{Rc<String>} around anywhere we'd have a \String. The 
signature of \code{foo} didn't change, but works just as well with either type. This example has two conversions: \code{Rc<String>} to 
\String\ and then \String\ to \code{\&str}. Rust will do this as many times as possible until the types match.

\blank

Another very common implementation provided by the standard library is:

\begin{rustc}
fn foo(s: &[i32]) {
    // borrow a slice for a second
}

// Vec<T> implements Deref<Target=[T]>
let owned = vec![1, 2, 3];

foo(&owned);
\end{rustc}

Vectors can \code{Deref} to a slice.

\subsection*{Deref and method calls}

\code{Deref} will also kick in when calling a method. Consider the following example.

\begin{rustc}
struct Foo;

impl Foo {
    fn foo(&self) { println!("Foo"); }
}

let f = &&Foo;

f.foo();
\end{rustc}

Even though \code{f} is a \code{\&\&Foo} and \code{foo} takes \code{\&self}, this works. That's because these things are the same:

\begin{rustc}
f.foo();
(&f).foo();
(&&f).foo();
(&&&&&&&&f).foo();
\end{rustc}

A value of type \code{\&\&\&\&\&\&\&\&\&\&\&\&\&\&\&\&Foo} can still have methods defined on \code{Foo} called, because the compiler 
will insert as many \code{*} operations as necessary to get it right. And since it's inserting \code{*}s, that uses \code{Deref}.


\subsection{Macros}
\label{sec:syntax_macros}
By now you've learned about many of the tools Rust provides for abstracting and reusing code. These units of code reuse have a rich 
semantic structure. For example, functions have a type signature, type parameters have trait bounds, and overloaded functions must 
belong to a particular trait.

\blank

This structure means that Rust's core abstractions have powerful compile-time correctness checking. But this comes at the price of 
flexibility. If you visually identify a pattern of repeated code, you may find it's difficult or cumbersome to express that pattern 
as a generic function, a trait, or anything else within Rust's semantics.

\blank

Macros allow us to abstract at a syntactic level. A macro invocation is shorthand for an \enquote{expanded} syntactic form. This 
expansion happens early in compilation, before any static checking. As a result, macros can capture many patterns of code reuse that 
Rust's core abstractions cannot.

\blank

The drawback is that macro-based code can be harder to understand, because fewer of the built-in rules apply. Like an ordinary 
function, a well-behaved macro can be used without understanding its implementation. However, it can be difficult to design a 
well-behaved macro! Additionally, compiler errors in macro code are harder to interpret, because they describe problems in the 
expanded code, not the source-level form that developers use.

\blank

These drawbacks make macros something of a \enquote{feature of last resort}. That's not to say that macros are bad; they are part of 
Rust because sometimes they're needed for truly concise, well-abstracted code. Just keep this tradeoff in mind.

\subsection*{Defining a macro}

You may have seen the \code{vec!} macro, used to initialize a vector with any number of elements.

\begin{rustc}
let x: Vec<u32> = vec![1, 2, 3];
\end{rustc}

This can't be an ordinary function, because it takes any number of arguments. But we can imagine it as syntactic shorthand for

\begin{rustc}
let x: Vec<u32> = {
    let mut temp_vec = Vec::new();
    temp_vec.push(1);
    temp_vec.push(2);
    temp_vec.push(3);
    temp_vec
};
\end{rustc}

We can implement this shorthand, using a macro:\footnote{The actual definition of \code{vec!} in libcollections differs from the one 
presented here, for reasons of efficiency and reusability.}

\begin{rustc}
macro_rules! vec {
    ( $( $x:expr ),* ) => {
        {
            let mut temp_vec = Vec::new();
            $(
                temp_vec.push($x);
            )*
            temp_vec
        }
    };
}
\end{rustc}

Whoa, that's a lot of new syntax! Let's break it down.

\begin{rustc}
macro_rules! vec { ... }
\end{rustc}

This says we're defining a macro named \code{vec}, much as \code{fn vec} would define a function named \code{vec}. In prose, we informally 
write a macro's name with an exclamation point, e.g. \code{vec!}. The exclamation point is part of the invocation syntax and serves to 
distinguish a macro from an ordinary function.

\myparagraph{Matching}

The macro is defined through a series of rules, which are pattern-matching cases. Above, we had

\begin{rustc}
( $( $x:expr ),* ) => { ... };
\end{rustc}

This is like a \code{match} expression arm, but the matching happens on Rust syntax trees, at compile time. The semicolon is optional
on the last (here, only) case. The \enquote{pattern} on the left-hand side of \code{=>} is known as a 'matcher'. These have 
\href{https://doc.rust-lang.org/reference.html#macros}{their own little grammar} within the language.

\blank

The matcher \code{\$x:expr} will match any Rust expression, binding that syntax tree to the 'metavariable' \code{\$x}. The identifier 
\code{expr} is a 'fragment specifier'; the full possibilities are enumerated later in this chapter. Surrounding the matcher with 
\code{\$(...),*} will match zero or more expressions, separated by commas.

\blank

Aside from the special matcher syntax, any Rust tokens that appear in a matcher must match exactly. For example,

\begin{rustc}
macro_rules! foo {
    (x => $e:expr) => (println!("mode X: {}", $e));
    (y => $e:expr) => (println!("mode Y: {}", $e));
}

fn main() {
    foo!(y => 3);
}
\end{rustc}

will print

\begin{verbatim}
mode Y: 3
\end{verbatim}

With

\begin{rustc}
foo!(z => 3);
\end{rustc}

we get the compiler error

\begin{verbatim}
error: no rules expected the token `z`
\end{verbatim}

\myparagraph{Expansion}

The right-hand side of a macro rule is ordinary Rust syntax, for the most part. But we can splice in bits of syntax captured by the matcher. 
From the original example:

\begin{rustc}
$(
    temp_vec.push($x);
)*
\end{rustc}

Each matched expression \code{\$x} will produce a single \code{push} statement in the macro expansion. The repetition in the expansion 
proceeds in \enquote{lockstep} with repetition in the matcher (more on this in a moment).

\blank

Because \code{\$x} was already declared as matching an expression, we don't repeat \code{:expr} on the right-hand side. Also, we don't 
include a separating comma as part of the repetition operator. Instead, we have a terminating semicolon within the repeated block.

\blank

Another detail: the \code{vec!} macro has \emph{two} pairs of braces on the right-hand side. They are often combined like so:

\begin{rustc}
macro_rules! foo {
    () => {{
        ...
    }}
}
\end{rustc}

The outer braces are part of the syntax of \code{macro\_rules!}. In fact, you can use \code{()} or \code{[]} instead. They simply delimit 
the right-hand side as a whole.

\blank

The inner braces are part of the expanded syntax. Remember, the \code{vec!} macro is used in an expression context. To write an expression 
with multiple statements, including \keylet-bindings, we use a block. If your macro expands to a single expression, you don't need this 
extra layer of braces.

\blank

Note that we never \emph{declared} that the macro produces an expression. In fact, this is not determined until we use the macro as 
an expression. With care, you can write a macro whose expansion works in several contexts. For example, shorthand for a data type could 
be valid as either an expression or a pattern.

\myparagraph{Repetition}

The repetition operator follows two principal rules:

\begin{enumerate}
  \item{\code{\$(...)*} walks through one \enquote{layer} of repetitions, for all of the \code{\$names} it contains, in lockstep, and}
  \item{each \code{\$name} must be under at least as many \code{\$(...)*}s as it was matched against. If it is under more, it'll be 
      duplicated, as appropriate.}
\end{enumerate}

This baroque macro illustrates the duplication of variables from outer repetition levels.

\begin{rustc}
macro_rules! o_O {
    (
        $(
            $x:expr; [ $( $y:expr ),* ]
        );*
    ) => {
        &[ $($( $x + $y ),*),* ]
    }
}

fn main() {
    let a: &[i32]
        = o_O!(10; [1, 2, 3];
               20; [4, 5, 6]);

    assert_eq!(a, [11, 12, 13, 24, 25, 26]);
}
\end{rustc}

That's most of the matcher syntax. These examples use \code{\$(...)*}, which is a \enquote{zero or more} match. Alternatively you can write 
\code{\$(...)+} for a \enquote{one or more} match. Both forms optionally include a separator, which can be any token except \code{+} or 
\code{*}.

\blank

This system is based on \href{https://www.cs.indiana.edu/ftp/techreports/TR206.pdf}{Macro-by-Example} (PDF link).

\subsection*{Hygiene}

Some languages implement macros using simple text substitution, which leads to various problems. For example, this C program prints 
\code{13} instead of the expected \code{25}.

\begin{minted}{c}
#define FIVE_TIMES(x) 5 * x

int main() {
    printf("%d\n", FIVE_TIMES(2 + 3));
    return 0;
}
\end{minted}

After expansion we have \code{5 * 2 + 3}, and multiplication has greater precedence than addition. If you've used C macros a lot, you 
probably know the standard idioms for avoiding this problem, as well as five or six others. In Rust, we don't have to worry about it.

\begin{rustc}
macro_rules! five_times {
    ($x:expr) => (5 * $x);
}

fn main() {
    assert_eq!(25, five_times!(2 + 3));
}
\end{rustc}

The metavariable \code{\$x} is parsed as a single expression node, and keeps its place in the syntax tree even after substitution.

\blank

Another common problem in macro systems is 'variable capture'. Here's a C macro, using 
\href{https://gcc.gnu.org/onlinedocs/gcc/Statement-Exprs.html}{a GNU C extension} to emulate Rust's expression blocks.

\begin{minted}{c}
#define LOG(msg) ({ \
    int state = get_log_state(); \
    if (state > 0) { \
        printf("log(%d): %s\n", state, msg); \
    } \
})
\end{minted}

Here's a simple use case that goes terribly wrong:

\begin{minted}{c}
const char *state = "reticulating splines";
LOG(state)
\end{minted}

This expands to

\begin{minted}{c}
const char *state = "reticulating splines";
{
    int state = get_log_state();
    if (state > 0) {
        printf("log(%d): %s\n", state, state);
    }
}
\end{minted}

The second variable named \code{state} shadows the first one. This is a problem because the print statement should refer to both of them.

\blank

The equivalent Rust macro has the desired behavior.

\begin{rustc}
macro_rules! log {
    ($msg:expr) => {{
        let state: i32 = get_log_state();
        if state > 0 {
            println!("log({}): {}", state, $msg);
        }
    }};
}

fn main() {
    let state: &str = "reticulating splines";
    log!(state);
}
\end{rustc}

This works because Rust has a \href{https://en.wikipedia.org/wiki/Hygienic_macro}{hygienic macro system}. Each macro expansion happens 
in a distinct 'syntax context', and each variable is tagged with the syntax context where it was introduced. It's as though the variable 
\code{state} inside \code{main} is painted a different \enquote{color} from the variable state inside the macro, and therefore they 
don't conflict.

\blank

This also restricts the ability of macros to introduce new bindings at the invocation site. Code such as the following will not work:

\begin{rustc}
macro_rules! foo {
    () => (let x = 3);
}

fn main() {
    foo!();
    println!("{}", x);
}
\end{rustc}

Instead you need to pass the variable name into the invocation, so it's tagged with the right syntax context.

\begin{rustc}
macro_rules! foo {
    ($v:ident) => (let $v = 3);
}

fn main() {
    foo!(x);
    println!("{}", x);
}
\end{rustc}

This holds for \keylet\ bindings and loop labels, but not for \href{https://doc.rust-lang.org/reference.html#items}{items}. So the 
following code does compile:

\begin{rustc}
macro_rules! foo {
    () => (fn x() { });
}

fn main() {
    foo!();
    x();
}
\end{rustc}

\subsection*{Recursive macros}

A macro's expansion can include more macro invocations, including invocations of the very same macro being expanded. These recursive 
macros are useful for processing tree-structured input, as illustrated by this (simplistic) HTML shorthand:

\begin{rustc}
macro_rules! write_html {
    ($w:expr, ) => (());

    ($w:expr, $e:tt) => (write!($w, "{}", $e));

    ($w:expr, $tag:ident [ $($inner:tt)* ] $($rest:tt)*) => {{
        write!($w, "<{}>", stringify!($tag));
        write_html!($w, $($inner)*);
        write!($w, "</{}>", stringify!($tag));
        write_html!($w, $($rest)*);
    }};
}

fn main() {
    use std::fmt::Write;
    let mut out = String::new();

    write_html!(&mut out,
        html[
            head[title["Macros guide"]]
            body[h1["Macros are the best!"]]
        ]);

    assert_eq!(out,
        "<html><head><title>Macros guide</title></head>\
         <body><h1>Macros are the best!</h1></body></html>");
}
\end{rustc}

\subsection*{Debugging macro code}

To see the results of expanding macros, run \code{rustc --pretty expanded}. The output represents a whole crate, so you can also 
feed it back in to \code{rustc}, which will sometimes produce better error messages than the original compilation. Note that the 
\code{--pretty expanded} output may have a different meaning if multiple variables of the same name (but different syntax contexts) 
are in play in the same scope. In this case \code{--pretty expanded,hygiene} will tell you about the syntax contexts.

\blank

\code{rustc} provides two syntax extensions that help with macro debugging. For now, they are unstable and require feature gates.

\begin{itemize}
  \item{\code{log\_syntax!(...)} will print its arguments to standard output, at compile time, and \enquote{expand} to nothing.}
  \item{\code{trace\_macros!(true)} will enable a compiler message every time a macro is expanded. Use \code{trace\_macros!(false)} 
      later in expansion to turn it off.}
\end{itemize}

\subsection*{Syntactic requirements}

Even when Rust code contains un-expanded macros, it can be parsed as a full syntax tree (see \nameref{sec:gloss_syntaxtree}). This 
property can be very useful for editors and other tools that process code. It also has a few consequences for the design of Rust's macro 
system.

\blank

One consequence is that Rust must determine, when it parses a macro invocation, whether the macro stands in for

\begin{itemize}
  \item{zero or more items,}
  \item{zero or more methods,}
  \item{an expression,}
  \item{a statement, or}
  \item{a pattern.}
\end{itemize}

A macro invocation within a block could stand for some items, or for an expression / statement. Rust uses a simple rule to resolve 
this ambiguity. A macro invocation that stands for items must be either

\begin{itemize}
  \item{delimited by curly braces, e.g. \code{foo! \{ ... \}}, or}
  \item{terminated by a semicolon, e.g. \code{foo!(...);}}
\end{itemize}

Another consequence of pre-expansion parsing is that the macro invocation must consist of valid Rust tokens. Furthermore, parentheses, 
brackets, and braces must be balanced within a macro invocation. For example, \code{foo!([)} is forbidden. This allows Rust to know 
where the macro invocation ends.

\blank

More formally, the macro invocation body must be a sequence of 'token trees'. A token tree is defined recursively as either

\begin{itemize}
  \item{a sequence of token trees surrounded by matching \code{()}, \code{[]}, or \code{\{\}}, or}
  \item{any other single token.}
\end{itemize}

Within a matcher, each metavariable has a 'fragment specifier', identifying which syntactic form it matches.

\begin{itemize}
  \item{\code{ident}: an identifier. Examples: \x; \code{foo}.}
  \item{\code{path}: a qualified name. Example: \code{T::SpecialA}.}
  \item{\code{expr}: an expression. Examples: \code{2 + 2}; \code{if true \{ 1 \} else \{ 2 \}}; \code{f(42)}.}
  \item{\code{ty}: a type. Examples: \itt; \code{Vec<(char, String)>}; \code{\&T}.}
  \item{\code{pat}: a pattern. Examples: \code{Some(t)}; \code{(17, 'a')}; \code{\_}.}
  \item{\code{stmt}: a single statement. Example: \code{let x = 3}.}
  \item{\code{block}: a brace-delimited sequence of statements. Example: \code{\{ log(error, "hi"); return 12; \}}.}
  \item{\code{item}: an \href{https://doc.rust-lang.org/reference.html\#items}{item}. Examples: \code{fn foo() \{ \}}; \code{struct Bar;}.}
  \item{\code{meta}: a "meta item", as found in attributes. Example: \code{cfg(target\_os = "windows")}.}
  \item{\code{tt}: a single token tree.}
\end{itemize}

There are additional rules regarding the next token after a metavariable:

\begin{itemize}
  \item{\code{expr} and \code{stmt} variables may only be followed by one of: \code{=> , ;}}
  \item{\code{ty} and \code{path} variables may only be followed by one of: \code{=> , = | ; : > [ \{ as where}}
  \item{\code{pat} variables may only be followed by one of: \code{=> , = | if in}}
  \item{Other variables may be followed by any token.}
\end{itemize}

These rules provide some flexibility for Rust's syntax to evolve without breaking existing macros.

\blank

The macro system does not deal with parse ambiguity at all. For example, the grammar \code{\$(\$i:ident)* \$e:expr} will always fail to 
parse, because the parser would be forced to choose between parsing \code{\$i} and parsing \code{\$e}. Changing the invocation syntax to 
put a distinctive token in front can solve the problem. In this case, you can write \code{\$(I \$i:ident)* E \$e:expr}.

\subsection*{Scoping and macro import/export}

Macros are expanded at an early stage in compilation, before name resolution. One downside is that scoping works differently for 
macros, compared to other constructs in the language.

\blank

Definition and expansion of macros both happen in a single depth-first, lexical-order traversal of a crate's source. So a macro 
defined at module scope is visible to any subsequent code in the same module, which includes the body of any subsequent child \code{mod} items.

\blank

A macro defined within the body of a single \code{fn}, or anywhere else not at module scope, is visible only within that item.

\blank

If a module has the \code{macro\_use} attribute, its macros are also visible in its parent module after the child's \code{mod} item. If 
the parent also has \code{macro\_use} then the macros will be visible in the grandparent after the parent's \code{mod} item, and so forth.

\blank

The \code{macro\_use} attribute can also appear on \code{extern crate}. In this context it controls which macros are loaded from the external 
crate, e.g.

\begin{rustc}
#[macro_use(foo, bar)]
extern crate baz;
\end{rustc}

If the attribute is given simply as \code{\#[macro\_use]}, all macros are loaded. If there is no \code{\#[macro\_use]} attribute then no 
macros are loaded. Only macros defined with the \code{\#[macro\_export]} attribute may be loaded.

\blank

To load a crate's macros without linking it into the output, use \code{\#[no\_link]} as well.

\blank

An example:

\begin{rustc}
macro_rules! m1 { () => (()) }

// visible here: m1

mod foo {
    // visible here: m1

    #[macro_export]
    macro_rules! m2 { () => (()) }

    // visible here: m1, m2
}

// visible here: m1

macro_rules! m3 { () => (()) }

// visible here: m1, m3

#[macro_use]
mod bar {
    // visible here: m1, m3

    macro_rules! m4 { () => (()) }

    // visible here: m1, m3, m4
}

// visible here: m1, m3, m4
\end{rustc}

When this library is loaded with \code{\#[macro\_use] extern crate}, only \code{m2} will be imported.

\blank

The Rust Reference has a \href{https://doc.rust-lang.org/reference.html\#macro-related-attributes}{listing of macro-related attributes}.

\subsection*{The variable \code{\$crate}}

A further difficulty occurs when a macro is used in multiple crates. Say that \code{mylib} defines

\begin{rustc}
pub fn increment(x: u32) -> u32 {
    x + 1
}

#[macro_export]
macro_rules! inc_a {
    ($x:expr) => ( ::increment($x) )
}

#[macro_export]
macro_rules! inc_b {
    ($x:expr) => ( ::mylib::increment($x) )
}
\end{rustc}

\code{inc\_a} only works within \code{mylib}, while \code{inc\_b} only works outside the library. Furthermore, \code{inc\_b} will break 
if the user imports \code{mylib} under another name.

\blank

Rust does not (yet) have a hygiene system for crate references, but it does provide a simple workaround for this problem. Within a 
macro imported from a crate named foo, the special macro variable \code{\$crate} will expand to \code{::foo}. By contrast, when a macro 
is defined and then used in the same crate, \code{\$crate} will expand to nothing. This means we can write

\begin{rustc}
#[macro_export]
macro_rules! inc {
    ($x:expr) => ( $crate::increment($x) )
}
\end{rustc}

to define a single macro that works both inside and outside our library. The function name will expand to either \code{::increment}
or \code{::mylib::increment}.

\blank

To keep this system simple and correct, \code{\#[macro\_use] extern crate ...} may only appear at the root of your crate, not 
inside \code{mod}.

\subsection*{The deep end}

The introductory chapter mentioned recursive macros, but it did not give the full story. Recursive macros are useful for another 
reason: Each recursive invocation gives you another opportunity to pattern-match the macro's arguments.

\blank

As an extreme example, it is possible, though hardly advisable, to implement the 
\href{https://esolangs.org/wiki/Bitwise_Cyclic_Tag}{Bitwise Cyclic Tag} automaton within Rust's macro system.

\begin{rustc}
macro_rules! bct {
    // cmd 0:  d ... => ...
    (0, $($ps:tt),* ; $_d:tt)
        => (bct!($($ps),*, 0 ; ));
    (0, $($ps:tt),* ; $_d:tt, $($ds:tt),*)
        => (bct!($($ps),*, 0 ; $($ds),*));

    // cmd 1p:  1 ... => 1 ... p
    (1, $p:tt, $($ps:tt),* ; 1)
        => (bct!($($ps),*, 1, $p ; 1, $p));
    (1, $p:tt, $($ps:tt),* ; 1, $($ds:tt),*)
        => (bct!($($ps),*, 1, $p ; 1, $($ds),*, $p));

    // cmd 1p:  0 ... => 0 ...
    (1, $p:tt, $($ps:tt),* ; $($ds:tt),*)
        => (bct!($($ps),*, 1, $p ; $($ds),*));

    // halt on empty data string
    ( $($ps:tt),* ; )
        => (());
}
\end{rustc}

Exercise: use macros to reduce duplication in the above definition of the \code{bct!} macro.

\subsection*{Common macros}

Here are some common macros you'll see in Rust code.

\myparagraph{panic!}

This macro causes the current thread to panic. You can give it a message to panic with:

\begin{rustc}
panic!("oh no!");
\end{rustc}

\myparagraph{vec!}

The \code{vec!} macro is used throughout the book, so you've probably seen it already. It creates \code{Vec<T>}s with ease:

\begin{rustc}
let v = vec![1, 2, 3, 4, 5];
\end{rustc}

It also lets you make vectors with repeating values. For example, a hundred zeroes:

\begin{rustc}
let v = vec![0; 100];
\end{rustc}

\myparagraph{assert! and assert\_eq!}

These two macros are used in tests. \code{assert!} takes a boolean. \code{assert\_eq!} takes two values and checks them for equality. 
\code{true} passes, \code{false} \panic s. Like this:

\begin{rustc}
// A-ok!

assert!(true);
assert_eq!(5, 3 + 2);

// nope :(

assert!(5 < 3);
assert_eq!(5, 3);
\end{rustc}

\myparagraph{try!}

\code{try!} is used for error handling. It takes something that can return a \code{Result<T, E>}, and gives \code{T} if it's a \code{Ok<T>},
and returns with the \code{Err(E)} if it's that. Like this:

\begin{rustc}
use std::fs::File;

fn foo() -> std::io::Result<()> {
    let f = try!(File::create("foo.txt"));

    Ok(())
}
\end{rustc}

This is cleaner than doing this:

\begin{rustc}
use std::fs::File;

fn foo() -> std::io::Result<()> {
    let f = File::create("foo.txt");

    let f = match f {
        Ok(t) => t,
        Err(e) => return Err(e),
    };

    Ok(())
}
\end{rustc}

\myparagraph{unreachable!}

This macro is used when you think some code should never execute:

\begin{rustc}
if false {
    unreachable!();
}
\end{rustc}

Sometimes, the compiler may make you have a different branch that you know will never, ever run. In these cases, use this macro, so 
that if you end up wrong, you'll get a \panic\ about it.

\begin{rustc}
let x: Option<i32> = None;

match x {
    Some(_) => unreachable!(),
    None => println!("I know x is None!"),
}
\end{rustc}

\myparagraph{unimplemented!}

The \code{unimplemented!} macro can be used when you're trying to get your functions to typecheck, and don't want to worry about writing 
out the body of the function. One example of this situation is implementing a trait with multiple required methods, where you want to tackle 
one at a time. Define the others as \code{unimplemented!} until you're ready to write them.

\subsection*{Procedural macros}

If Rust's macro system can't do what you need, you may want to write a compiler plugin instead (see \nameref{sec:nightly_compilerPlugins}). 
Compared to \code{macro\_rules!} macros, this is significantly more work, the interfaces are much less stable, and bugs can be much 
harder to track down. In exchange you get the flexibility of running arbitrary Rust code within the compiler. Syntax extension plugins 
are sometimes called 'procedural macros' for this reason.

