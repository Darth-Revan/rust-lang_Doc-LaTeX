So you've learned how to write some Rust code. But there's a difference between writing any Rust code and writing good Rust code.

\blank

This chapter consists of relatively independent tutorials which show you how to take your Rust to the next level. Common patterns 
and standard library features will be introduced. Read these sections in any order of your choosing.

\section{The Stack and the Heap}
\label{sec:effective_stackAndHeap}
As a systems language, Rust operates at a low level. If you're coming from a high-level language, there are some aspects of systems
programming that you may not be familiar with. The most important one is how memory works, with a stack and a heap. If you're familiar
with how C-like languages use stack allocation, this chapter will be a refresher. If you're not, you'll learn about this more general
concept, but with a Rust-y focus.

\blank

As with most things, when learning about them, we'll use a simplified model to start. This lets you get a handle on the basics, without
getting bogged down with details which are, for now, irrelevant. The examples we'll use aren't 100\% accurate, but are representative for
the level we're trying to learn at right now. Once you have the basics down, learning more about how allocators are implemented, virtual
memory, and other advanced topics will reveal the leaks in this particular abstraction.

\subsubsection*{Memory management}

These two terms are about memory management. The stack and the heap are abstractions that help you determine when to allocate and
deallocate memory.

\blank

Here's a high-level comparison:

\blank

The stack is very fast, and is where memory is allocated in Rust by default. But the allocation is local to a function call, and
is limited in size. The heap, on the other hand, is slower, and is explicitly allocated by your program. But it's effectively
unlimited in size, and is globally accessible.

\subsubsection*{The Stack}

Let's talk about this Rust program:

\begin{rustc}
fn main() {
    let x = 42;
}
\end{rustc}

This program has one variable binding, \x. This memory needs to be allocated from somewhere. Rust 'stack allocates' by default, which means that basic values 'go on the stack'. What does that mean?

\blank

Well, when a function gets called, some memory gets allocated for all of its local variables and some other information. This is called a 
'stack frame', and for the purpose of this tutorial, we're going to ignore the extra information and only consider the local variables 
we're allocating. So in this case, when \code{main()} is run, we'll allocate a single 32-bit integer for our stack frame. This is 
automatically handled for you, as you can see; we didn't have to write any special Rust code or anything.

\blank

When the function exits, its stack frame gets deallocated. This happens automatically as well.

\blank

That's all there is for this simple program. The key thing to understand here is that stack allocation is very, very fast. Since we know 
all the local variables we have ahead of time, we can grab the memory all at once. And since we'll throw them all away at the same time as 
well, we can get rid of it very fast too.

\blank

The downside is that we can't keep values around if we need them for longer than a single function. We also haven't talked about what 
the word, 'stack', means. To do that, we need a slightly more complicated example:

\begin{rustc}
fn foo() {
    let y = 5;
    let z = 100;
}

fn main() {
    let x = 42;

    foo();
}
\end{rustc}

This program has three variables total: two in \code{foo()}, one in \code{main()}. Just as before, when \code{main()} is called, a single 
integer is allocated for its stack frame. But before we can show what happens when \code{foo()} is called, we need to visualize what's 
going on with memory. Your operating system presents a view of memory to your program that's pretty simple: a huge list of addresses, from 
0 to a large number, representing how much RAM your computer has. For example, if you have a gigabyte of RAM, your addresses go from \code{0} 
to \code{1,073,741,823}. That number comes from $2^{30}$, the number of bytes in a gigabyte.\footnote{'Gigabyte' can mean two things: 
$10^{9}$, or $2^{30}$. The SI standard resolved this by stating that 'gigabyte' is $10^{9}$, and 'gibibyte' is $2^{30}$. However, very 
few people use this terminology, and rely on context to differentiate. We follow in that tradition here.}

\blank

This memory is kind of like a giant array: addresses start at zero and go up to the final number. So here's a diagram of our first 
stack frame:

\begin{table}[H]
  \begin{tabular}{|l|l|l|}
    \hline
    \textbf{Address} & \textbf{Name} & \textbf{Value} \\
    \hline
    0 & x & 42 \\
    \hline
  \end{tabular}
\end{table}

We've got \x\ located at address \code{0}, with the value \code{42}.

When \code{foo()} is called, a new stack frame is allocated:

\begin{table}[H]
  \begin{tabular}{|l|l|l|}
    \hline
    \textbf{Address} & \textbf{Name} & \textbf{Value} \\
    \hline
    2 & z & 100 \\
    \hline
    1 & y & 5 \\
    \hline
    0 & x & 42 \\
    \hline
  \end{tabular}
\end{table}

Because \code{0} was taken by the first frame, \code{1} and \code{2} are used for \code{foo()}'s stack frame. It grows upward, the more 
functions we call.

\blank

There are some important things we have to take note of here. The numbers 0, 1, and 2 are all solely for illustrative purposes, and 
bear no relationship to the address values the computer will use in reality. In particular, the series of addresses are in reality 
going to be separated by some number of bytes that separate each address, and that separation may even exceed the size of the value 
being stored.

\blank

After \code{foo()} is over, its frame is deallocated:

\begin{table}[H]
  \begin{tabular}{|l|l|l|}
    \hline
    \textbf{Address} & \textbf{Name} & \textbf{Value} \\
    \hline
    0 & x & 42 \\
    \hline
  \end{tabular}
\end{table}

And then, after \code{main()}, even this last value goes away. Easy!

\blank

It's called a 'stack' because it works like a stack of dinner plates: the first plate you put down is the last plate to pick back up. 
Stacks are sometimes called 'last in, first out queues' for this reason, as the last value you put on the stack is the first one you 
retrieve from it.

\blank

Let's try a three-deep example:

\begin{rustc}
fn italic() {
    let i = 6;
}

fn bold() {
    let a = 5;
    let b = 100;
    let c = 1;

    italic();
}

fn main() {
    let x = 42;

    bold();
}
\end{rustc}

We have some kooky function names to make the diagrams clearer.

\blank

Okay, first, we call \code{main()}:

\begin{table}[H]
  \begin{tabular}{|l|l|l|}
    \hline
    \textbf{Address} & \textbf{Name} & \textbf{Value} \\
    \hline
    0 & x & 42 \\
    \hline
  \end{tabular}
\end{table}

Next up, \code{main()} calls \code{bold()}:

\begin{table}[H]
  \begin{tabular}{|l|l|l|}
    \hline
    \textbf{Address} & \textbf{Name} & \textbf{Value} \\
    \hline
    \textbf{3} & \textbf{c} & \textbf{1} \\
    \hline
    \textbf{2} & \textbf{b} & \textbf{100} \\
    \hline
    \textbf{1} & \textbf{a} & \textbf{5} \\
    \hline
    0 & x & 42 \\
    \hline
  \end{tabular}
\end{table}

And then \code{bold()} calls \code{italic()}:

\begin{table}[H]
  \begin{tabular}{|l|l|l|}
    \hline
    \textbf{Address} & \textbf{Name} & \textbf{Value} \\
    \hline
    \textit{4} & \textit{i} & \textit{6} \\
    \hline
    \textbf{3} & \textbf{c} & \textbf{1} \\
    \hline
    \textbf{2} & \textbf{b} & \textbf{100} \\
    \hline
    \textbf{1} & \textbf{a} & \textbf{5} \\
    \hline
    0 & x & 42 \\
    \hline
  \end{tabular}
\end{table}

Whew! Our stack is growing tall.

\blank

After \code{italic()} is over, its frame is deallocated, leaving only \code{bold()} and \code{main()}:

\begin{table}[H]
  \begin{tabular}{|l|l|l|}
    \hline
    \textbf{Address} & \textbf{Name} & \textbf{Value} \\
    \hline
    \textbf{3} & \textbf{c} & \textbf{1} \\
    \hline
    \textbf{2} & \textbf{b} & \textbf{100} \\
    \hline
    \textbf{1} & \textbf{a} & \textbf{5} \\
    \hline
    0 & x & 42 \\
    \hline
  \end{tabular}
\end{table}

And then \code{bold()} ends, leaving only \code{main()}:

\begin{table}[H]
  \begin{tabular}{|l|l|l|}
    \hline
    \textbf{Address} & \textbf{Name} & \textbf{Value} \\
    \hline
    0 & x & 42 \\
    \hline
  \end{tabular}
\end{table}

And then we're done. Getting the hang of it? It's like piling up dishes: you add to the top, you take away from the top.


\subsubsection*{The Heap}

Now, this works pretty well, but not everything can work like this. Sometimes, you need to pass some memory between different 
functions, or keep it alive for longer than a single function's execution. For this, we can use the heap.

\blank

In Rust, you can allocate memory on the heap with the \href{https://doc.rust-lang.org/std/boxed/}{Box<T> type}. Here's an example:

\begin{rustc}
fn main() {
    let x = Box::new(5);
    let y = 42;
}
\end{rustc}

Here's what happens in memory when \code{main()} is called:

\begin{table}[H]
  \begin{tabular}{|l|l|l|}
    \hline
    \textbf{Address} & \textbf{Name} & \textbf{Value} \\
    \hline
    1 & y & 42 \\
    \hline
    0 & x & ?????? \\
    \hline
  \end{tabular}
\end{table}

We allocate space for two variables on the stack. \y\ is \code{42}, as it always has been, but what about \x? Well, \x\ is a 
\code{Box<i32>}, and boxes allocate memory on the heap. The actual value of the box is a structure which has a pointer to 'the heap'. 
When we start executing the function, and \code{Box::new()} is called, it allocates some memory for the heap, and puts \code{5} there. 
The memory now looks like this:

\begin{table}[H]
  \begin{tabular}{|l|l|l|}
    \hline
    \textbf{Address} & \textbf{Name} & \textbf{Value} \\
    \hline
    ($2^{30}$) - 1 & & 5 \\
    \hline
    \ldots & \ldots & \ldots \\
    \hline
    1 & y & 42 \\
    \hline
    0 & x & 42 \\
    \hline
  \end{tabular}
\end{table}

% TODO make raw pointer a hyperref
We have ($2^{30}$) - 1 addresses in our hypothetical computer with 1GB of RAM. And since our stack grows from zero, the easiest place 
to allocate memory is from the other end. So our first value is at the highest place in memory. And the value of the struct at \x\ has a 
raw pointer to the place we've allocated on the heap, so the value of \x\ is ($2^{30}$) - 1, the memory location we've asked for.

\blank

We haven't really talked too much about what it actually means to allocate and deallocate memory in these contexts. Getting into very 
deep detail is out of the scope of this tutorial, but what's important to point out here is that the heap isn't a stack that grows from 
the opposite end. We'll have an example of this later in the book, but because the heap can be allocated and freed in any order, it can 
end up with 'holes'. Here's a diagram of the memory layout of a program which has been running for a while now:

\begin{table}[H]
  \begin{tabular}{|l|l|l|}
    \hline
    \textbf{Address} & \textbf{Name} & \textbf{Value} \\
    \hline
    ($2^{30}$) - 1 & & 5 \\
    \hline
    ($2^{30}$) - 2 & & \\
    \hline
    ($2^{30}$) - 3 & & \\
    \hline
    ($2^{30}$) - 4 & & 40 \\
    \hline
    \ldots & \ldots & \ldots \\
    \hline
    3 & y & $\rightarrow(2^{30})$-4 \\
    \hline
    2 & y & 42 \\
    \hline
    1 & y & 42 \\
    \hline
    0 & x & $\rightarrow(2^{30})$-1 \\
    \hline
  \end{tabular}
\end{table}

In this case, we've allocated four things on the heap, but deallocated two of them. There's a gap between ($2^{30}$) - 1 and ($2^{30}$) - 4 
which isn't currently being used. The specific details of how and why this happens depends on what kind of strategy you use to manage the 
heap. Different programs can use different 'memory allocators', which are libraries that manage this for you. Rust programs use 
\href{http://www.canonware.com/jemalloc/}{jemalloc} for this purpose.

\blank

Anyway, back to our example. Since this memory is on the heap, it can stay alive longer than the function which allocates the box. In 
this case, however, it doesn't.\footnote{We can make the memory live longer by transferring ownership, sometimes called 'moving out of 
the box'. More complex examples will be covered later.} When the function is over, we need to free the stack frame for \code{main()}. 
\code{Box<T>}, though, has a trick up its sleeve: Drop (see \nameref{sec:syntax_drop}). The implementation of \code{Drop} for \code{Box}
deallocates the memory that was allocated when it was created. Great! So when \x\ goes away, it first frees the memory allocated on the 
heap:

\begin{table}[H]
  \begin{tabular}{|l|l|l|}
    \hline
    \textbf{Address} & \textbf{Name} & \textbf{Value} \\
    \hline
    1 & y & 42 \\
    \hline
    0 & x & ?????? \\
    \hline
  \end{tabular}
\end{table}

And then the stack frame goes away, freeing all of our memory.

\subsubsection*{Arguments and borrowing}

We've got some basic examples with the stack and the heap going, but what about function arguments and borrowing? Here's a small 
Rust program:

\begin{rustc}
fn foo(i: &i32) {
    let z = 42;
}

fn main() {
    let x = 5;
    let y = &x;

    foo(y);
}
\end{rustc}

When we enter \code{main()}, memory looks like this:

\begin{table}[H]
  \begin{tabular}{|l|l|l|}
    \hline
    \textbf{Address} & \textbf{Name} & \textbf{Value} \\
    \hline
    1 & y & $\rightarrow$ 0 \\
    \hline
    0 & x & 5 \\
    \hline
  \end{tabular}
\end{table}

\x\ is a plain old \code{5}, and \y\ is a reference to \x. So its value is the memory location that \x\ lives at, which in this 
case is \code{0}.

\blank

What about when we call \code{foo()}, passing \y\ as an argument?

\begin{table}[H]
  \begin{tabular}{|l|l|l|}
    \hline
    \textbf{Address} & \textbf{Name} & \textbf{Value} \\
    \hline
    3 & z & 42 \\
    \hline
    2 & i & $\rightarrow$ 0 \\
    \hline
    1 & y & $\rightarrow$ 0 \\
    \hline
    0 & x & 5 \\
    \hline
  \end{tabular}
\end{table}

Stack frames aren't only for local bindings, they're for arguments too. So in this case, we need to have both \code{i}, our argument, 
and \z, our local variable binding. \code{i} is a copy of the argument, \y. Since \code{y}'s value is \code{0}, so is \code{i}'s.

\blank

This is one reason why borrowing a variable doesn't deallocate any memory: the value of a reference is a pointer to a memory location. 
If we got rid of the underlying memory, things wouldn't work very well.

\subsubsection*{A complex example}

Okay, let's go through this complex program step-by-step:

\begin{rustc}
fn foo(x: &i32) {
    let y = 10;
    let z = &y;

    baz(z);
    bar(x, z);
}

fn bar(a: &i32, b: &i32) {
    let c = 5;
    let d = Box::new(5);
    let e = &d;

    baz(e);
}

fn baz(f: &i32) {
    let g = 100;
}

fn main() {
    let h = 3;
    let i = Box::new(20);
    let j = &h;

    foo(j);
}
\end{rustc}

First, we call \code{main()}:

\begin{table}[H]
  \begin{tabular}{|l|l|l|}
    \hline
    \textbf{Address} & \textbf{Name} & \textbf{Value} \\
    \hline
    ($2^{30}$) - 1 & & 20 \\
    \hline
    \ldots & \ldots & \ldots \\
    \hline
    2 & j & $\rightarrow$ 0 \\
    \hline
    1 & i & $\rightarrow$ ($2^{30}$)-1 \\
    \hline
    0 & h & 3 \\
    \hline
  \end{tabular}
\end{table}

We allocate memory for \code{j}, \code{i}, and \code{h}. \code{i} is on the heap, and so has a value pointing there.

\blank

Next, at the end of \code{main()}, \code{foo()} gets called:

\begin{table}[H]
  \begin{tabular}{|l|l|l|}
    \hline
    \textbf{Address} & \textbf{Name} & \textbf{Value} \\
    \hline
    ($2^{30}$) - 1 & & 20 \\
    \hline
    \ldots & \ldots & \ldots \\
    \hline
    5 & z & $\rightarrow$ 4 \\
    \hline
    4 & y & 10 \\
    \hline
    3 & x & $\rightarrow$ 0 \\
    \hline
    2 & j & $\rightarrow$ 0 \\
    \hline
    1 & i & $\rightarrow$ ($2^{30}$)-1 \\
    \hline
    0 & h & 3 \\
    \hline
  \end{tabular}
\end{table}

Space gets allocated for \x, \y, and \z. The argument \x\ has the same value as \code{j}, since that's what we passed it in. It's 
a pointer to the \code{0} address, since \code{j} points at \code{h}.

\blank

Next, \code{foo()} calls \code{baz()}, passing \z:

\begin{table}[H]
  \begin{tabular}{|l|l|l|}
    \hline
    \textbf{Address} & \textbf{Name} & \textbf{Value} \\
    \hline
    ($2^{30}$) - 1 & & 20 \\
    \hline
    \ldots & \ldots & \ldots \\
    \hline
    7 & g & 100 \\
    \hline
    6 & f & $\rightarrow$ 4 \\
    \hline
    5 & z & $\rightarrow$ 4 \\
    \hline
    4 & y & 10 \\
    \hline
    3 & x & $\rightarrow$ 0 \\
    \hline
    2 & j & $\rightarrow$ 0 \\
    \hline
    1 & i & $\rightarrow$ ($2^{30}$)-1 \\
    \hline
    0 & h & 3 \\
    \hline
  \end{tabular}
\end{table}

We've allocated memory for \code{f} and \code{g}. \code{baz()} is very short, so when it's over, we get rid of its stack frame:

\begin{table}[H]
  \begin{tabular}{|l|l|l|}
    \hline
    \textbf{Address} & \textbf{Name} & \textbf{Value} \\
    \hline
    ($2^{30}$) - 1 & & 20 \\
    \hline
    \ldots & \ldots & \ldots \\
    \hline
    5 & z & $\rightarrow$ 4 \\
    \hline
    4 & y & 10 \\
    \hline
    3 & x & $\rightarrow$ 0 \\
    \hline
    2 & j & $\rightarrow$ 0 \\
    \hline
    1 & i & $\rightarrow$ ($2^{30}$)-1 \\
    \hline
    0 & h & 3 \\
    \hline
  \end{tabular}
\end{table}

Next, \code{foo()} calls \code{bar()} with \x\ and \z:

\begin{table}[H]
  \begin{tabular}{|l|l|l|}
    \hline
    \textbf{Address} & \textbf{Name} & \textbf{Value} \\
    \hline
    ($2^{30}$) - 1 & & 20 \\
    \hline
    ($2^{30}$) - 2 & & 5 \\
    \hline
    \hline
    \ldots & \ldots & \ldots \\
    \hline
    10 & e & $\rightarrow$ 9 \\
    \hline
    9 & d & $\rightarrow$ ($2^{30}$)-2 \\
    \hline
    8 & c & 5 \\
    \hline
    7 & b & $\rightarrow$ 4 \\
    \hline
    6 & a & $\rightarrow$ 0 \\
    \hline
    5 & z & $\rightarrow$ 4 \\
    \hline
    4 & y & 10 \\
    \hline
    3 & x & $\rightarrow$ 0 \\
    \hline
    2 & j & $\rightarrow$ 0 \\
    \hline
    1 & i & $\rightarrow$ ($2^{30}$)-1 \\
    \hline
    0 & h & 3 \\
    \hline
  \end{tabular}
\end{table}

We end up allocating another value on the heap, and so we have to subtract one from ($2^{30}$) - 1. It's easier to write that than 
\code{1,073,741,822}. In any case, we set up the variables as usual.

At the end of \code{bar()}, it calls \code{baz()}:

\begin{table}[H]
  \begin{tabular}{|l|l|l|}
    \hline
    \textbf{Address} & \textbf{Name} & \textbf{Value} \\
    \hline
    ($2^{30}$) - 1 & & 20 \\
    \hline
    ($2^{30}$) - 2 & & 5 \\
    \hline
    \ldots & \ldots & \ldots \\
    12 & g & 100 \\
    \hline
    11 & f & $\rightarrow$ ($2^{30}$)-2\\
    \hline
    10 & e & $\rightarrow$ 9 \\
    \hline
    9 & d & $\rightarrow$ ($2^{30}$)-2 \\
    \hline
    8 & c & 5 \\
    \hline
    7 & b & $\rightarrow$ 4 \\
    \hline
    6 & a & $\rightarrow$ 0 \\
    \hline
    5 & z & $\rightarrow$ 4 \\
    \hline
    4 & y & 10 \\
    \hline
    3 & x & $\rightarrow$ 0 \\
    \hline
    2 & j & $\rightarrow$ 0 \\
    \hline
    1 & i & $\rightarrow$ ($2^{30}$)-1 \\
    \hline
    0 & h & 3 \\
    \hline
  \end{tabular}
\end{table}

With this, we're at our deepest point! Whew! Congrats for following along this far.

\blank

After \code{baz()} is over, we get rid of \code{f} and \code{g}:

\begin{table}[H]
  \begin{tabular}{|l|l|l|}
    \hline
    \textbf{Address} & \textbf{Name} & \textbf{Value} \\
    \hline
    ($2^{30}$) - 1 & & 20 \\
    \hline
    ($2^{30}$) - 2 & & 5 \\
    \hline
    \ldots & \ldots & \ldots \\
    \hline
    10 & e & $\rightarrow$ 9 \\
    \hline
    9 & d & $\rightarrow$ ($2^{30}$)-2 \\
    \hline
    8 & c & 5 \\
    \hline
    7 & b & $\rightarrow$ 4 \\
    \hline
    6 & a & $\rightarrow$ 0 \\
    \hline
    5 & z & $\rightarrow$ 4 \\
    \hline
    4 & y & 10 \\
    \hline
    3 & x & $\rightarrow$ 0 \\
    \hline
    2 & j & $\rightarrow$ 0 \\
    \hline
    1 & i & $\rightarrow$ ($2^{30}$)-1 \\
    \hline
    0 & h & 3 \\
    \hline
  \end{tabular}
\end{table}

Next, we return from \code{bar()}. \code{d} in this case is a \code{Box<T>}, so it also frees what it points to: ($2^{30}$) - 2.

\begin{table}[H]
  \begin{tabular}{|l|l|l|}
    \hline
    \textbf{Address} & \textbf{Name} & \textbf{Value} \\
    \hline
    ($2^{30}$) - 1 & & 20 \\
    \hline
    \ldots & \ldots & \ldots \\
    \hline
    5 & z & $\rightarrow$ 4 \\
    \hline
    4 & y & 10 \\
    \hline
    3 & x & $\rightarrow$ 0 \\
    \hline
    2 & j & $\rightarrow$ 0 \\
    \hline
    1 & i & $\rightarrow$ ($2^{30}$)-1 \\
    \hline
    0 & h & 3 \\
    \hline
  \end{tabular}
\end{table}

And after that, \code{foo()} returns:

\begin{table}[H]
  \begin{tabular}{|l|l|l|}
    \hline
    \textbf{Address} & \textbf{Name} & \textbf{Value} \\
    \hline
    ($2^{30}$) - 1 & & 20 \\
    \hline
    \ldots & \ldots & \ldots \\
    \hline
    2 & j & $\rightarrow$ 0 \\
    \hline
    1 & i & $\rightarrow$ ($2^{30}$)-1 \\
    \hline
    0 & h & 3 \\
    \hline
  \end{tabular}
\end{table}

And then, finally, \code{main()}, which cleans the rest up. When \code{i} is \code{Drop}ped, it will clean up the last of the heap too.

\subsubsection*{What do other languages do?}

Most languages with a garbage collector heap-allocate by default. This means that every value is boxed. There are a number of reasons 
why this is done, but they're out of scope for this tutorial. There are some possible optimizations that don't make it true 100\% of 
the time, too. Rather than relying on the stack and \code{Drop} to clean up memory, the garbage collector deals with the heap instead.

\subsubsection*{Which to use?}

So if the stack is faster and easier to manage, why do we need the heap? A big reason is that Stack-allocation alone means you only 
have 'Last In First Out (LIFO)' semantics for reclaiming storage. Heap-allocation is strictly more general, allowing storage to be 
taken from and returned to the pool in arbitrary order, but at a complexity cost.


Generally, you should prefer stack allocation, and so, Rust stack-allocates by default. The LIFO model of the stack is simpler, 
at a fundamental level. This has two big impacts: runtime efficiency and semantic impact.

\myparagraph{Runtime Efficiency}

Managing the memory for the stack is trivial: The machine increments or decrements a single value, the so-called \enquote{stack pointer}.
Managing memory for the heap is non-trivial: heap-allocated memory is freed at arbitrary points, and each block of heap-allocated memory 
can be of arbitrary size, so the memory manager must generally work much harder to identify memory for reuse.

\blank

If you'd like to dive into this topic in greater detail, \href{http://citeseerx.ist.psu.edu/viewdoc/summary?doi=10.1.1.143.4688}{this paper}
is a great introduction.

\myparagraph{Semantic impact}

Stack-allocation impacts the Rust language itself, and thus the developer's mental model. The LIFO semantics is what drives how the 
Rust language handles automatic memory management. Even the deallocation of a uniquely-owned heap-allocated box can be driven by the 
stack-based LIFO semantics, as discussed throughout this chapter. The flexibility (i.e. expressiveness) of non LIFO-semantics means that 
in general the compiler cannot automatically infer at compile-time where memory should be freed; it has to rely on dynamic protocols, 
potentially from outside the language itself, to drive deallocation (reference counting, as used by \code{Rc<T>} and \code{Arc<T>}, is 
one example of this).

\blank

When taken to the extreme, the increased expressive power of heap allocation comes at the cost of either significant runtime support 
(e.g. in the form of a garbage collector) or significant programmer effort (in the form of explicit memory management calls that 
require verification not provided by the Rust compiler).


\section{Testing}
\label{sec:effective_testing}

\begin{myquote}
Program testing can be a very effective way to show the presence of bugs, but it is hopelessly inadequate for showing their absence.

\blank

Edsger W. Dijkstra, \enquote{The Humble Programmer} (1972)
\end{myquote}

Let's talk about how to test Rust code. What we will not be talking about is the right way to test Rust code. There are many schools 
of thought regarding the right and wrong way to write tests. All of these approaches use the same basic tools, and so we'll show you 
the syntax for using them.

\subsubsection*{The \code{test} attribute}

At its simplest, a test in Rust is a function that's annotated with the \code{test} attribute. Let's make a new project with Cargo 
called \code{adder}:

\begin{verbatim}
$ cargo new adder
$ cd adder
\end{verbatim}

Cargo will automatically generate a simple test when you make a new project. Here's the contents of \code{src/lib.rs}:

\begin{rustc}
#[test]
fn it_works() {
}
\end{rustc}

Note the \code{\#[test]}. This attribute indicates that this is a test function. It currently has no body. That's good enough to pass! 
We can run the tests with \code{cargo test}:

\begin{verbatim}
$ cargo test
   Compiling adder v0.0.1 (file:///home/you/projects/adder)
     Running target/adder-91b3e234d4ed382a

running 1 test
test it_works ... ok

test result: ok. 1 passed; 0 failed; 0 ignored; 0 measured

   Doc-tests adder

running 0 tests

test result: ok. 0 passed; 0 failed; 0 ignored; 0 measured
\end{verbatim}

Cargo compiled and ran our tests. There are two sets of output here: one for the test we wrote, and another for documentation tests. 
We'll talk about those later. For now, see this line:

\begin{verbatim}
test it_works ... ok
\end{verbatim}

Note the \code{it\_works}. This comes from the name of our function:

\begin{rustc}
fn it_works() {
\end{rustc}

We also get a summary line:

\begin{verbatim}
test result: ok. 1 passed; 0 failed; 0 ignored; 0 measured
\end{verbatim}

So why does our do-nothing test pass? Any test which doesn't \panic\ passes, and any test that does \panic\ fails. Let's make our test fail:

\begin{rustc}
#[test]
fn it_works() {
    assert!(false);
}
\end{rustc}

\code{assert!} is a macro provided by Rust which takes one argument: if the argument is \code{true}, nothing happens. If the argument 
is \code{false}, it \panic s. Let's run our tests again:

\begin{verbatim}
$ cargo test
   Compiling adder v0.0.1 (file:///home/you/projects/adder)
     Running target/adder-91b3e234d4ed382a

running 1 test
test it_works ... FAILED

failures:

---- it_works stdout ----
        thread 'it_works' panicked at 'assertion failed: false', /home/steve/tmp/adder/src/lib.rs:3



failures:
    it_works

test result: FAILED. 0 passed; 1 failed; 0 ignored; 0 measured

thread '<main>' panicked at 'Some tests failed', /home/steve/src/rust/src/libtest/lib.rs:247
\end{verbatim}

Rust indicates that our test failed:

\begin{verbatim}
test it_works ... FAILED
\end{verbatim}

And that's reflected in the summary line:

\begin{verbatim}
test result: FAILED. 0 passed; 1 failed; 0 ignored; 0 measured
\end{verbatim}

We also get a non-zero status code. We can use \code{\$?} on OS X and Linux:

\begin{verbatim}
$ echo $?
101
\end{verbatim}

On Windows, if you’re using \code{cmd}:

\begin{verbatim}
> echo %ERRORLEVEL%
\end{verbatim}

And if you’re using PowerShell:

\begin{verbatim}
> echo $LASTEXITCODE # the code itself
> echo $? # a boolean, fail or succeed
\end{verbatim}

This is useful if you want to integrate \code{cargo test} into other tooling.

\blank

We can invert our test's failure with another attribute: \code{should\_panic}:

\begin{rustc}
#[test]
#[should_panic]
fn it_works() {
    assert!(false);
}
\end{rustc}

This test will now succeed if we \panic\ and fail if we complete. Let's try it:

\begin{verbatim}
$ cargo test
   Compiling adder v0.0.1 (file:///home/you/projects/adder)
     Running target/adder-91b3e234d4ed382a

running 1 test
test it_works ... ok

test result: ok. 1 passed; 0 failed; 0 ignored; 0 measured

   Doc-tests adder

running 0 tests

test result: ok. 0 passed; 0 failed; 0 ignored; 0 measured
\end{verbatim}

Rust provides another macro, \code{assert\_eq!}, that compares two arguments for equality:

\begin{rustc}
#[test]
#[should_panic]
fn it_works() {
    assert_eq!("Hello", "world");
}
\end{rustc}

Does this test pass or fail? Because of the \code{should\_panic} attribute, it passes:

\begin{verbatim}
$ cargo test
   Compiling adder v0.0.1 (file:///home/you/projects/adder)
     Running target/adder-91b3e234d4ed382a

running 1 test
test it_works ... ok

test result: ok. 1 passed; 0 failed; 0 ignored; 0 measured

   Doc-tests adder

running 0 tests

test result: ok. 0 passed; 0 failed; 0 ignored; 0 measured
\end{verbatim}

\code{should\_panic} tests can be fragile, as it's hard to guarantee that the test didn't fail for an unexpected reason. To help 
with this, an optional \code{expected} parameter can be added to the \code{should\_panic} attribute. The test harness will make 
sure that the failure message contains the provided text. A safer version of the example above would be:

\begin{rustc}
#[test]
#[should_panic(expected = "assertion failed")]
fn it_works() {
    assert_eq!("Hello", "world");
}
\end{rustc}

That's all there is to the basics! Let's write one 'real' test:

\begin{rustc}
pub fn add_two(a: i32) -> i32 {
    a + 2
}

#[test]
fn it_works() {
    assert_eq!(4, add_two(2));
}
\end{rustc}

This is a very common use of \code{assert\_eq!}: call some function with some known arguments and compare it to the expected output.

\subsubsection*{The \code{ignore} attribute}

Sometimes a few specific tests can be very time-consuming to execute. These can be disabled by default by using the \code{ignore} attribute:

\begin{rustc}
#[test]
fn it_works() {
    assert_eq!(4, add_two(2));
}

#[test]
#[ignore]
fn expensive_test() {
    // code that takes an hour to run
}
\end{rustc}

Now we run our tests and see that \code{it\_works} is run, but \code{expensive\_test} is not:

\begin{verbatim}
$ cargo test
   Compiling adder v0.0.1 (file:///home/you/projects/adder)
     Running target/adder-91b3e234d4ed382a

running 2 tests
test expensive_test ... ignored
test it_works ... ok

test result: ok. 1 passed; 0 failed; 1 ignored; 0 measured

   Doc-tests adder

running 0 tests

test result: ok. 0 passed; 0 failed; 0 ignored; 0 measured
\end{verbatim}

The expensive tests can be run explicitly using \code{cargo test -- --ignored}:

\begin{verbatim}
$ cargo test -- --ignored
     Running target/adder-91b3e234d4ed382a

running 1 test
test expensive_test ... ok

test result: ok. 1 passed; 0 failed; 0 ignored; 0 measured

   Doc-tests adder

running 0 tests

test result: ok. 0 passed; 0 failed; 0 ignored; 0 measured
\end{verbatim}

The \code{--ignored} argument is an argument to the test binary, and not to Cargo, which is why the command is 
\code{cargo test -- --ignored}.

\subsubsection*{The \code{tests} module}

There is one way in which our existing example is not idiomatic: it's missing the \code{tests} module. The idiomatic way of 
writing our example looks like this:

\begin{rustc}
pub fn add_two(a: i32) -> i32 {
    a + 2
}

#[cfg(test)]
mod tests {
    use super::add_two;

    #[test]
    fn it_works() {
        assert_eq!(4, add_two(2));
    }
}
\end{rustc}

There's a few changes here. The first is the introduction of a \code{mod tests} with a \code{cfg} attribute. The module allows 
us to group all of our tests together, and to also define helper functions if needed, that don't become a part of the rest of 
our crate. The \code{cfg} attribute only compiles our test code if we're currently trying to run the tests. This can save compile 
time, and also ensures that our tests are entirely left out of a normal build.

\blank

The second change is the \code{use} declaration. Because we're in an inner module, we need to bring our test function into scope. 
This can be annoying if you have a large module, and so this is a common use of globs. Let's change our \code{src/lib.rs} to make 
use of it:

\begin{rustc}
pub fn add_two(a: i32) -> i32 {
    a + 2
}

#[cfg(test)]
mod tests {
    use super::*;

    #[test]
    fn it_works() {
        assert_eq!(4, add_two(2));
    }
}
\end{rustc}

Note the different \code{use} line. Now we run our tests:

\begin{verbatim}
$ cargo test
    Updating registry `https://github.com/rust-lang/crates.io-index`
   Compiling adder v0.0.1 (file:///home/you/projects/adder)
     Running target/adder-91b3e234d4ed382a

running 1 test
test tests::it_works ... ok

test result: ok. 1 passed; 0 failed; 0 ignored; 0 measured

   Doc-tests adder

running 0 tests

test result: ok. 0 passed; 0 failed; 0 ignored; 0 measured
\end{verbatim}

It works!

\blank

The current convention is to use the \code{tests} module to hold your \enquote{unit-style} tests. Anything that tests one 
small bit of functionality makes sense to go here. But what about \enquote{integration-style} \code{tests} instead? For that, 
we have the tests directory.

\subsubsection*{The \code{tests} directory}

To write an integration test, let's make a \code{tests} directory, and put a \code{tests/lib.rs} file inside, with this as its contents:

\begin{rustc}
extern crate adder;

#[test]
fn it_works() {
    assert_eq!(4, adder::add_two(2));
}
\end{rustc}

This looks similar to our previous tests, but slightly different. We now have an \code{extern crate adder} at the top. This is 
because the tests in the \code{tests} directory are an entirely separate crate, and so we need to import our library. This is also 
why \code{tests} is a suitable place to write integration-style tests: they use the library like any other consumer of it would.

\blank

Let's run them:

\begin{verbatim}
$ cargo test
   Compiling adder v0.0.1 (file:///home/you/projects/adder)
     Running target/adder-91b3e234d4ed382a

running 1 test
test tests::it_works ... ok

test result: ok. 1 passed; 0 failed; 0 ignored; 0 measured

     Running target/lib-c18e7d3494509e74

running 1 test
test it_works ... ok

test result: ok. 1 passed; 0 failed; 0 ignored; 0 measured

   Doc-tests adder

running 0 tests

test result: ok. 0 passed; 0 failed; 0 ignored; 0 measured
\end{verbatim}

Now we have three sections: our previous test is also run, as well as our new one.

\blank

That's all there is to the \code{tests} directory. The \code{tests} module isn't needed here, since the whole thing is focused on tests.

\blank

Let's finally check out that third section: documentation tests.

\subsubsection*{Documentation tests}

Nothing is better than documentation with examples. Nothing is worse than examples that don't actually work, because the code has 
changed since the documentation has been written. To this end, Rust supports automatically running examples in your documentation 
(\textbf{note}: this only works in library crates, not binary crates). Here's a fleshed-out \code{src/lib.rs} with examples:

\begin{rustc}
//! The `adder` crate provides functions that add numbers to other numbers.
//!
//! # Examples
//!
//! ```
//! assert_eq!(4, adder::add_two(2));
//! ```

/// This function adds two to its argument.
///
/// # Examples
///
/// ```
/// use adder::add_two;
///
/// assert_eq!(4, add_two(2));
/// ```
pub fn add_two(a: i32) -> i32 {
    a + 2
}

#[cfg(test)]
mod tests {
    use super::*;

    #[test]
    fn it_works() {
        assert_eq!(4, add_two(2));
    }
}
\end{rustc}

Note the module-level documentation with \code{//!} and the function-level documentation with \code{///}. Rust's documentation 
supports Markdown in comments, and so triple graves mark code blocks. It is conventional to include the \code{\# Examples} section, 
exactly like that, with examples following.

\blank

Let's run the tests again:

\begin{verbatim}
$ cargo test
   Compiling adder v0.0.1 (file:///home/steve/tmp/adder)
     Running target/adder-91b3e234d4ed382a

running 1 test
test tests::it_works ... ok

test result: ok. 1 passed; 0 failed; 0 ignored; 0 measured

     Running target/lib-c18e7d3494509e74

running 1 test
test it_works ... ok

test result: ok. 1 passed; 0 failed; 0 ignored; 0 measured

   Doc-tests adder

running 2 tests
test add_two_0 ... ok
test _0 ... ok

test result: ok. 2 passed; 0 failed; 0 ignored; 0 measured
\end{verbatim}

Now we have all three kinds of tests running! Note the names of the documentation tests: the \code{\_0} is generated for the module 
test, and \code{add\_two\_0} for the function test. These will auto increment with names like \code{add\_two\_1} as you add more examples.

\blank

% TODO make documentation chapter a hyperref
We haven’t covered all of the details with writing documentation tests. For more, please see the Documentation chapter.

\blank

One final note: documentation tests \emph{cannot} be run on binary crates. To see more on file arrangement see the Crates and Modules section
(see \nameref{sec:syntax_cratesAndModules}).


\section{Conditional Compilation}
\label{sec:effective_conditionalCompilation}
Rust has a special attribute, \code{\#[cfg]}, which allows you to compile code based on a flag passed to the compiler. It has two forms:

\begin{rustc}
#[cfg(foo)]

#[cfg(bar = "baz")]
\end{rustc}

They also have some helpers:

\begin{rustc}
#[cfg(any(unix, windows))]

#[cfg(all(unix, target_pointer_width = "32"))]

#[cfg(not(foo))]
\end{rustc}

These can nest arbitrarily:

\begin{rustc}
#[cfg(any(not(unix), all(target_os="macos", target_arch = "powerpc")))]
\end{rustc}

As for how to enable or disable these switches, if you’re using Cargo, they get set in the 
\href{http://doc.crates.io/manifest.html\#the-features-section}{[features] section} of your \code{Cargo.toml}:

\begin{verbatim}
[features]
# no features by default
default = []

# The “secure-password” feature depends on the bcrypt package.
secure-password = ["bcrypt"]
\end{verbatim}

When you do this, Cargo passes along a flag to \code{rustc}:

\begin{verbatim}
--cfg feature="${feature_name}"
\end{verbatim}

The sum of these \code{cfg} flags will determine which ones get activated, and therefore, which code gets compiled. Let’s take this code:

\begin{rustc}
#[cfg(feature = "foo")]
mod foo {
}
\end{rustc}

If we compile it with \code{cargo build --features "foo"}, it will send the \code{--cfg feature="foo"} flag to \code{rustc}, and the 
output will have the \code{mod foo} in it. If we compile it with a regular \code{cargo build}, no extra flags get passed on, and so, 
no \code{foo} module will exist.

\subsection*{cfg\_attr}

You can also set another attribute based on a \code{cfg} variable with \code{cfg\_attr}:

\begin{rustc}
#[cfg_attr(a, b)]
\end{rustc}

Will be the same as \code{\#[b]} if a is set by \code{cfg} attribute, and nothing otherwise.

\subsection*{cfg!}

The \code{cfg!} syntax extension (see \nameref{sec:nightly_compilerPlugins}) lets you use these kinds of flags elsewhere in your code, too:

\begin{rustc}
if cfg!(target_os = "macos") || cfg!(target_os = "ios") {
    println!("Think Different!");
}
\end{rustc}

These will be replaced by a \code{true} or \code{false} at compile-time, depending on the configuration settings.


\section{Documentation}
\label{sec:effective_documentation}
Documentation is an important part of any software project, and it's first-class in Rust. Let's talk about the tooling Rust gives 
you to document your project.

\subsection*{About \code{rustdoc}}

The Rust distribution includes a tool, \code{rustdoc}, that generates documentation. \code{rustdoc} is also used by Cargo through 
\code{cargo doc}.

\blank

Documentation can be generated in two ways: from source code, and from standalone Markdown files.

\subsection*{Documenting source code}

The primary way of documenting a Rust project is through annotating the source code. You can use documentation comments for this purpose:

\begin{rustc}
/// Constructs a new `Rc<T>`.
///
/// # Examples
///
/// ```
/// use std::rc::Rc;
///
/// let five = Rc::new(5);
/// ```
pub fn new(value: T) -> Rc<T> {
    // implementation goes here
}
\end{rustc}

This code generates documentation that looks \href{https://doc.rust-lang.org/nightly/std/rc/struct.Rc.html#method.new}{like this}. 
I've left the implementation out, with a regular comment in its place.

\blank

The first thing to notice about this annotation is that it uses \code{///} instead of \code{//}. The triple slash indicates a 
documentation comment.

\blank

Documentation comments are written in Markdown.

\blank

Rust keeps track of these comments, and uses them when generating documentation. This is important when documenting things like enums:

\begin{rustc}
/// The `Option` type. See [the module level documentation](index.html) for more.
enum Option<T> {
    /// No value
    None,
    /// Some value `T`
    Some(T),
}
\end{rustc}

The above works, but this does not:

\begin{rustc}
/// The `Option` type. See [the module level documentation](index.html) for more.
enum Option<T> {
    None, /// No value
    Some(T), /// Some value `T`
}
\end{rustc}

You'll get an error:

\begin{verbatim}
hello.rs:4:1: 4:2 error: expected ident, found `}`
hello.rs:4 }
           ^
\end{verbatim}

This \href{https://github.com/rust-lang/rust/issues/22547}{unfortunate error} is correct; documentation comments apply to the thing 
after them, and there's nothing after that last comment.

\myparagraph{Writing documentation comments}

Anyway, let's cover each part of this comment in detail:

\begin{rustc}
/// Constructs a new `Rc<T>`.
\end{rustc}

The first line of a documentation comment should be a short summary of its functionality. One sentence. Just the basics. High level.

\begin{rustc}
///
/// Other details about constructing `Rc<T>`s, maybe describing complicated
/// semantics, maybe additional options, all kinds of stuff.
///
\end{rustc}

Our original example had just a summary line, but if we had more things to say, we could have added more explanation in 
a new paragraph.

\myparagraph{Special sections}

Next, are special sections. These are indicated with a header, \code{\#}. There are four kinds of headers that are commonly used. 
They aren't special syntax, just convention, for now.

\begin{rustc}
/// # Panics
\end{rustc}

Unrecoverable misuses of a function (i.e. programming errors) in Rust are usually indicated by panics, which kill the whole 
current thread at the very least. If your function has a non-trivial contract like this, that is detected/enforced by panics, 
documenting it is very important.

\begin{rustc}
/// # Failures
\end{rustc}

If your function or method returns a \code{Result<T, E>}, then describing the conditions under which it returns \code{Err(E)} is 
a nice thing to do. This is slightly less important than \code{Panics}, because failure is encoded into the type system, but it's 
still a good thing to do.

\begin{rustc}
/// # Safety
\end{rustc}

If your function is \code{unsafe}, you should explain which invariants the caller is responsible for upholding.

\begin{rustc}
/// # Examples
///
/// ```
/// use std::rc::Rc;
///
/// let five = Rc::new(5);
/// ```
\end{rustc}

Fourth, \code{Examples}. Include one or more examples of using your function or method, and your users will love you for it. 
These examples go inside of code block annotations, which we'll talk about in a moment, and can have more than one section:

\begin{rustc}
/// # Examples
///
/// Simple `&str` patterns:
///
/// ```
/// let v: Vec<&str> = "Mary had a little lamb".split(' ').collect();
/// assert_eq!(v, vec!["Mary", "had", "a", "little", "lamb"]);
/// ```
///
/// More complex patterns with a lambda:
///
/// ```
/// let v: Vec<&str> = "abc1def2ghi".split(|c: char| c.is_numeric()).collect();
/// assert_eq!(v, vec!["abc", "def", "ghi"]);
/// ```
\end{rustc}

Let's discuss the details of these code blocks.

\myparagraph{Code block annotations}

To write some Rust code in a comment, use the triple graves:

\begin{rustc}
/// ```
/// println!("Hello, world");
/// ```
\end{rustc}

If you want something that's not Rust code, you can add an annotation:

\begin{rustc}
/// ```c
/// printf("Hello, world\n");
/// ```
\end{rustc}

This will highlight according to whatever language you're showing off. If you're only showing plain text, choose \code{text}.

\blank

It's important to choose the correct annotation here, because \code{rustdoc} uses it in an interesting way: It can be used to 
actually test your examples in a library crate, so that they don't get out of date. If you have some C code but \code{rustdoc} 
thinks it's Rust because you left off the annotation, \code{rustdoc} will complain when trying to generate the documentation.

\subsection*{Documentation as tests}

Let's discuss our sample example documentation:

\begin{rustc}
/// ```
/// println!("Hello, world");
/// ```
\end{rustc}

You'll notice that you don't need a \code{fn main()} or anything here. \code{rustdoc} will automatically add a \code{main()} 
wrapper around your code, using heuristics to attempt to put it in the right place. For example:

\begin{rustc}
/// ```
/// use std::rc::Rc;
///
/// let five = Rc::new(5);
/// ```
\end{rustc}

This will end up testing:

\begin{rustc}
fn main() {
    use std::rc::Rc;
    let five = Rc::new(5);
}
\end{rustc}

Here's the full algorithm rustdoc uses to preprocess examples:

\begin{enumerate}
  \item{Any leading \code{\#![foo]} attributes are left intact as crate attributes.}
  \item{Some common \code{allow} attributes are inserted, including \code{unused\_variables}, \code{unused\_assignments}, \code{unused\_mut}, 
      \code{unused\_attributes}, and \code{dead\_code}. Small examples often trigger these lints.}
  \item{If the example does not contain \code{extern crate}, then \code{extern crate <mycrate>}; is inserted (note the lack of 
      \code{\#[macro\_use]}).}
  \item{Finally, if the example does not contain \code{fn main}, the remainder of the text is wrapped in \code{fn main() \{ your\_code \}}.}
\end{enumerate}

This generated \code{fn main} can be a problem! If you have \code{extern crate} or a \code{mod} statements in the example code that 
are referred to by \code{use} statements, they will fail to resolve unless you include at least \code{fn main() \{\}} to inhibit step 4. 
\code{\#[macro\_use] extern crate} also does not work except at the crate root, so when testing macros an explicit \code{main} is 
always required. It doesn't have to clutter up your docs, though -- keep reading!

\blank

Sometimes this algorithm isn't enough, though. For example, all of these code samples with \code{///} we've been talking about? The raw text:

\begin{rustc}
/// Some documentation.
# fn foo() {}
\end{rustc}

looks different than the output:

\begin{rustc}
/// Some documentation.
\end{rustc}

Yes, that's right: you can add lines that start with \code{\#}, and they will be hidden from the output, but will be used when 
compiling your code. You can use this to your advantage. In this case, documentation comments need to apply to some kind of function, 
so if I want to show you just a documentation comment, I need to add a little function definition below it. At the same time, it's 
only there to satisfy the compiler, so hiding it makes the example more clear. You can use this technique to explain longer examples 
in detail, while still preserving the testability of your documentation.

\blank

For example, imagine that we wanted to document this code:

\begin{rustc}
let x = 5;
let y = 6;
println!("{}", x + y);
\end{rustc}

We might want the documentation to end up looking like this:

\begin{myquote}
First, we set \x\ to five:

\begin{rustc}
let x = 5;
\end{rustc}

Next, we set \y\ to six:

\begin{rustc}
let y = 6;
\end{rustc}

Finally, we print the sum of \x\ and \y:

\begin{rustc}
println!("{}", x + y);
\end{rustc}
\end{myquote}

To keep each code block testable, we want the whole program in each block, but we don't want the reader to see every line every time. 
Here's what we put in our source code:

\begin{verbatim}
    First, we set `x` to five:

    ```text
    let x = 5;
    # let y = 6;
    # println!("{}", x + y);
    ```

    Next, we set `y` to six:

    ```text
    # let x = 5;
    let y = 6;
    # println!("{}", x + y);
    ```

    Finally, we print the sum of `x` and `y`:

    ```text
    # let x = 5;
    # let y = 6;
    println!("{}", x + y);
    ```
\end{verbatim}

By repeating all parts of the example, you can ensure that your example still compiles, while only showing the parts that are relevant 
to that part of your explanation.

\subsection*{Documenting macros}

Here’s an example of documenting a macro:

\begin{rustc}
/// Panic with a given message unless an expression evaluates to true.
///
/// # Examples
///
/// ```
/// # #[macro_use] extern crate foo;
/// # fn main() {
/// panic_unless!(1 + 1 == 2, “Math is broken.”);
/// # }
/// ```
///
/// ```should_panic
/// # #[macro_use] extern crate foo;
/// # fn main() {
/// panic_unless!(true == false, “I’m broken.”);
/// # }
/// ```
#[macro_export]
macro_rules! panic_unless {
    ($condition:expr, $($rest:expr),+) => ({ if ! $condition { panic!($($rest),+); } });
}
\end{rustc}

You’ll note three things: we need to add our own \code{extern crate} line, so that we can add the \code{\#[macro\_use]} attribute. 
Second, we’ll need to add our own \code{main()} as well (for reasons discussed above). Finally, a judicious use of \code{\#} to 
comment out those two things, so they don’t show up in the output.

\blank

Another case where the use of \code{\#} is handy is when you want to ignore error handling. Lets say you want the following,

\begin{rustc}
/// use std::io;
/// let mut input = String::new();
/// try!(io::stdin().read_line(&mut input));
\end{rustc}

The problem is that \code{try!} returns a \code{Result<T, E>} and test functions don't return anything so this will give a 
mismatched types error.

\begin{rustc}
/// A doc test using try!
///
/// ```
/// use std::io;
/// # fn foo() -> io::Result<()> {
/// let mut input = String::new();
/// try!(io::stdin().read_line(&mut input));
/// # Ok(())
/// # }
/// ```
\end{rustc}

You can get around this by wrapping the code in a function. This catches and swallows the \code{Result<T, E>} when running tests on 
the docs. This pattern appears regularly in the standard library.

\subsection*{Running documentation tests}

To run the tests, either:

\begin{verbatim}
$ rustdoc --test path/to/my/crate/root.rs
# or
$ cargo test
\end{verbatim}

That's right, \code{cargo test} tests embedded documentation too. \textbf{However, \code{cargo test} will not test binary crates, 
only library ones.} This is due to the way \code{rustdoc} works: it links against the library to be tested, but with a binary, 
there’s nothing to link to.

\blank

There are a few more annotations that are useful to help \code{rustdoc} do the right thing when testing your code:

\begin{rustc}
/// ```ignore
/// fn foo() {
/// ```
\end{rustc}

The \code{ignore} directive tells Rust to ignore your code. This is almost never what you want, as it's the most generic. Instead, 
consider annotating it with \code{text} if it's not code, or using \code{\#}s to get a working example that only shows the part 
you care about.

\begin{rustc}
/// ```should_panic
/// assert!(false);
/// ```
\end{rustc}

\code{should\_panic} tells \code{rustdoc} that the code should compile correctly, but not actually pass as a test.

\begin{rustc}
/// ```no_run
/// loop {
///     println!("Hello, world");
/// }
/// ```
\end{rustc}

The \code{no\_run} attribute will compile your code, but not run it. This is important for examples such as \enquote{Here's how to 
start up a network service,} which you would want to make sure compile, but might run in an infinite loop!

\subsection*{Documenting modules}

Rust has another kind of doc comment, \code{//!}. This comment doesn't document the next item, but the enclosing item. In other words:

\begin{rustc}
mod foo {
    //! This is documentation for the `foo` module.
    //!
    //! # Examples

    // ...
}
\end{rustc}

This is where you'll see \code{//!} used most often: for module documentation. If you have a module in \code{foo.rs}, you'll often 
open its code and see this:

\begin{rustc}
//! A module for using `foo`s.
//!
//! The `foo` module contains a lot of useful functionality blah blah blah
\end{rustc}

\subsection*{Documentation comment style}

Check out \href{https://github.com/rust-lang/rfcs/blob/master/text/0505-api-comment-conventions.md}{RFC 505} for full conventions 
around the style and format of documentation.

\subsection*{Other documentation}

All of this behavior works in non-Rust source files too. Because comments are written in Markdown, they're often \code{.md} files.

\blank

When you write documentation in Markdown files, you don't need to prefix the documentation with comments. For example:

\begin{rustc}
/// # Examples
///
/// ```
/// use std::rc::Rc;
///
/// let five = Rc::new(5);
/// ```
\end{rustc}

is:

\begin{verbatim}
# Examples

```
use std::rc::Rc;

let five = Rc::new(5);
```
\end{verbatim}

when it's in a Markdown file. There is one wrinkle though: Markdown files need to have a title like this:

\begin{verbatim}
% The title

This is the example documentation.
\end{verbatim}

This \code{\%} line needs to be the very first line of the file.

\subsection*{\code{doc} attributes}

At a deeper level, documentation comments are syntactic sugar for documentation attributes:

\begin{rustc}
/// this

#[doc="this"]
\end{rustc}

are the same, as are these:

\begin{rustc}
//! this

#![doc="this"]
\end{rustc}

You won't often see this attribute used for writing documentation, but it can be useful when changing some options, or when writing a macro.

\subsection*{Re-exports}

\code{rustdoc} will show the documentation for a public re-export in both places:

\begin{rustc}
extern crate foo;

pub use foo::bar;
\end{rustc}

This will create documentation for \code{bar} both inside the documentation for the crate \code{foo}, as well as the documentation 
for your crate. It will use the same documentation in both places.

\blank

This behavior can be suppressed with \code{no\_inline}:

\begin{rustc}
extern crate foo;

#[doc(no_inline)]
pub use foo::bar;
\end{rustc}

\subsection*{Missing documentation}

Sometimes you want to make sure that every single public thing in your project is documented, especially when you are working on 
a library. Rust allows you to to generate warnings or errors, when an item is missing documentation. To generate warnings you use 
\code{warn}:

\begin{rustc}
#![warn(missing_docs)]
\end{rustc}

And to generate errors you use deny:

\begin{rustc}
#![deny(missing_docs)]
\end{rustc}

There are cases where you want to disable these warnings/errors to explicitly leave something undocumented. This is done by using \code{allow}:

\begin{rustc}
#[allow(missing_docs)]
struct Undocumented;
\end{rustc}

You might even want to hide items from the documentation completely:

\begin{rustc}
#[doc(hidden)]
struct Hidden;
\end{rustc}

\subsection*{Controlling HTML}

You can control a few aspects of the HTML that \code{rustdoc} generates through the \code{\#![doc]} version of the attribute:

\begin{rustc}
#![doc(html_logo_url = "https://www.rust-lang.org/logos/rust-logo-128x128-blk-v2.png",
       html_favicon_url = "https://www.rust-lang.org/favicon.ico",
       html_root_url = "https://doc.rust-lang.org/")]
\end{rustc}

This sets a few different options, with a logo, favicon, and a root URL.

\subsection*{Configuring documentation tests}

You can also configure the way that \code{rustdoc} tests your documentation examples through the \code{\#![doc(test(..))]} attribute.

\begin{rustc}
#![doc(test(attr(allow(unused_variables), deny(warnings))))]
\end{rustc}

This allows unused variables within the examples, but will fail the test for any other lint warning thrown.

\subsection*{Generation options}

\code{rustdoc} also contains a few other options on the command line, for further customization:

\begin{itemize}
  \item{\code{--html-in-header FILE}: includes the contents of FILE at the end of the \code{<head>...</head>} section.}
  \item{\code{--html-before-content FILE}: includes the contents of FILE directly after \code{<body>}, before the rendered 
      content (including the search bar).}
  \item{\code{--html-after-content FILE}: includes the contents of FILE after all the rendered content.}
\end{itemize}

\subsection*{Security note}

The Markdown in documentation comments is placed without processing into the final webpage. Be careful with literal HTML:

\begin{rustc}
/// <script>alert(document.cookie)</script>
\end{rustc}


\section{Iterators}
\label{sec:effective_iterators}
Let's talk about loops.

\blank

Remember Rust's \code{for} loop? Here's an example:

\begin{rustc}
for x in 0..10 {
    println!("{}", x);
}
\end{rustc}

Now that you know more Rust, we can talk in detail about how this works. Ranges (the \code{0..10}) are 'iterators'. An iterator is 
something that we can call the \code{.next()} method on repeatedly, and it gives us a sequence of things.

\blank

Like this:

\begin{rustc}
let mut range = 0..10;

loop {
    match range.next() {
        Some(x) => {
            println!("{}", x);
        },
        None => { break }
    }
}
\end{rustc}

We make a mutable binding to the range, which is our iterator. We then \code{loop}, with an inner \code{match}. This \code{match} is 
used on the result of \code{range.next()}, which gives us a reference to the next value of the iterator. \code{next} returns an 
\code{Option<i32>}, in this case, which will be \code{Some(i32)} when we have a value and \code{None} once we run out. If we get 
\code{Some(i32)}, we print it out, and if we get \code{None}, we \code{break} out of the loop.

\blank

This code sample is basically the same as our \code{for} loop version. The \code{for} loop is a handy way to write this 
\code{loop}/\code{match}/\code{break} construct.

\blank

\code{for} loops aren't the only thing that uses iterators, however. Writing your own iterator involves implementing the \code{Iterator} 
trait. While doing that is outside of the scope of this guide, Rust provides a number of useful iterators to accomplish various tasks. 
But first, a few notes about limitations of ranges.

\blank

Ranges are very primitive, and we often can use better alternatives. Consider the following Rust anti-pattern: using ranges to emulate a 
C-style \code{for} loop. Let’s suppose you needed to iterate over the contents of a vector. You may be tempted to write this:

\begin{rustc}
let nums = vec![1, 2, 3];

for i in 0..nums.len() {
    println!("{}", nums[i]);
}
\end{rustc}

This is strictly worse than using an actual iterator. You can iterate over vectors directly, so write this:

\begin{rustc}
let nums = vec![1, 2, 3];

for num in &nums {
    println!("{}", num);
}
\end{rustc}

There are two reasons for this. First, this more directly expresses what we mean. We iterate through the entire vector, rather than 
iterating through indexes, and then indexing the vector. Second, this version is more efficient: the first version will have extra 
bounds checking because it used indexing, \code{nums[i]}. But since we yield a reference to each element of the vector in turn with 
the iterator, there's no bounds checking in the second example. This is very common with iterators: we can ignore unnecessary bounds 
checks, but still know that we're safe.

\blank

There's another detail here that's not 100\% clear because of how \println\ works. \code{num} is actually of type \code{\&i32}. That is, 
it's a reference to an \itt, not an \itt\ itself. \println\ handles the dereferencing for us, so we don't see it. This code works fine too:

\begin{rustc}
let nums = vec![1, 2, 3];

for num in &nums {
    println!("{}", *num);
}
\end{rustc}

Now we're explicitly dereferencing \code{num}. Why does \code{\&nums} give us references? Firstly, because we explicitly asked it to 
with \code{\&}. Secondly, if it gave us the data itself, we would have to be its owner, which would involve making a copy of the data 
and giving us the copy. With references, we're only borrowing a reference to the data, and so it's only passing a reference, without 
needing to do the move.

\blank

So, now that we've established that ranges are often not what you want, let's talk about what you do want instead.

\blank

There are three broad classes of things that are relevant here: iterators, \emph{iterator adaptors}, and \emph{consumers}. Here's 
some definitions:

\begin{itemize}
  \item{\emph{iterators} give you a sequence of values.}
  \item{\emph{iterator adaptors} operate on an iterator, producing a new iterator with a different output sequence.}
  \item{\emph{consumers} operate on an iterator, producing some final set of values.}
\end{itemize}

Let's talk about consumers first, since you've already seen an iterator, ranges.

\subsection*{Consumers}

A \emph{consumer} operates on an iterator, returning some kind of value or values. The most common consumer is \code{collect()}. This 
code doesn't quite compile, but it shows the intention:

\begin{rustc}
let one_to_one_hundred = (1..101).collect();
\end{rustc}

As you can see, we call \code{collect()} on our iterator. \code{collect()} takes as many values as the iterator will give it, 
and returns a collection of the results. So why won't this compile? Rust can't determine what type of things you want to collect, 
and so you need to let it know. Here's the version that does compile:

\begin{rustc}
let one_to_one_hundred = (1..101).collect::<Vec<i32>>();
\end{rustc}

If you remember, the \code{::<>} syntax allows us to give a type hint, and so we tell it that we want a vector of integers. You 
don't always need to use the whole type, though. Using a \code{\_} will let you provide a partial hint:

\begin{rustc}
let one_to_one_hundred = (1..101).collect::<Vec<_>>();
\end{rustc}

This says \enquote{Collect into a \code{Vec<T>}, please, but infer what the \code{T} is for me.} \code{\_} is sometimes called a 
\enquote{type placeholder} for this reason.

\blank

\code{collect()} is the most common consumer, but there are others too. \code{find()} is one:

\begin{rustc}
let greater_than_forty_two = (0..100)
                             .find(|x| *x > 42);

match greater_than_forty_two {
    Some(_) => println!("Found a match!"),
    None => println!("No match found :("),
}
\end{rustc}

\code{find} takes a closure, and works on a reference to each element of an iterator. This closure returns \code{true} if the 
element is the element we're looking for, and \code{false} otherwise. \code{find} returns the first element satisfying the 
specified predicate. Because we might not find a matching element, \code{find} returns an \code{Option} rather than the element itself.

\blank

Another important consumer is \code{fold}. Here's what it looks like:

\begin{rustc}
let sum = (1..4).fold(0, |sum, x| sum + x);
\end{rustc}

\code{fold()} is a consumer that looks like this: \code{fold(base, |accumulator, element| ...)}. It takes two arguments: the first 
is an element called the \emph{base}. The second is a closure that itself takes two arguments: the first is called the \emph{accumulator}, 
and the second is an \emph{element}. Upon each iteration, the closure is called, and the result is the value of the accumulator on the 
next iteration. On the first iteration, the base is the value of the accumulator.

\blank

Okay, that's a bit confusing. Let's examine the values of all of these things in this iterator:

\begin{table}[H]
  \begin{tabular}{|l|l|l|l|}
  \hline
  \textbf{base} & \textbf{accumulator} & \textbf{element} & \textbf{closure result} \\
  \hline
  0 & 0 & 1 & 1 \\
  \hline
  0 & 1 & 2 & 3 \\
  \hline
  0 & 3 & 3 & 6 \\
  \hline
  \end{tabular}
\end{table}

We called \code{fold()} with these arguments:

\begin{rustc}
.fold(0, |sum, x| sum + x);
\end{rustc}

So, \code{0} is our base, \code{sum} is our accumulator, and \x\ is our element. On the first iteration, we set \code{sum} to \code{0}, 
and \x\ is the first element of \code{nums}, \code{1}. We then add \code{sum} and \x, which gives us \code{0 + 1 = 1}. On the second 
iteration, that value becomes our accumulator, \code{sum}, and the element is the second element of the array, \code{2}. \code{1 + 2 = 3}, 
and so that becomes the value of the accumulator for the last iteration. On that iteration, \x\ is the last element, \code{3}, and 
\code{3 + 3 = 6}, which is our final result for our \code{sum}. \code{1 + 2 + 3 = 6}, and that's the result we got.

\blank

Whew. \code{fold} can be a bit strange the first few times you see it, but once it clicks, you can use it all over the place. 
Any time you have a list of things, and you want a single result, \code{fold} is appropriate.

\blank

Consumers are important due to one additional property of iterators we haven't talked about yet: laziness. Let's talk some more 
about iterators, and you'll see why consumers matter.

\subsection*{Iterators}

As we've said before, an iterator is something that we can call the \code{.next()} method on repeatedly, and it gives us a sequence 
of things. Because you need to call the method, this means that iterators can be \emph{lazy} and not generate all of the values upfront. 
This code, for example, does not actually generate the numbers \code{1-99}, instead creating a value that merely represents the sequence:

\begin{rustc}
let nums = 1..100;
\end{rustc}

Since we didn't do anything with the range, it didn't generate the sequence. Let's add the consumer:

\begin{rustc}
let nums = (1..100).collect::<Vec<i32>>();
\end{rustc}

Now, \code{collect()} will require that the range gives it some numbers, and so it will do the work of generating the sequence.

\blank

Ranges are one of two basic iterators that you'll see. The other is \code{iter()}. \code{iter()} can turn a vector into a simple 
iterator that gives you each element in turn:

\begin{rustc}
let nums = vec![1, 2, 3];

for num in nums.iter() {
   println!("{}", num);
}
\end{rustc}

These two basic iterators should serve you well. There are some more advanced iterators, including ones that are infinite.

\blank

That's enough about iterators. Iterator adaptors are the last concept we need to talk about with regards to iterators. Let's get to it!

\subsection*{Iterator adaptors}

\emph{Iterator adaptors} take an iterator and modify it somehow, producing a new iterator. The simplest one is called \code{map}:

\begin{rustc}
(1..100).map(|x| x + 1);
\end{rustc}

\code{map} is called upon another iterator, and produces a new iterator where each element reference has the closure it's been given 
as an argument called on it. So this would give us the numbers from \code{2-100}. Well, almost! If you compile the example, you'll 
get a warning:

\begin{verbatim}
warning: unused result which must be used: iterator adaptors are lazy and
         do nothing unless consumed, #[warn(unused_must_use)] on by default
(1..100).map(|x| x + 1);
 ^~~~~~~~~~~~~~~~~~~~~~~~~~~~~~~~
\end{verbatim}

Laziness strikes again! That closure will never execute. This example doesn't print any numbers:

\begin{rustc}
(1..100).map(|x| println!("{}", x));
\end{rustc}

If you are trying to execute a closure on an iterator for its side effects, use \code{for} instead.

\blank

There are tons of interesting iterator adaptors. \code{take(n)} will return an iterator over the next \code{n} elements of the 
original iterator. Let's try it out with an infinite iterator:

\begin{rustc}
for i in (1..).take(5) {
    println!("{}", i);
}
\end{rustc}

This will print

\begin{verbatim}
1
2
3
4
5
\end{verbatim}

\code{filter()} is an adapter that takes a closure as an argument. This closure returns \code{true} or \code{false}. The new iterator 
\code{filter()} produces only the elements that the closure returns \code{true} for:

\begin{rustc}
for i in (1..100).filter(|&x| x % 2 == 0) {
    println!("{}", i);
}
\end{rustc}

This will print all of the even numbers between one and a hundred. (Note that because \code{filter} doesn't consume the 
elements that are being iterated over, it is passed a reference to each element, and thus the filter predicate uses the 
\code{\&x} pattern to extract the integer itself.)

\blank

You can chain all three things together: start with an iterator, adapt it a few times, and then consume the result. Check it out:

\begin{rustc}
(1..)
    .filter(|&x| x % 2 == 0)
    .filter(|&x| x % 3 == 0)
    .take(5)
    .collect::<Vec<i32>>();
\end{rustc}

This will give you a vector containing \code{6}, \code{12}, \code{18}, \code{24}, and \code{30}.

\blank

This is just a small taste of what iterators, iterator adaptors, and consumers can help you with. There are a number of really 
useful iterators, and you can write your own as well. Iterators provide a safe, efficient way to manipulate all kinds of lists. 
They're a little unusual at first, but if you play with them, you'll get hooked. For a full list of the different iterators and 
consumers, check out the \href{https://doc.rust-lang.org/std/iter/}{iterator module documentation}.


\section{Concurrency}
\label{sec:effective_concurrency}
\input{src/effective_rust/concurrency.tex}

\section{Error Handling}
\label{sec:effective_errorHandling}
Like most programming languages, Rust encourages the programmer to handle errors in a particular way. Generally speaking, 
error handling is divided into two broad categories: exceptions and return values. Rust opts for return values.

\blank

In this section, we intend to provide a comprehensive treatment of how to deal with errors in Rust. More than that, we will 
attempt to introduce error handling one piece at a time so that you'll come away with a solid working knowledge of how everything 
fits together.

\blank

When done naively, error handling in Rust can be verbose and annoying. This section will explore those stumbling blocks and 
demonstrate how to use the standard library to make error handling concise and ergonomic.

\subsection*{Table of Contents}

This section is very long, mostly because we start at the very beginning with sum types and combinators, and try to motivate 
the way Rust does error handling incrementally. As such, programmers with experience in other expressive type systems may 
want to jump around.
%
%     The Basics
%         Unwrapping explained
%         The Option type
%             Composing Option<T> values
%         The Result type
%             Parsing integers
%             The Result type alias idiom
%         A brief interlude: unwrapping isn't evil
%     Working with multiple error types
%         Composing Option and Result
%         The limits of combinators
%         Early returns
%         The try! macro
%         Defining your own error type
%     Standard library traits used for error handling
%         The Error trait
%         The From trait
%         The real try! macro
%         Composing custom error types
%         Advice for library writers
%     Case study: A program to read population data
%         Initial setup
%         Argument parsing
%         Writing the logic
%         Error handling with Box<Error>
%         Reading from stdin
%         Error handling with a custom type
%         Adding functionality
%     The short story
%
\subsection*{The Basics}

You can think of error handling as using case analysis to determine whether a computation was successful or not. As 
you will see, the key to ergonomic error handling is reducing the amount of explicit case analysis the programmer has 
to do while keeping code composable.

\blank

Keeping code composable is important, because without that requirement, we could 
\href{https://doc.rust-lang.org/std/macro.panic!.html}{panic} whenever we come across something unexpected. (\code{panic} 
causes the current task to unwind, and in most cases, the entire program aborts.) Here's an example:

\begin{rustc}
// Guess a number between 1 and 10.
// If it matches the number we had in mind, return true. Else, return false.
fn guess(n: i32) -> bool {
    if n < 1 || n > 10 {
        panic!("Invalid number: {}", n);
    }
    n == 5
}

fn main() {
    guess(11);
}
\end{rustc}

If you try running this code, the program will crash with a message like this:

\begin{verbatim}
thread '<main>' panicked at 'Invalid number: 11', src/bin/panic-simple.rs:5
\end{verbatim}

Here's another example that is slightly less contrived. A program that accepts an integer as an argument, doubles 
it and prints it.

\begin{rustc}
use std::env;

fn main() {
    let mut argv = env::args();
    let arg: String = argv.nth(1).unwrap(); // error 1
    let n: i32 = arg.parse().unwrap(); // error 2
    println!("{}", 2 * n);
}
\end{rustc}

If you give this program zero arguments (error 1) or if the first argument isn't an integer (error 2), the program will 
panic just like in the first example.

\blank

You can think of this style of error handling as similar to a bull running through a china shop. The bull will get to where 
it wants to go, but it will trample everything in the process.

\subsubsection*{Unwrapping explained}

In the previous example, we claimed that the program would simply panic if it reached one of the two error conditions, 
yet, the program does not include an explicit call to \code{panic} like the first example. This is because the \code{panic} 
is embedded in the calls to \code{unwrap}.

\blank

To \enquote{unwrap} something in Rust is to say, \enquote{Give me the result of the computation, and if there was an error, 
panic and stop the program.} It would be better if we showed the code for unwrapping because it is so simple, but to do that, 
we will first need to explore the \option\ and \result\ types. Both of these types have a method called \code{unwrap} 
defined on them.

\subsubsection*{The \option\ type}

The \option\ type is \href{https://doc.rust-lang.org/std/option/enum.Option.html}{defined in the standard library}:

\begin{rustc}
enum Option<T> {
    None,
    Some(T),
}
\end{rustc}

The \option\ type is a way to use Rust's type system to express the \emph{possibility of absence}. Encoding the 
possibility of absence into the type system is an important concept because it will cause the compiler to force the 
programmer to handle that absence. Let's take a look at an example that tries to find a character in a string:

\begin{rustc}
// Searches `haystack` for the Unicode character `needle`. If one is found, the
// byte offset of the character is returned. Otherwise, `None` is returned.
fn find(haystack: &str, needle: char) -> Option<usize> {
    for (offset, c) in haystack.char_indices() {
        if c == needle {
            return Some(offset);
        }
    }
    None
}
\end{rustc}

Notice that when this function finds a matching character, it doesn't only return the \code{offset}. Instead, it returns 
\code{Some(offset)}. \code{Some} is a variant or a \emph{value constructor} for the \option\ type. You can think of 
it as a function with the type \code{fn<T>(value: T) -> Option<T>}. Correspondingly, \none\ is also a value constructor, 
except it has no arguments. You can think of \none\ as a function with the type \code{fn<T>() -> Option<T>}.

\blank

This might seem like much ado about nothing, but this is only half of the story. The other half is \emph{using} the 
\code{find} function we've written. Let's try to use it to find the extension in a file name.

\begin{rustc}
fn main() {
    let file_name = "foobar.rs";
    match find(file_name, '.') {
        None => println!("No file extension found."),
        Some(i) => println!("File extension: {}", &file_name[i+1..]),
    }
}
\end{rustc}

This code uses pattern matching (see \nameref{sec:syntax_patterns}) to do \emph{case analysis} on the \code{Option<usize>} 
returned by the \code{find} function. In fact, case analysis is the only way to get at the value stored inside an 
\code{Option<T>}. This means that you, as the programmer, must handle the case when an \code{Option<T>} is \none\ 
instead of \code{Some(t)}.

\blank

But wait, what about \code{unwrap}, which we used previously? There was no case analysis there! Instead, the case analysis 
was put inside the \code{unwrap} method for you. You could define it yourself if you want:

\begin{rustc}
enum Option<T> {
    None,
    Some(T),
}

impl<T> Option<T> {
    fn unwrap(self) -> T {
        match self {
            Option::Some(val) => val,
            Option::None =>
              panic!("called `Option::unwrap()` on a `None` value"),
        }
    }
}
\end{rustc}

The \code{unwrap} method \emph{abstracts away the case analysis}. This is precisely the thing that makes \code{unwrap} 
ergonomic to use. Unfortunately, that \panic\ means that \code{unwrap} is not composable: it is the bull in the china shop.

\parag{Composing \code{Option<T>} values}

In an example from before, we saw how to use \code{find} to discover the extension in a file name. Of course, not all file 
names have a \code{.} in them, so it's possible that the file name has no extension. This \emph{possibility of absence} is 
encoded into the types using \code{Option<T>}. In other words, the compiler will force us to address the possibility that 
an extension does not exist. In our case, we only print out a message saying as such.

\blank

Getting the extension of a file name is a pretty common operation, so it makes sense to put it into a function:

\begin{rustc}
// Returns the extension of the given file name, where the extension is defined
// as all characters proceeding the first `.`.
// If `file_name` has no `.`, then `None` is returned.
fn extension_explicit(file_name: &str) -> Option<&str> {
    match find(file_name, '.') {
        None => None,
        Some(i) => Some(&file_name[i+1..]),
    }
}
\end{rustc}

(Pro-tip: don't use this code. Use the \href{https://doc.rust-lang.org/std/path/struct.Path.html#method.extension}{extension} 
method in the standard library instead.)

\blank

The code stays simple, but the important thing to notice is that the type of \code{find} forces us to consider the 
possibility of absence. This is a good thing because it means the compiler won't let us accidentally forget about the 
case where a file name doesn't have an extension. On the other hand, doing explicit case analysis like we've done 
in \code{extension\_explicit} every time can get a bit tiresome.

\blank

In fact, the case analysis in \code{extension\_explicit} follows a very common pattern: \emph{map} a function on to 
the value inside of an \code{Option<T>}, unless the option is \none, in which case, return \none.

\blank

Rust has parametric polymorphism, so it is very easy to define a combinator that abstracts this pattern:

\begin{rustc}
fn map<F, T, A>(option: Option<T>, f: F) -> Option<A> where F: FnOnce(T) -> A {
    match option {
        None => None,
        Some(value) => Some(f(value)),
    }
}
\end{rustc}

Indeed, \code{map} is \href{https://doc.rust-lang.org/std/option/enum.Option.html#method.map}{defined as a method} 
on \code{Option<T>} in the standard library.

\blank

Armed with our new combinator, we can rewrite our \code{extension\_explicit} method to get rid of the case analysis:

\begin{rustc}
// Returns the extension of the given file name, where the extension is defined
// as all characters proceeding the first `.`.
// If `file_name` has no `.`, then `None` is returned.
fn extension(file_name: &str) -> Option<&str> {
    find(file_name, '.').map(|i| &file_name[i+1..])
}
\end{rustc}

One other pattern we commonly find is assigning a default value to the case when an \option\ value is \none. 
For example, maybe your program assumes that the extension of a file is \code{rs} even if none is present. As you might 
imagine, the case analysis for this is not specific to file extensions - it can work with any \code{Option<T>}:

\begin{rustc}
fn unwrap_or<T>(option: Option<T>, default: T) -> T {
    match option {
        None => default,
        Some(value) => value,
    }
}
\end{rustc}

The trick here is that the default value must have the same type as the value that might be inside the \code{Option<T>}. 
Using it is dead simple in our case:

\begin{rustc}
fn main() {
    assert_eq!(extension("foobar.csv").unwrap_or("rs"), "csv");
    assert_eq!(extension("foobar").unwrap_or("rs"), "rs");
}
\end{rustc}

(Note that \code{unwrap\_or} is \href{https://doc.rust-lang.org/std/option/enum.Option.html\#method.unwrap\_or}{defined as a method} 
on \code{Option<T>} in the standard library, so we use that here instead of the free-standing function we defined above. 
Don't forget to check out the more general 
\href{https://doc.rust-lang.org/std/option/enum.Option.html\#method.unwrap\_or\_else}{unwrap\_or\_else method}.)

\blank

There is one more combinator that we think is worth paying special attention to: \code{and\_then}. It makes it easy to 
compose distinct computations that admit the \emph{possibility of absence}. For example, much of the code in this section 
is about finding an extension given a file name. In order to do this, you first need the file name which is typically 
extracted from a file path. While most file paths have a file name, not all of them do. For example, \code{.}, \code{..} or 
\code{/}.

\blank

So, we are tasked with the challenge of finding an extension given a file path. Let's start with explicit case analysis:

\begin{rustc}
fn file_path_ext_explicit(file_path: &str) -> Option<&str> {
    match file_name(file_path) {
        None => None,
        Some(name) => match extension(name) {
            None => None,
            Some(ext) => Some(ext),
        }
    }
}

fn file_name(file_path: &str) -> Option<&str> {
  // implementation elided
  unimplemented!()
}
\end{rustc}

You might think that we could use the \code{map} combinator to reduce the case analysis, but its type doesn't quite fit. 
Namely, \code{map} takes a function that does something only with the inner value. The result of that function is then 
\emph{always} rewrapped with \code{Some}. Instead, we need something like \code{map}, but which allows the caller to 
return another \option. Its generic implementation is even simpler than \code{map}:

\begin{rustc}
fn and_then<F, T, A>(option: Option<T>, f: F) -> Option<A>
        where F: FnOnce(T) -> Option<A> {
    match option {
        None => None,
        Some(value) => f(value),
    }
}
\end{rustc}

Now we can rewrite our \code{file\_path\_ext} function without explicit case analysis:

\begin{rustc}
fn file_path_ext(file_path: &str) -> Option<&str> {
    file_name(file_path).and_then(extension)
}
\end{rustc}

The \option\ type has many other combinators \href{https://doc.rust-lang.org/std/option/enum.Option.html}{defined in 
the standard library}. It is a good idea to skim this list and familiarize yourself with what's available—they can often 
reduce case analysis for you. Familiarizing yourself with these combinators will pay dividends because many of them are also 
defined (with similar semantics) for \result, which we will talk about next.

\blank

Combinators make using types like \option\ ergonomic because they reduce explicit case analysis. They are also 
composable because they permit the caller to handle the possibility of absence in their own way. Methods like \code{unwrap} 
remove choices because they will panic if \code{Option<T>} is \none.

\subsubsection*{The \result\ type}

The \result\ type is also \href{https://doc.rust-lang.org/std/result/}{defined in the standard library}:

\begin{rustc}
enum Result<T, E> {
    Ok(T),
    Err(E),
}
\end{rustc}

The \result\ type is a richer version of \option. Instead of expressing the \emph{possibility of absence} like 
\option\ does, \result\ expresses the \emph{possibility of error}. Usually, the error is used to explain why the 
execution of some computation failed. This is a strictly more general form of \option. Consider the following type 
alias, which is semantically equivalent to the real \code{Option<T}> in every way:

\begin{rustc}
type Option<T> = Result<T, ()>;
\end{rustc}

This fixes the second type parameter of \result\ to always be \code{()} (pronounced \enquote{unit} or \enquote{empty 
tuple}). Exactly one value inhabits the \code{()} type: \code{()}. (Yup, the type and value level terms have the same notation!)

\blank

The \result\ type is a way of representing one of two possible outcomes in a computation. By convention, one outcome is 
meant to be expected or \enquote{\code{Ok}} while the other outcome is meant to be unexpected or \enquote{\code{Err}}.

\blank

Just like \option, the \result\ type also has an 
\href{https://doc.rust-lang.org/std/result/enum.Result.html\#method.unwrap}{unwrap method defined} in the standard library. 
Let's define it:

\begin{rustc}
impl<T, E: ::std::fmt::Debug> Result<T, E> {
    fn unwrap(self) -> T {
        match self {
            Result::Ok(val) => val,
            Result::Err(err) =>
              panic!("called `Result::unwrap()` on an `Err` value: {:?}", err),
        }
    }
}
\end{rustc}

This is effectively the same as our definition for \code{Option::unwrap}, except it includes the error value in the \panic\ 
message. This makes debugging easier, but it also requires us to add a 
\href{https://doc.rust-lang.org/std/fmt/trait.Debug.html}{Debug} constraint on the \code{E} type parameter (which represents 
our error type). Since the vast majority of types should satisfy the \code{Debug} constraint, this tends to work out in 
practice. (\code{Debug} on a type simply means that there's a reasonable way to print a human readable description of values 
with that type.)

\blank

OK, let's move on to an example.

\parag{Parsing integers}

The Rust standard library makes converting strings to integers dead simple. It's so easy in fact, that it is very tempting 
to write something like the following:

\begin{rustc}
fn double_number(number_str: &str) -> i32 {
    2 * number_str.parse::<i32>().unwrap()
}

fn main() {
    let n: i32 = double_number("10");
    assert_eq!(n, 20);
}
\end{rustc}

At this point, you should be skeptical of calling \code{unwrap}. For example, if the string doesn't parse as a number, 
you'll get a panic:

\begin{verbatim}
thread '<main>' panicked at 'called `Result::unwrap()` on an `Err` value: ParseIntError { kind: InvalidDigit }', /home/rustbuild/src/rust-buildbot/slave/beta-dist-rustc-linux/build/src/libcore/result.rs:729
\end{verbatim}

This is rather unsightly, and if this happened inside a library you're using, you might be understandably annoyed. Instead, 
we should try to handle the error in our function and let the caller decide what to do. This means changing the return type 
of \code{double\_number}. But to what? Well, that requires looking at the signature of the \code{parse} method in the 
\href{https://doc.rust-lang.org/std/primitive.str.html\#method.parse}{standard library}:

\begin{rustc}
impl str {
    fn parse<F: FromStr>(&self) -> Result<F, F::Err>;
}
\end{rustc}

Hmm. So we at least know that we need to use a \result. Certainly, it's possible that this could have returned an 
\option. After all, a string either parses as a number or it doesn't, right? That's certainly a reasonable way to go, 
but the implementation internally distinguishes why the string didn't parse as an integer. (Whether it's an empty string, an 
invalid digit, too big or too small.) Therefore, using a \result\ makes sense because we want to provide more information 
than simply \enquote{absence.} We want to say why the parsing failed. You should try to emulate this line of reasoning when 
faced with a choice between \option\ and \result. If you can provide detailed error information, then you probably 
should. (We'll see more on this later.)

\blank

OK, but how do we write our return type? The \code{parse} method as defined above is generic over all the different number 
types defined in the standard library. We could (and probably should) also make our function generic, but let's favor 
explicitness for the moment. We only care about \itt, so we need to 
\href{https://doc.rust-lang.org/std/primitive.i32.html}{find its implementation of FromStr} (do a \code{CTRL-F} in your 
browser for \enquote{FromStr}) and look at its associated type \code{Err}. We did this so we can find the concrete error type. 
In this case, it's \href{https://doc.rust-lang.org/std/num/struct.ParseIntError.html}{std::num::ParseIntError}. Finally, 
we can rewrite our function:

\begin{rustc}
use std::num::ParseIntError;

fn double_number(number_str: &str) -> Result<i32, ParseIntError> {
    match number_str.parse::<i32>() {
        Ok(n) => Ok(2 * n),
        Err(err) => Err(err),
    }
}

fn main() {
    match double_number("10") {
        Ok(n) => assert_eq!(n, 20),
        Err(err) => println!("Error: {:?}", err),
    }
}
\end{rustc}

This is a little better, but now we've written a lot more code! The case analysis has once again bitten us.

\blank

Combinators to the rescue! Just like \option, \result\ has lots of combinators defined as methods. There is 
a large intersection of common combinators between \result\ and \option. In particular, map is part of that 
intersection:

\begin{rustc}
use std::num::ParseIntError;

fn double_number(number_str: &str) -> Result<i32, ParseIntError> {
    number_str.parse::<i32>().map(|n| 2 * n)
}

fn main() {
    match double_number("10") {
        Ok(n) => assert_eq!(n, 20),
        Err(err) => println!("Error: {:?}", err),
    }
}
\end{rustc}

The usual suspects are all there for \result, including 
\href{https://doc.rust-lang.org/std/result/enum.Result.html\#method.unwrap\_or}{unwrap\_or} and 
\href{https://doc.rust-lang.org/std/result/enum.Result.html\#method.and\_then}{and\_then}. Additionally, since \result\ 
has a second type parameter, there are combinators that affect only the error type, such as 
\href{https://doc.rust-lang.org/std/result/enum.Result.html\#method.map\_err}{map\_err} (instead of \code{map}) and 
\href{https://doc.rust-lang.org/std/result/enum.Result.html\#method.or\_else}{or\_else} (instead of \code{and\_then}).

\parag{The \result\ type alias idiom}

In the standard library, you may frequently see types like \code{Result<i32>}. But wait, we defined \result\ to have 
two type parameters. How can we get away with only specifying one? The key is to define a \result\ type alias that 
fixes one of the type parameters to a particular type. Usually the fixed type is the error type. For example, our previous 
example parsing integers could be rewritten like this:

\begin{rustc}
use std::num::ParseIntError;
use std::result;

type Result<T> = result::Result<T, ParseIntError>;

fn double_number(number_str: &str) -> Result<i32> {
    unimplemented!();
}
\end{rustc}

Why would we do this? Well, if we have a lot of functions that could return \code{ParseIntError}, then it's much more 
convenient to define an alias that always uses \code{ParseIntError} so that we don't have to write it out all the time.

\blank

The most prominent place this idiom is used in the standard library is with 
\href{https://doc.rust-lang.org/std/io/type.Result.html}{io::Result}. Typically, one writes \code{io::Result<T>}, which 
makes it clear that you're using the \code{io} module's type alias instead of the plain definition from \code{std::result}. 
(This idiom is also used for \href{https://doc.rust-lang.org/std/fmt/type.Result.html}{fmt::Result}.)

\subsubsection*{A brief interlude: unwrapping isn't evil}

If you've been following along, you might have noticed that I've taken a pretty hard line against calling methods like 
\code{unwrap} that could \code{panic} and abort your program. \emph{Generally speaking}, this is good advice.

\blank

However, \code{unwrap} can still be used judiciously. What exactly justifies use of \code{unwrap} is somewhat of a grey 
area and reasonable people can disagree. I'll summarize some of my \emph{opinions} on the matter.

\begin{itemize}
  \item{\textbf{In examples and quick 'n' dirty code.} Sometimes you're writing examples or a quick program, and error 
      handling simply isn't important. Beating the convenience of \code{unwrap} can be hard in such scenarios, so it is 
      very appealing.}
  \item{\textbf{When panicking indicates a bug in the program.} When the invariants of your code should prevent a certain 
      case from happening (like, say, popping from an empty stack), then panicking can be permissible. This is because it 
      exposes a bug in your program. This can be explicit, like from an \code{assert!} failing, or it could be because your 
      index into an array was out of bounds.}
\end{itemize}

This is probably not an exhaustive list. Moreover, when using an \option, it is often better to use its 
\href{https://doc.rust-lang.org/std/option/enum.Option.html\#method.expect}{expect} method. \code{expect} does exactly the 
same thing as \code{unwrap}, except it prints a message you give to \code{expect}. This makes the resulting panic a bit nicer 
to deal with, since it will show your message instead of \enquote{called \code{unwrap} on a \none\ value.}

\blank

My advice boils down to this: use good judgment. There's a reason why the words \enquote{never do X} or \enquote{Y is 
considered harmful} don't appear in my writing. There are trade offs to all things, and it is up to you as the programmer 
to determine what is acceptable for your use cases. My goal is only to help you evaluate trade offs as accurately as possible.

\blank

Now that we've covered the basics of error handling in Rust, and explained unwrapping, let's start exploring more of the 
standard library.

\subsection*{Working with multiple error types}

Thus far, we've looked at error handling where everything was either an \code{Option<T>} or a \code{Result<T, SomeError>}. 
But what happens when you have both an \option\ and a \result? Or what if you have a \code{Result<T, Error1>} 
and a \code{Result<T, Error2>}? Handling composition of distinct error types is the next challenge in front of us, and it 
will be the major theme throughout the rest of this section.

\subsubsection*{Composing \option\ and \result}

So far, I've talked about combinators defined for \option\ and combinators defined for \result. We can use 
these combinators to compose results of different computations without doing explicit case analysis.

\blank

Of course, in real code, things aren't always as clean. Sometimes you have a mix of \option\ and \result\ types. 
Must we resort to explicit case analysis, or can we continue using combinators?

\blank

For now, let's revisit one of the first examples in this section:

\begin{rustc}
use std::env;

fn main() {
    let mut argv = env::args();
    let arg: String = argv.nth(1).unwrap(); // error 1
    let n: i32 = arg.parse().unwrap(); // error 2
    println!("{}", 2 * n);
}
\end{rustc}

Given our new found knowledge of \option, \result\ and their various combinators, we should try to rewrite this so that 
errors are handled properly and the program doesn't panic if there's an error.

\blank

The tricky aspect here is that \code{argv.nth(1)} produces an \option\ while \code{arg.parse()} produces a \result. These 
aren't directly composable. When faced with both an \option\ and a \result, the solution is usually to convert the \option\ 
to a \result. In our case, the absence of a command line parameter (from \code{env::args()}) means the user didn't invoke 
the program correctly. We could use a \String\ to describe the error. Let's try:

\begin{rustc}
use std::env;

fn double_arg(mut argv: env::Args) -> Result<i32, String> {
    argv.nth(1)
        .ok_or("Please give at least one argument".to_owned())
        .and_then(|arg| arg.parse::<i32>().map_err(|err| err.to_string()))
        .map(|n| 2 * n)
}

fn main() {
    match double_arg(env::args()) {
        Ok(n) => println!("{}", n),
        Err(err) => println!("Error: {}", err),
    }
}
\end{rustc}

There are a couple new things in this example. The first is the use of the 
\href{https://doc.rust-lang.org/std/option/enum.Option.html\#method.ok\_or}{Option::ok\_or} combinator. This is one way to 
convert an \option\ into a \result. The conversion requires you to specify what error to use if \option\ is \none. Like 
the other combinators we've seen, its definition is very simple:

\begin{rustc}
fn ok_or<T, E>(option: Option<T>, err: E) -> Result<T, E> {
    match option {
        Some(val) => Ok(val),
        None => Err(err),
    }
}
\end{rustc}

The other new combinator used here is \href{https://doc.rust-lang.org/std/result/enum.Result.html\#method.map\_err}
{Result::map\_err}. This is like \code{Result::map}, except it maps a function on to the error portion of a \result\ value. 
If the \result\ is an \code{Ok(...)} value, then it is returned unmodified.

\blank

We use \code{map\_err} here because it is necessary for the error types to remain the same (because of our use of 
\code{and\_then}). Since we chose to convert the \code{Option<String>} (from \code{argv.nth(1)}) to a \code{Result<String, String>},
we must also convert the \code{ParseIntError} from \code{arg.parse()} to a \String.

\subsubsection*{The limits of combinators}

Doing IO and parsing input is a very common task, and it's one that I personally have done a lot of in Rust. Therefore, we 
will use (and continue to use) IO and various parsing routines to exemplify error handling.

\blank

Let's start simple. We are tasked with opening a file, reading all of its contents and converting its contents to a number. 
Then we multiply it by \code{2} and print the output.

\blank

Although I've tried to convince you not to use \code{unwrap}, it can be useful to first write your code using \code{unwrap}. 
It allows you to focus on your problem instead of the error handling, and it exposes the points where proper error handling 
need to occur. Let's start there so we can get a handle on the code, and then refactor it to use better error handling.

\begin{rustc}
use std::fs::File;
use std::io::Read;
use std::path::Path;

fn file_double<P: AsRef<Path>>(file_path: P) -> i32 {
    let mut file = File::open(file_path).unwrap(); // error 1
    let mut contents = String::new();
    file.read_to_string(&mut contents).unwrap(); // error 2
    let n: i32 = contents.trim().parse().unwrap(); // error 3
    2 * n
}

fn main() {
    let doubled = file_double("foobar");
    println!("{}", doubled);
}
\end{rustc}

(N.B. The \code{AsRef<Path>} is used because those are the 
\href{https://doc.rust-lang.org/std/fs/struct.File.html\#method.open}{same bounds used on std::fs::File::open}. This makes 
it ergonomic to use any kind of string as a file path.)

\blank

There are three different errors that can occur here:

\begin{enumerate}
  \item{A problem opening the file.}
  \item{A problem reading data from the file.}
  \item{A problem parsing the data as a number.}
\end{enumerate}

The first two problems are described via the \href{https://doc.rust-lang.org/std/io/struct.Error.html}{std::io::Error} type. 
We know this because of the return types of 
\href{https://doc.rust-lang.org/std/fs/struct.File.html\#method.open}{std::fs::File::open} and 
\href{https://doc.rust-lang.org/std/io/trait.Read.html\#method.read\_to\_string}{std::io::Read::read\_to\_string}. (Note that 
they both use the \result\ type alias idiom described previously. If you click on the \result\ type, you'll 
\href{https://doc.rust-lang.org/std/io/type.Result.html}{see the type alias}, and consequently, the underlying \code{io::Error} 
type.) The third problem is described by the 
\href{https://doc.rust-lang.org/std/num/struct.ParseIntError.html}{std::num::ParseIntError} type. The \code{io::Error} type 
in particular is \emph{pervasive} throughout the standard library. You will see it again and again.

\blank

Let's start the process of refactoring the \code{file\_double} function. To make this function composable with other 
components of the program, it should not panic if any of the above error conditions are met. Effectively, this means t
hat the function should \emph{return an error} if any of its operations fail. Our problem is that the return type of 
\code{file\_double} is \itt, which does not give us any useful way of reporting an error. Thus, we must start by changing 
the return type from \itt\ to something else.

\blank

The first thing we need to decide: should we use \option\ or \result? We certainly could use \option\ very easily. If any 
of the three errors occur, we could simply return \none. This will work and it is better than panicking, but we can do a 
lot better. Instead, we should pass some detail about the error that occurred. Since we want to express the possibility of 
error, we should use \code{Result<i32, E>}. But what should \code{E} be? Since two different types of errors can occur, we 
need to convert them to a common type. One such type is \String. Let's see how that impacts our code:

\begin{rustc}
use std::fs::File;
use std::io::Read;
use std::path::Path;

fn file_double<P: AsRef<Path>>(file_path: P) -> Result<i32, String> {
    File::open(file_path)
         .map_err(|err| err.to_string())
         .and_then(|mut file| {
              let mut contents = String::new();
              file.read_to_string(&mut contents)
                  .map_err(|err| err.to_string())
                  .map(|_| contents)
         })
         .and_then(|contents| {
              contents.trim().parse::<i32>()
                      .map_err(|err| err.to_string())
         })
         .map(|n| 2 * n)
}

fn main() {
    match file_double("foobar") {
        Ok(n) => println!("{}", n),
        Err(err) => println!("Error: {}", err),
    }
}
\end{rustc}

This code looks a bit hairy. It can take quite a bit of practice before code like this becomes easy to write. The way we 
write it is by following the types. As soon as we changed the return type of \code{file\_double} to \code{Result<i32, String>}, 
we had to start looking for the right combinators. In this case, we only used three different combinators: 
\code{and\_then}, \code{map} and \code{map\_err}.

\blank

\code{and\_then} is used to chain multiple computations where each computation could return an error. After opening the file, 
there are two more computations that could fail: reading from the file and parsing the contents as a number. Correspondingly, 
there are two calls to \code{and\_then}.

\blank

\code{map} is used to apply a function to the \code{Ok(...)} value of a \result. For example, the very last call to \code{map}
multiplies the \code{Ok(...)} value (which is an \itt) by \code{2}. If an error had occurred before that point, this operation 
would have been skipped because of how \code{map} is defined.

\blank

\code{map\_err} is the trick that makes all of this work. \code{map\_err} is like \code{map}, except it applies a function to 
the \code{Err(...)} value of a \result. In this case, we want to convert all of our errors to one type: \String. Since both 
\code{io::Error} and \code{num::ParseIntError} implement \code{ToString}, we can call the \code{to\_string()} method to 
convert them.

\blank

With all of that said, the code is still hairy. Mastering use of combinators is important, but they have their limits. 
Let's try a different approach: early returns.

\subsubsection*{Early returns}

I'd like to take the code from the previous section and rewrite it using \emph{early returns}. Early returns let you exit 
the function early. We can't return early in \code{file\_double} from inside another closure, so we'll need to revert back 
to explicit case analysis.

\begin{rustc}
use std::fs::File;
use std::io::Read;
use std::path::Path;

fn file_double<P: AsRef<Path>>(file_path: P) -> Result<i32, String> {
    let mut file = match File::open(file_path) {
        Ok(file) => file,
        Err(err) => return Err(err.to_string()),
    };
    let mut contents = String::new();
    if let Err(err) = file.read_to_string(&mut contents) {
        return Err(err.to_string());
    }
    let n: i32 = match contents.trim().parse() {
        Ok(n) => n,
        Err(err) => return Err(err.to_string()),
    };
    Ok(2 * n)
}

fn main() {
    match file_double("foobar") {
        Ok(n) => println!("{}", n),
        Err(err) => println!("Error: {}", err),
    }
}
\end{rustc}

Reasonable people can disagree over whether this code is better than the code that uses combinators, but if you aren't 
familiar with the combinator approach, this code looks simpler to read to me. It uses explicit case analysis with \code{match} 
and \code{if let}. If an error occurs, it simply stops executing the function and returns the error (by converting it to a 
string).

\blank

Isn't this a step backwards though? Previously, we said that the key to ergonomic error handling is reducing explicit case 
analysis, yet we've reverted back to explicit case analysis here. It turns out, there are \emph{multiple} ways to reduce 
explicit case analysis. Combinators aren't the only way.

\subsubsection*{The \code{try!} macro}

A cornerstone of error handling in Rust is the \code{try!} macro. The \code{try!} macro abstracts case analysis like 
combinators, but unlike combinators, it also abstracts control flow. Namely, it can abstract the \emph{early return 
pattern} seen above.

\blank

Here is a simplified definition of a \code{try!} macro:

\begin{rustc}
macro_rules! try {
    ($e:expr) => (match $e {
        Ok(val) => val,
        Err(err) => return Err(err),
    });
}
\end{rustc}

(The \href{https://doc.rust-lang.org/std/macro.try!.html}{real definition} is a bit more sophisticated. We will address 
that later.)

\blank

Using the \code{try!} macro makes it very easy to simplify our last example. Since it does the case analysis and the early 
return for us, we get tighter code that is easier to read:

\begin{rustc}
use std::fs::File;
use std::io::Read;
use std::path::Path;

fn file_double<P: AsRef<Path>>(file_path: P) -> Result<i32, String> {
    let mut file = try!(File::open(file_path).map_err(|e| e.to_string()));
    let mut contents = String::new();
    try!(file.read_to_string(&mut contents).map_err(|e| e.to_string()));
    let n = try!(contents.trim().parse::<i32>().map_err(|e| e.to_string()));
    Ok(2 * n)
}

fn main() {
    match file_double("foobar") {
        Ok(n) => println!("{}", n),
        Err(err) => println!("Error: {}", err),
    }
}
\end{rustc}

The \code{map\_err} calls are still necessary given our definition of \code{try!}. This is because the error types still 
need to be converted to \String. The good news is that we will soon learn how to remove those \code{map\_err} calls! The bad 
news is that we will need to learn a bit more about a couple important traits in the standard library before we can remove 
the \code{map\_err} calls.

\subsubsection*{Defining your own error type}

Before we dive into some of the standard library error traits, I'd like to wrap up this section by removing the use of \String\ 
as our error type in the previous examples.

\blank

Using \String\ as we did in our previous examples is convenient because it's easy to convert errors to strings, or even make 
up your own errors as strings on the spot. However, using \String\ for your errors has some downsides.

\blank

The first downside is that the error messages tend to clutter your code. It's possible to define the error messages elsewhere, 
but unless you're unusually disciplined, it is very tempting to embed the error message into your code. Indeed, we did exactly 
this in a previous example.

\blank

The second and more important downside is that \String s are \emph{lossy}. That is, if all errors are converted to strings, 
then the errors we pass to the caller become completely opaque. The only reasonable thing the caller can do with a \String\ 
error is show it to the user. Certainly, inspecting the string to determine the type of error is not robust. (Admittedly, 
this downside is far more important inside of a library as opposed to, say, an application.)

\blank

For example, the \code{io::Error} type embeds an \href{https://doc.rust-lang.org/std/io/enum.ErrorKind.html}{io::ErrorKind}, 
which is structured data that represents what went wrong during an IO operation. This is important because you might want to 
react differently depending on the error. (e.g., A \code{BrokenPipe} error might mean quitting your program gracefully while 
a \code{NotFound} error might mean exiting with an error code and showing an error to the user.) With \code{io::ErrorKind}, 
the caller can examine the type of an error with case analysis, which is strictly superior to trying to tease out the details 
of an error inside of a \String.

\blank

Instead of using a \String\ as an error type in our previous example of reading an integer from a file, we can define our 
own error type that represents errors with \emph{structured data}. We endeavor to not drop information from underlying errors 
in case the caller wants to inspect the details.

\blank

The ideal way to represent \emph{one of many possibilities} is to define our own sum type using \enum. In our case, an error 
is either an \code{io::Error} or a \code{num::ParseIntError}, so a natural definition arises:

\begin{rustc}
use std::io;
use std::num;

// We derive `Debug` because all types should probably derive `Debug`.
// This gives us a reasonable human readable description of `CliError` values.
#[derive(Debug)]
enum CliError {
    Io(io::Error),
    Parse(num::ParseIntError),
}
\end{rustc}

Tweaking our code is very easy. Instead of converting errors to strings, we simply convert them to our \code{CliError} type 
using the corresponding value constructor:

\begin{rustc}
use std::fs::File;
use std::io::Read;
use std::path::Path;

fn file_double<P: AsRef<Path>>(file_path: P) -> Result<i32, CliError> {
    let mut file = try!(File::open(file_path).map_err(CliError::Io));
    let mut contents = String::new();
    try!(file.read_to_string(&mut contents).map_err(CliError::Io));
    let n: i32 = try!(contents.trim().parse().map_err(CliError::Parse));
    Ok(2 * n)
}

fn main() {
    match file_double("foobar") {
        Ok(n) => println!("{}", n),
        Err(err) => println!("Error: {:?}", err),
    }
}
\end{rustc}

The only change here is switching \code{map\_err(|e| e.to\_string())} (which converts errors to strings) to 
\code{map\_err(CliError::Io)} or \code{map\_err(CliError::Parse)}. The caller gets to decide the level of detail to report 
to the user. In effect, using a \String\ as an error type removes choices from the caller while using a custom \enum\ error 
type like \code{CliError} gives the caller all of the conveniences as before in addition to structured data describing the 
error.

\blank

A rule of thumb is to define your own error type, but a \String\ error type will do in a pinch, particularly if you're 
writing an application. If you're writing a library, defining your own error type should be strongly preferred so that 
you don't remove choices from the caller unnecessarily.

\subsection*{Standard library traits used for error handling}

The standard library defines two integral traits for error handling: 
\href{https://doc.rust-lang.org/std/error/trait.Error.html}{std::error::Error} and 
\href{https://doc.rust-lang.org/std/convert/trait.From.html}{std::convert::From}. While \code{Error} is designed 
specifically for generically describing errors, the \code{From} trait serves a more general role for converting values 
between two distinct types.

\subsubsection*{The \code{Error} trait}

The \code{Error} trait is \href{https://doc.rust-lang.org/std/error/trait.Error.html}{defined in the standard library}:

\begin{rustc}
use std::fmt::{Debug, Display};

trait Error: Debug + Display {
  /// A short description of the error.
  fn description(&self) -> &str;

  /// The lower level cause of this error, if any.
  fn cause(&self) -> Option<&Error> { None }
}
\end{rustc}

This trait is super generic because it is meant to be implemented for all types that represent errors. This will prove 
useful for writing composable code as we'll see later. Otherwise, the trait allows you to do at least the following things:

\begin{itemize}
  \item{Obtain a \code{Debug} representation of the error.}
  \item{Obtain a user-facing \code{Display} representation of the error.}
  \item{Obtain a short description of the error (via the \code{description} method).}
  \item{Inspect the causal chain of an error, if one exists (via the \code{cause} method).}
\end{itemize}

The first two are a result of \code{Error} requiring \code{impl}s for both \code{Debug} and \code{Display}. The latter two 
are from the two methods defined on \code{Error}. The power of \code{Error} comes from the fact that all error types impl
\code{Error}, which means errors can be existentially quantified as a trait object (see \nameref{sec:syntax_traitObjects}). 
This manifests as either \code{Box<Error>} or \code{\&Error}. Indeed, the cause method returns an \code{\&Error}, which is 
itself a trait object. We'll revisit the \code{Error} trait's utility as a trait object later.

\blank

For now, it suffices to show an example implementing the \code{Error} trait. Let's use the error type we defined in the 
previous section:

\begin{rustc}
use std::io;
use std::num;

// We derive `Debug` because all types should probably derive `Debug`.
// This gives us a reasonable human readable description of `CliError` values.
#[derive(Debug)]
enum CliError {
    Io(io::Error),
    Parse(num::ParseIntError),
}
\end{rustc}

This particular error type represents the possibility of two types of errors occurring: an error dealing with I/O or an 
error converting a string to a number. The error could represent as many error types as you want by adding new variants to 
the \enum\ definition.

\blank

Implementing \code{Error} is pretty straight-forward. It's mostly going to be a lot explicit case analysis.

\begin{rustc}
use std::error;
use std::fmt;

impl fmt::Display for CliError {
    fn fmt(&self, f: &mut fmt::Formatter) -> fmt::Result {
        match *self {
            // Both underlying errors already impl `Display`, so we defer to
            // their implementations.
            CliError::Io(ref err) => write!(f, "IO error: {}", err),
            CliError::Parse(ref err) => write!(f, "Parse error: {}", err),
        }
    }
}

impl error::Error for CliError {
    fn description(&self) -> &str {
        // Both underlying errors already impl `Error`, so we defer to their
        // implementations.
        match *self {
            CliError::Io(ref err) => err.description(),
            CliError::Parse(ref err) => err.description(),
        }
    }

    fn cause(&self) -> Option<&error::Error> {
        match *self {
            // N.B. Both of these implicitly cast `err` from their concrete
            // types (either `&io::Error` or `&num::ParseIntError`)
            // to a trait object `&Error`. This works because both error types
            // implement `Error`.
            CliError::Io(ref err) => Some(err),
            CliError::Parse(ref err) => Some(err),
        }
    }
}
\end{rustc}

We note that this is a very typical implementation of \code{Error}: match on your different error types and satisfy the 
contracts defined for \code{description} and \code{cause}.

\subsubsection*{The \code{From} trait}

The \code{std::convert::From} trait is \href{https://doc.rust-lang.org/std/convert/trait.From.html}{defined in the 
standard library}:

\begin{rustc}
trait From<T> {
    fn from(T) -> Self;
}
\end{rustc}

Deliciously simple, yes? \code{From} is very useful because it gives us a generic way to talk about conversion from a 
particular type \code{T} to some other type (in this case, \enquote{some other type} is the subject of the impl, or 
\code{Self}). The crux of \code{From} is the \href{https://doc.rust-lang.org/std/convert/trait.From.html}{set of 
implementations provided by the standard library}.

\blank

Here are a few simple examples demonstrating how \code{From} works:

\begin{rustc}
let string: String = From::from("foo");
let bytes: Vec<u8> = From::from("foo");
let cow: ::std::borrow::Cow<str> = From::from("foo");
\end{rustc}

OK, so \code{From} is useful for converting between strings. But what about errors? It turns out, there is one critical impl:

\begin{rustc}
impl<'a, E: Error + 'a> From<E> for Box<Error + 'a>
\end{rustc}

This impl says that for \emph{any} type that impls \code{Error}, we can convert it to a trait object \code{Box<Error>}. This 
may not seem terribly surprising, but it is useful in a generic context.

\blank

Remember the two errors we were dealing with previously? Specifically, \code{io::Error} and \code{num::ParseIntError}. Since 
both impl \code{Error}, they work with \code{From}:

\begin{rustc}
use std::error::Error;
use std::fs;
use std::io;
use std::num;

// We have to jump through some hoops to actually get error values.
let io_err: io::Error = io::Error::last_os_error();
let parse_err: num::ParseIntError = "not a number".parse::<i32>().unwrap_err();

// OK, here are the conversions.
let err1: Box<Error> = From::from(io_err);
let err2: Box<Error> = From::from(parse_err);
\end{rustc}

There is a really important pattern to recognize here. Both \code{err1} and \code{err2} have the same type. This is because 
they are existentially quantified types, or trait objects. In particular, their underlying type is erased from the 
compiler's knowledge, so it truly sees \code{err1} and \code{err2} as exactly the same. Additionally, we constructed \code{err1} 
and \code{err2} using precisely the same function call: \code{From::from}. This is because \code{From::from} is overloaded 
on both its argument and its return type.

\blank

This pattern is important because it solves a problem we had earlier: it gives us a way to reliably convert errors to the 
same type using the same function.

\blank

Time to revisit an old friend; the \code{try!} macro.

\subsubsection*{The real \code{try!} macro}

Previously, we presented this definition of \code{try!}:

\begin{rustc}
macro_rules! try {
    ($e:expr) => (match $e {
        Ok(val) => val,
        Err(err) => return Err(err),
    });
}
\end{rustc}

This is not its real definition. Its real definition is \href{https://doc.rust-lang.org/std/macro.try!.html}{in the 
standard library}:

\begin{rustc}
macro_rules! try {
    ($e:expr) => (match $e {
        Ok(val) => val,
        Err(err) => return Err(::std::convert::From::from(err)),
    });
}
\end{rustc}

There's one tiny but powerful change: the error value is passed through \code{From::from}. This makes the \code{try!} macro 
a lot more powerful because it gives you automatic type conversion for free.

\blank

Armed with our more powerful \code{try!} macro, let's take a look at code we wrote previously to read a file and convert 
its contents to an integer:

\begin{rustc}
use std::fs::File;
use std::io::Read;
use std::path::Path;

fn file_double<P: AsRef<Path>>(file_path: P) -> Result<i32, String> {
    let mut file = try!(File::open(file_path).map_err(|e| e.to_string()));
    let mut contents = String::new();
    try!(file.read_to_string(&mut contents).map_err(|e| e.to_string()));
    let n = try!(contents.trim().parse::<i32>().map_err(|e| e.to_string()));
    Ok(2 * n)
}
\end{rustc}

Earlier, we promised that we could get rid of the \code{map\_err} calls. Indeed, all we have to do is pick a type that 
\code{From} works with. As we saw in the previous section, \code{From} has an impl that lets it convert any error type into 
a \code{Box<Error>}:

\begin{rustc}
use std::error::Error;
use std::fs::File;
use std::io::Read;
use std::path::Path;

fn file_double<P: AsRef<Path>>(file_path: P) -> Result<i32, Box<Error>> {
    let mut file = try!(File::open(file_path));
    let mut contents = String::new();
    try!(file.read_to_string(&mut contents));
    let n = try!(contents.trim().parse::<i32>());
    Ok(2 * n)
}
\end{rustc}

We are getting very close to ideal error handling. Our code has very little overhead as a result from error handling because 
the \code{try!} macro encapsulates three things simultaneously:

\begin{enumerate}
  \item{Case analysis.}
  \item{Control flow.}
  \item{Error type conversion.}
\end{enumerate}

When all three things are combined, we get code that is unencumbered by combinators, calls to \code{unwrap} or case analysis.

\blank

There's one little nit left: the \code{Box<Error>} type is \emph{opaque}. If we return a \code{Box<Error>} to the caller, 
the caller can't (easily) inspect underlying error type. The situation is certainly better than \String\ because the caller 
can call methods like 
\href{https://doc.rust-lang.org/std/error/trait.Error.html\#tymethod.description}{description} and 
\href{https://doc.rust-lang.org/std/error/trait.Error.html\#method.cause}{cause}, but the limitation remains: \code{Box<Error>} 
is opaque. (N.B. This isn't entirely true because Rust does have runtime reflection, which is useful in some scenarios that are
\href{https://crates.io/crates/error}{beyond the scope of this section}.)

\blank

It's time to revisit our custom \code{CliError} type and tie everything together.

\subsubsection*{Composing custom error types}

In the last section, we looked at the real \code{try!} macro and how it does automatic type conversion for us by calling 
\code{From::from} on the error value. In particular, we converted errors to \code{Box<Error>}, which works, but the type 
is opaque to callers.

\blank

To fix this, we use the same remedy that we're already familiar with: a custom error type. Once again, here is the code 
that reads the contents of a file and converts it to an integer:

\begin{rustc}
use std::fs::File;
use std::io::{self, Read};
use std::num;
use std::path::Path;

// We derive `Debug` because all types should probably derive `Debug`.
// This gives us a reasonable human readable description of `CliError` values.
#[derive(Debug)]
enum CliError {
    Io(io::Error),
    Parse(num::ParseIntError),
}

fn file_double_verbose<P: AsRef<Path>>(file_path: P) -> Result<i32, CliError> {
    let mut file = try!(File::open(file_path).map_err(CliError::Io));
    let mut contents = String::new();
    try!(file.read_to_string(&mut contents).map_err(CliError::Io));
    let n: i32 = try!(contents.trim().parse().map_err(CliError::Parse));
    Ok(2 * n)
}
\end{rustc}

Notice that we still have the calls to \code{map\_err}. Why? Well, recall the definitions of \code{try!} and \code{From}. 
The problem is that there is no \code{From} impl that allows us to convert from error types like \code{io::Error} and 
\code{num::ParseIntError} to our own custom \code{CliError}. Of course, it is easy to fix this! Since we defined 
\code{CliError}, we can impl \code{From} with it:

\begin{rustc}
use std::io;
use std::num;

impl From<io::Error> for CliError {
    fn from(err: io::Error) -> CliError {
        CliError::Io(err)
    }
}

impl From<num::ParseIntError> for CliError {
    fn from(err: num::ParseIntError) -> CliError {
        CliError::Parse(err)
    }
}
\end{rustc}

All these impls are doing is teaching \code{From} how to create a \code{CliError} from other error types. In our case, 
construction is as simple as invoking the corresponding value constructor. Indeed, it is typically this easy.

\blank

We can finally rewrite \code{file\_double}:

\begin{rustc}
use std::fs::File;
use std::io::Read;
use std::path::Path;

fn file_double<P: AsRef<Path>>(file_path: P) -> Result<i32, CliError> {
    let mut file = try!(File::open(file_path));
    let mut contents = String::new();
    try!(file.read_to_string(&mut contents));
    let n: i32 = try!(contents.trim().parse());
    Ok(2 * n)
}
\end{rustc}

The only thing we did here was remove the calls to \code{map\_err}. They are no longer needed because the \code{try!} 
macro invokes \code{From::from} on the error value. This works because we've provided \code{From} impls for all the error 
types that could appear.

\blank

If we modified our \code{file\_double} function to perform some other operation, say, convert a string to a float, then 
we'd need to add a new variant to our error type:

\begin{rustc}
use std::io;
use std::num;

enum CliError {
    Io(io::Error),
    ParseInt(num::ParseIntError),
    ParseFloat(num::ParseFloatError),
}
\end{rustc}

And add a new \code{From} impl:

\begin{rustc}
use std::num;

impl From<num::ParseFloatError> for CliError {
    fn from(err: num::ParseFloatError) -> CliError {
        CliError::ParseFloat(err)
    }
}
\end{rustc}

And that's it!

\subsubsection*{Advice for library writers}

If your library needs to report custom errors, then you should probably define your own error type. It's up to you whether 
or not to expose its representation (like \href{https://doc.rust-lang.org/std/io/enum.ErrorKind.html}{ErrorKind}) or keep it 
hidden (like \href{https://doc.rust-lang.org/std/num/struct.ParseIntError.html}{ParseIntError}). Regardless of how you do it, 
it's usually good practice to at least provide some information about the error beyond its \String\ representation. But 
certainly, this will vary depending on use cases.

\blank

At a minimum, you should probably implement the \href{https://doc.rust-lang.org/std/error/trait.Error.html}{Error} trait. 
This will give users of your library some minimum flexibility for composing errors. Implementing the Error trait also means 
that users are guaranteed the ability to obtain a string representation of an error (because it requires impls for both 
\code{fmt::Debug} and \code{fmt::Display}).

\blank

Beyond that, it can also be useful to provide implementations of \code{From} on your error types. This allows you (the 
library author) and your users to compose more detailed errors. For example, 
\href{http://burntsushi.net/rustdoc/csv/enum.Error.html}{csv::Error} provides \code{From} impls for both \code{io::Error} 
and \code{byteorder::Error}.

\blank

Finally, depending on your tastes, you may also want to define a \result\ type alias, particularly if your library defines a 
single error type. This is used in the standard library for \href{https://doc.rust-lang.org/std/io/type.Result.html}{io::Result} 
and \href{https://doc.rust-lang.org/std/fmt/type.Result.html}{fmt::Result}.

\subsection*{Case study: A program to read population data}

This section was long, and depending on your background, it might be rather dense. While there is plenty of example code 
to go along with the prose, most of it was specifically designed to be pedagogical. So, we're going to do something new: a 
case study.

\blank

For this, we're going to build up a command line program that lets you query world population data. The objective is simple: 
you give it a location and it will tell you the population. Despite the simplicity, there is a lot that can go wrong!

\blank

The data we'll be using comes from the \href{https://github.com/petewarden/dstkdata}{Data Science Toolkit}. I've prepared 
some data from it for this exercise. You can either grab the 
\href{http://burntsushi.net/stuff/worldcitiespop.csv.gz}{world population data} (41MB gzip compressed, 145MB uncompressed) 
or only the \href{http://burntsushi.net/stuff/uscitiespop.csv.gz}{US population data} (2.2MB gzip compressed, 7.2MB uncompressed).

\blank

Up until now, we've kept the code limited to Rust's standard library. For a real task like this though, we'll want to at 
least use something to parse CSV data, parse the program arguments and decode that stuff into Rust types automatically. For 
that, we'll use the \href{https://crates.io/crates/csv}{csv}, and 
\href{https://crates.io/crates/rustc-serialize}{rustc-serialize} crates.

\subsubsection*{Initial setup}

We're not going to spend a lot of time on setting up a project with Cargo because it is already covered well in the Cargo section 
(see \nameref{sec:gettingstarted_helloCargo}) and \href{http://doc.crates.io/guide.html}{Cargo's documentation}.

\blank

To get started from scratch, run \code{cargo new --bin city-pop} and make sure your \code{Cargo.toml} looks something like this:

\begin{verbatim}
[package]
name = "city-pop"
version = "0.1.0"
authors = ["Andrew Gallant <jamslam@gmail.com>"]

[[bin]]
name = "city-pop"

[dependencies]
csv = "0.*"
rustc-serialize = "0.*"
getopts = "0.*"
\end{verbatim}

You should already be able to run:

\begin{verbatim}
cargo build --release
./target/release/city-pop
# Outputs: Hello, world!
\end{verbatim}

\subsubsection*{Argument parsing}

Let's get argument parsing out of the way. We won't go into too much detail on Getopts, but there is 
\href{http://doc.rust-lang.org/getopts/getopts/}{some good documentation} describing it. The short story is that Getopts 
generates an argument parser and a help message from a vector of options (The fact that it is a vector is hidden behind a 
struct and a set of methods). Once the parsing is done, we can decode the program arguments into a Rust struct. From there, 
we can get information about the flags, for instance, whether they were passed in, and what arguments they had. Here's our
 program with the appropriate \code{extern crate} statements, and the basic argument setup for Getopts:

\begin{rustc}
extern crate getopts;
extern crate rustc_serialize;

use getopts::Options;
use std::env;

fn print_usage(program: &str, opts: Options) {
    println!("{}", opts.usage(&format!("Usage: {} [options] <data-path> <city>", program)));
}

fn main() {
    let args: Vec<String> = env::args().collect();
    let program = args[0].clone();

    let mut opts = Options::new();
    opts.optflag("h", "help", "Show this usage message.");

    let matches = match opts.parse(&args[1..]) {
        Ok(m)  => { m }
        Err(e) => { panic!(e.to_string()) }
    };
    if matches.opt_present("h") {
        print_usage(&program, opts);
        return;
    }
    let data_path = args[1].clone();
    let city = args[2].clone();

    // Do stuff with information
}
\end{rustc}

First, we get a vector of the arguments passed into our program. We then store the first one, knowing that it is our 
program's name. Once that's done, we set up our argument flags, in this case a simplistic help message flag. Once we 
have the argument flags set up, we use \code{Options.parse} to parse the argument vector (starting from index one, because 
index 0 is the program name). If this was successful, we assign matches to the parsed object, if not, we panic. Once past 
that, we test if the user passed in the help flag, and if so print the usage message. The option help messages are constructed 
by Getopts, so all we have to do to print the usage message is tell it what we want it to print for the program name and 
template. If the user has not passed in the help flag, we assign the proper variables to their corresponding arguments.

\subsubsection*{Writing the logic}

We all write code differently, but error handling is usually the last thing we want to think about. This isn't great for 
the overall design of a program, but it can be useful for rapid prototyping. Because Rust forces us to be explicit about 
error handling (by making us call \code{unwrap}), it is easy to see which parts of our program can cause errors.

\blank

In this case study, the logic is really simple. All we need to do is parse the CSV data given to us and print out a field 
in matching rows. Let's do it. (Make sure to add \code{extern crate csv}; to the top of your file.)

\begin{rustc}
use std::fs::File;
use std::path::Path;

// This struct represents the data in each row of the CSV file.
// Type based decoding absolves us of a lot of the nitty gritty error
// handling, like parsing strings as integers or floats.
#[derive(Debug, RustcDecodable)]
struct Row {
    country: String,
    city: String,
    accent_city: String,
    region: String,

    // Not every row has data for the population, latitude or longitude!
    // So we express them as `Option` types, which admits the possibility of
    // absence. The CSV parser will fill in the correct value for us.
    population: Option<u64>,
    latitude: Option<f64>,
    longitude: Option<f64>,
}

fn print_usage(program: &str, opts: Options) {
    println!("{}", opts.usage(&format!("Usage: {} [options] <data-path> <city>", program)));
}

fn main() {
    let args: Vec<String> = env::args().collect();
    let program = args[0].clone();

    let mut opts = Options::new();
    opts.optflag("h", "help", "Show this usage message.");

    let matches = match opts.parse(&args[1..]) {
        Ok(m)  => { m }
        Err(e) => { panic!(e.to_string()) }
    };

    if matches.opt_present("h") {
        print_usage(&program, opts);
        return;
    }

    let data_file = args[1].clone();
    let data_path = Path::new(&data_file);
    let city = args[2].clone();

    let file = File::open(data_path).unwrap();
    let mut rdr = csv::Reader::from_reader(file);

    for row in rdr.decode::<Row>() {
        let row = row.unwrap();

        if row.city == city {
            println!("{}, {}: {:?}",
                row.city, row.country,
                row.population.expect("population count"));
        }
    }
}
\end{rustc}

Let's outline the errors. We can start with the obvious: the three places that \code{unwrap} is called:

\begin{enumerate}
  \item{\href{https://doc.rust-lang.org/std/fs/struct.File.html\#method.open}{File::open} can return an 
      \href{https://doc.rust-lang.org/std/io/struct.Error.html}{io::Error}.}
  \item{\href{http://burntsushi.net/rustdoc/csv/struct.Reader.html\#method.decode}{csv::Reader::decode} decodes one record at 
      a time, and \href{http://burntsushi.net/rustdoc/csv/struct.DecodedRecords.html}{decoding a record} (look at the 
      \code{Item} associated type on the \code{Iterator} impl) can produce a 
      \href{http://burntsushi.net/rustdoc/csv/enum.Error.html}{csv::Error}.}
  \item{If \code{row.population} is \none, then calling \code{expect} will panic.}
\end{enumerate}

Are there any others? What if we can't find a matching city? Tools like \code{grep} will return an error code, so we 
probably should too. So we have logic errors specific to our problem, IO errors and CSV parsing errors. We're going to 
explore two different ways to approach handling these errors.

\blank

I'd like to start with \code{Box<Error>}. Later, we'll see how defining our own error type can be useful.

\subsubsection*{Error handling with \code{Box<Error>}}

\code{Box<Error>} is nice because it \emph{just works}. You don't need to define your own error types and you don't need 
any \code{From} implementations. The downside is that since \code{Box<Error>} is a trait object, it erases the type, which 
means the compiler can no longer reason about its underlying type.

\blank

Previously we started refactoring our code by changing the type of our function from \code{T} to 
\code{Result<T, OurErrorType>}. In this case, \code{OurErrorType} is only \code{Box<Error>}. But what's \code{T}? And can we 
add a return type to \code{main}?

\blank

The answer to the second question is no, we can't. That means we'll need to write a new function. But what is \code{T}? 
The simplest thing we can do is to return a list of matching \code{Row} values as a \code{Vec<Row>}. (Better code would 
return an iterator, but that is left as an exercise to the reader.)

\blank

Let's refactor our code into its own function, but keep the calls to \code{unwrap}. Note that we opt to handle the 
possibility of a missing population count by simply ignoring that row.

\begin{rustc}
struct Row {
    // unchanged
}

struct PopulationCount {
    city: String,
    country: String,
    // This is no longer an `Option` because values of this type are only
    // constructed if they have a population count.
    count: u64,
}

fn print_usage(program: &str, opts: Options) {
    println!("{}", opts.usage(&format!("Usage: {} [options] <data-path> <city>", program)));
}

fn search<P: AsRef<Path>>(file_path: P, city: &str) -> Vec<PopulationCount> {
    let mut found = vec![];
    let file = File::open(file_path).unwrap();
    let mut rdr = csv::Reader::from_reader(file);
    for row in rdr.decode::<Row>() {
        let row = row.unwrap();
        match row.population {
            None => { } // skip it
            Some(count) => if row.city == city {
                found.push(PopulationCount {
                    city: row.city,
                    country: row.country,
                    count: count,
                });
            },
        }
    }
    found
}

fn main() {
    let args: Vec<String> = env::args().collect();
    let program = args[0].clone();

    let mut opts = Options::new();
    opts.optflag("h", "help", "Show this usage message.");

    let matches = match opts.parse(&args[1..]) {
        Ok(m)  => { m }
        Err(e) => { panic!(e.to_string()) }
    };
    if matches.opt_present("h") {
        print_usage(&program, opts);
        return;
    }

    let data_file = args[1].clone();
    let data_path = Path::new(&data_file);
    let city = args[2].clone();
    for pop in search(&data_path, &city) {
        println!("{}, {}: {:?}", pop.city, pop.country, pop.count);
    }
}
\end{rustc}

While we got rid of one use of \code{expect} (which is a nicer variant of \code{unwrap}), we still should handle the 
absence of any search results.

\blank

To convert this to proper error handling, we need to do the following:

\begin{enumerate}
  \item{Change the return type of \code{search} to be \code{Result<Vec<PopulationCount>, Box<Error>>}.}
  \item{Use the \code{try!} macro so that errors are returned to the caller instead of panicking the program.}
  \item{Handle the error in \code{main}.}
\end{enumerate}

Let's try it:

\begin{rustc}
use std::error::Error;

// The rest of the code before this is unchanged

fn search<P: AsRef<Path>>
         (file_path: P, city: &str)
         -> Result<Vec<PopulationCount>, Box<Error+Send+Sync>> {
    let mut found = vec![];
    let file = try!(File::open(file_path));
    let mut rdr = csv::Reader::from_reader(file);
    for row in rdr.decode::<Row>() {
        let row = try!(row);
        match row.population {
            None => { } // skip it
            Some(count) => if row.city == city {
                found.push(PopulationCount {
                    city: row.city,
                    country: row.country,
                    count: count,
                });
            },
        }
    }
    if found.is_empty() {
        Err(From::from("No matching cities with a population were found."))
    } else {
        Ok(found)
    }
}
\end{rustc}

Instead of \code{x.unwrap()}, we now have \code{try!(x)}. Since our function returns a \code{Result<T, E>}, the 
\code{try!} macro will return early from the function if an error occurs.

\blank

There is one big gotcha in this code: we used \code{Box<Error + Send + Sync>} instead of \code{Box<Error>}. We did this 
so we could convert a plain string to an error type. We need these extra bounds so that we can use the 
\href{https://doc.rust-lang.org/std/convert/trait.From.html}{corresponding From impls}:

\begin{rustc}
// We are making use of this impl in the code above, since we call `From::from`
// on a `&'static str`.
impl<'a, 'b> From<&'b str> for Box<Error + Send + Sync + 'a>

// But this is also useful when you need to allocate a new string for an
// error message, usually with `format!`.
impl From<String> for Box<Error + Send + Sync>
\end{rustc}

Since \code{search} now returns a \code{Result<T, E>}, main should use case analysis when calling search:

\begin{rustc}
...
match search(&data_file, &city) {
    Ok(pops) => {
        for pop in pops {
            println!("{}, {}: {:?}", pop.city, pop.country, pop.count);
        }
    }
    Err(err) => println!("{}", err)
}
...
\end{rustc}

Now that we've seen how to do proper error handling with \code{Box<Error>}, let's try a different approach with our own 
custom error type. But first, let's take a quick break from error handling and add support for reading from \code{stdin}.

\subsubsection*{Reading from \code{stdin}}

In our program, we accept a single file for input and do one pass over the data. This means we probably should be able to 
accept input on stdin. But maybe we like the current format too—so let's have both!

\blank

Adding support for stdin is actually quite easy. There are only three things we have to do:

\begin{enumerate}
  \item{Tweak the program arguments so that a single parameter—the city—can be accepted while the population data is read 
      from stdin.}
  \item{Modify the program so that an option \code{-f} can take the file, if it is not passed into stdin.}
  \item{Modify the \code{search} function to take an \emph{optional} file path. When None, it should know to read from 
      stdin.}
\end{enumerate}

First, here's the new usage:

\begin{rustc}
fn print_usage(program: &str, opts: Options) {
    println!("{}", opts.usage(&format!("Usage: {} [options] <city>", program)));
}
\end{rustc}

The next part is going to be only a little harder:

\begin{rustc}
...
let mut opts = Options::new();
opts.optopt("f", "file", "Choose an input file, instead of using STDIN.", "NAME");
opts.optflag("h", "help", "Show this usage message.");
...
let file = matches.opt_str("f");
let data_file = file.as_ref().map(Path::new);

let city = if !matches.free.is_empty() {
    matches.free[0].clone()
} else {
    print_usage(&program, opts);
    return;
};

match search(&data_file, &city) {
    Ok(pops) => {
        for pop in pops {
            println!("{}, {}: {:?}", pop.city, pop.country, pop.count);
        }
    }
    Err(err) => println!("{}", err)
}
...
\end{rustc}

In this piece of code, we take \code{file} (which has the type \code{Option<String>}), and convert it to a type that 
\code{search} can use, in this case, \code{\&Option<AsRef<Path>>}. To do this, we take a reference of file, and map 
\code{Path::new} onto it. In this case, \code{as\_ref()} converts the \code{Option<String>} into an \code{Option<\&str>}, 
and from there, we can execute \code{Path::new} to the content of the optional, and return the optional of the new value. 
Once we have that, it is a simple matter of getting the \code{city} argument and executing \code{search}.

\blank

Modifying \code{search} is slightly trickier. The \code{csv} crate can build a parser out of 
\href{http://burntsushi.net/rustdoc/csv/struct.Reader.html\#method.from\_reader}{any type that implements io::Read}. But 
how can we use the same code over both types? There's actually a couple ways we could go about this. One way is to write 
\code{search} such that it is generic on some type parameter \code{R} that satisfies \code{io::Read}. Another way is to 
use trait objects:

\begin{rustc}
use std::io;

// The rest of the code before this is unchanged

fn search<P: AsRef<Path>>
         (file_path: &Option<P>, city: &str)
         -> Result<Vec<PopulationCount>, Box<Error+Send+Sync>> {
    let mut found = vec![];
    let input: Box<io::Read> = match *file_path {
        None => Box::new(io::stdin()),
        Some(ref file_path) => Box::new(try!(File::open(file_path))),
    };
    let mut rdr = csv::Reader::from_reader(input);
    // The rest remains unchanged!
}
\end{rustc}

\subsubsection*{Error handling with a custom type}

Previously, we learned how to compose errors using a custom error type. We did this by defining our error type as an 
\enum\ and implementing \code{Error} and \code{From}.

\blank

Since we have three distinct errors (IO, CSV parsing and not found), let's define an \enum\ with three variants:

\begin{rustc}
#[derive(Debug)]
enum CliError {
    Io(io::Error),
    Csv(csv::Error),
    NotFound,
}
\end{rustc}

And now for impls on \code{Display} and \code{Error}:

\begin{rustc}
impl fmt::Display for CliError {
    fn fmt(&self, f: &mut fmt::Formatter) -> fmt::Result {
        match *self {
            CliError::Io(ref err) => err.fmt(f),
            CliError::Csv(ref err) => err.fmt(f),
            CliError::NotFound => write!(f, "No matching cities with a \
                                             population were found."),
        }
    }
}

impl Error for CliError {
    fn description(&self) -> &str {
        match *self {
            CliError::Io(ref err) => err.description(),
            CliError::Csv(ref err) => err.description(),
            CliError::NotFound => "not found",
        }
    }
}
\end{rustc}

Before we can use our \code{CliError} type in our \code{search} function, we need to provide a couple \code{From} impls. 
How do we know which impls to provide? Well, we'll need to convert from both \code{io::Error} and \code{csv::Error} to 
\code{CliError}. Those are the only external errors, so we'll only need two \code{From} impls for now:

\begin{rustc}
impl From<io::Error> for CliError {
    fn from(err: io::Error) -> CliError {
        CliError::Io(err)
    }
}

impl From<csv::Error> for CliError {
    fn from(err: csv::Error) -> CliError {
        CliError::Csv(err)
    }
}
\end{rustc}

The \code{From} impls are important because of how \code{try!} is defined. In particular, if an error occurs, 
\code{From::from} is called on the error, which in this case, will convert it to our own error type \code{CliError}.

\blank

With the \code{From} impls done, we only need to make two small tweaks to our \code{search} function: the return type and 
the \enquote{not found} error. Here it is in full:

\begin{rustc}
fn search<P: AsRef<Path>>
         (file_path: &Option<P>, city: &str)
         -> Result<Vec<PopulationCount>, CliError> {
    let mut found = vec![];
    let input: Box<io::Read> = match *file_path {
        None => Box::new(io::stdin()),
        Some(ref file_path) => Box::new(try!(File::open(file_path))),
    };
    let mut rdr = csv::Reader::from_reader(input);
    for row in rdr.decode::<Row>() {
        let row = try!(row);
        match row.population {
            None => { } // skip it
            Some(count) => if row.city == city {
                found.push(PopulationCount {
                    city: row.city,
                    country: row.country,
                    count: count,
                });
            },
        }
    }
    if found.is_empty() {
        Err(CliError::NotFound)
    } else {
        Ok(found)
    }
}
\end{rustc}

No other changes are necessary.

\subsubsection*{Adding functionality}

Writing generic code is great, because generalizing stuff is cool, and it can then be useful later. But sometimes, the 
juice isn't worth the squeeze. Look at what we just did in the previous step:

\begin{enumerate}
  \item{Defined a new error type.}
  \item{Added impls for \code{Error}, \code{Display} and two for \code{From}.}
\end{enumerate}

The big downside here is that our program didn't improve a whole lot. There is quite a bit of overhead to representing 
errors with \enum s, especially in short programs like this.

\blank

One useful aspect of using a custom error type like we've done here is that the \code{main} function can now choose to 
handle errors differently. Previously, with \code{Box<Error>}, it didn't have much of a choice: just print the message. 
We're still doing that here, but what if we wanted to, say, add a \code{--quiet} flag? The \code{--quiet} flag should 
silence any verbose output.

\blank

Right now, if the program doesn't find a match, it will output a message saying so. This can be a little clumsy, especially 
if you intend for the program to be used in shell scripts.

\blank

So let's start by adding the flags. Like before, we need to tweak the usage string and add a flag to the Option variable. 
Once we've done that, Getopts does the rest:

\begin{rustc}
...
let mut opts = Options::new();
opts.optopt("f", "file", "Choose an input file, instead of using STDIN.", "NAME");
opts.optflag("h", "help", "Show this usage message.");
opts.optflag("q", "quiet", "Silences errors and warnings.");
...
\end{rustc}

Now we only need to implement our \enquote{quiet} functionality. This requires us to tweak the case analysis in \code{main}:

\begin{rustc}
match search(&args.arg_data_path, &args.arg_city) {
    Err(CliError::NotFound) if args.flag_quiet => process::exit(1),
    Err(err) => panic!("{}", err),
    Ok(pops) => for pop in pops {
        println!("{}, {}: {:?}", pop.city, pop.country, pop.count);
    }
}
\end{rustc}

Certainly, we don't want to be quiet if there was an IO error or if the data failed to parse. Therefore, we use case 
analysis to check if the error type is \code{NotFound} \emph{and} if \code{--quiet} has been enabled. If the search failed, 
we still quit with an exit code (following \code{grep}'s convention).

\blank

If we had stuck with \code{Box<Error>}, then it would be pretty tricky to implement the \code{--quiet} functionality.

\blank

This pretty much sums up our case study. From here, you should be ready to go out into the world and write your own 
programs and libraries with proper error handling.

\subsection*{The Short Story}

Since this section is long, it is useful to have a quick summary for error handling in Rust. These are some good 
\enquote{rules of thumb.} They are emphatically not commandments. There are probably good reasons to break every one 
of these heuristics!

\begin{itemize}
  \item{If you're writing short example code that would be overburdened by error handling, it's probably fine to use 
      \code{unwrap} (whether that's \href{https://doc.rust-lang.org/std/result/enum.Result.html\#method.unwrap}{Result::unwrap}, 
      \href{https://doc.rust-lang.org/std/option/enum.Option.html\#method.unwrap}{Option::unwrap} or preferably 
      \href{https://doc.rust-lang.org/std/option/enum.Option.html\#method.expect}{Option::expect}). Consumers of your code 
      should know to use proper error handling. (If they don't, send them here!)}
  \item{If you're writing a quick 'n' dirty program, don't feel ashamed if you use \code{unwrap}. Be warned: if it winds 
      up in someone else's hands, don't be surprised if they are agitated by poor error messages!}
  \item{If you're writing a quick 'n' dirty program and feel ashamed about panicking anyway, then use either a \String\ or 
      a \code{Box<Error + Send + Sync>} for your error type (the \code{Box<Error + Send + Sync>} type is because of the 
      available \code{From} impls).}
  \item{Otherwise, in a program, define your own error types with appropriate 
      \href{https://doc.rust-lang.org/std/convert/trait.From.html}{From} and 
      \href{https://doc.rust-lang.org/std/error/trait.Error.html}{Error} impls to make the 
      \href{https://doc.rust-lang.org/std/macro.try!.html}{try!} macro more ergonomic.}
  \item{If you're writing a library and your code can produce errors, define your own error type and implement the 
      \href{https://doc.rust-lang.org/std/error/trait.Error.html}{std::error::Error} trait. Where appropriate, implement 
      \href{https://doc.rust-lang.org/std/convert/trait.From.html}{From} to make both your library code and the caller's 
      code easier to write. (Because of Rust's coherence rules, callers will not be able to impl \code{From} on your error 
      type, so your library should do it.)}
  \item{Learn the combinators defined on \href{https://doc.rust-lang.org/std/option/enum.Option.html}{Option} and 
      \href{https://doc.rust-lang.org/std/result/enum.Result.html}{Result}. Using them exclusively can be a bit tiring at 
      times, but I've personally found a healthy mix of \code{try!} and combinators to be quite appealing. \code{and\_then}, 
      \code{map} and \code{unwrap\_or} are my favorites.}
\end{itemize}


\section{Choosing your Guarantees}
\label{sec:effective_choosingYourGuarantees}
One important feature of Rust is that it lets us control the costs and guarantees of a program.

\blank

There are various \enquote{wrapper type} abstractions in the Rust standard library which embody a multitude of tradeoffs 
between cost, ergonomics, and guarantees. Many let one choose between run time and compile time enforcement. This section 
will explain a few selected abstractions in detail.

\blank

Before proceeding, it is highly recommended that one reads about ownership (see \nameref{sec:syntax_ownership}) and borrowing 
(see \nameref{sec:syntax_referencesBorrowing}) in Rust.

\subsection*{Basic pointer types}

\subsubsection*{\code{Box<T>}}

\href{https://doc.rust-lang.org/std/boxed/struct.Box.html}{Box<T>} is an \enquote{owned} pointer, or a \enquote{box}. While 
it can hand out references to the contained data, it is the only owner of the data. In particular, consider the following:

\begin{rustc}
let x = Box::new(1);
let y = x;
// x no longer accessible here
\end{rustc}

Here, the box was moved into \y. As \x\ no longer owns it, the compiler will no longer allow the programmer to use \x\ 
after this. A box can similarly be moved out of a function by returning it.

\blank

When a box (that hasn't been moved) goes out of scope, destructors are run. These destructors take care of deallocating 
the inner data.

\blank

This is a zero-cost abstraction for dynamic allocation. If you want to allocate some memory on the heap and safely pass 
around a pointer to that memory, this is ideal. Note that you will only be allowed to share references to this by the 
regular borrowing rules, checked at compile time.

\subsubsection*{\code{\&T} and \code{\&mut T}}

These are immutable and mutable references respectively. They follow the \enquote{read-write lock} pattern, such that one 
may either have only one mutable reference to some data, or any number of immutable ones, but not both. This guarantee is 
enforced at compile time, and has no visible cost at runtime. In most cases these two pointer types suffice for sharing 
cheap references between sections of code.

\blank

These pointers cannot be copied in such a way that they outlive the lifetime associated with them.

\subsubsection*{\code{*const T} and \code{*mut T}}

These are C-like raw pointers with no lifetime or ownership attached to them. They point to some location in memory 
with no other restrictions. The only guarantee that these provide is that they cannot be dereferenced except in code 
marked \code{unsafe}.

\blank

These are useful when building safe, low cost abstractions like \code{Vec<T>}, but should be avoided in safe code.

\subsubsection*{\code{Rc<T>}}

This is the first wrapper we will cover that has a runtime cost.

\blank

\href{https://doc.rust-lang.org/std/rc/struct.Rc.html}{Rc<T>} is a reference counted pointer. In other words, this lets 
us have multiple \enquote{owning} pointers to the same data, and the data will be dropped (destructors will be run) when 
all pointers are out of scope.

\blank

Internally, it contains a shared \enquote{reference count} (also called \enquote{refcount}), which is incremented each 
time the \code{Rc} is cloned, and decremented each time one of the \code{Rc}s goes out of scope. The main responsibility 
of \code{Rc<T>} is to ensure that destructors are called for shared data.

\blank

The internal data here is immutable, and if a cycle of references is created, the data will be leaked. If we want data 
that doesn't leak when there are cycles, we need a garbage collector.

\parag{Guarantees}

The main guarantee provided here is that the data will not be destroyed until all references to it are out of scope.

\blank

This should be used when we wish to dynamically allocate and share some data (read-only) between various portions of your 
program, where it is not certain which portion will finish using the pointer last. It's a viable alternative to \code{\&T} 
when \code{\&T} is either impossible to statically check for correctness, or creates extremely unergonomic code where the 
programmer does not wish to spend the development cost of working with.

\blank

This pointer is not thread safe, and Rust will not let it be sent or shared with other threads. This lets one avoid the 
cost of atomics in situations where they are unnecessary.

\blank

There is a sister smart pointer to this one, \code{Weak<T>}. This is a non-owning, but also non-borrowed, smart pointer. It 
is also similar to \code{\&T}, but it is not restricted in lifetime—a \code{Weak<T>} can be held on to forever. However, it 
is possible that an attempt to access the inner data may fail and return \none, since this can outlive the owned \code{Rc}s. 
This is useful for cyclic data structures and other things.

\parag{Cost}

As far as memory goes, \code{Rc<T>} is a single allocation, though it will allocate two extra words (i.e. two \code{usize} 
values) as compared to a regular \code{Box<T>} (for \enquote{strong} and \enquote{weak} refcounts).

\blank

\code{Rc<T>} has the computational cost of incrementing/decrementing the refcount whenever it is cloned or goes out of 
scope respectively. Note that a clone will not do a deep copy, rather it will simply increment the inner reference count 
and return a copy of the \code{Rc<T>}.

\subsection*{Cell types}

\code{Cell}s provide interior mutability. In other words, they contain data which can be manipulated even if the type 
cannot be obtained in a mutable form (for example, when it is behind an \code{\&}-ptr or \code{Rc<T>}).

\blank

\href{https://doc.rust-lang.org/std/cell/}{The documentation for the cell} module has a pretty good explanation for these.

\blank

These types are generally found in struct fields, but they may be found elsewhere too.

\subsubsection*{\code{Cell<T>}}

\href{https://doc.rust-lang.org/std/cell/struct.Cell.html}{Cell<T>} is a type that provides zero-cost interior mutability, 
but only for \code{Copy} types. Since the compiler knows that all the data owned by the contained value is on the stack, 
there's no worry of leaking any data behind references (or worse!) by simply replacing the data.

\blank

It is still possible to violate your own invariants using this wrapper, so be careful when using it. If a field is wrapped 
in \code{Cell}, it's a nice indicator that the chunk of data is mutable and may not stay the same between the time you first 
read it and when you intend to use it.

\begin{rustc}
use std::cell::Cell;

let x = Cell::new(1);
let y = &x;
let z = &x;
x.set(2);
y.set(3);
z.set(4);
println!("{}", x.get());
\end{rustc}

Note that here we were able to mutate the same value from various immutable references.

\blank

This has the same runtime cost as the following:

\begin{rustc}
let mut x = 1;
let y = &mut x;
let z = &mut x;
x = 2;
*y = 3;
*z = 4;
println!("{}", x);
\end{rustc}

but it has the added benefit of actually compiling successfully.

\parag{Guarantees}

This relaxes the \enquote{no aliasing with mutability} restriction in places where it's unnecessary. However, this also 
relaxes the guarantees that the restriction provides; so if your invariants depend on data stored within \code{Cell}, you 
should be careful.

\blank

This is useful for mutating primitives and other \code{Copy} types when there is no easy way of doing it in line with the 
static rules of \code{\&} and \code{\&mut}.

\blank

\code{Cell} does not let you obtain interior references to the data, which makes it safe to freely mutate.

\parag{Cost}

There is no runtime cost to using \code{Cell<T>,} however if you are using it to wrap larger (\code{Copy}) structs, 
it might be worthwhile to instead wrap individual fields in \code{Cell<T>} since each write is otherwise a full copy of 
the struct.

\subsubsection*{RefCell<T>}

\href{https://doc.rust-lang.org/std/cell/struct.RefCell.html}{RefCell<T>} also provides interior mutability, but isn't 
restricted to \code{Copy} types.

\blank

Instead, it has a runtime cost. \code{RefCell<T>} enforces the read-write lock pattern at runtime (it's like a single-threaded 
mutex), unlike \code{\&T}/\code{\&mut T} which do so at compile time. This is done by the \code{borrow()} and 
\code{borrow\_mut()} functions, which modify an internal reference count and return smart pointers which can be dereferenced 
immutably and mutably respectively. The refcount is restored when the smart pointers go out of scope. With this system, we 
can dynamically ensure that there are never any other borrows active when a mutable borrow is active. If the programmer attempts 
to make such a borrow, the thread will panic.

\begin{rustc}
use std::cell::RefCell;

let x = RefCell::new(vec![1,2,3,4]);
{
    println!("{:?}", *x.borrow())
}

{
    let mut my_ref = x.borrow_mut();
    my_ref.push(1);
}
\end{rustc}

Similar to \code{Cell}, this is mainly useful for situations where it's hard or impossible to satisfy the borrow checker. 
Generally we know that such mutations won't happen in a nested form, but it's good to check.

\blank

For large, complicated programs, it becomes useful to put some things in \code{RefCell}s to make things simpler. For example, 
a lot of the maps in \href{https://doc.rust-lang.org/rustc/middle/ty/struct.ctxt.html}{the ctxt struct} in the Rust compiler 
internals are inside this wrapper. These are only modified once (during creation, which is not right after initialization) or 
a couple of times in well-separated places. However, since this struct is pervasively used everywhere, juggling mutable and 
immutable pointers would be hard (perhaps impossible) and probably form a soup of \code{\&}-ptrs which would be hard to extend. 
On the other hand, the \code{RefCell} provides a cheap (not zero-cost) way of safely accessing these. In the future, if someone 
adds some code that attempts to modify the cell when it's already borrowed, it will cause a (usually deterministic) panic which 
can be traced back to the offending borrow.

\blank

Similarly, in Servo's DOM there is a lot of mutation, most of which is local to a DOM type, but some of which crisscrosses 
the DOM and modifies various things. Using \code{RefCell} and \code{Cell} to guard all mutation lets us avoid worrying 
about mutability everywhere, and it simultaneously highlights the places where mutation is \emph{actually} happening.

\blank

Note that \code{RefCell} should be avoided if a mostly simple solution is possible with \code{\&} pointers.

\parag{Guarantees}

\code{RefCell} relaxes the \emph{static} restrictions preventing aliased mutation, and replaces them with \emph{dynamic} 
ones. As such the guarantees have not changed.

\parag{Cost}

\code{RefCell} does not allocate, but it contains an additional \enquote{borrow state} indicator (one word in size) 
along with the data.

\blank

At runtime each borrow causes a modification/check of the refcount.

\subsection*{Synchronous types}

Many of the types above cannot be used in a threadsafe manner. Particularly, \code{Rc<T>} and \code{RefCell<T>}, which both 
use non-atomic reference counts (atomic reference counts are those which can be incremented from multiple threads without 
causing a data race), cannot be used this way. This makes them cheaper to use, but we need thread safe versions of these too. 
They exist, in the form of \code{Arc<T>} and \code{Mutex<T>}/\code{RwLock<T>}.

\blank

Note that the non-threadsafe types \emph{cannot} be sent between threads, and this is checked at compile time.

\blank

There are many useful wrappers for concurrent programming in the \href{https://doc.rust-lang.org/std/sync/}{sync} module, 
but only the major ones will be covered below.

\subsubsection*{Arc<T>}

\href{https://doc.rust-lang.org/std/sync/struct.Arc.html}{Arc<T>} is a version of \code{Rc<T>} that uses an atomic 
reference count (hence, \enquote{Arc}). This can be sent freely between threads.

\blank

C++'s \code{shared\_ptr} is similar to \code{Arc}, however in the case of C++ the inner data is always mutable. For 
semantics similar to that from C++, we should use \code{Arc<Mutex<T>>}, \code{Arc<RwLock<T>>}, or 
\code{Arc<UnsafeCell<T>>}\footnote{\code{Arc<UnsafeCell<T>>} actually won't compile since \code{UnsafeCell<T>} isn't 
\code{Send} or \code{Sync}, but we can wrap it in a type and implement \code{Send}/\code{Sync} for it manually to get 
\code{Arc<Wrapper<T>>} where \code{Wrapper} is \code{struct Wrapper<T>(UnsafeCell<T>}).} (\code{UnsafeCell<T>} is a cell 
type that can be used to hold any data and has no runtime cost, but accessing it requires \code{unsafe} blocks). The last 
one should only be used if we are certain that the usage won't cause any memory unsafety. Remember that writing to a struct 
is not an atomic operation, and many functions like \code{vec.push()} can reallocate internally and cause unsafe behavior, 
so even monotonicity may not be enough to justify \code{UnsafeCell}.

\parag{Guarantees}

Like \code{Rc}, this provides the (thread safe) guarantee that the destructor for the internal data will be run when the 
last \code{Arc} goes out of scope (barring any cycles).

\parag{Cost}

This has the added cost of using atomics for changing the refcount (which will happen whenever it is cloned or goes out 
of scope). When sharing data from an \code{Arc} in a single thread, it is preferable to share \code{\&} pointers whenever 
possible.

\subsubsection*{\code{Mutex<T>} and \code{RwLock<T>}}

\href{https://doc.rust-lang.org/std/sync/struct.Mutex.html}{Mutex<T>} and 
\href{https://doc.rust-lang.org/std/sync/struct.RwLock.html}{RwLock<T>} provide mutual-exclusion via RAII guards (guards 
are objects which maintain some state, like a lock, until their destructor is called). For both of these, the mutex is 
opaque until we call \code{lock()} on it, at which point the thread will block until a lock can be acquired, and then a 
guard will be returned. This guard can be used to access the inner data (mutably), and the lock will be released when the 
guard goes out of scope.

\begin{rustc}
{
    let guard = mutex.lock();
    // guard dereferences mutably to the inner type
    *guard += 1;
} // lock released when destructor runs
\end{rustc}

\code{RwLock} has the added benefit of being efficient for multiple reads. It is always safe to have multiple readers to 
shared data as long as there are no writers; and \code{RwLock} lets readers acquire a \enquote{read lock}. Such locks can 
be acquired concurrently and are kept track of via a reference count. Writers must obtain a \enquote{write lock} which can 
only be obtained when all readers have gone out of scope.

\parag{Guarantees}

Both of these provide safe shared mutability across threads, however they are prone to deadlocks. Some level of additional 
protocol safety can be obtained via the type system.

\parag{Costs}

These use internal atomic-like types to maintain the locks, which are pretty costly (they can block all memory reads across 
processors till they're done). Waiting on these locks can also be slow when there's a lot of concurrent access happening.

\subsection*{Composition}

A common gripe when reading Rust code is with types like \code{Rc<RefCell<Vec<T>>>} (or even more complicated compositions 
of such types). It's not always clear what the composition does, or why the author chose one like this (and when one should 
be using such a composition in one's own code)

\blank

Usually, it's a case of composing together the guarantees that you need, without paying for stuff that is unnecessary.

\blank

For example, \code{Rc<RefCell<T>>} is one such composition. \code{Rc<T>} itself can't be dereferenced mutably; because 
\code{Rc<T>} provides sharing and shared mutability can lead to unsafe behavior, so we put \code{RefCell<T>} inside to 
get dynamically verified shared mutability. Now we have shared mutable data, but it's shared in a way that there can only 
be one mutator (and no readers) or multiple readers.

\blank

Now, we can take this a step further, and have \code{Rc<RefCell<Vec<T>>>} or \code{Rc<Vec<RefCell<T>>>}. These are both 
shareable, mutable vectors, but they're not the same.

\blank

With the former, the \code{RefCell<T>} is wrapping the \code{Vec<T>}, so the \code{Vec<T>} in its entirety is mutable. 
At the same time, there can only be one mutable borrow of the whole \code{Vec} at a given time. This means that your code 
cannot simultaneously work on different elements of the vector from different \code{Rc} handles. However, we are able to 
push and pop from the \code{Vec<T>} at will. This is similar to a \code{\&mut Vec<T>} with the borrow checking done at runtime.

\blank

With the latter, the borrowing is of individual elements, but the overall vector is immutable. Thus, we can independently 
borrow separate elements, but we cannot push or pop from the vector. This is similar to a \code{\&mut [T]}\footnote{\code{\&[T]} 
and \code{\&mut [T]} are slices; they consist of a pointer and a length and can refer to a portion of a vector or array. 
\code{\&mut [T]} can have its elements mutated, however its length cannot be touched.}, but, again, the borrow checking is 
at runtime.

\blank

In concurrent programs, we have a similar situation with \code{Arc<Mutex<T>>}, which provides shared mutability and ownership.

\blank

When reading code that uses these, go in step by step and look at the guarantees/costs provided.

\blank

When choosing a composed type, we must do the reverse; figure out which guarantees we want, and at which point of the 
composition we need them. For example, if there is a choice between \code{Vec<RefCell<T>>} and \code{RefCell<Vec<T>>}, we 
should figure out the tradeoffs as done above and pick one.


\section{Foreign Function Interface}
\label{sec:effective_FFI}
\input{src/effective_rust/ffi.tex}

\section{Borrow and AsRef}
\label{sec:effective_borrowAndAsRef}
The \href{https://doc.rust-lang.org/std/borrow/trait.Borrow.html}{Borrow} and 
\href{https://doc.rust-lang.org/std/convert/trait.AsRef.html}{AsRef} traits are very similar, but different. Here's a 
quick refresher on what these two traits mean.

\subsection*{Borrow}

The \code{Borrow} trait is used when you're writing a datastructure, and you want to use either an owned or borrowed 
type as synonymous for some purpose.

\blank

For example, \href{https://doc.rust-lang.org/std/collections/struct.HashMap.html}{HashMap} has a 
\href{https://doc.rust-lang.org/std/collections/struct.HashMap.html\#method.get}{get method} which uses \code{Borrow}:

\begin{rustc}
fn get<Q: ?Sized>(&self, k: &Q) -> Option<&V>
    where K: Borrow<Q>,
          Q: Hash + Eq
\end{rustc}

This signature is pretty complicated. The \code{K} parameter is what we're interested in here. It refers to a parameter 
of the \code{HashMap} itself:

\begin{rustc}
struct HashMap<K, V, S = RandomState> {
\end{rustc}

The \code{K} parameter is the type of key the \code{HashMap} uses. So, looking at the signature of \code{get()} again, 
we can use \code{get()} when the key implements \code{Borrow<Q>}. That way, we can make a \code{HashMap} which uses \String\ 
keys, but use \code{\&str}s when we're searching:

\begin{rustc}
use std::collections::HashMap;

let mut map = HashMap::new();
map.insert("Foo".to_string(), 42);

assert_eq!(map.get("Foo"), Some(&42));
\end{rustc}

This is because the standard library has \code{impl Borrow<str> for String}.

\blank

For most types, when you want to take an owned or borrowed type, a \code{\&T} is enough. But one area where \code{Borrow} 
is effective is when there's more than one kind of borrowed value. This is especially true of references and slices: you can 
have both an \code{\&T} or a \code{\&mut T}. If we wanted to accept both of these types, \code{Borrow} is up for it:

\begin{rustc}
use std::borrow::Borrow;
use std::fmt::Display;

fn foo<T: Borrow<i32> + Display>(a: T) {
    println!("a is borrowed: {}", a);
}

let mut i = 5;

foo(&i);
foo(&mut i);
\end{rustc}

This will print out \code{a is borrowed: 5} twice.

\subsection*{AsRef}

The \code{AsRef} trait is a conversion trait. It's used for converting some value to a reference in generic code. Like this:

\begin{rustc}
let s = "Hello".to_string();

fn foo<T: AsRef<str>>(s: T) {
    let slice = s.as_ref();
}
\end{rustc}

\subsection*{Which should I use?}

We can see how they're kind of the same: they both deal with owned and borrowed versions of some type. However, they're a 
bit different.

\blank

Choose \code{Borrow} when you want to abstract over different kinds of borrowing, or when you're building a datastructure 
that treats owned and borrowed values in equivalent ways, such as hashing and comparison.

\blank

Choose \code{AsRef} when you want to convert something to a reference directly, and you're writing generic code.


\section{Release Channels}
\label{sec:effective_releaseChannels}
The Rust project uses a concept called 'release channels' to manage releases. It's important to understand this process 
to choose which version of Rust your project should use.

\subsection*{Overview}

There are three channels for Rust releases:

\begin{itemize}
  \item{Nightly}
  \item{Beta}
  \item{Stable}
\end{itemize}

New nightly releases are created once a day. Every six weeks, the latest nightly release is promoted to 'Beta'. At that 
point, it will only receive patches to fix serious errors. Six weeks later, the beta is promoted to 'Stable', and becomes 
the next release of \code{1.x}.

\blank

This process happens in parallel. So every six weeks, on the same day, nightly goes to beta, beta goes to stable. When 
\code{1.x} is released, at the same time, \code{1.(x + 1)-beta} is released, and the nightly becomes the first version of 
\code{1.(x + 2)-nightly}.

\subsection*{Choosing a version}

Generally speaking, unless you have a specific reason, you should be using the stable release channel. These releases 
are intended for a general audience.

\blank

However, depending on your interest in Rust, you may choose to use nightly instead. The basic tradeoff is this: in the 
nightly channel, you can use unstable, new Rust features. However, unstable features are subject to change, and so any 
new nightly release may break your code. If you use the stable release, you cannot use experimental features, but the next 
release of Rust will not cause significant issues through breaking changes.

\subsection*{Helping the ecosystem through CI}

What about beta? We encourage all Rust users who use the stable release channel to also test against the beta channel 
in their continuous integration systems. This will help alert the team in case there's an accidental regression.

\blank

Additionally, testing against nightly can catch regressions even sooner, and so if you don't mind a third build, we'd 
appreciate testing against all channels.

\blank

As an example, many Rust programmers use \href{https://travis-ci.org/}{Travis} to test their crates, which is free for 
open source projects. Travis \href{http://docs.travis-ci.com/user/languages/rust/}{supports Rust directly}, and you can 
use a \code{.travis.yml} file like this to test on all channels:

\begin{verbatim}
language: rust
rust:
  - nightly
  - beta
  - stable

matrix:
  allow_failures:
    - rust: nightly

\end{verbatim}

With this configuration, Travis will test all three channels, but if something breaks on nightly, it won't fail your 
build. A similar configuration is recommended for any CI system, check the documentation of the one you're using for 
more details.


\section{Using Rust without the Standard Library}
\label{sec:effective_rustWithoutStdLib}
Rust's standard library provides a lot of useful functionality, but assumes support for various features of its host 
system: threads, networking, heap allocation, and others. There are systems that do not have these features, however, and 
Rust can work with those too! To do so, we tell Rust that we don't want to use the standard library via an attribute: 
\code{\#![no\_std]}.

\begin{myquote}
Note: This feature is technically stable, but there are some caveats. For one, you can build a \code{\#![no\_std]} library 
on stable, but not a binary. For details on binaries without the standard library, see the nightly chapter on 
\code{\#![no\_std]} (see \nameref{sec:nightly_nostdlib}).
\end{myquote}

To use \code{\#![no\_std]}, add a it to your crate root:

\begin{rustc}
#![no_std]

fn plus_one(x: i32) -> i32 {
    x + 1
}
\end{rustc}

Much of the functionality that's exposed in the standard library is also available via the 
\href{https://doc.rust-lang.org/core/}{core crate}. When we're using the standard library, Rust automatically brings 
\code{std} into scope, allowing you to use its features without an explicit import. By the same token, when using 
\code{!\#[no\_std]}, Rust will bring \code{core} into scope for you, as well as 
\href{https://doc.rust-lang.org/core/prelude/v1/}{its prelude}. This means that a lot of code will Just Work:

\begin{rustc}
#![no_std]

fn may_fail(failure: bool) -> Result<(), &'static str> {
    if failure {
        Err("this didn't work!")
    } else {
        Ok(())
    }
}
\end{rustc}

