Not every Rustacean has a background in systems programming, nor in computer science, so we've added explanations of terms that might be 
unfamiliar.

\subsection*{Abstract Syntax Tree}
\label{sec:gloss_syntaxtree}

When a compiler is compiling your program, it does a number of different things. One of the things that it does is turn 
the text of your program into an ‘abstract syntax tree’, or ‘AST’. This tree is a representation of the structure of your 
program. For example, \code{2 + 3} can be turned into a tree:

\begin{verbatim}
  +
 / \
2   3
\end{verbatim}

And \code{2 + (3 * 4)} would look like this:

\begin{verbatim}
  +
 / \
2   *
   / \
  3   4
\end{verbatim}

\subsection*{Arity}
\label{sec:gloss_arity}

Arity refers to the number of arguments a function or operation takes.

\begin{rustc}
let x = (2, 3);
let y = (4, 6);
let z = (8, 2, 6);
\end{rustc}

In the example above \code{x} and \code{y} have arity 2. \code{z} has arity 3.

\subsection*{Bounds}
\label{sec:gloss_bounds}

% TODO make traits a hyperref to traits section
Bounds are constraints on a type or trait. For example, if a bound is placed on the argument a function takes, 
types passed to that function must abide by that constraint.

\subsection*{DST (Dynamically Sized Type)}
\label{sec:gloss_dst}

A type without a statically known size or alignment. (\href{https://doc.rust-lang.org/nomicon/exotic-sizes.html#dynamically-sized-types-dsts}{more info})

\subsection*{Expression}
\label{sec:gloss_expression}

In computer programming, an expression is a combination of values, constants, variables, operators and functions that 
evaluate to a single value. For example, \code{2 + (3 * 4)} is an expression that returns the value 14. It is worth 
noting that expressions can have side-effects. For example, a function included in an expression might perform actions 
other than simply returning a value.

\subsection*{Expression-Oriented Language}
\label{sec:gloss_expressionorientedlang}

In early programming languages, \nameref{sec:gloss_expression} and \nameref{sec:gloss_statements} were two separate syntactic categories: 
expressions had a value and statements did things. However, later languages blurred this distinction, allowing expressions to do things 
and statements to have a value. In an expression-oriented language, (nearly) every statement is an expression and therefore returns a value.
 Consequently, these expression statements can themselves form part of larger expressions.

\subsection*{Statement}
\label{sec:gloss_statements}

In computer programming, a statement is the smallest standalone element of a programming language that commands a computer to perform 
an action.
