With the \code{associated\_consts} feature, you can define constants like this:

\begin{rustc}
#![feature(associated_consts)]

trait Foo {
    const ID: i32;
}

impl Foo for i32 {
    const ID: i32 = 1;
}

fn main() {
    assert_eq!(1, i32::ID);
}
\end{rustc}

Any implementor of \code{Foo} will have to define \code{ID}. Without the definition:

\begin{rustc}
#![feature(associated_consts)]

trait Foo {
    const ID: i32;
}

impl Foo for i32 {
}
\end{rustc}

gives

\begin{verbatim}
error: not all trait items implemented, missing: `ID` [E0046]
     impl Foo for i32 {
     }
\end{verbatim}

A default value can be implemented as well:

\begin{rustc}
#![feature(associated_consts)]

trait Foo {
    const ID: i32 = 1;
}

impl Foo for i32 {
}

impl Foo for i64 {
    const ID: i32 = 5;
}

fn main() {
    assert_eq!(1, i32::ID);
    assert_eq!(5, i64::ID);
}
\end{rustc}

As you can see, when implementing \code{Foo}, you can leave it unimplemented, as with \itt. It will then use the default value. 
But, as in \code{i64}, we can also add our own definition.

\blank

Associated constants don't have to be associated with a trait. An \code{impl} block for a \struct\ or an \code{enum} works fine too:

\begin{rustc}
#![feature(associated_consts)]

struct Foo;

impl Foo {
    const FOO: u32 = 3;
}
\end{rustc}
