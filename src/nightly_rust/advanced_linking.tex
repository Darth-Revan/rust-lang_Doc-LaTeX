The common cases of linking with Rust have been covered earlier in this book, but supporting the range of linking 
possibilities made available by other languages is important for Rust to achieve seamless interaction with native 
libraries.

\subsection*{Link args}

There is one other way to tell \code{rustc} how to customize linking, and that is via the \code{link\_args} attribute. 
This attribute is applied to \code{extern} blocks and specifies raw flags which need to get passed to the linker when 
producing an artifact. An example usage would be:

\begin{rustc}
#![feature(link_args)]

#[link_args = "-foo -bar -baz"]
extern {}
\end{rustc}

Note that this feature is currently hidden behind the \code{feature(link\_args)} gate because this is not a sanctioned way 
of performing linking. Right now \code{rustc} shells out to the system linker (\code{gcc} on most systems, \code{link.exe} 
on MSVC), so it makes sense to provide extra command line arguments, but this will not always be the case. In the future 
\code{rustc} may use LLVM directly to link native libraries, in which case \code{link\_args} will have no meaning. You can 
achieve the same effect as the \code{link\_args} attribute with the \code{-C link-args} argument to \code{rustc}.

\blank

It is highly recommended to \emph{not} use this attribute, and rather use the more formal \code{\#[link(...)]} attribute on 
\code{extern} blocks instead.

\subsection*{Static linking}

Static linking refers to the process of creating output that contains all required libraries and so doesn't need libraries 
installed on every system where you want to use your compiled project. Pure-Rust dependencies are statically linked by default 
so you can use created binaries and libraries without installing Rust everywhere. By contrast, native libraries (e.g. \code{libc} 
and \code{libm}) are usually dynamically linked, but it is possible to change this and statically link them as well.

\blank

Linking is a very platform-dependent topic, and static linking may not even be possible on some platforms! This section assumes 
some basic familiarity with linking on your platform of choice.

\subsubsection*{Linux}

By default, all Rust programs on Linux will link to the system \code{libc} along with a number of other libraries. Let's look 
at an example on a 64-bit Linux machine with GCC and \code{glibc} (by far the most common \code{libc} on Linux):

\begin{verbatim}
$ cat example.rs
fn main() {}
$ rustc example.rs
$ ldd example
        linux-vdso.so.1 =>  (0x00007ffd565fd000)
        libdl.so.2 => /lib/x86_64-linux-gnu/libdl.so.2 (0x00007fa81889c000)
        libpthread.so.0 => /lib/x86_64-linux-gnu/libpthread.so.0 (0x00007fa81867e000)
        librt.so.1 => /lib/x86_64-linux-gnu/librt.so.1 (0x00007fa818475000)
        libgcc_s.so.1 => /lib/x86_64-linux-gnu/libgcc_s.so.1 (0x00007fa81825f000)
        libc.so.6 => /lib/x86_64-linux-gnu/libc.so.6 (0x00007fa817e9a000)
        /lib64/ld-linux-x86-64.so.2 (0x00007fa818cf9000)
        libm.so.6 => /lib/x86_64-linux-gnu/libm.so.6 (0x00007fa817b93000)
\end{verbatim}

Dynamic linking on Linux can be undesirable if you wish to use new library features on old systems or target systems which 
do not have the required dependencies for your program to run.

\blank

Static linking is supported via an alternative \code{libc}, \href{http://www.musl-libc.org/}{musl}. You can compile your own 
version of Rust with musl enabled and install it into a custom directory with the instructions below:

\begin{verbatim}
$ mkdir musldist
$ PREFIX=$(pwd)/musldist
$
$ # Build musl
$ curl -O http://www.musl-libc.org/releases/musl-1.1.10.tar.gz
$ tar xf musl-1.1.10.tar.gz
$ cd musl-1.1.10/
musl-1.1.10 $ ./configure --disable-shared --prefix=$PREFIX
musl-1.1.10 $ make
musl-1.1.10 $ make install
musl-1.1.10 $ cd ..
$ du -h musldist/lib/libc.a
2.2M    musldist/lib/libc.a
$
$ # Build libunwind.a
$ curl -O http://llvm.org/releases/3.7.0/llvm-3.7.0.src.tar.xz
$ tar xf llvm-3.7.0.src.tar.xz
$ cd llvm-3.7.0.src/projects/
llvm-3.7.0.src/projects $ curl http://llvm.org/releases/3.7.0/libunwind-3.7.0.src.tar.xz | tar xJf -
llvm-3.7.0.src/projects $ mv libunwind-3.7.0.src libunwind
llvm-3.7.0.src/projects $ mkdir libunwind/build
llvm-3.7.0.src/projects $ cd libunwind/build
llvm-3.7.0.src/projects/libunwind/build $ cmake -DLLVM_PATH=../../.. -DLIBUNWIND_ENABLE_SHARED=0 ..
llvm-3.7.0.src/projects/libunwind/build $ make
llvm-3.7.0.src/projects/libunwind/build $ cp lib/libunwind.a $PREFIX/lib/
llvm-3.7.0.src/projects/libunwind/build $ cd ../../../../
$ du -h musldist/lib/libunwind.a
164K    musldist/lib/libunwind.a
$
$ # Build musl-enabled rust
$ git clone https://github.com/rust-lang/rust.git muslrust
$ cd muslrust
muslrust $ ./configure --target=x86_64-unknown-linux-musl --musl-root=$PREFIX --prefix=$PREFIX
muslrust $ make
muslrust $ make install
muslrust $ cd ..
$ du -h musldist/bin/rustc
12K     musldist/bin/rustc
\end{verbatim}

You now have a build of a \code{musl}-enabled Rust! Because we've installed it to a custom prefix we need to make sure 
our system can find the binaries and appropriate libraries when we try and run it:

\begin{verbatim}
$ export PATH=$PREFIX/bin:$PATH
$ export LD_LIBRARY_PATH=$PREFIX/lib:$LD_LIBRARY_PATH
\end{verbatim}

Let's try it out!

\begin{verbatim}
$ echo 'fn main() { println!("hi!"); panic!("failed"); }' > example.rs
$ rustc --target=x86_64-unknown-linux-musl example.rs
$ ldd example
        not a dynamic executable
$ ./example
hi!
thread '<main>' panicked at 'failed', example.rs:1
\end{verbatim}

Success! This binary can be copied to almost any Linux machine with the same machine architecture and run without issues.

\blank

\code{cargo build} also permits the \code{--target} option so you should be able to build your crates as normal. However, you 
may need to recompile your native libraries against \code{musl} before they can be linked against.
