For extremely low-level manipulations and performance reasons, one might wish to control the CPU directly. Rust supports 
using inline assembly to do this via the \code{asm!} macro. The syntax roughly matches that of GCC \& Clang:

\begin{rustc}
asm!(assembly template
   : output operands
   : input operands
   : clobbers
   : options
   );
\end{rustc}

Any use of \code{asm} is feature gated (requires \code{\#![feature(asm)]} on the crate to allow) and of course requires 
an \code{unsafe} block.

\begin{myquote}
Note: the examples here are given in x86/x86-64 assembly, but all platforms are supported.
\end{myquote}

\subsection*{Assembly template}

The \code{assembly template} is the only required parameter and must be a literal string (i.e. \code{""})

\begin{rustc}
#![feature(asm)]

#[cfg(any(target_arch = "x86", target_arch = "x86_64"))]
fn foo() {
    unsafe {
        asm!("NOP");
    }
}

// other platforms
#[cfg(not(any(target_arch = "x86", target_arch = "x86_64")))]
fn foo() { /* ... */ }

fn main() {
    // ...
    foo();
    // ...
}
\end{rustc}

(The \code{feature(asm)} and \code{\#[cfg]}s are omitted from now on.)

\blank

Output operands, input operands, clobbers and options are all optional but you must add the right number of \code{:} 
if you skip them:

\begin{rustc}
asm!("xor %eax, %eax"
    :
    :
    : "{eax}"
   );
\end{rustc}

Whitespace also doesn't matter:

\begin{rustc}
asm!("xor %eax, %eax" ::: "{eax}");
\end{rustc}

\subsection*{Operands}

Input and output operands follow the same format: \code{: \enquote{constraints1}(expr1), \enquote{constraints2}(expr2), ..."}. 
Output operand expressions must be mutable lvalues, or not yet assigned:

\begin{rustc}
fn add(a: i32, b: i32) -> i32 {
    let c: i32;
    unsafe {
        asm!("add $2, $0"
             : "=r"(c)
             : "0"(a), "r"(b)
             );
    }
    c
}

fn main() {
    assert_eq!(add(3, 14159), 14162)
}
\end{rustc}

If you would like to use real operands in this position, however, you are required to put curly braces \code{\{\}} around the 
register that you want, and you are required to put the specific size of the operand. This is useful for very low level programming, 
where which register you use is important:

\begin{rustc}
let result: u8;
asm!("in %dx, %al" : "={al}"(result) : "{dx}"(port));
result
\end{rustc}

\subsection*{Clobbers}

Some instructions modify registers which might otherwise have held different values so we use the clobbers list to indicate 
to the compiler not to assume any values loaded into those registers will stay valid.

\begin{rustc}
// Put the value 0x200 in eax
asm!("mov $$0x200, %eax" : /* no outputs */ : /* no inputs */ : "{eax}");
\end{rustc}

Input and output registers need not be listed since that information is already communicated by the given constraints. 
Otherwise, any other registers used either implicitly or explicitly should be listed.

\blank

If the assembly changes the condition code register \code{cc} should be specified as one of the clobbers. Similarly, if 
the assembly modifies memory, \code{memory} should also be specified.

\subsection*{Options}

The last section, \code{options} is specific to Rust. The format is comma separated literal strings (i.e. 
\code{:\enquote{foo}, \enquote{bar}, \enquote{baz}}). It's used to specify some extra info about the inline assembly:

\blank

Current valid options are:

\begin{enumerate}
  \item{\emph{volatile} - specifying this is analogous to \code{\_\_asm\_\_ \_\_volatile\_\_ (...)} in gcc/clang.}
  \item{\emph{alignstack} - certain instructions expect the stack to be aligned a certain way (i.e. SSE) and specifying this 
      indicates to the compiler to insert its usual stack alignment code}
  \item{\emph{intel} - use intel syntax instead of the default AT\&T.}
\end{enumerate}

\begin{rustc}
let result: i32;
unsafe {
   asm!("mov eax, 2" : "={eax}"(result) : : : "intel")
}
println!("eax is currently {}", result);
\end{rustc}

\subsection*{More Information}

The current implementation of the \code{asm!} macro is a direct binding to 
\href{http://llvm.org/docs/LangRef.html\#inline-assembler-expressions}{LLVM's inline assembler expressions}, so be sure to 
check out \href{http://llvm.org/docs/LangRef.html\#inline-assembler-expressions}{their documentation} as well for more information 
about clobbers, constraints, etc.
