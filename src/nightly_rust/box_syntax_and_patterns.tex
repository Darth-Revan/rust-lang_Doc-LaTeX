Currently the only stable way to create a \code{Box} is via the \code{Box::new} method. Also it is not possible in stable 
Rust to destructure a \code{Box} in a match pattern. The unstable \code{box} keyword can be used to both create and destructure a 
\code{Box}. An example usage would be:

\begin{rustc}
#![feature(box_syntax, box_patterns)]

fn main() {
    let b = Some(box 5);
    match b {
        Some(box n) if n < 0 => {
            println!("Box contains negative number {}", n);
        },
        Some(box n) if n >= 0 => {
            println!("Box contains non-negative number {}", n);
        },
        None => {
            println!("No box");
        },
        _ => unreachable!()
    }
}
\end{rustc}

Note that these features are currently hidden behind the \code{box\_syntax} (box creation) and \code{box\_patterns} 
(destructuring and pattern matching) gates because the syntax may still change in the future.

\subsection*{Returning Pointers}

In many languages with pointers, you'd return a pointer from a function so as to avoid copying a large data structure. 
For example:

\begin{rustc}
struct BigStruct {
    one: i32,
    two: i32,
    // etc
    one_hundred: i32,
}

fn foo(x: Box<BigStruct>) -> Box<BigStruct> {
    Box::new(*x)
}

fn main() {
    let x = Box::new(BigStruct {
        one: 1,
        two: 2,
        one_hundred: 100,
    });

    let y = foo(x);
}
\end{rustc}

The idea is that by passing around a box, you're only copying a pointer, rather than the hundred \itt s that make up the \code{BigStruct}.

\blank

This is an antipattern in Rust. Instead, write this:

\begin{rustc}
#![feature(box_syntax)]

struct BigStruct {
    one: i32,
    two: i32,
    // etc
    one_hundred: i32,
}

fn foo(x: Box<BigStruct>) -> BigStruct {
    *x
}

fn main() {
    let x = Box::new(BigStruct {
        one: 1,
        two: 2,
        one_hundred: 100,
    });

    let y: Box<BigStruct> = box foo(x);
}
\end{rustc}

This gives you flexibility without sacrificing performance.

\blank

You may think that this gives us terrible performance: return a value and then immediately box it up ?! Isn't this pattern the 
worst of both worlds? Rust is smarter than that. There is no copy in this code. \code{main} allocates enough room for the \code{box}, 
passes a pointer to that memory into \code{foo} as \x, and then \code{foo} writes the value straight into the \code{Box<T>}.

\blank

This is important enough that it bears repeating: pointers are not for optimizing returning values from your code. Allow the caller 
to choose how they want to use your output.
