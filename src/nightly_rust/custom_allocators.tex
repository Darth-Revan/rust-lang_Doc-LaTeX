Allocating memory isn't always the easiest thing to do, and while Rust generally takes care of this by default it often 
becomes necessary to customize how allocation occurs. The compiler and standard library currently allow switching out the 
default global allocator in use at compile time. The design is currently spelled out in 
\href{https://github.com/rust-lang/rfcs/blob/master/text/1183-swap-out-jemalloc.md}{RFC 1183} but this will walk you through 
how to get your own allocator up and running.

\subsection*{Default Allocator}

The compiler currently ships two default allocators: \code{allo\_system} and \code{alloc\_jemalloc} (some targets don't have 
\code{jemalloc}, however). These allocators are normal Rust crates and contain an implementation of the routines to allocate 
and deallocate memory. The standard library is not compiled assuming either one, and the compiler will decide which allocator 
is in use at compile-time depending on the type of output artifact being produced.

\blank

Binaries generated by the compiler will use \code{alloc\_jemalloc} by default (where available). In this situation the 
compiler \enquote{controls the world} in the sense of it has power over the final link. Primarily this means that the 
allocator decision can be left up the compiler.

\blank

Dynamic and static libraries, however, will use \code{alloc\_system} by default. Here Rust is typically a 'guest' in another 
application or another world where it cannot authoritatively decide what allocator is in use. As a result it resorts back to 
the standard APIs (e.g. \code{malloc} and \code{free}) for acquiring and releasing memory.

\subsection*{Switching Allocators}

Although the compiler's default choices may work most of the time, it's often necessary to tweak certain aspects. Overriding 
the compiler's decision about which allocator is in use is done simply by linking to the desired allocator:

\begin{rustc}
#![feature(alloc_system)]

extern crate alloc_system;

fn main() {
    let a = Box::new(4); // allocates from the system allocator
    println!("{}", a);
}
\end{rustc}

In this example the binary generated will not link to jemalloc by default but instead use the system allocator. Conversely 
to generate a dynamic library which uses jemalloc by default one would write:

\begin{rustc}
#![feature(alloc_jemalloc)]
#![crate_type = "dylib"]

extern crate alloc_jemalloc;

pub fn foo() {
    let a = Box::new(4); // allocates from jemalloc
    println!("{}", a);
}
\end{rustc}

\subsection*{Writing a custom allocator}

Sometimes even the choices of jemalloc vs the system allocator aren't enough and an entirely new custom allocator is required. 
In this you'll write your own crate which implements the allocator API (e.g. the same as \code{alloc\_system} or 
\code{alloc\_jemalloc}). As an example, let's take a look at a simplified and annotated version of \code{alloc\_system}.

\begin{rustc}
// The compiler needs to be instructed that this crate is an allocator in order
// to realize that when this is linked in another allocator like jemalloc should
// not be linked in
#![feature(allocator)]
#![allocator]

// Allocators are not allowed to depend on the standard library which in turn
// requires an allocator in order to avoid circular dependencies. This crate,
// however, can use all of libcore.
#![no_std]

// Let's give a unique name to our custom allocator
#![crate_name = "my_allocator"]
#![crate_type = "rlib"]

// Our system allocator will use the in-tree libc crate for FFI bindings. Note
// that currently the external (crates.io) libc cannot be used because it links
// to the standard library (e.g. `#![no_std]` isn't stable yet), so that's why
// this specifically requires the in-tree version.
#![feature(libc)]
extern crate libc;

// Listed below are the five allocation functions currently required by custom
// allocators. Their signatures and symbol names are not currently typechecked
// by the compiler, but this is a future extension and are required to match
// what is found below.
//
// Note that the standard `malloc` and `realloc` functions do not provide a way
// to communicate alignment so this implementation would need to be improved
// with respect to alignment in that aspect.

#[no_mangle]
pub extern fn __rust_allocate(size: usize, _align: usize) -> *mut u8 {
    unsafe { libc::malloc(size as libc::size_t) as *mut u8 }
}

#[no_mangle]
pub extern fn __rust_deallocate(ptr: *mut u8, _old_size: usize, _align: usize) {
    unsafe { libc::free(ptr as *mut libc::c_void) }
}

#[no_mangle]
pub extern fn __rust_reallocate(ptr: *mut u8, _old_size: usize, size: usize,
                                _align: usize) -> *mut u8 {
    unsafe {
        libc::realloc(ptr as *mut libc::c_void, size as libc::size_t) as *mut u8
    }
}

#[no_mangle]
pub extern fn __rust_reallocate_inplace(_ptr: *mut u8, old_size: usize,
                                        _size: usize, _align: usize) -> usize {
    old_size // this api is not supported by libc
}

#[no_mangle]
pub extern fn __rust_usable_size(size: usize, _align: usize) -> usize {
    size
}
\end{rustc}

After we compile this crate, it can be used as follows:

\begin{rustc}
extern crate my_allocator;

fn main() {
    let a = Box::new(8); // allocates memory via our custom allocator crate
    println!("{}", a);
}
\end{rustc}

\subsection*{Custom allocator limitations}

There are a few restrictions when working with custom allocators which may cause compiler errors:

\begin{itemize}
  \item{Any one artifact may only be linked to at most one allocator. Binaries, dylibs, and staticlibs must link to exactly one 
      allocator, and if none have been explicitly chosen the compiler will choose one. On the other hand rlibs do not need to 
      link to an allocator (but still can).}
  \item{A consumer of an allocator is tagged with \code{\#![needs\_allocator]} (e.g. the \code{liballoc} crate currently) and 
      an \code{\#[allocator]} crate cannot transitively depend on a crate which needs an allocator (e.g. circular dependencies 
      are not allowed). This basically means that allocators must restrict themselves to libcore currently.}
\end{itemize}
