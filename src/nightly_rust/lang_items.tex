\begin{myquote}
Note: lang items are often provided by crates in the Rust distribution, and lang items themselves have an unstable 
interface. It is recommended to use officially distributed crates instead of defining your own lang items.
\end{myquote}

The \code{rustc} compiler has certain pluggable operations, that is, functionality that isn't hard-coded into the language, 
but is implemented in libraries, with a special marker to tell the compiler it exists. The marker is the attribute 
\code{\#[lang = \enquote{...}]} and there are various different values of \code{...}, i.e. various different 'lang items'.

\blank

For example, \code{Box} pointers require two lang items, one for allocation and one for deallocation. A freestanding program 
that uses the \code{Box} sugar for dynamic allocations via \code{malloc} and \code{free}:

\begin{rustc}
#![feature(lang_items, box_syntax, start, libc)]
#![no_std]

extern crate libc;

extern {
    fn abort() -> !;
}

#[lang = "owned_box"]
pub struct Box<T>(*mut T);

#[lang = "exchange_malloc"]
unsafe fn allocate(size: usize, _align: usize) -> *mut u8 {
    let p = libc::malloc(size as libc::size_t) as *mut u8;

    // malloc failed
    if p as usize == 0 {
        abort();
    }

    p
}

#[lang = "exchange_free"]
unsafe fn deallocate(ptr: *mut u8, _size: usize, _align: usize) {
    libc::free(ptr as *mut libc::c_void)
}

#[lang = "box_free"]
unsafe fn box_free<T>(ptr: *mut T) {
    deallocate(ptr as *mut u8, ::core::mem::size_of::<T>(), ::core::mem::align_of::<T>());
}

#[start]
fn main(argc: isize, argv: *const *const u8) -> isize {
    let x = box 1;

    0
}

#[lang = "eh_personality"] extern fn eh_personality() {}
#[lang = "panic_fmt"] fn panic_fmt() -> ! { loop {} }
\end{rustc}

Note the use of \code{abort}: the \code{exchange\_malloc} lang item is assumed to return a valid pointer, and so needs to 
do the check internally.

\blank

Other features provided by lang items include:

\begin{itemize}
  \item{overloadable operators via traits: the traits corresponding to the \code{==}, \code{<}, dereferencing (\code{*}) 
      and \code{+} (etc.) operators are all marked with lang items; those specific four are \code{eq}, \code{ord}, \code{deref}, 
      and \code{add} respectively.}
  \item{stack unwinding and general failure; the \code{eh\_personality}, \code{fail} and \code{fail\_bounds\_checks} lang items.}
  \item{the traits in \code{std::marker} used to indicate types of various kinds; lang items \code{send}, \code{sync} and \code{copy}.}
  \item{the marker types and variance indicators found in \code{std::marker}; lang items \code{covariant\_type}, 
      \code{contravariant\_lifetime}, etc.}
\end{itemize}

Lang items are loaded lazily by the compiler; e.g. if one never uses \code{Box} then there is no need to define functions 
for \code{exchange\_malloc} and \code{exchange\_free}. \code{rustc} will emit an error when an item is needed but not found 
in the current crate or any that it depends on.
