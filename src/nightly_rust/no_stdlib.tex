Rust's standard library provides a lot of useful functionality, but assumes support for various features of its host system: 
threads, networking, heap allocation, and others. There are systems that do not have these features, however, and Rust can work 
with those too! To do so, we tell Rust that we don't want to use the standard library via an attribute: \code{\#![no\_std]}.

\begin{myquote}
Note: This feature is technically stable, but there are some caveats. For one, you can build a \code{\#![no\_std]} \emph{library} 
on stable, but not a \emph{binary}. For details on libraries without the standard library, see the chapter on \code{\#![no\_std]} (
see \nameref{sec:effective_rustWithoutStdLib}).
\end{myquote}

Obviously there's more to life than just libraries: one can use \code{\#[no\_std]} with an executable, controlling the entry 
point is possible in two ways: the \code{\#[start]} attribute, or overriding the default shim for the C \code{main} function 
with your own.

\blank

The function marked \code{\#[start]} is passed the command line parameters in the same format as C:

\begin{rustc}
#![feature(lang_items)]
#![feature(start)]
#![no_std]

// Pull in the system libc library for what crt0.o likely requires
extern crate libc;

// Entry point for this program
#[start]
fn start(_argc: isize, _argv: *const *const u8) -> isize {
    0
}

// These functions and traits are used by the compiler, but not
// for a bare-bones hello world. These are normally
// provided by libstd.
#[lang = "eh_personality"] extern fn eh_personality() {}
#[lang = "panic_fmt"] fn panic_fmt() -> ! { loop {} }
\end{rustc}

To override the compiler-inserted \code{main} shim, one has to disable it with \code{\#![no\_main]} and then create the 
appropriate symbol with the correct ABI and the correct name, which requires overriding the compiler's name mangling too:

\begin{rustc}
#![feature(lang_items)]
#![feature(start)]
#![no_std]
#![no_main]

extern crate libc;

#[no_mangle] // ensure that this symbol is called `main` in the output
pub extern fn main(argc: i32, argv: *const *const u8) -> i32 {
    0
}

#[lang = "eh_personality"] extern fn eh_personality() {}
#[lang = "panic_fmt"] fn panic_fmt() -> ! { loop {} }
\end{rustc}

The compiler currently makes a few assumptions about symbols which are available in the executable to call. Normally 
these functions are provided by the standard library, but without it you must define your own.

\blank

The first of these two functions, \code{eh\_personality}, is used by the failure mechanisms of the compiler. This is often 
mapped to GCC's personality function (see the 
\href{https://github.com/rust-lang/rust/blob/master/src/libstd/sys/common/unwind/gcc.rs}{libstd implementation} for more 
information), but crates which do not trigger a panic can be assured that this function is never called. The second function, 
\code{panic\_fmt}, is also used by the failure mechanisms of the compiler.
