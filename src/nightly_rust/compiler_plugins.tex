\subsection*{Introduction}

\code{rustc} can load compiler plugins, which are user-provided libraries that extend the compiler's behavior with new syntax 
extensions, lint checks, etc.

\blank

A plugin is a dynamic library crate with a designated registrar function that registers extensions with \code{rustc}. Other 
crates can load these extensions using the crate attribute \code{\#![plugin(...)]}. See the 
\href{https://doc.rust-lang.org/rustc\_plugin/}{rustc\_plugin} documentation for more about the mechanics of defining and 
loading a plugin.

\blank

If present, arguments passed as \code{\#![plugin(foo(... args ...))]} are not interpreted by \code{rustc} itself. They are 
provided to the plugin through the \code{Registry}'s 
\href{https://doc.rust-lang.org/rustc\_plugin/registry/struct.Registry.html\#method.args}{args method}.

\blank

In the vast majority of cases, a plugin should \emph{only} be used through \code{\#![plugin]} and not through an \code{extern crate} 
item. Linking a plugin would pull in all of \code{libsyntax} and \code{librustc} as dependencies of your crate. This is generally 
unwanted unless you are building another plugin. The \code{plugin\_as\_library} lint checks these guidelines.

\blank

The usual practice is to put compiler plugins in their own crate, separate from any \code{macro\_rules!} macros or ordinary Rust 
code meant to be used by consumers of a library.

\subsection*{Syntax extensions}

Plugins can extend Rust's syntax in various ways. One kind of syntax extension is the procedural macro. These are invoked 
the same way as ordinary macros (see \nameref{sec:syntax_macros}), but the expansion is performed by arbitrary Rust code 
that manipulates \href{https://doc.rust-lang.org/syntax/ast/}{syntax trees} at compile time.

\blank

Let's write a plugin \href{https://github.com/rust-lang/rust/tree/master/src/test/auxiliary/roman\_numerals.rs}{roman\_numerals.rs} 
that implements Roman numeral integer literals.

\begin{rustc}
#![crate_type="dylib"]
#![feature(plugin_registrar, rustc_private)]

extern crate syntax;
extern crate rustc;
extern crate rustc_plugin;

use syntax::codemap::Span;
use syntax::parse::token;
use syntax::ast::TokenTree;
use syntax::ext::base::{ExtCtxt, MacResult, DummyResult, MacEager};
use syntax::ext::build::AstBuilder;  // trait for expr_usize
use rustc_plugin::Registry;

fn expand_rn(cx: &mut ExtCtxt, sp: Span, args: &[TokenTree])
        -> Box<MacResult + 'static> {

    static NUMERALS: &'static [(&'static str, usize)] = &[
        ("M", 1000), ("CM", 900), ("D", 500), ("CD", 400),
        ("C",  100), ("XC",  90), ("L",  50), ("XL",  40),
        ("X",   10), ("IX",   9), ("V",   5), ("IV",   4),
        ("I",    1)];

    if args.len() != 1 {
        cx.span_err(
            sp,
            &format!("argument should be a single identifier, but got {} arguments", args.len()));
        return DummyResult::any(sp);
    }

    let text = match args[0] {
        TokenTree::Token(_, token::Ident(s, _)) => s.to_string(),
        _ => {
            cx.span_err(sp, "argument should be a single identifier");
            return DummyResult::any(sp);
        }
    };

    let mut text = &*text;
    let mut total = 0;
    while !text.is_empty() {
        match NUMERALS.iter().find(|&&(rn, _)| text.starts_with(rn)) {
            Some(&(rn, val)) => {
                total += val;
                text = &text[rn.len()..];
            }
            None => {
                cx.span_err(sp, "invalid Roman numeral");
                return DummyResult::any(sp);
            }
        }
    }

    MacEager::expr(cx.expr_usize(sp, total))
}

#[plugin_registrar]
pub fn plugin_registrar(reg: &mut Registry) {
    reg.register_macro("rn", expand_rn);
}
\end{rustc}

Then we can use \code{rn!()} like any other macro:

\begin{rustc}
#![feature(plugin)]
#![plugin(roman_numerals)]

fn main() {
    assert_eq!(rn!(MMXV), 2015);
}
\end{rustc}

The advantages over a simple \code{fn(\&str) -> u32} are:

\begin{itemize}
  \item{The (arbitrarily complex) conversion is done at compile time.}
  \item{Input validation is also performed at compile time.}
  \item{It can be extended to allow use in patterns, which effectively gives a way to define new literal syntax for any data type.}
\end{itemize}

In addition to procedural macros, you can define new \href{https://doc.rust-lang.org/reference.html\#derive}{derive}-like 
attributes and other kinds of extensions. See 
\href{https://doc.rust-lang.org/rustc\_plugin/registry/struct.Registry.html\#method.register\_syntax\_extension}
{Registry::register\_syntax\_extension} and the \href{https://doc.rust-lang.org/syntax/ext/base/enum.SyntaxExtension.html}
{SyntaxExtension enum}. For a more involved macro example, see 
\href{https://github.com/rust-lang/regex/blob/master/regex\_macros/src/lib.rs}{regex\_macros}.

\subsection*{Tips and tricks}

Some of the macro debugging tips are applicable.

\blank

You can use \href{https://doc.rust-lang.org/syntax/parse/}{syntax::parse} to turn token trees into higher-level syntax elements 
like expressions:

\begin{rustc}
fn expand_foo(cx: &mut ExtCtxt, sp: Span, args: &[TokenTree])
        -> Box<MacResult+'static> {

    let mut parser = cx.new_parser_from_tts(args);

    let expr: P<Expr> = parser.parse_expr();
\end{rustc}

Looking through \href{https://github.com/rust-lang/rust/blob/master/src/libsyntax/parse/parser.rs}{libsyntax parser code} will 
give you a feel for how the parsing infrastructure works.

\blank

Keep the \href{https://doc.rust-lang.org/syntax/codemap/struct.Span.html}{Spans} of everything you parse, for better error 
reporting. You can wrap \href{https://doc.rust-lang.org/syntax/codemap/struct.Spanned.html}{Spanned} around your custom data structures.

\blank

Calling \href{https://doc.rust-lang.org/syntax/ext/base/struct.ExtCtxt.html\#method.span\_fatal}{ExtCtxt::span\_fatal} will 
immediately abort compilation. It's better to instead call 
\href{https://doc.rust-lang.org/syntax/ext/base/struct.ExtCtxt.html\#method.span\_err}{ExtCtxt::span\_err} and return 
\href{https://doc.rust-lang.org/syntax/ext/base/struct.DummyResult.html}{DummyResult}, so that the compiler can continue and 
find further errors.

\blank

To print syntax fragments for debugging, you can use 
\href{https://doc.rust-lang.org/syntax/ext/base/struct.ExtCtxt.html\#method.span\_note}{span\_note} together with 
\href{https://doc.rust-lang.org/syntax/print/pprust/\#functions}{syntax::print::pprust::*\_to\_string}.

\blank

The example above produced an integer literal using 
\href{https://doc.rust-lang.org/syntax/ext/build/trait.AstBuilder.html\#tymethod.expr\_usize}{AstBuilder::expr\_usize}. As 
an alternative to the \code{AstBuilder} trait, \code{libsyntax} provides a set of 
\href{https://doc.rust-lang.org/syntax/ext/quote/}{quasiquote macros}. They are undocumented and very rough around the edges. 
However, the implementation may be a good starting point for an improved quasiquote as an ordinary plugin library.

\subsection*{Lint plugins}

Plugins can extend \href{https://doc.rust-lang.org/reference.html\#lint-check-attributes}{Rust's lint infrastructure} with 
additional checks for code style, safety, etc. Now let's write a plugin 
\href{https://github.com/rust-lang/rust/blob/master/src/test/auxiliary/lint\_plugin\_test.rs}{lint\_plugin\_test.rs} that warns 
about any item named lintme.

\begin{rustc}
#![feature(plugin_registrar)]
#![feature(box_syntax, rustc_private)]

extern crate syntax;

// Load rustc as a plugin to get macros
#[macro_use]
extern crate rustc;
extern crate rustc_plugin;

use rustc::lint::{EarlyContext, LintContext, LintPass, EarlyLintPass,
                  EarlyLintPassObject, LintArray};
use rustc_plugin::Registry;
use syntax::ast;

declare_lint!(TEST_LINT, Warn, "Warn about items named 'lintme'");

struct Pass;

impl LintPass for Pass {
    fn get_lints(&self) -> LintArray {
        lint_array!(TEST_LINT)
    }
}

impl EarlyLintPass for Pass {
    fn check_item(&mut self, cx: &EarlyContext, it: &ast::Item) {
        if it.ident.name.as_str() == "lintme" {
            cx.span_lint(TEST_LINT, it.span, "item is named 'lintme'");
        }
    }
}

#[plugin_registrar]
pub fn plugin_registrar(reg: &mut Registry) {
    reg.register_early_lint_pass(box Pass as EarlyLintPassObject);
}
\end{rustc}

Then code like

\begin{rustc}
#![plugin(lint_plugin_test)]

fn lintme() { }
\end{rustc}

will produce a compiler warning:

\begin{verbatim}
foo.rs:4:1: 4:16 warning: item is named 'lintme', #[warn(test_lint)] on by default
foo.rs:4 fn lintme() { }
         ^~~~~~~~~~~~~~~
\end{verbatim}

The components of a lint plugin are:

\begin{itemize}
  \item{one or more \code{declare\_lint!} invocations, which define static 
      \href{https://doc.rust-lang.org/rustc/lint/struct.Lint.html}{Lint} structs;}
  \item{a struct holding any state needed by the lint pass (here, none);}
  \item{a \href{https://doc.rust-lang.org/rustc/lint/trait.LintPass.html}{LintPass} implementation defining how to check each 
      syntax element. A single \code{LintPass} may call \code{span\_lint} for several different \code{Lint}s, but should register 
      them all through the \code{get\_lints} method.}
\end{itemize}

Lint passes are syntax traversals, but they run at a late stage of compilation where type information is available. 
\code{rustc}'s \href{https://github.com/rust-lang/rust/blob/master/src/librustc/lint/builtin.rs}{built-in lints} mostly use 
the same infrastructure as lint plugins, and provide examples of how to access type information.

\blank

Lints defined by plugins are controlled by the usual \href{https://doc.rust-lang.org/reference.html\#lint-check-attributes}{attributes 
and compiler flags}, e.g. \code{\#[allow(test\_lint)]} or \code{-A test-lint}. These identifiers are derived from the first argument to 
\code{declare\_lint!}, with appropriate case and punctuation conversion.

\blank

You can run \code{rustc -W help foo.rs} to see a list of lints known to \code{rustc}, including those provided by plugins loaded 
by \code{foo.rs}.
