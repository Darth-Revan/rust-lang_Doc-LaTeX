Rust provides three distribution channels for Rust: nightly, beta, and stable. Unstable features are only available 
on nightly Rust. For more details on this process, see \href{http://blog.rust-lang.org/2014/10/30/Stability.html}{Stability 
as a deliverable}.

\blank

To install nightly Rust, you can use \code{rustup.sh}:

\begin{verbatim}
$ curl -s https://static.rust-lang.org/rustup.sh | sh -s -- --channel=nightly
\end{verbatim}

If you're concerned about the \href{http://curlpipesh.tumblr.com/}{potential insecurity} of using \code{curl | sh}, 
please keep reading and see our disclaimer below. And feel free to use a two-step version of the installation and examine 
our installation script:

\begin{verbatim}
$ curl -f -L https://static.rust-lang.org/rustup.sh -O
$ sh rustup.sh --channel=nightly
\end{verbatim}

If you're on Windows, please download either the 
\href{https://static.rust-lang.org/dist/rust-nightly-i686-pc-windows-gnu.msi}{32-bit installer} or the 
\href{https://static.rust-lang.org/dist/rust-nightly-x86_64-pc-windows-gnu.msi}{64-bit installer} and run it.

\subsection*{Uninstalling}

If you decide you don't want Rust anymore, we'll be a bit sad, but that's okay. Not every programming language is great 
for everyone. Just run the uninstall script:

\begin{verbatim}
$ sudo /usr/local/lib/rustlib/uninstall.sh
\end{verbatim}

If you used the Windows installer, re-run the \code{.msi} and it will give you an uninstall option.

\blank

Some people, and somewhat rightfully so, get very upset when we tell you to \code{curl | sh}. Basically, when you do this, 
you are trusting that the good people who maintain Rust aren't going to hack your computer and do bad things. That's a good 
instinct! If you're one of those people, please check out the documentation on 
\href{https://github.com/rust-lang/rust\#building-from-source}{building Rust from Source}, or 
\href{https://www.rust-lang.org/install.html}{the official binary downloads}.

\blank

Oh, we should also mention the officially supported platforms:

\begin{itemize}
  \item{Windows (7, 8, Server 2008 R2)}
  \item{Linux (2.6.18 or later, various distributions), x86 and x86-64}
  \item{OSX 10.7 (Lion) or greater, x86 and x86-64}
\end{itemize}

We extensively test Rust on these platforms, and a few others, too, like Android. But these are the ones most likely 
to work, as they have the most testing.

\blank

Finally, a comment about Windows. Rust considers Windows to be a first-class platform upon release, but if we're honest, 
the Windows experience isn't as integrated as the Linux/OS X experience is. We're working on it! If anything does not work, 
it is a bug. Please let us know if that happens. Each and every commit is tested against Windows like any other platform.

\blank

If you've got Rust installed, you can open up a shell, and type this:

\begin{verbatim}
$ rustc --version
\end{verbatim}

You should see the version number, commit hash, commit date and build date:

\begin{verbatim}
rustc 1.0.0-nightly (f11f3e7ba 2015-01-04) (built 2015-01-06)
\end{verbatim}

If you did, Rust has been installed successfully! Congrats!

\blank

This installer also installs a copy of the documentation locally, so you can read it offline. On UNIX systems, 
\code{/usr/local/share/doc/rust} is the location. On Windows, it's in a \code{share/doc} directory, inside wherever you 
installed Rust to.

\blank

If not, there are a number of places where you can get help. The easiest is the 
\href{irc://irc.mozilla.org/\#rust}{\#rust IRC channel on irc.mozilla.org}, which you can access through 
\href{http://chat.mibbit.com/?server=irc.mozilla.org\&channel=\%23rust}{Mibbit}. Click that link, and you'll be chatting with 
other Rustaceans (a silly nickname we call ourselves), and we can help you out. Other great resources include the 
\href{https://users.rust-lang.org/}{user's forum}, and \href{http://stackoverflow.com/questions/tagged/rust}{Stack Overflow}.

\section{Compiler Plugins}
\label{sec:nightly_compilerPlugins}
\subsection*{Introduction}

\code{rustc} can load compiler plugins, which are user-provided libraries that extend the compiler's behavior with new syntax 
extensions, lint checks, etc.

\blank

A plugin is a dynamic library crate with a designated registrar function that registers extensions with \code{rustc}. Other 
crates can load these extensions using the crate attribute \code{\#![plugin(...)]}. See the 
\href{https://doc.rust-lang.org/rustc\_plugin/}{rustc\_plugin} documentation for more about the mechanics of defining and 
loading a plugin.

\blank

If present, arguments passed as \code{\#![plugin(foo(... args ...))]} are not interpreted by \code{rustc} itself. They are 
provided to the plugin through the \code{Registry}'s 
\href{https://doc.rust-lang.org/rustc\_plugin/registry/struct.Registry.html\#method.args}{args method}.

\blank

In the vast majority of cases, a plugin should \emph{only} be used through \code{\#![plugin]} and not through an \code{extern crate} 
item. Linking a plugin would pull in all of \code{libsyntax} and \code{librustc} as dependencies of your crate. This is generally 
unwanted unless you are building another plugin. The \code{plugin\_as\_library} lint checks these guidelines.

\blank

The usual practice is to put compiler plugins in their own crate, separate from any \code{macro\_rules!} macros or ordinary Rust 
code meant to be used by consumers of a library.

\subsection*{Syntax extensions}

Plugins can extend Rust's syntax in various ways. One kind of syntax extension is the procedural macro. These are invoked 
the same way as ordinary macros (see \nameref{sec:syntax_macros}), but the expansion is performed by arbitrary Rust code 
that manipulates \href{https://doc.rust-lang.org/syntax/ast/}{syntax trees} at compile time.

\blank

Let's write a plugin \href{https://github.com/rust-lang/rust/tree/master/src/test/auxiliary/roman\_numerals.rs}{roman\_numerals.rs} 
that implements Roman numeral integer literals.

\begin{rustc}
#![crate_type="dylib"]
#![feature(plugin_registrar, rustc_private)]

extern crate syntax;
extern crate rustc;
extern crate rustc_plugin;

use syntax::codemap::Span;
use syntax::parse::token;
use syntax::ast::TokenTree;
use syntax::ext::base::{ExtCtxt, MacResult, DummyResult, MacEager};
use syntax::ext::build::AstBuilder;  // trait for expr_usize
use rustc_plugin::Registry;

fn expand_rn(cx: &mut ExtCtxt, sp: Span, args: &[TokenTree])
        -> Box<MacResult + 'static> {

    static NUMERALS: &'static [(&'static str, usize)] = &[
        ("M", 1000), ("CM", 900), ("D", 500), ("CD", 400),
        ("C",  100), ("XC",  90), ("L",  50), ("XL",  40),
        ("X",   10), ("IX",   9), ("V",   5), ("IV",   4),
        ("I",    1)];

    if args.len() != 1 {
        cx.span_err(
            sp,
            &format!("argument should be a single identifier, but got {} arguments", args.len()));
        return DummyResult::any(sp);
    }

    let text = match args[0] {
        TokenTree::Token(_, token::Ident(s, _)) => s.to_string(),
        _ => {
            cx.span_err(sp, "argument should be a single identifier");
            return DummyResult::any(sp);
        }
    };

    let mut text = &*text;
    let mut total = 0;
    while !text.is_empty() {
        match NUMERALS.iter().find(|&&(rn, _)| text.starts_with(rn)) {
            Some(&(rn, val)) => {
                total += val;
                text = &text[rn.len()..];
            }
            None => {
                cx.span_err(sp, "invalid Roman numeral");
                return DummyResult::any(sp);
            }
        }
    }

    MacEager::expr(cx.expr_usize(sp, total))
}

#[plugin_registrar]
pub fn plugin_registrar(reg: &mut Registry) {
    reg.register_macro("rn", expand_rn);
}
\end{rustc}

Then we can use \code{rn!()} like any other macro:

\begin{rustc}
#![feature(plugin)]
#![plugin(roman_numerals)]

fn main() {
    assert_eq!(rn!(MMXV), 2015);
}
\end{rustc}

The advantages over a simple \code{fn(\&str) -> u32} are:

\begin{itemize}
  \item{The (arbitrarily complex) conversion is done at compile time.}
  \item{Input validation is also performed at compile time.}
  \item{It can be extended to allow use in patterns, which effectively gives a way to define new literal syntax for any data type.}
\end{itemize}

In addition to procedural macros, you can define new \href{https://doc.rust-lang.org/reference.html\#derive}{derive}-like 
attributes and other kinds of extensions. See 
\href{https://doc.rust-lang.org/rustc\_plugin/registry/struct.Registry.html\#method.register\_syntax\_extension}
{Registry::register\_syntax\_extension} and the \href{https://doc.rust-lang.org/syntax/ext/base/enum.SyntaxExtension.html}
{SyntaxExtension enum}. For a more involved macro example, see 
\href{https://github.com/rust-lang/regex/blob/master/regex\_macros/src/lib.rs}{regex\_macros}.

\subsection*{Tips and tricks}

Some of the macro debugging tips are applicable.

\blank

You can use \href{https://doc.rust-lang.org/syntax/parse/}{syntax::parse} to turn token trees into higher-level syntax elements 
like expressions:

\begin{rustc}
fn expand_foo(cx: &mut ExtCtxt, sp: Span, args: &[TokenTree])
        -> Box<MacResult+'static> {

    let mut parser = cx.new_parser_from_tts(args);

    let expr: P<Expr> = parser.parse_expr();
\end{rustc}

Looking through \href{https://github.com/rust-lang/rust/blob/master/src/libsyntax/parse/parser.rs}{libsyntax parser code} will 
give you a feel for how the parsing infrastructure works.

\blank

Keep the \href{https://doc.rust-lang.org/syntax/codemap/struct.Span.html}{Spans} of everything you parse, for better error 
reporting. You can wrap \href{https://doc.rust-lang.org/syntax/codemap/struct.Spanned.html}{Spanned} around your custom data structures.

\blank

Calling \href{https://doc.rust-lang.org/syntax/ext/base/struct.ExtCtxt.html\#method.span\_fatal}{ExtCtxt::span\_fatal} will 
immediately abort compilation. It's better to instead call 
\href{https://doc.rust-lang.org/syntax/ext/base/struct.ExtCtxt.html\#method.span\_err}{ExtCtxt::span\_err} and return 
\href{https://doc.rust-lang.org/syntax/ext/base/struct.DummyResult.html}{DummyResult}, so that the compiler can continue and 
find further errors.

\blank

To print syntax fragments for debugging, you can use 
\href{https://doc.rust-lang.org/syntax/ext/base/struct.ExtCtxt.html\#method.span\_note}{span\_note} together with 
\href{https://doc.rust-lang.org/syntax/print/pprust/\#functions}{syntax::print::pprust::*\_to\_string}.

\blank

The example above produced an integer literal using 
\href{https://doc.rust-lang.org/syntax/ext/build/trait.AstBuilder.html\#tymethod.expr\_usize}{AstBuilder::expr\_usize}. As 
an alternative to the \code{AstBuilder} trait, \code{libsyntax} provides a set of 
\href{https://doc.rust-lang.org/syntax/ext/quote/}{quasiquote macros}. They are undocumented and very rough around the edges. 
However, the implementation may be a good starting point for an improved quasiquote as an ordinary plugin library.

\subsection*{Lint plugins}

Plugins can extend \href{https://doc.rust-lang.org/reference.html\#lint-check-attributes}{Rust's lint infrastructure} with 
additional checks for code style, safety, etc. Now let's write a plugin 
\href{https://github.com/rust-lang/rust/blob/master/src/test/auxiliary/lint\_plugin\_test.rs}{lint\_plugin\_test.rs} that warns 
about any item named lintme.

\begin{rustc}
#![feature(plugin_registrar)]
#![feature(box_syntax, rustc_private)]

extern crate syntax;

// Load rustc as a plugin to get macros
#[macro_use]
extern crate rustc;
extern crate rustc_plugin;

use rustc::lint::{EarlyContext, LintContext, LintPass, EarlyLintPass,
                  EarlyLintPassObject, LintArray};
use rustc_plugin::Registry;
use syntax::ast;

declare_lint!(TEST_LINT, Warn, "Warn about items named 'lintme'");

struct Pass;

impl LintPass for Pass {
    fn get_lints(&self) -> LintArray {
        lint_array!(TEST_LINT)
    }
}

impl EarlyLintPass for Pass {
    fn check_item(&mut self, cx: &EarlyContext, it: &ast::Item) {
        if it.ident.name.as_str() == "lintme" {
            cx.span_lint(TEST_LINT, it.span, "item is named 'lintme'");
        }
    }
}

#[plugin_registrar]
pub fn plugin_registrar(reg: &mut Registry) {
    reg.register_early_lint_pass(box Pass as EarlyLintPassObject);
}
\end{rustc}

Then code like

\begin{rustc}
#![plugin(lint_plugin_test)]

fn lintme() { }
\end{rustc}

will produce a compiler warning:

\begin{verbatim}
foo.rs:4:1: 4:16 warning: item is named 'lintme', #[warn(test_lint)] on by default
foo.rs:4 fn lintme() { }
         ^~~~~~~~~~~~~~~
\end{verbatim}

The components of a lint plugin are:

\begin{itemize}
  \item{one or more \code{declare\_lint!} invocations, which define static 
      \href{https://doc.rust-lang.org/rustc/lint/struct.Lint.html}{Lint} structs;}
  \item{a struct holding any state needed by the lint pass (here, none);}
  \item{a \href{https://doc.rust-lang.org/rustc/lint/trait.LintPass.html}{LintPass} implementation defining how to check each 
      syntax element. A single \code{LintPass} may call \code{span\_lint} for several different \code{Lint}s, but should register 
      them all through the \code{get\_lints} method.}
\end{itemize}

Lint passes are syntax traversals, but they run at a late stage of compilation where type information is available. 
\code{rustc}'s \href{https://github.com/rust-lang/rust/blob/master/src/librustc/lint/builtin.rs}{built-in lints} mostly use 
the same infrastructure as lint plugins, and provide examples of how to access type information.

\blank

Lints defined by plugins are controlled by the usual \href{https://doc.rust-lang.org/reference.html\#lint-check-attributes}{attributes 
and compiler flags}, e.g. \code{\#[allow(test\_lint)]} or \code{-A test-lint}. These identifiers are derived from the first argument to 
\code{declare\_lint!}, with appropriate case and punctuation conversion.

\blank

You can run \code{rustc -W help foo.rs} to see a list of lints known to \code{rustc}, including those provided by plugins loaded 
by \code{foo.rs}.


\section{Inline Assembly}
\label{sec:nightly_inlineAssembly}
For extremely low-level manipulations and performance reasons, one might wish to control the CPU directly. Rust supports 
using inline assembly to do this via the \code{asm!} macro. The syntax roughly matches that of GCC \& Clang:

\begin{rustc}
asm!(assembly template
   : output operands
   : input operands
   : clobbers
   : options
   );
\end{rustc}

Any use of \code{asm} is feature gated (requires \code{\#![feature(asm)]} on the crate to allow) and of course requires 
an \code{unsafe} block.

\begin{myquote}
Note: the examples here are given in x86/x86-64 assembly, but all platforms are supported.
\end{myquote}

\subsection*{Assembly template}

The \code{assembly template} is the only required parameter and must be a literal string (i.e. \code{""})

\begin{rustc}
#![feature(asm)]

#[cfg(any(target_arch = "x86", target_arch = "x86_64"))]
fn foo() {
    unsafe {
        asm!("NOP");
    }
}

// other platforms
#[cfg(not(any(target_arch = "x86", target_arch = "x86_64")))]
fn foo() { /* ... */ }

fn main() {
    // ...
    foo();
    // ...
}
\end{rustc}

(The \code{feature(asm)} and \code{\#[cfg]}s are omitted from now on.)

\blank

Output operands, input operands, clobbers and options are all optional but you must add the right number of \code{:} 
if you skip them:

\begin{rustc}
asm!("xor %eax, %eax"
    :
    :
    : "{eax}"
   );
\end{rustc}

Whitespace also doesn't matter:

\begin{rustc}
asm!("xor %eax, %eax" ::: "{eax}");
\end{rustc}

\subsection*{Operands}

Input and output operands follow the same format: \code{: \enquote{constraints1}(expr1), \enquote{constraints2}(expr2), ..."}. 
Output operand expressions must be mutable lvalues, or not yet assigned:

\begin{rustc}
fn add(a: i32, b: i32) -> i32 {
    let c: i32;
    unsafe {
        asm!("add $2, $0"
             : "=r"(c)
             : "0"(a), "r"(b)
             );
    }
    c
}

fn main() {
    assert_eq!(add(3, 14159), 14162)
}
\end{rustc}

If you would like to use real operands in this position, however, you are required to put curly braces \code{\{\}} around the 
register that you want, and you are required to put the specific size of the operand. This is useful for very low level programming, 
where which register you use is important:

\begin{rustc}
let result: u8;
asm!("in %dx, %al" : "={al}"(result) : "{dx}"(port));
result
\end{rustc}

\subsection*{Clobbers}

Some instructions modify registers which might otherwise have held different values so we use the clobbers list to indicate 
to the compiler not to assume any values loaded into those registers will stay valid.

\begin{rustc}
// Put the value 0x200 in eax
asm!("mov $$0x200, %eax" : /* no outputs */ : /* no inputs */ : "{eax}");
\end{rustc}

Input and output registers need not be listed since that information is already communicated by the given constraints. 
Otherwise, any other registers used either implicitly or explicitly should be listed.

\blank

If the assembly changes the condition code register \code{cc} should be specified as one of the clobbers. Similarly, if 
the assembly modifies memory, \code{memory} should also be specified.

\subsection*{Options}

The last section, \code{options} is specific to Rust. The format is comma separated literal strings (i.e. 
\code{:\enquote{foo}, \enquote{bar}, \enquote{baz}}). It's used to specify some extra info about the inline assembly:

\blank

Current valid options are:

\begin{enumerate}
  \item{\emph{volatile} - specifying this is analogous to \code{\_\_asm\_\_ \_\_volatile\_\_ (...)} in gcc/clang.}
  \item{\emph{alignstack} - certain instructions expect the stack to be aligned a certain way (i.e. SSE) and specifying this 
      indicates to the compiler to insert its usual stack alignment code}
  \item{\emph{intel} - use intel syntax instead of the default AT\&T.}
\end{enumerate}

\begin{rustc}
let result: i32;
unsafe {
   asm!("mov eax, 2" : "={eax}"(result) : : : "intel")
}
println!("eax is currently {}", result);
\end{rustc}

\subsection*{More Information}

The current implementation of the \code{asm!} macro is a direct binding to 
\href{http://llvm.org/docs/LangRef.html\#inline-assembler-expressions}{LLVM's inline assembler expressions}, so be sure to 
check out \href{http://llvm.org/docs/LangRef.html\#inline-assembler-expressions}{their documentation} as well for more information 
about clobbers, constraints, etc.


\section{No stdlib}
\label{sec:nightly_nostdlib}
Rust's standard library provides a lot of useful functionality, but assumes support for various features of its host system: 
threads, networking, heap allocation, and others. There are systems that do not have these features, however, and Rust can work 
with those too! To do so, we tell Rust that we don't want to use the standard library via an attribute: \code{\#![no\_std]}.

\begin{myquote}
Note: This feature is technically stable, but there are some caveats. For one, you can build a \code{\#![no\_std]} \emph{library} 
on stable, but not a \emph{binary}. For details on libraries without the standard library, see the chapter on \code{\#![no\_std]} (
see \nameref{sec:effective_rustWithoutStdLib}).
\end{myquote}

Obviously there's more to life than just libraries: one can use \code{\#[no\_std]} with an executable, controlling the entry 
point is possible in two ways: the \code{\#[start]} attribute, or overriding the default shim for the C \code{main} function 
with your own.

\blank

The function marked \code{\#[start]} is passed the command line parameters in the same format as C:

\begin{rustc}
#![feature(lang_items)]
#![feature(start)]
#![no_std]

// Pull in the system libc library for what crt0.o likely requires
extern crate libc;

// Entry point for this program
#[start]
fn start(_argc: isize, _argv: *const *const u8) -> isize {
    0
}

// These functions and traits are used by the compiler, but not
// for a bare-bones hello world. These are normally
// provided by libstd.
#[lang = "eh_personality"] extern fn eh_personality() {}
#[lang = "panic_fmt"] fn panic_fmt() -> ! { loop {} }
\end{rustc}

To override the compiler-inserted \code{main} shim, one has to disable it with \code{\#![no\_main]} and then create the 
appropriate symbol with the correct ABI and the correct name, which requires overriding the compiler's name mangling too:

\begin{rustc}
#![feature(lang_items)]
#![feature(start)]
#![no_std]
#![no_main]

extern crate libc;

#[no_mangle] // ensure that this symbol is called `main` in the output
pub extern fn main(argc: i32, argv: *const *const u8) -> i32 {
    0
}

#[lang = "eh_personality"] extern fn eh_personality() {}
#[lang = "panic_fmt"] fn panic_fmt() -> ! { loop {} }
\end{rustc}

The compiler currently makes a few assumptions about symbols which are available in the executable to call. Normally 
these functions are provided by the standard library, but without it you must define your own.

\blank

The first of these two functions, \code{eh\_personality}, is used by the failure mechanisms of the compiler. This is often 
mapped to GCC's personality function (see the 
\href{https://github.com/rust-lang/rust/blob/master/src/libstd/sys/common/unwind/gcc.rs}{libstd implementation} for more 
information), but crates which do not trigger a panic can be assured that this function is never called. The second function, 
\code{panic\_fmt}, is also used by the failure mechanisms of the compiler.


\section{Intrinsics}
\label{sec:nightly_intrinsics}
\begin{myquote}
Note: intrinsics will forever have an unstable interface, it is recommended to use the stable interfaces of libcore rather 
than intrinsics directly.
\end{myquote}

These are imported as if they were FFI functions, with the special \code{rust-intrinsic} ABI. For example, if one was in a 
freestanding context, but wished to be able to \code{transmute} between types, and perform efficient pointer arithmetic, one 
would import those functions via a declaration like

\begin{rustc}
#![feature(intrinsics)]

extern "rust-intrinsic" {
    fn transmute<T, U>(x: T) -> U;

    fn offset<T>(dst: *const T, offset: isize) -> *const T;
}
\end{rustc}

As with any other FFI functions, these are always \code{unsafe} to call.


\section{Lang items}
\label{sec:nightly_langItems}
\begin{myquote}
Note: lang items are often provided by crates in the Rust distribution, and lang items themselves have an unstable 
interface. It is recommended to use officially distributed crates instead of defining your own lang items.
\end{myquote}

The \code{rustc} compiler has certain pluggable operations, that is, functionality that isn't hard-coded into the language, 
but is implemented in libraries, with a special marker to tell the compiler it exists. The marker is the attribute 
\code{\#[lang = \enquote{...}]} and there are various different values of \code{...}, i.e. various different 'lang items'.

\blank

For example, \code{Box} pointers require two lang items, one for allocation and one for deallocation. A freestanding program 
that uses the \code{Box} sugar for dynamic allocations via \code{malloc} and \code{free}:

\begin{rustc}
#![feature(lang_items, box_syntax, start, libc)]
#![no_std]

extern crate libc;

extern {
    fn abort() -> !;
}

#[lang = "owned_box"]
pub struct Box<T>(*mut T);

#[lang = "exchange_malloc"]
unsafe fn allocate(size: usize, _align: usize) -> *mut u8 {
    let p = libc::malloc(size as libc::size_t) as *mut u8;

    // malloc failed
    if p as usize == 0 {
        abort();
    }

    p
}

#[lang = "exchange_free"]
unsafe fn deallocate(ptr: *mut u8, _size: usize, _align: usize) {
    libc::free(ptr as *mut libc::c_void)
}

#[lang = "box_free"]
unsafe fn box_free<T>(ptr: *mut T) {
    deallocate(ptr as *mut u8, ::core::mem::size_of::<T>(), ::core::mem::align_of::<T>());
}

#[start]
fn main(argc: isize, argv: *const *const u8) -> isize {
    let x = box 1;

    0
}

#[lang = "eh_personality"] extern fn eh_personality() {}
#[lang = "panic_fmt"] fn panic_fmt() -> ! { loop {} }
\end{rustc}

Note the use of \code{abort}: the \code{exchange\_malloc} lang item is assumed to return a valid pointer, and so needs to 
do the check internally.

\blank

Other features provided by lang items include:

\begin{itemize}
  \item{overloadable operators via traits: the traits corresponding to the \code{==}, \code{<}, dereferencing (\code{*}) 
      and \code{+} (etc.) operators are all marked with lang items; those specific four are \code{eq}, \code{ord}, \code{deref}, 
      and \code{add} respectively.}
  \item{stack unwinding and general failure; the \code{eh\_personality}, \code{fail} and \code{fail\_bounds\_checks} lang items.}
  \item{the traits in \code{std::marker} used to indicate types of various kinds; lang items \code{send}, \code{sync} and \code{copy}.}
  \item{the marker types and variance indicators found in \code{std::marker}; lang items \code{covariant\_type}, 
      \code{contravariant\_lifetime}, etc.}
\end{itemize}

Lang items are loaded lazily by the compiler; e.g. if one never uses \code{Box} then there is no need to define functions 
for \code{exchange\_malloc} and \code{exchange\_free}. \code{rustc} will emit an error when an item is needed but not found 
in the current crate or any that it depends on.


\section{Advanced Linking}
\label{sec:nightly_advancedLinking}
The common cases of linking with Rust have been covered earlier in this book, but supporting the range of linking 
possibilities made available by other languages is important for Rust to achieve seamless interaction with native 
libraries.

\subsection*{Link args}

There is one other way to tell \code{rustc} how to customize linking, and that is via the \code{link\_args} attribute. 
This attribute is applied to \code{extern} blocks and specifies raw flags which need to get passed to the linker when 
producing an artifact. An example usage would be:

\begin{rustc}
#![feature(link_args)]

#[link_args = "-foo -bar -baz"]
extern {}
\end{rustc}

Note that this feature is currently hidden behind the \code{feature(link\_args)} gate because this is not a sanctioned way 
of performing linking. Right now \code{rustc} shells out to the system linker (\code{gcc} on most systems, \code{link.exe} 
on MSVC), so it makes sense to provide extra command line arguments, but this will not always be the case. In the future 
\code{rustc} may use LLVM directly to link native libraries, in which case \code{link\_args} will have no meaning. You can 
achieve the same effect as the \code{link\_args} attribute with the \code{-C link-args} argument to \code{rustc}.

\blank

It is highly recommended to \emph{not} use this attribute, and rather use the more formal \code{\#[link(...)]} attribute on 
\code{extern} blocks instead.

\subsection*{Static linking}

Static linking refers to the process of creating output that contains all required libraries and so doesn't need libraries 
installed on every system where you want to use your compiled project. Pure-Rust dependencies are statically linked by default 
so you can use created binaries and libraries without installing Rust everywhere. By contrast, native libraries (e.g. \code{libc} 
and \code{libm}) are usually dynamically linked, but it is possible to change this and statically link them as well.

\blank

Linking is a very platform-dependent topic, and static linking may not even be possible on some platforms! This section assumes 
some basic familiarity with linking on your platform of choice.

\subsubsection*{Linux}

By default, all Rust programs on Linux will link to the system \code{libc} along with a number of other libraries. Let's look 
at an example on a 64-bit Linux machine with GCC and \code{glibc} (by far the most common \code{libc} on Linux):

\begin{verbatim}
$ cat example.rs
fn main() {}
$ rustc example.rs
$ ldd example
        linux-vdso.so.1 =>  (0x00007ffd565fd000)
        libdl.so.2 => /lib/x86_64-linux-gnu/libdl.so.2 (0x00007fa81889c000)
        libpthread.so.0 => /lib/x86_64-linux-gnu/libpthread.so.0 (0x00007fa81867e000)
        librt.so.1 => /lib/x86_64-linux-gnu/librt.so.1 (0x00007fa818475000)
        libgcc_s.so.1 => /lib/x86_64-linux-gnu/libgcc_s.so.1 (0x00007fa81825f000)
        libc.so.6 => /lib/x86_64-linux-gnu/libc.so.6 (0x00007fa817e9a000)
        /lib64/ld-linux-x86-64.so.2 (0x00007fa818cf9000)
        libm.so.6 => /lib/x86_64-linux-gnu/libm.so.6 (0x00007fa817b93000)
\end{verbatim}

Dynamic linking on Linux can be undesirable if you wish to use new library features on old systems or target systems which 
do not have the required dependencies for your program to run.

\blank

Static linking is supported via an alternative \code{libc}, \href{http://www.musl-libc.org/}{musl}. You can compile your own 
version of Rust with musl enabled and install it into a custom directory with the instructions below:

\begin{verbatim}
$ mkdir musldist
$ PREFIX=$(pwd)/musldist
$
$ # Build musl
$ curl -O http://www.musl-libc.org/releases/musl-1.1.10.tar.gz
$ tar xf musl-1.1.10.tar.gz
$ cd musl-1.1.10/
musl-1.1.10 $ ./configure --disable-shared --prefix=$PREFIX
musl-1.1.10 $ make
musl-1.1.10 $ make install
musl-1.1.10 $ cd ..
$ du -h musldist/lib/libc.a
2.2M    musldist/lib/libc.a
$
$ # Build libunwind.a
$ curl -O http://llvm.org/releases/3.7.0/llvm-3.7.0.src.tar.xz
$ tar xf llvm-3.7.0.src.tar.xz
$ cd llvm-3.7.0.src/projects/
llvm-3.7.0.src/projects $ curl http://llvm.org/releases/3.7.0/libunwind-3.7.0.src.tar.xz | tar xJf -
llvm-3.7.0.src/projects $ mv libunwind-3.7.0.src libunwind
llvm-3.7.0.src/projects $ mkdir libunwind/build
llvm-3.7.0.src/projects $ cd libunwind/build
llvm-3.7.0.src/projects/libunwind/build $ cmake -DLLVM_PATH=../../.. -DLIBUNWIND_ENABLE_SHARED=0 ..
llvm-3.7.0.src/projects/libunwind/build $ make
llvm-3.7.0.src/projects/libunwind/build $ cp lib/libunwind.a $PREFIX/lib/
llvm-3.7.0.src/projects/libunwind/build $ cd ../../../../
$ du -h musldist/lib/libunwind.a
164K    musldist/lib/libunwind.a
$
$ # Build musl-enabled rust
$ git clone https://github.com/rust-lang/rust.git muslrust
$ cd muslrust
muslrust $ ./configure --target=x86_64-unknown-linux-musl --musl-root=$PREFIX --prefix=$PREFIX
muslrust $ make
muslrust $ make install
muslrust $ cd ..
$ du -h musldist/bin/rustc
12K     musldist/bin/rustc
\end{verbatim}

You now have a build of a \code{musl}-enabled Rust! Because we've installed it to a custom prefix we need to make sure 
our system can find the binaries and appropriate libraries when we try and run it:

\begin{verbatim}
$ export PATH=$PREFIX/bin:$PATH
$ export LD_LIBRARY_PATH=$PREFIX/lib:$LD_LIBRARY_PATH
\end{verbatim}

Let's try it out!

\begin{verbatim}
$ echo 'fn main() { println!("hi!"); panic!("failed"); }' > example.rs
$ rustc --target=x86_64-unknown-linux-musl example.rs
$ ldd example
        not a dynamic executable
$ ./example
hi!
thread '<main>' panicked at 'failed', example.rs:1
\end{verbatim}

Success! This binary can be copied to almost any Linux machine with the same machine architecture and run without issues.

\blank

\code{cargo build} also permits the \code{--target} option so you should be able to build your crates as normal. However, you 
may need to recompile your native libraries against \code{musl} before they can be linked against.


\section{Benchmark Tests}
\label{sec:nightly_benchmarkTests}
Rust supports benchmark tests, which can test the performance of your code. Let's make our \code{src/lib.rs} look like 
this (comments elided):

\begin{rustc}
#![feature(test)]

extern crate test;

pub fn add_two(a: i32) -> i32 {
    a + 2
}

#[cfg(test)]
mod tests {
    use super::*;
    use test::Bencher;

    #[test]
    fn it_works() {
        assert_eq!(4, add_two(2));
    }

    #[bench]
    fn bench_add_two(b: &mut Bencher) {
        b.iter(|| add_two(2));
    }
}
\end{rustc}

Note the \code{test} feature gate, which enables this unstable feature.

\blank

We've imported the \code{test} crate, which contains our benchmarking support. We have a new function as well, with the 
\code{bench} attribute. Unlike regular tests, which take no arguments, benchmark tests take a \code{\&mut Bencher}. This 
Bencher \code{provides} an \code{iter} method, which takes a closure. This closure contains the code we'd like to benchmark.

\blank

We can run benchmark tests with \code{cargo bench}:

\begin{verbatim}
$ cargo bench
   Compiling adder v0.0.1 (file:///home/steve/tmp/adder)
     Running target/release/adder-91b3e234d4ed382a

running 2 tests
test tests::it_works ... ignored
test tests::bench_add_two ... bench:         1 ns/iter (+/- 0)

test result: ok. 0 passed; 0 failed; 1 ignored; 1 measured
\end{verbatim}

Our non-benchmark test was ignored. You may have noticed that \code{cargo bench} takes a bit longer than \code{cargo test}. 
This is because Rust runs our benchmark a number of times, and then takes the average. Because we're doing so little work in 
this example, we have a \code{1 ns/iter (+/- 0)}, but this would show the variance if there was one.

\blank

Advice on writing benchmarks:

\begin{itemize}
  \item{Move setup code outside the \code{iter} loop; only put the part you want to measure inside}
  \item{Make the code do \enquote{the same thing} on each iteration; do not accumulate or change state}
  \item{Make the outer function idempotent too; the benchmark runner is likely to run it many times}
  \item{Make the inner \code{iter} loop short and fast so benchmark runs are fast and the calibrator can adjust the 
      run-length at fine resolution}
  \item{Make the code in the \code{iter} loop do something simple, to assist in pinpointing performance improvements 
      (or regressions)}
\end{itemize}

\subsection*{Gotcha: optimizations}

There's another tricky part to writing benchmarks: benchmarks compiled with optimizations activated can be dramatically 
changed by the optimizer so that the benchmark is no longer benchmarking what one expects. For example, the compiler might 
recognize that some calculation has no external effects and remove it entirely.

\begin{rustc}
#![feature(test)]

extern crate test;
use test::Bencher;

#[bench]
fn bench_xor_1000_ints(b: &mut Bencher) {
    b.iter(|| {
        (0..1000).fold(0, |old, new| old ^ new);
    });
}
\end{rustc}

gives the following results

\begin{verbatim}
running 1 test
test bench_xor_1000_ints ... bench:         0 ns/iter (+/- 0)

test result: ok. 0 passed; 0 failed; 0 ignored; 1 measured
\end{verbatim}

The benchmarking runner offers two ways to avoid this. Either, the closure that the \code{iter} method receives can return 
an arbitrary value which forces the optimizer to consider the result used and ensures it cannot remove the computation entirely. 
This could be done for the example above by adjusting the \code{b.iter} call to

\begin{rustc}
b.iter(|| {
    // note lack of `;` (could also use an explicit `return`).
    (0..1000).fold(0, |old, new| old ^ new)
});
\end{rustc}

Or, the other option is to call the generic \code{test::black\_box} function, which is an opaque \enquote{black box} to the 
optimizer and so forces it to consider any argument as used.

\begin{rustc}
#![feature(test)]

extern crate test;

b.iter(|| {
    let n = test::black_box(1000);

    (0..n).fold(0, |a, b| a ^ b)
})
\end{rustc}

Neither of these read or modify the value, and are very cheap for small values. Larger values can be passed indirectly to 
reduce overhead (e.g. \code{black\_box(\&huge\_struct)}).

\blank

Performing either of the above changes gives the following benchmarking results

\begin{verbatim}
running 1 test
test bench_xor_1000_ints ... bench:       131 ns/iter (+/- 3)

test result: ok. 0 passed; 0 failed; 0 ignored; 1 measured
\end{verbatim}

However, the optimizer can still modify a testcase in an undesirable manner even when using either of the above.


\section{Box Syntax and Patterns}
\label{sec:nightly_boxSyntaxAndPatterns}
Currently the only stable way to create a \code{Box} is via the \code{Box::new} method. Also it is not possible in stable 
Rust to destructure a \code{Box} in a match pattern. The unstable \code{box} keyword can be used to both create and destructure a 
\code{Box}. An example usage would be:

\begin{rustc}
#![feature(box_syntax, box_patterns)]

fn main() {
    let b = Some(box 5);
    match b {
        Some(box n) if n < 0 => {
            println!("Box contains negative number {}", n);
        },
        Some(box n) if n >= 0 => {
            println!("Box contains non-negative number {}", n);
        },
        None => {
            println!("No box");
        },
        _ => unreachable!()
    }
}
\end{rustc}

Note that these features are currently hidden behind the \code{box\_syntax} (box creation) and \code{box\_patterns} 
(destructuring and pattern matching) gates because the syntax may still change in the future.

\subsection*{Returning Pointers}

In many languages with pointers, you'd return a pointer from a function so as to avoid copying a large data structure. 
For example:

\begin{rustc}
struct BigStruct {
    one: i32,
    two: i32,
    // etc
    one_hundred: i32,
}

fn foo(x: Box<BigStruct>) -> Box<BigStruct> {
    Box::new(*x)
}

fn main() {
    let x = Box::new(BigStruct {
        one: 1,
        two: 2,
        one_hundred: 100,
    });

    let y = foo(x);
}
\end{rustc}

The idea is that by passing around a box, you're only copying a pointer, rather than the hundred \itt s that make up the \code{BigStruct}.

\blank

This is an antipattern in Rust. Instead, write this:

\begin{rustc}
#![feature(box_syntax)]

struct BigStruct {
    one: i32,
    two: i32,
    // etc
    one_hundred: i32,
}

fn foo(x: Box<BigStruct>) -> BigStruct {
    *x
}

fn main() {
    let x = Box::new(BigStruct {
        one: 1,
        two: 2,
        one_hundred: 100,
    });

    let y: Box<BigStruct> = box foo(x);
}
\end{rustc}

This gives you flexibility without sacrificing performance.

\blank

You may think that this gives us terrible performance: return a value and then immediately box it up ?! Isn't this pattern the 
worst of both worlds? Rust is smarter than that. There is no copy in this code. \code{main} allocates enough room for the \code{box}, 
passes a pointer to that memory into \code{foo} as \x, and then \code{foo} writes the value straight into the \code{Box<T>}.

\blank

This is important enough that it bears repeating: pointers are not for optimizing returning values from your code. Allow the caller 
to choose how they want to use your output.


\section{Slice patterns}
\label{sec:nightly_slicePatterns}
If you want to match against a slice or array, you can use \code{\&} with the \code{slice\_patterns} feature:

\begin{rustc}
#![feature(slice_patterns)]

fn main() {
    let v = vec!["match_this", "1"];

    match &v[..] {
        ["match_this", second] => println!("The second element is {}", second),
        _ => {},
    }
}
\end{rustc}

The \code{advanced\_slice\_patterns} gate lets you use \code{..} to indicate any number of elements inside a pattern matching a 
slice. This wildcard can only be used once for a given array. If there's an identifier before the \code{..}, the result of the slice 
will be bound to that name. For example:

\begin{rustc}
#![feature(advanced_slice_patterns, slice_patterns)]

fn is_symmetric(list: &[u32]) -> bool {
    match list {
        [] | [_] => true,
        [x, inside.., y] if x == y => is_symmetric(inside),
        _ => false
    }
}

fn main() {
    let sym = &[0, 1, 4, 2, 4, 1, 0];
    assert!(is_symmetric(sym));

    let not_sym = &[0, 1, 7, 2, 4, 1, 0];
    assert!(!is_symmetric(not_sym));
}
\end{rustc}


\section{Associated Constants}
\label{sec:nightly_associatedConstants}
With the \code{associated\_consts} feature, you can define constants like this:

\begin{rustc}
#![feature(associated_consts)]

trait Foo {
    const ID: i32;
}

impl Foo for i32 {
    const ID: i32 = 1;
}

fn main() {
    assert_eq!(1, i32::ID);
}
\end{rustc}

Any implementor of \code{Foo} will have to define \code{ID}. Without the definition:

\begin{rustc}
#![feature(associated_consts)]

trait Foo {
    const ID: i32;
}

impl Foo for i32 {
}
\end{rustc}

gives

\begin{verbatim}
error: not all trait items implemented, missing: `ID` [E0046]
     impl Foo for i32 {
     }
\end{verbatim}

A default value can be implemented as well:

\begin{rustc}
#![feature(associated_consts)]

trait Foo {
    const ID: i32 = 1;
}

impl Foo for i32 {
}

impl Foo for i64 {
    const ID: i32 = 5;
}

fn main() {
    assert_eq!(1, i32::ID);
    assert_eq!(5, i64::ID);
}
\end{rustc}

As you can see, when implementing \code{Foo}, you can leave it unimplemented, as with \itt. It will then use the default value. 
But, as in \code{i64}, we can also add our own definition.

\blank

Associated constants don't have to be associated with a trait. An \code{impl} block for a \struct\ or an \code{enum} works fine too:

\begin{rustc}
#![feature(associated_consts)]

struct Foo;

impl Foo {
    const FOO: u32 = 3;
}
\end{rustc}


\section{Custom Allocators}
\label{sec:nightly_customAllocators}
Allocating memory isn't always the easiest thing to do, and while Rust generally takes care of this by default it often 
becomes necessary to customize how allocation occurs. The compiler and standard library currently allow switching out the 
default global allocator in use at compile time. The design is currently spelled out in 
\href{https://github.com/rust-lang/rfcs/blob/master/text/1183-swap-out-jemalloc.md}{RFC 1183} but this will walk you through 
how to get your own allocator up and running.

\subsection*{Default Allocator}

The compiler currently ships two default allocators: \code{allo\_system} and \code{alloc\_jemalloc} (some targets don't have 
\code{jemalloc}, however). These allocators are normal Rust crates and contain an implementation of the routines to allocate 
and deallocate memory. The standard library is not compiled assuming either one, and the compiler will decide which allocator 
is in use at compile-time depending on the type of output artifact being produced.

\blank

Binaries generated by the compiler will use \code{alloc\_jemalloc} by default (where available). In this situation the 
compiler \enquote{controls the world} in the sense of it has power over the final link. Primarily this means that the 
allocator decision can be left up the compiler.

\blank

Dynamic and static libraries, however, will use \code{alloc\_system} by default. Here Rust is typically a 'guest' in another 
application or another world where it cannot authoritatively decide what allocator is in use. As a result it resorts back to 
the standard APIs (e.g. \code{malloc} and \code{free}) for acquiring and releasing memory.

\subsection*{Switching Allocators}

Although the compiler's default choices may work most of the time, it's often necessary to tweak certain aspects. Overriding 
the compiler's decision about which allocator is in use is done simply by linking to the desired allocator:

\begin{rustc}
#![feature(alloc_system)]

extern crate alloc_system;

fn main() {
    let a = Box::new(4); // allocates from the system allocator
    println!("{}", a);
}
\end{rustc}

In this example the binary generated will not link to jemalloc by default but instead use the system allocator. Conversely 
to generate a dynamic library which uses jemalloc by default one would write:

\begin{rustc}
#![feature(alloc_jemalloc)]
#![crate_type = "dylib"]

extern crate alloc_jemalloc;

pub fn foo() {
    let a = Box::new(4); // allocates from jemalloc
    println!("{}", a);
}
\end{rustc}

\subsection*{Writing a custom allocator}

Sometimes even the choices of jemalloc vs the system allocator aren't enough and an entirely new custom allocator is required. 
In this you'll write your own crate which implements the allocator API (e.g. the same as \code{alloc\_system} or 
\code{alloc\_jemalloc}). As an example, let's take a look at a simplified and annotated version of \code{alloc\_system}.

\begin{rustc}
// The compiler needs to be instructed that this crate is an allocator in order
// to realize that when this is linked in another allocator like jemalloc should
// not be linked in
#![feature(allocator)]
#![allocator]

// Allocators are not allowed to depend on the standard library which in turn
// requires an allocator in order to avoid circular dependencies. This crate,
// however, can use all of libcore.
#![no_std]

// Let's give a unique name to our custom allocator
#![crate_name = "my_allocator"]
#![crate_type = "rlib"]

// Our system allocator will use the in-tree libc crate for FFI bindings. Note
// that currently the external (crates.io) libc cannot be used because it links
// to the standard library (e.g. `#![no_std]` isn't stable yet), so that's why
// this specifically requires the in-tree version.
#![feature(libc)]
extern crate libc;

// Listed below are the five allocation functions currently required by custom
// allocators. Their signatures and symbol names are not currently typechecked
// by the compiler, but this is a future extension and are required to match
// what is found below.
//
// Note that the standard `malloc` and `realloc` functions do not provide a way
// to communicate alignment so this implementation would need to be improved
// with respect to alignment in that aspect.

#[no_mangle]
pub extern fn __rust_allocate(size: usize, _align: usize) -> *mut u8 {
    unsafe { libc::malloc(size as libc::size_t) as *mut u8 }
}

#[no_mangle]
pub extern fn __rust_deallocate(ptr: *mut u8, _old_size: usize, _align: usize) {
    unsafe { libc::free(ptr as *mut libc::c_void) }
}

#[no_mangle]
pub extern fn __rust_reallocate(ptr: *mut u8, _old_size: usize, size: usize,
                                _align: usize) -> *mut u8 {
    unsafe {
        libc::realloc(ptr as *mut libc::c_void, size as libc::size_t) as *mut u8
    }
}

#[no_mangle]
pub extern fn __rust_reallocate_inplace(_ptr: *mut u8, old_size: usize,
                                        _size: usize, _align: usize) -> usize {
    old_size // this api is not supported by libc
}

#[no_mangle]
pub extern fn __rust_usable_size(size: usize, _align: usize) -> usize {
    size
}
\end{rustc}

After we compile this crate, it can be used as follows:

\begin{rustc}
extern crate my_allocator;

fn main() {
    let a = Box::new(8); // allocates memory via our custom allocator crate
    println!("{}", a);
}
\end{rustc}

\subsection*{Custom allocator limitations}

There are a few restrictions when working with custom allocators which may cause compiler errors:

\begin{itemize}
  \item{Any one artifact may only be linked to at most one allocator. Binaries, dylibs, and staticlibs must link to exactly one 
      allocator, and if none have been explicitly chosen the compiler will choose one. On the other hand rlibs do not need to 
      link to an allocator (but still can).}
  \item{A consumer of an allocator is tagged with \code{\#![needs\_allocator]} (e.g. the \code{liballoc} crate currently) and 
      an \code{\#[allocator]} crate cannot transitively depend on a crate which needs an allocator (e.g. circular dependencies 
      are not allowed). This basically means that allocators must restrict themselves to libcore currently.}
\end{itemize}

