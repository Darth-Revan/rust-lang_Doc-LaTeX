Rust, with its focus on safety, provides two different ways of casting different types between each other. The first, \code{as}, is 
for safe casts. In contrast, \code{transmute} allows for arbitrary casting, and is one of the most dangerous features of Rust!

\subsection*{Coercion}

Coercion between types is implicit and has no syntax of its own, but can be spelled out with \code{as}.

\blank

Coercion occurs in \keylet, \code{const}, and \code{static} statements; in function call arguments; in field values in struct 
initialization; and in a function result.

\blank

The most common case of coercion is removing mutability from a reference:

\begin{itemize}
  \item{\code{\&mut T} to \code{\&T}}
\end{itemize}

An analogous conversion is to remove mutability from a raw pointer (see \nameref{sec:syntax_rawPointers}):

\begin{itemize}
  \item{\code{*mut T} to \code{*const T}}
\end{itemize}

References can also be coerced to raw pointers:

\begin{itemize}
  \item{\code{\&T} to \code{*const T}}
  \item{\code{\&mut T} to \code{*mut T}}
\end{itemize}

Custom coercions may be defined using Deref (see \nameref{sec:syntax_derefCoercions}).

\blank

Coercion is transitive.

\subsection*{\code{as}}

The \code{as} keyword does safe casting:

\begin{rustc}
let x: i32 = 5;

let y = x as i64;
\end{rustc}

There are three major categories of safe cast: explicit coercions, casts between numeric types, and pointer casts.

\blank

Casting is not transitive: even if \code{e as U1 as U2} is a valid expression, \code{e as U2} is not necessarily so (in fact it will 
only be valid if \code{U1} coerces to \code{U2}).

\myparagraph{Explicit coercions}

A cast \code{e as U} is valid if \code{e} has type \code{T} and \code{T} \emph{coerces} to \code{U}.

\myparagraph{Numeric casts}

A cast \code{e as U} is also valid in any of the following cases:

\begin{itemize}
  \item{\code{e} has type \code{T} and \code{T} and \code{U} are any numeric types; \emph{numeric-cast}}
  \item{\code{e} is a C-like enum (with no data attached to the variants), and \code{U} is an integer type; \emph{enum-cast}}
  \item{\code{e} has type \code{bool} or \varchar\ and \code{U} is an integer type; \emph{prim-int-cast}}
  \item{\code{e} has type \code{u8} and \code{U} is \varchar; \emph{u8-char-cast}}
\end{itemize}

For example

\begin{rustc}
let one = true as u8;
let at_sign = 64 as char;
let two_hundred = -56i8 as u8;
\end{rustc}

The semantics of numeric casts are:

\begin{itemize}
  \item{Casting between two integers of the same size (e.g. i32 -> u32) is a no-op}
  \item{Casting from a larger integer to a smaller integer (e.g. u32 -> u8) will truncate}
  \item{Casting from a smaller integer to a larger integer (e.g. u8 -> u32) will}
  \begin{itemize}
    \item{zero-extend if the source is unsigned}
    \item{sign-extend if the source is signed}
  \end{itemize}
  \item{Casting from a float to an integer will round the float towards zero}
  \begin{itemize}
    \item{\href{https://github.com/rust-lang/rust/issues/10184}{NOTE: currently this will cause Undefined Behavior if the rounded 
        value cannot be represented by the target integer type.} This includes Inf and NaN. This is a bug and will be fixed.}
  \end{itemize}
  \item{Casting from an integer to float will produce the floating point representation of the integer, rounded if necessary 
      (rounding strategy unspecified)}
  \item{Casting from an f32 to an f64 is perfect and lossless}
  \item{Casting from an f64 to an f32 will produce the closest possible value (rounding strategy unspecified)}
  \begin{itemize}
    \item{\href{https://github.com/rust-lang/rust/issues/15536}{NOTE: currently this will cause Undefined Behavior if the value 
        is finite but larger or smaller than the largest or smallest finite value representable by f32.} This is a bug and will be fixed.}
  \end{itemize}
\end{itemize}

\myparagraph{Pointer casts}

Perhaps surprisingly, it is safe to cast raw pointers to and from integers, and to cast between pointers to different types subject 
to some constraints (see \nameref{sec:syntax_rawPointers}). It is only unsafe to dereference the pointer:

\begin{rustc}
let a = 300 as *const char; // a pointer to location 300
let b = a as u32;
\end{rustc}

\code{e as U} is a valid pointer cast in any of the following cases:

\begin{itemize}
  \item{\code{e} has type \code{*T}, \code{U} has type \code{*U\_0}, and either \code{U\_0: Sized} or \code{unsize\_kind(T) == unsize\_kind(U\_0)}; 
      \emph{a ptr-ptr-cast}}
  \item{\code{e} has type \code{*T} and \code{U} is a numeric type, while \code{T: Sized}; \emph{ptr-addr-cast}}
  \item{\code{e} is an integer and \code{U} is \code{*U\_0}, while \code{U\_0: Sized}; \emph{addr-ptr-cast}}
  \item{\code{e} has type \code{\&[T; n]} and \code{U} is \code{*const T}; \emph{array-ptr-cast}}
  \item{\code{e} is a function pointer type and \code{U} has type \code{*T}, while \code{T: Sized}; \emph{fptr-ptr-cast}}
  \item{\code{e} is a function pointer type and \code{U} is an integer; \emph{fptr-addr-cast}}
\end{itemize}

\subsection*{\code{transmute}}

\code{as} only allows safe casting, and will for example reject an attempt to cast four bytes into a \code{u32}:

\begin{rustc}
let a = [0u8, 0u8, 0u8, 0u8];

let b = a as u32; // four eights makes 32
\end{rustc}

This errors with:

\begin{verbatim}
error: non-scalar cast: `[u8; 4]` as `u32`
let b = a as u32; // four eights makes 32
        ^~~~~~~~
\end{verbatim}

This is a 'non-scalar cast' because we have multiple values here: the four elements of the array. These kinds of casts are very 
dangerous, because they make assumptions about the way that multiple underlying structures are implemented. For this, we need 
something more dangerous.

\blank

The \code{transmute} function is provided by a compiler intrinsic (see \nameref{sec:nightly_intrinsics}), and what it does is very 
simple, but very scary. It tells Rust to treat a value of one type as though it were another type. It does this regardless of the 
typechecking system, and completely trusts you.

\blank

In our previous example, we know that an array of four \code{u8}s represents a \code{u32} properly, and so we want to do the cast. 
Using \code{transmute} instead of \code{as}, Rust lets us:

\begin{rustc}
use std::mem;

unsafe {
    let a = [0u8, 0u8, 0u8, 0u8];

    let b = mem::transmute::<[u8; 4], u32>(a);
}
\end{rustc}

We have to wrap the operation in an \code{unsafe} block for this to compile successfully. Technically, only the \code{mem::transmute} call 
itself needs to be in the block, but it's nice in this case to enclose everything related, so you know where to look. In this case, the 
details about \code{a} are also important, and so they're in the block. You'll see code in either style, sometimes the context is too far away, 
and wrapping all of the code in \code{unsafe} isn't a great idea.

\blank

While \code{transmute} does very little checking, it will at least make sure that the types are the same size. This errors:

\begin{rustc}
use std::mem;

unsafe {
    let a = [0u8, 0u8, 0u8, 0u8];

    let b = mem::transmute::<[u8; 4], u64>(a);
}
\end{rustc}

with:

\begin{verbatim}
error: transmute called with differently sized types: [u8; 4] (32 bits) to u64
(64 bits)
\end{verbatim}

Other than that, you're on your own!
