Sometimes, when writing a function or data type, we may want it to work for multiple types of arguments. In Rust, we can do this with 
generics. Generics are called 'parametric polymorphism' in type theory, which means that they are types or functions that have multiple 
forms ('poly' is multiple, 'morph' is form) over a given parameter ('parametric').

\blank

Anyway, enough type theory, let's check out some generic code. Rust's standard library provides a type, \code{Option<T>}, that's generic:

\begin{rustc}
enum Option<T> {
    Some(T),
    None,
}
\end{rustc}

The \code{<T>} part, which you've seen a few times before, indicates that this is a generic data type. Inside the declaration of our \enum, 
wherever we see a \code{T}, we substitute that type for the same type used in the generic. Here's an example of using \code{Option<T>}, with 
some extra type annotations:

\begin{rustc}
let x: Option<i32> = Some(5);
\end{rustc}

In the type declaration, we say \code{Option<i32>}. Note how similar this looks to \code{Option<T>}. So, in this particular Option, \code{T} has 
the value of \itt. On the right-hand side of the binding, we make a \code{Some(T)}, where \code{T} is \code{5}. Since that's an \itt, the two 
sides match, and Rust is happy. If they didn't match, we'd get an error:

\begin{rustc}
let x: Option<f64> = Some(5);
// error: mismatched types: expected `core::option::Option<f64>`,
// found `core::option::Option<_>` (expected f64 but found integral variable)
\end{rustc}

That doesn't mean we can't make \code{Option<T>}s that hold an \code{f64}! They have to match up:

\begin{rustc}
let x: Option<i32> = Some(5);
let y: Option<f64> = Some(5.0f64);
\end{rustc}

This is just fine. One definition, multiple uses.

\blank

Generics don't have to only be generic over one type. Consider another type from Rust's standard library that's similar, \code{Result<T, E>}:

\begin{rustc}
enum Result<T, E> {
    Ok(T),
    Err(E),
}
\end{rustc}

This type is generic over two types: \code{T} and \code{E}. By the way, the capital letters can be any letter you'd like. We could define 
\code{Result<T, E>} as:

\begin{rustc}
enum Result<A, Z> {
    Ok(A),
    Err(Z),
}
\end{rustc}

if we wanted to. Convention says that the first generic parameter should be \code{T}, for 'type', and that we use \code{E} for 'error'. Rust 
doesn't care, however.

\blank

The \code{Result<T, E>} type is intended to be used to return the result of a computation, and to have the ability to return an error if it 
didn't work out.

\subsubsection*{Generic functions}

We can write functions that take generic types with a similar syntax:

\begin{rustc}
fn takes_anything<T>(x: T) {
    // do something with x
}
\end{rustc}

The syntax has two parts: the \code{<T>} says \enquote{this function is generic over one type, \code{T}}, and the \code{x: T} says 
\enquote{\x\ has the type \code{T}.}

\blank

Multiple arguments can have the same generic type:

\begin{rustc}
fn takes_two_of_the_same_things<T>(x: T, y: T) {
    // ...
}
\end{rustc}

We could write a version that takes multiple types:

\begin{rustc}
fn takes_two_things<T, U>(x: T, y: U) {
    // ...
}
\end{rustc}

\subsubsection*{Generic structs}

You can store a generic type in a \struct\ as well:

\begin{rustc}
struct Point<T> {
    x: T,
    y: T,
}

let int_origin = Point { x: 0, y: 0 };
let float_origin = Point { x: 0.0, y: 0.0 };
\end{rustc}

Similar to functions, the \code{<T>} is where we declare the generic parameters, and we then use \code{x: T} in the type declaration, too.

\blank

When you want to add an implementation for the generic \struct, you declare the type parameter after the \code{impl}:

\begin{rustc}
impl<T> Point<T> {
    fn swap(&mut self) {
        std::mem::swap(&mut self.x, &mut self.y);
    }
}
\end{rustc}

% TODO make trait bounds a hyperref
So far you've seen generics that take absolutely any type. These are useful in many cases: you've already seen \code{Option<T>}, and later 
you'll meet universal container types like \href{https://doc.rust-lang.org/std/vec/struct.Vec.html}{Vec<T>}. On the other hand, often you 
want to trade that flexibility for increased expressive power. Read about trait bounds to see why and how.
