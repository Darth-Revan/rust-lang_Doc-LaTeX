This guide is one of three presenting Rust's ownership system. This is one of Rust's most unique and compelling features, with 
which Rust developers should become quite acquainted. Ownership is how Rust achieves its largest goal, memory safety. There are 
a few distinct concepts, each with its own chapter:

% TODO make borrowing and ownership hyperrefs
\begin{itemize}
  \item{ownership, which you're reading now}
  \item{borrowing, and their associated feature 'references'}
  \item{lifetimes, an advanced concept of borrowing}
\end{itemize}

These three chapters are related, and in order. You'll need all three to fully understand the ownership system.

\subsubsection*{Meta}

Before we get to the details, two important notes about the ownership system.

\blank

Rust has a focus on safety and speed. It accomplishes these goals through many 'zero-cost abstractions', which means that in 
Rust, abstractions cost as little as possible in order to make them work. The ownership system is a prime example of a zero-cost
abstraction. All of the analysis we'll talk about in this guide is \emph{done at compile time}. You do not pay any run-time cost 
for any of these features.

\blank

However, this system does have a certain cost: learning curve. Many new users to Rust experience something we like to call 
'fighting with the borrow checker', where the Rust compiler refuses to compile a program that the author thinks is valid. 
This often happens because the programmer's mental model of how ownership should work doesn't match the actual rules that Rust 
implements. You probably will experience similar things at first. There is good news, however: more experienced Rust developers 
report that once they work with the rules of the ownership system for a period of time, they fight the borrow checker less and less.

\blank

With that in mind, let's learn about lifetimes.

\subsubsection*{Lifetimes}

Lending out a reference to a resource that someone else owns can be complicated. For example, imagine this set of operations:

\begin{enumerate}
  \item{I acquire a handle to some kind of resource.}
  \item{I lend you a reference to the resource.}
  \item{I decide I'm done with the resource, and deallocate it, while you still have your reference.}
  \item{You decide to use the resource.}
\end{enumerate}

Uh oh! Your reference is pointing to an invalid resource. This is called a dangling pointer or 'use after free', when the resource 
is memory.

\blank

To fix this, we have to make sure that step four never happens after step three. The ownership system in Rust does this through a 
concept called lifetimes, which describe the scope that a reference is valid for.

\blank

When we have a function that takes a reference by argument, we can be implicit or explicit about the lifetime of the reference:

\begin{rustc}
// implicit
fn foo(x: &i32) {
}

// explicit
fn bar<'a>(x: &'a i32) {
}
\end{rustc}

The \code{'a} reads 'the lifetime a'. Technically, every reference has some lifetime associated with it, but the compiler lets 
you elide (i.e. omit, see \enquote{\nameref{paragraph:lifetime_elision}} below) them in common cases. Before we get to that, though, let's break the explicit 
example down:

\begin{rustc}
fn bar<'a>(...)
\end{rustc}

% TODO make later in the book a hyperref
We previously talked a little about function syntax (\nameref{sec:syntax_functions}), but we didn't discuss the \code{<>}s after 
a function's name. A function can have 'generic parameters' between the \code{<>}s, of which lifetimes are one kind. We'll discuss 
other kinds of generics later in the book, but for now, let's focus on the lifetimes aspect.

\blank

We use \code{<>} to declare our lifetimes. This says that bar has one lifetime, \code{'a}. If we had two reference parameters, it 
would look like this:

\begin{rustc}
fn bar<'a, 'b>(...)
\end{rustc}

Then in our parameter list, we use the lifetimes we've named:

\begin{rustc}
...(x: &'a i32)
\end{rustc}

If we wanted a \code{\&mut} reference, we'd do this:

\begin{rustc}
...(x: &'a mut i32)
\end{rustc}

If you compare \code{\&mut i32} to \code{\&'a mut i32}, they're the same, it's that the lifetime \code{'a} has snuck in between the 
\code{\&} and the \code{mut i32}. We read \code{\&mut i32} as 'a mutable reference to an \itt' and \code{\&'a mut i32} as 
'a mutable reference to an \itt\ with the lifetime \code{'a}'.

\subsubsection*{In \code{struct}s}

% TODO make structs a hyperref
You'll also need explicit lifetimes when working with structs that contain references:

\begin{rustc}
struct Foo<'a> {
    x: &'a i32,
}

fn main() {
    let y = &5; // this is the same as `let _y = 5; let y = &_y;`
    let f = Foo { x: y };

    println!("{}", f.x);
}
\end{rustc}

As you can see, \struct s can also have lifetimes. In a similar way to functions,

\begin{rustc}
struct Foo<'a> {
\end{rustc}

declares a lifetime, and

\begin{rustc}
x: &'a i32,
\end{rustc}

uses it. So why do we need a lifetime here? We need to ensure that any reference to a \code{Foo} cannot outlive the reference to an 
\itt\ it contains.

\myparagraph{\code{impl} blocks}

Let's implement a method on \code{Foo}:

\begin{rustc}
struct Foo<'a> {
    x: &'a i32,
}

impl<'a> Foo<'a> {
    fn x(&self) -> &'a i32 { self.x }
}

fn main() {
    let y = &5; // this is the same as `let _y = 5; let y = &_y;`
    let f = Foo { x: y };

    println!("x is: {}", f.x());
}
\end{rustc}

As you can see, we need to declare a lifetime for \code{Foo} in the \code{impl} line. We repeat \code{'a} twice, like on functions:
\code{impl<'a>} defines a lifetime \code{'a}, and \code{Foo<'a>} uses it.

\myparagraph{Multiple lifetimes}

If you have multiple references, you can use the same lifetime multiple times:

\begin{rustc}
fn x_or_y<'a>(x: &'a str, y: &'a str) -> &'a str {
\end{rustc}

This says that \x\ and \y\ both are alive for the same scope, and that the return value is also alive for that scope. 
If you wanted \x\ and \y\ to have different lifetimes, you can use multiple lifetime parameters:

\begin{rustc}
fn x_or_y<'a, 'b>(x: &'a str, y: &'b str) -> &'a str {
\end{rustc}

In this example, \x\ and \y\ have different valid scopes, but the return value has the same lifetime as \x.

\myparagraph{Thinking in scopes}

A way to think about lifetimes is to visualize the scope that a reference is valid for. For example:

\begin{rustc}
fn main() {
    let y = &5;     // -+ y goes into scope
                    //  |
    // stuff        //  |
                    //  |
}                   // -+ y goes out of scope
\end{rustc}

Adding in our \code{Foo}:

\begin{rustc}
struct Foo<'a> {
    x: &'a i32,
}

fn main() {
    let y = &5;           // -+ y goes into scope
    let f = Foo { x: y }; // -+ f goes into scope
    // stuff              //  |
                          //  |
}                         // -+ f and y go out of scope
\end{rustc}

Our \code{f} lives within the scope of \y, so everything works. What if it didn't? This code won't work:

\begin{rustc}
struct Foo<'a> {
    x: &'a i32,
}

fn main() {
    let x;                    // -+ x goes into scope
                              //  |
    {                         //  |
        let y = &5;           // ---+ y goes into scope
        let f = Foo { x: y }; // ---+ f goes into scope
        x = &f.x;             //  | | error here
    }                         // ---+ f and y go out of scope
                              //  |
    println!("{}", x);        //  |
}                             // -+ x goes out of scope
\end{rustc}

Whew! As you can see here, the scopes of \code{f} and \y\ are smaller than the scope of \x. But when we do 
\code{x = \&f.x}, we make \x\ a reference to something that's about to go out of scope.

\blank

Named lifetimes are a way of giving these scopes a name. Giving something a name is the first step towards being able to talk 
about it.

\myparagraph{'static}

The lifetime named 'static' is a special lifetime. It signals that something has the lifetime of the entire program. Most 
Rust programmers first come across \code{'static} when dealing with strings:

\begin{rustc}
let x: &'static str = "Hello, world.";
\end{rustc}

String literals have the type \code{\&'static str} because the reference is always alive: they are baked into the data segment of 
the final binary. Another example are globals:

\begin{rustc}
static FOO: i32 = 5;
let x: &'static i32 = &FOO;
\end{rustc}

This adds an \itt\ to the data segment of the binary, and \x\ is a reference to it.

\myparagraph{Lifetime Elision}
\label{paragraph:lifetime_elision}

Rust supports powerful local type inference in function bodies, but it's forbidden in item signatures to allow reasoning about the 
types based on the item signature alone. However, for ergonomic reasons a very restricted secondary inference algorithm called 
\enquote{lifetime elision} applies in function signatures. It infers only based on the signature components themselves and not based 
on the body of the function, only infers lifetime parameters, and does this with only three easily memorizable and unambiguous rules. 
This makes lifetime elision a shorthand for writing an item signature, while not hiding away the actual types involved as full local
inference would if applied to it.

\blank

When talking about lifetime elision, we use the term \emph{input lifetime} and \emph{output lifetime}. An \emph{input lifetime} is 
a lifetime associated with a parameter of a function, and an \emph{output lifetime} is a lifetime associated with the return value 
of a function. For example, this function has an input lifetime:

\begin{rustc}
fn foo<'a>(bar: &'a str)
\end{rustc}

This one has an output lifetime:

\begin{rustc}
fn foo<'a>() -> &'a str
\end{rustc}

This one has a lifetime in both positions:

\begin{rustc}
fn foo<'a>(bar: &'a str) -> &'a str
\end{rustc}

Here are the three rules:

\begin{itemize}
  \item{Each elided lifetime in a function's arguments becomes a distinct lifetime parameter.}
  \item{If there is exactly one input lifetime, elided or not, that lifetime is assigned to all elided lifetimes in the return 
      values of that function.}
  \item{If there are multiple input lifetimes, but one of them is \code{\&self} or \code{\&mut self}, the lifetime of self is 
      assigned to all elided output lifetimes.}
\end{itemize}

Otherwise, it is an error to elide an output lifetime.

\myparagraph{Examples}

Here are some examples of functions with elided lifetimes. We've paired each example of an elided lifetime with its expanded form.

\begin{rustc}
fn print(s: &str); // elided
fn print<'a>(s: &'a str); // expanded

fn debug(lvl: u32, s: &str); // elided
fn debug<'a>(lvl: u32, s: &'a str); // expanded

// In the preceding example, `lvl` doesn't need a lifetime because it's not a
// reference (`&`). Only things relating to references (such as a `struct`
// which contains a reference) need lifetimes.

fn substr(s: &str, until: u32) -> &str; // elided
fn substr<'a>(s: &'a str, until: u32) -> &'a str; // expanded

fn get_str() -> &str; // ILLEGAL, no inputs

fn frob(s: &str, t: &str) -> &str; // ILLEGAL, two inputs
fn frob<'a, 'b>(s: &'a str, t: &'b str) -> &str; // Expanded: Output lifetime is ambiguous

fn get_mut(&mut self) -> &mut T; // elided
fn get_mut<'a>(&'a mut self) -> &'a mut T; // expanded

fn args<T: ToCStr>(&mut self, args: &[T]) -> &mut Command; // elided
fn args<'a, 'b, T: ToCStr>(&'a mut self, args: &'b [T]) -> &'a mut Command; // expanded

fn new(buf: &mut [u8]) -> BufWriter; // elided
fn new<'a>(buf: &'a mut [u8]) -> BufWriter<'a>; // expanded
\end{rustc}
