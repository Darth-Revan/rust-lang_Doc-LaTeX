Rust has a way of defining constants with the \code{const} keyword:

\begin{rustc}
const N: i32 = 5;
\end{rustc}

Unlike \keylet\ bindings (see \nameref{sec:syntax_variableBindings}), you must annotate the type of a \code{const}.

\blank

Constants live for the entire lifetime of a program. More specifically, constants in Rust have no fixed address in memory. This 
is because they're effectively inlined to each place that they're used. References to the same constant are not necessarily guaranteed 
to refer to the same memory address for this reason.

\subsection*{\code{static}}
\label{paragraph:static}

Rust provides a 'global variable' sort of facility in static items. They're similar to constants, but static items aren't inlined upon 
use. This means that there is only one instance for each value, and it's at a fixed location in memory.

\blank

Here's an example:

\begin{rustc}
static N: i32 = 5;
\end{rustc}

Unlike \keylet\ bindings, you must annotate the type of a \code{static}.

\blank

Statics live for the entire lifetime of a program, and therefore any reference stored in a constant has a \code{'static} lifetime
(see \nameref{sec:syntax_lifetimes}):

\begin{rustc}
static NAME: &'static str = "Steve";
\end{rustc}

\subsection*{Mutability}

You can introduce mutability with the \mut\ keyword:

\begin{rustc}
static mut N: i32 = 5;
\end{rustc}

% TODO make unsafe a hyperref
Because this is mutable, one thread could be updating \code{N} while another is reading it, causing memory unsafety. As such both 
accessing and mutating a \code{static mut} is unsafe, and so must be done in an \code{unsafe} block:

\begin{rustc}
unsafe {
    N += 1;

    println!("N: {}", N);
}
\end{rustc}

Furthermore, any type stored in a \code{static} must be \code{Sync}, and may not have a \code{Drop} implementation (see \nameref{sec:syntax_drop}).

\subsection*{Initializing}

Both \code{const} and \code{static} have requirements for giving them a value. They may only be given a value that's a constant 
expression. In other words, you cannot use the result of a function call or anything similarly complex or at runtime.

\subsection*{Which construct should I use?}

Almost always, if you can choose between the two, choose \code{const}. It's pretty rare that you actually want a memory location 
associated with your constant, and using a const allows for optimizations like constant propagation not only in your crate but 
downstream crates.
