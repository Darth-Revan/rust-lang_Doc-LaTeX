This guide is one of three presenting Rust's ownership system. This is one of Rust's most unique and compelling features, with 
which Rust developers should become quite acquainted. Ownership is how Rust achieves its largest goal, memory safety. There are 
a few distinct concepts, each with its own chapter:

% TODO make ownership and lifetimes hyperrefs
\begin{itemize}
  \item{ownership, which you're reading now}
  \item{borrowing, and their associated feature 'references'}
  \item{lifetimes, an advanced concept of borrowing}
\end{itemize}

These three chapters are related, and in order. You'll need all three to fully understand the ownership system.

\subsection*{Meta}

Before we get to the details, two important notes about the ownership system.

\blank

Rust has a focus on safety and speed. It accomplishes these goals through many 'zero-cost abstractions', which means that in 
Rust, abstractions cost as little as possible in order to make them work. The ownership system is a prime example of a zero-cost
abstraction. All of the analysis we'll talk about in this guide is \emph{done at compile time}. You do not pay any run-time cost 
for any of these features.

\blank

However, this system does have a certain cost: learning curve. Many new users to Rust experience something we like to call 
'fighting with the borrow checker', where the Rust compiler refuses to compile a program that the author thinks is valid. 
This often happens because the programmer's mental model of how ownership should work doesn't match the actual rules that Rust 
implements. You probably will experience similar things at first. There is good news, however: more experienced Rust developers 
report that once they work with the rules of the ownership system for a period of time, they fight the borrow checker less and less.

\blank

With that in mind, let's learn about borrowing.

\subsection*{Borrowing}

At the end of the \nameref{sec:syntax_ownership} section, we had a nasty function that looked like this:

\begin{rustc}
fn foo(v1: Vec<i32>, v2: Vec<i32>) -> (Vec<i32>, Vec<i32>, i32) {
    // do stuff with v1 and v2

    // hand back ownership, and the result of our function
    (v1, v2, 42)
}

let v1 = vec![1, 2, 3];
let v2 = vec![1, 2, 3];

let (v1, v2, answer) = foo(v1, v2);
\end{rustc}

This is not idiomatic Rust, however, as it doesn't take advantage of borrowing. Here's the first step:

\begin{rustc}
fn foo(v1: &Vec<i32>, v2: &Vec<i32>) -> i32 {
    // do stuff with v1 and v2

    // return the answer
    42
}

let v1 = vec![1, 2, 3];
let v2 = vec![1, 2, 3];

let answer = foo(&v1, &v2);

// we can use v1 and v2 here!

\end{rustc}

Instead of taking \code{Vec<i32>}s as our arguments, we take a reference: \code{\&Vec<i32>}. And instead of passing \code{v1} and 
\code{v2} directly, we pass \code{\&v1} and \code{\&v2}. We call the \code{\&T} type a 'reference', and rather than owning the 
resource, it borrows ownership. A binding that borrows something does not deallocate the resource when it goes out of scope. This 
means that after the call to \code{foo()}, we can use our original bindings again.

\blank

References are immutable, like bindings. This means that inside of \code{foo()}, the vectors can't be changed at all:

\begin{rustc}
fn foo(v: &Vec<i32>) {
     v.push(5);
}

let v = vec![];

foo(&v);
\end{rustc}

errors with:

\begin{verbatim}
error: cannot borrow immutable borrowed content `*v` as mutable
v.push(5);
^
\end{verbatim}

Pushing a value mutates the vector, and so we aren't allowed to do it.

\subsection*{\&mut references}

There's a second kind of reference: \code{\&mut T}. A 'mutable reference' allows you to mutate the resource you're borrowing. 
For example:

\begin{rustc}
let mut x = 5;
{
    let y = &mut x;
    *y += 1;
}
println!("{}", x);
\end{rustc}

This will print \code{6}. We make \y\ a mutable reference to \x\, then add one to the thing \y\ points at. You'll 
notice that \x\ had to be marked \mut\ as well. If it wasn't, we couldn't take a mutable borrow to an immutable value.

\blank

You'll also notice we added an asterisk (\code{*}) in front of \y, making it \code{*y}, this is because \y\ is a 
\code{\&mut} reference. You'll also need to use them for accessing the contents of a reference as well.

\blank

Otherwise, \code{\&mut} references are like references. There is a large difference between the two, and how they interact, though. 
You can tell something is fishy in the above example, because we need that extra scope, with the \code{\{} and \code{\}}. If we 
remove them, we get an error:

\begin{verbatim}
error: cannot borrow `x` as immutable because it is also borrowed as mutable
    println!("{}", x);
                   ^
note: previous borrow of `x` occurs here; the mutable borrow prevents
subsequent moves, borrows, or modification of `x` until the borrow ends
        let y = &mut x;
                     ^
note: previous borrow ends here
fn main() {

}
^
\end{verbatim}

As it turns out, there are rules.

\subsection*{The Rules}

Here's the rules about borrowing in Rust:

First, any borrow must last for a scope no greater than that of the owner. Second, you may have one or the other of these two 
kinds of borrows, but not both at the same time:

\begin{itemize}
  \item{one or more references (\code{\&T}) to a resource,}
  \item{exactly one mutable reference (\code{\&mut T}).}
\end{itemize}

You may notice that this is very similar, though not exactly the same as, to the definition of a data race:

\begin{myquote}
There is a 'data race' when two or more pointers access the same memory location at the same time, where at least one of 
them is writing, and the operations are not synchronized.
\end{myquote}

With references, you may have as many as you'd like, since none of them are writing. However, as we can only have one \code{\&mut}
at a time, it is impossible to have a data race. This is how Rust prevents data races at compile time: we'll get errors if we break 
the rules.

\blank

With this in mind, let's consider our example again.

\myparagraph{Thinking in scopes}

Here's the code:

\begin{rustc}
let mut x = 5;
let y = &mut x;

*y += 1;

println!("{}", x);
\end{rustc}

This code gives us this error:

\begin{verbatim}
error: cannot borrow `x` as immutable because it is also borrowed as mutable
    println!("{}", x);
                   ^
\end{verbatim}

This is because we've violated the rules: we have a \code{\&mut T} pointing to \x, and so we aren't allowed to create any 
\code{\&T}s. One or the other. The note hints at how to think about this problem:

\begin{verbatim}
note: previous borrow ends here
fn main() {

}
^
\end{verbatim}

In other words, the mutable borrow is held through the rest of our example. What we want is for the mutable borrow to end before 
we try to call \println\ and make an immutable borrow. In Rust, borrowing is tied to the scope that the borrow is valid for. 
And our scopes look like this:

\begin{rustc}
let mut x = 5;

let y = &mut x;    // -+ &mut borrow of x starts here
                   //  |
*y += 1;           //  |
                   //  |
println!("{}", x); // -+ - try to borrow x here
                   // -+ &mut borrow of x ends here
\end{rustc}

The scopes conflict: we can't make an \code{\&x} while \y\ is in scope.

\blank

So when we add the curly braces:

\begin{rustc}
let mut x = 5;

{
    let y = &mut x; // -+ &mut borrow starts here
    *y += 1;        //  |
}                   // -+ ... and ends here

println!("{}", x);  // <- try to borrow x here
\end{rustc}

There's no problem. Our mutable borrow goes out of scope before we create an immutable one. But scope is the key to seeing how 
long a borrow lasts for.

\myparagraph{Issues borrowing prevents}

Why have these restrictive rules? Well, as we noted, these rules prevent data races. What kinds of issues do data races cause? 
Here's a few.

\myparagraph{Iterator invalidation}

One example is 'iterator invalidation', which happens when you try to mutate a collection that you're iterating over. Rust's 
borrow checker prevents this from happening:

\begin{rustc}
let mut v = vec![1, 2, 3];

for i in &v {
    println!("{}", i);
}
\end{rustc}

This prints out one through three. As we iterate through the vector, we're only given references to the elements. And \code{v} is 
itself borrowed as immutable, which means we can't change it while we're iterating:

\begin{rustc}
let mut v = vec![1, 2, 3];

for i in &v {
    println!("{}", i);
    v.push(34);
}
\end{rustc}

Here's the error:

\begin{verbatim}
error: cannot borrow `v` as mutable because it is also borrowed as immutable
    v.push(34);
    ^
note: previous borrow of `v` occurs here; the immutable borrow prevents
subsequent moves or mutable borrows of `v` until the borrow ends
for i in &v {
          ^
note: previous borrow ends here
for i in &v {
    println!(“{}”, i);
    v.push(34);
}
^
\end{verbatim}

We can't modify \code{v} because it's borrowed by the loop.

\myparagraph{use after free}

References must not live longer than the resource they refer to. Rust will check the scopes of your references to ensure that 
this is true.

\blank

If Rust didn't check this property, we could accidentally use a reference which was invalid. For example:

\begin{rustc}
let y: &i32;
{
    let x = 5;
    y = &x;
}

println!("{}", y);
\end{rustc}

We get this error:

\begin{verbatim}
error: `x` does not live long enough
    y = &x;
         ^
note: reference must be valid for the block suffix following statement 0 at
2:16...
let y: &i32;
{
    let x = 5;
    y = &x;
}

note: ...but borrowed value is only valid for the block suffix following
statement 0 at 4:18
    let x = 5;
    y = &x;
}
\end{verbatim}

In other words, \y\ is only valid for the scope where \x\ exists. As soon as \x\ goes away, it becomes invalid to 
refer to it. As such, the error says that the borrow 'doesn't live long enough' because it's not valid for the right amount of time.

\blank

The same problem occurs when the reference is declared before the variable it refers to. This is because resources within the same 
scope are freed in the opposite order they were declared:

\begin{rustc}
let y: &i32;
let x = 5;
y = &x;

println!("{}", y);
\end{rustc}

We get this error:

\begin{verbatim}
error: `x` does not live long enough
y = &x;
     ^
note: reference must be valid for the block suffix following statement 0 at
2:16...
    let y: &i32;
    let x = 5;
    y = &x;

    println!("{}", y);
}

note: ...but borrowed value is only valid for the block suffix following
statement 1 at 3:14
    let x = 5;
    y = &x;

    println!("{}", y);
}
\end{verbatim}

In the above example, \y\ is declared before \x, meaning that \y\ lives longer than \x, which is not allowed.
