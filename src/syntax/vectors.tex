A 'vector' is a dynamic or 'growable' array, implemented as the standard library type \href{https://doc.rust-lang.org/std/vec/}{Vec<T>}. The 
\code{T} means that we can have vectors of any type (see the chapter on \nameref{sec:syntax_generics} for more). Vectors always allocate 
their data on the heap. You can create them with the \code{vec!} macro:

\begin{rustc}
let v = vec![1, 2, 3, 4, 5]; // v: Vec<i32>
\end{rustc}

(Notice that unlike the \println\ macro we've used in the past, we use square brackets \code{[]} with \code{vec!} macro. Rust allows you to 
use either in either situation, this is just convention.)

\blank

There's an alternate form of \code{vec!} for repeating an initial value:

\begin{rustc}
let v = vec![0; 10]; // ten zeroes
\end{rustc}

\subsection*{Accessing elements}

To get the value at a particular index in the vector, we use \code{[]}s:

\begin{rustc}
let v = vec![1, 2, 3, 4, 5];

println!("The third element of v is {}", v[2]);
\end{rustc}

The indices count from \code{0}, so the third element is \code{v[2]}.

\blank

It's also important to note that you must index with the \code{usize} type:

\begin{rustc}
let v = vec![1, 2, 3, 4, 5];

let i: usize = 0;
let j: i32 = 0;

// works
v[i];

// doesn't
v[j];
\end{rustc}

Indexing with a non-\code{usize} type gives an error that looks like this:

\begin{verbatim}
error: the trait `core::ops::Index<i32>` is not implemented for the type
`collections::vec::Vec<_>` [E0277]
v[j];
^~~~
note: the type `collections::vec::Vec<_>` cannot be indexed by `i32`
error: aborting due to previous error
\end{verbatim}

There's a lot of punctuation in that message, but the core of it makes sense: you cannot index with an \itt.

\subsection*{Out-of-bounds Access}

If you try to access an index that doesn't exist:

\begin{rustc}
let v = vec![1, 2, 3];
println!("Item 7 is {}", v[7]);
\end{rustc}

then the current thread will panic (see \nameref{sec:effective_concurrency}) with a message like this:

\begin{verbatim}
thread '<main>' panicked at 'index out of bounds: the len is 3 but the index is 7'
\end{verbatim}

If you want to handle out-of-bounds errors without panicking, you can use methods like 
\href{http://doc.rust-lang.org/std/vec/struct.Vec.html#method.get}{get} or 
\href{http://doc.rust-lang.org/std/vec/struct.Vec.html#method.get_mut}{get\_mut} that return \code{None} when given an invalid index:

\begin{rustc}
let v = vec![1, 2, 3];
match v.get(7) {
    Some(x) => println!("Item 7 is {}", x),
    None => println!("Sorry, this vector is too short.")
}
\end{rustc}

\subsection*{Iterating}

Once you have a vector, you can iterate through its elements with \code{for}. There are three versions:

\begin{rustc}
let mut v = vec![1, 2, 3, 4, 5];

for i in &v {
    println!("A reference to {}", i);
}

for i in &mut v {
    println!("A mutable reference to {}", i);
}

for i in v {
    println!("Take ownership of the vector and its element {}", i);
}
\end{rustc}

Vectors have many more useful methods, which you can read about in \href{https://doc.rust-lang.org/std/vec/}{their API documentation}.
