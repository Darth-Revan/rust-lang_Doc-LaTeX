Now that we've discussed traits, let's talk about a particular trait provided by the Rust standard library, 
\href{https://doc.rust-lang.org/std/ops/trait.Drop.html}{Drop}. The \code{Drop} trait provides a way to run some code when a value goes out of 
scope. For example:

\begin{rustc}
struct HasDrop;

impl Drop for HasDrop {
    fn drop(&mut self) {
        println!("Dropping!");
    }
}

fn main() {
    let x = HasDrop;

    // do stuff

} // x goes out of scope here
\end{rustc}

When \x\ goes out of scope at the end of \code{main()}, the code for \code{Drop} will run. \code{Drop} has one method, which is also called 
\code{drop()}. It takes a mutable reference to \code{self}.

\blank

That's it! The mechanics of \code{Drop} are very simple, but there are some subtleties. For example, values are dropped in the opposite 
order they are declared. Here's another example:

\begin{rustc}
struct Firework {
    strength: i32,
}

impl Drop for Firework {
    fn drop(&mut self) {
        println!("BOOM times {}!!!", self.strength);
    }
}

fn main() {
    let firecracker = Firework { strength: 1 };
    let tnt = Firework { strength: 100 };
}
\end{rustc}

This will output:

\begin{verbatim}
BOOM times 100!!!
BOOM times 1!!!
\end{verbatim}

The TNT goes off before the firecracker does, because it was declared afterwards. Last in, first out.

\blank

So what is \code{Drop} good for? Generally, \code{Drop} is used to clean up any resources associated with a \struct. For example, the 
\href{https://doc.rust-lang.org/std/sync/struct.Arc.html}{Arc<T>} type is a reference-counted type. When \code{Drop} is called, it will 
decrement the reference count, and if the total number of references is zero, will clean up the underlying value.
