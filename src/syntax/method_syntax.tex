Functions are great, but if you want to call a bunch of them on some data, it can be awkward. Consider this code:

\begin{rustc}
baz(bar(foo));
\end{rustc}

We would read this left-to-right, and so we see 'baz bar foo'. But this isn't the order that the functions would get called in, that's 
inside-out: 'foo bar baz'. Wouldn't it be nice if we could do this instead?

\begin{rustc}
foo.bar().baz();
\end{rustc}

Luckily, as you may have guessed with the leading question, you can! Rust provides the ability to use this 'method call syntax' via the 
\code{impl} keyword.

\subsubsection*{Method calls}

Here's how it works:

\begin{rustc}
struct Circle {
    x: f64,
    y: f64,
    radius: f64,
}

impl Circle {
    fn area(&self) -> f64 {
        std::f64::consts::PI * (self.radius * self.radius)
    }
}

fn main() {
    let c = Circle { x: 0.0, y: 0.0, radius: 2.0 };
    println!("{}", c.area());
}
\end{rustc}

This will print \code{12.566371}.

We've made a \struct\ that represents a circle. We then write an \code{impl} block, and inside it, define a method, \code{area}.

\blank

Methods take a special first parameter, of which there are three variants: \code{self}, \code{\&self}, and \code{\&mut self}. You can think 
of this first parameter as being the \code{foo} in \code{foo.bar()}. The three variants correspond to the three kinds of things \code{foo} 
could be: \code{self} if it's a value on the stack, \code{\&self} if it's a reference, and \code{\&mut self} if it's a mutable reference. 
Because we took the \code{\&self} parameter to \code{area}, we can use it like any other parameter. Because we know it's a \code{Circle}, 
we can access the \code{radius} like we would with any other \struct.

\blank

We should default to using \code{\&self}, as you should prefer borrowing over taking ownership, as well as taking immutable references over 
mutable ones. Here's an example of all three variants:

\begin{rustc}
struct Circle {
    x: f64,
    y: f64,
    radius: f64,
}

impl Circle {
    fn reference(&self) {
       println!("taking self by reference!");
    }

    fn mutable_reference(&mut self) {
       println!("taking self by mutable reference!");
    }

    fn takes_ownership(self) {
       println!("taking ownership of self!");
    }
}
\end{rustc}

You can use as many \code{impl} blocks as you'd like. The previous example could have also been written like this:

\begin{rustc}
struct Circle {
    x: f64,
    y: f64,
    radius: f64,
}

impl Circle {
    fn reference(&self) {
       println!("taking self by reference!");
    }
}

impl Circle {
    fn mutable_reference(&mut self) {
       println!("taking self by mutable reference!");
    }
}

impl Circle {
    fn takes_ownership(self) {
       println!("taking ownership of self!");
    }
}
\end{rustc}

\subsubsection*{Chaining method calls}

So, now we know how to call a method, such as \code{foo.bar()}. But what about our original example, \code{foo.bar().baz()}? This is called 
'method chaining'. Let's look at an example:

\begin{rustc}
struct Circle {
    x: f64,
    y: f64,
    radius: f64,
}

impl Circle {
    fn area(&self) -> f64 {
        std::f64::consts::PI * (self.radius * self.radius)
    }

    fn grow(&self, increment: f64) -> Circle {
        Circle { x: self.x, y: self.y, radius: self.radius + increment }
    }
}

fn main() {
    let c = Circle { x: 0.0, y: 0.0, radius: 2.0 };
    println!("{}", c.area());

    let d = c.grow(2.0).area();
    println!("{}", d);
}
\end{rustc}

Check the return type:

\begin{rustc}
fn grow(&self, increment: f64) -> Circle {
\end{rustc}

We say we're returning a \code{Circle}. With this method, we can grow a new \code{Circle} to any arbitrary size.

\subsubsection*{Associated functions}

You can also define associated functions that do not take a \code{self} parameter. Here's a pattern that's very common in Rust code:

\begin{rustc}
struct Circle {
    x: f64,
    y: f64,
    radius: f64,
}

impl Circle {
    fn new(x: f64, y: f64, radius: f64) -> Circle {
        Circle {
            x: x,
            y: y,
            radius: radius,
        }
    }
}

fn main() {
    let c = Circle::new(0.0, 0.0, 2.0);
}
\end{rustc}

This 'associated function' builds a new \code{Circle} for us. Note that associated functions are called with the \code{Struct::function()} 
syntax, rather than the \code{ref.method()} syntax. Some other languages call associated functions 'static methods'.

\subsubsection*{Builder Pattern}

Let's say that we want our users to be able to create \code{Circles}, but we will allow them to only set the properties they care about. 
Otherwise, the \x\ and \y\ attributes will be \code{0.0}, and the \code{radius} will be \code{1.0}. Rust doesn't have method overloading, 
named arguments, or variable arguments. We employ the builder pattern instead. It looks like this:

\begin{rustc}
struct Circle {
    x: f64,
    y: f64,
    radius: f64,
}

impl Circle {
    fn area(&self) -> f64 {
        std::f64::consts::PI * (self.radius * self.radius)
    }
}

struct CircleBuilder {
    x: f64,
    y: f64,
    radius: f64,
}

impl CircleBuilder {
    fn new() -> CircleBuilder {
        CircleBuilder { x: 0.0, y: 0.0, radius: 1.0, }
    }

    fn x(&mut self, coordinate: f64) -> &mut CircleBuilder {
        self.x = coordinate;
        self
    }

    fn y(&mut self, coordinate: f64) -> &mut CircleBuilder {
        self.y = coordinate;
        self
    }

    fn radius(&mut self, radius: f64) -> &mut CircleBuilder {
        self.radius = radius;
        self
    }

    fn finalize(&self) -> Circle {
        Circle { x: self.x, y: self.y, radius: self.radius }
    }
}

fn main() {
    let c = CircleBuilder::new()
                .x(1.0)
                .y(2.0)
                .radius(2.0)
                .finalize();

    println!("area: {}", c.area());
    println!("x: {}", c.x);
    println!("y: {}", c.y);
}
\end{rustc}

What we've done here is make another \struct, \code{CircleBuilder}. We've defined our builder methods on it. We've also defined our \code{area()}
method on \code{Circle}. We also made one more method on \code{CircleBuilder}: \code{finalize()}. This method creates our final \code{Circle} 
from the builder. Now, we've used the type system to enforce our concerns: we can use the methods on \code{CircleBuilder} to constrain making
\code{Circles} in any way we choose.
