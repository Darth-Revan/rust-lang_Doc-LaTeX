Sometimes it is useful to wrap up a function and \emph{free variables} for better clarity and reuse. The free variables that can be used 
come from the enclosing scope and are 'closed over' when used in the function. From this, we get the name 'closures' and Rust provides a 
really great implementation of them, as we'll see.

\subsubsection*{Syntax}

Closures look like this:

\begin{rustc}
let plus_one = |x: i32| x + 1;

assert_eq!(2, plus_one(1));
\end{rustc}

We create a binding, \code{plus\_one}, and assign it to a closure. The closure's arguments go between the pipes (\code{|}), and the body 
is an expression, in this case, \code{x + 1}. Remember that \code{\{ \}} is an expression, so we can have multi-line closures too:

\begin{rustc}
let plus_two = |x| {
    let mut result: i32 = x;

    result += 1;
    result += 1;

    result
};

assert_eq!(4, plus_two(2));
\end{rustc}

You'll notice a few things about closures that are a bit different from regular named functions defined with \code{fn}. The first is that 
we did not need to annotate the types of arguments the closure takes or the values it returns. We can:

\begin{rustc}
let plus_one = |x: i32| -> i32 { x + 1 };

assert_eq!(2, plus_one(1));
\end{rustc}

But we don't have to. Why is this? Basically, it was chosen for ergonomic reasons. While specifying the full type for named functions 
is helpful with things like documentation and type inference, the full type signatures of closures are rarely documented since they're 
anonymous, and they don't cause the kinds of error-at-a-distance problems that inferring named function types can.

\blank

The second is that the syntax is similar, but a bit different. I've added spaces here for easier comparison:

\begin{rustc}
fn  plus_one_v1   (x: i32) -> i32 { x + 1 }
let plus_one_v2 = |x: i32| -> i32 { x + 1 };
let plus_one_v3 = |x: i32|          x + 1  ;
\end{rustc}

Small differences, but they're similar.

\subsubsection*{Closures and their environment}

The environment for a closure can include bindings from its enclosing scope in addition to parameters and local bindings. It looks like this:

\begin{rustc}
let num = 5;
let plus_num = |x: i32| x + num;

assert_eq!(10, plus_num(5));
\end{rustc}

This closure, \code{plus\_num}, refers to a \keylet\ binding in its scope: \code{num}. More specifically, it borrows the binding. If we 
do something that would conflict with that binding, we get an error. Like this one:

\begin{rustc}
let mut num = 5;
let plus_num = |x: i32| x + num;

let y = &mut num;
\end{rustc}

Which errors with:

\begin{verbatim}
error: cannot borrow `num` as mutable because it is also borrowed as immutable
    let y = &mut num;
                 ^~~
note: previous borrow of `num` occurs here due to use in closure; the immutable
  borrow prevents subsequent moves or mutable borrows of `num` until the borrow
  ends
    let plus_num = |x| x + num;
                   ^~~~~~~~~~~
note: previous borrow ends here
fn main() {
    let mut num = 5;
    let plus_num = |x| x + num;

    let y = &mut num;
}
^
\end{verbatim}

A verbose yet helpful error message! As it says, we can't take a mutable borrow on \code{num} because the closure is already borrowing 
it. If we let the closure go out of scope, we can:

\begin{rustc}
let mut num = 5;
{
    let plus_num = |x: i32| x + num;

} // plus_num goes out of scope, borrow of num ends

let y = &mut num;
\end{rustc}

If your closure requires it, however, Rust will take ownership and move the environment instead. This doesn't work:

\begin{rustc}
let nums = vec![1, 2, 3];

let takes_nums = || nums;

println!("{:?}", nums);
\end{rustc}

We get this error:

\begin{verbatim}
note: `nums` moved into closure environment here because it has type
  `[closure(()) -> collections::vec::Vec<i32>]`, which is non-copyable
let takes_nums = || nums;
                 ^~~~~~~
\end{verbatim}

\code{Vec<T>} has ownership over its contents, and therefore, when we refer to it in our closure, we have to take ownership of \code{nums}. 
It's the same as if we'd passed \code{nums} to a function that took ownership of it.

\subsubsection*{\code{move} closures}

We can force our closure to take ownership of its environment with the \code{move} keyword:

\begin{rustc}
let num = 5;

let owns_num = move |x: i32| x + num;
\end{rustc}

Now, even though the keyword is \code{move}, the variables follow normal move semantics. In this case, \code{5} implements \code{Copy}, 
and so \code{owns\_num} takes ownership of a copy of \code{num}. So what's the difference?

\begin{rustc}
let mut num = 5;

{
    let mut add_num = |x: i32| num += x;

    add_num(5);
}

assert_eq!(10, num);
\end{rustc}

So in this case, our closure took a mutable reference to \code{num}, and then when we called \code{add\_num}, it mutated the underlying 
value, as we'd expect. We also needed to declare \code{add\_num} as \mut\ too, because we're mutating its environment.

\blank

If we change to a \code{move} closure, it's different:

\begin{rustc}
let mut num = 5;

{
    let mut add_num = move |x: i32| num += x;

    add_num(5);
}

assert_eq!(5, num);
\end{rustc}

We only get \code{5}. Rather than taking a mutable borrow out on our \code{num}, we took ownership of a copy.

\blank

Another way to think about \code{move} closures: they give a closure its own stack frame. Without \code{move}, a closure may be tied 
to the stack frame that created it, while a \code{move} closure is self-contained. This means that you cannot generally return a non-\code{move} 
closure from a function, for example.

\blank

But before we talk about taking and returning closures, we should talk some more about the way that closures are implemented. As a systems 
language, Rust gives you tons of control over what your code does, and closures are no different.

\subsubsection*{Closure implementation}

Rust's implementation of closures is a bit different than other languages. They are effectively syntax sugar for traits. You'll want to 
make sure to have read the traits section (see \nameref{sec:syntax_traits}) before this one, as well as the section on 
trait objects (see \nameref{sec:syntax_traitObjects}).

\blank

Got all that? Good.

\blank

The key to understanding how closures work under the hood is something a bit strange: Using \code{()} to call a function, like \code{foo()}, 
is an overloadable operator. From this, everything else clicks into place. In Rust, we use the trait system to overload operators. Calling 
functions is no different. We have three separate traits to overload with:

\begin{rustc}
pub trait Fn<Args> : FnMut<Args> {
    extern "rust-call" fn call(&self, args: Args) -> Self::Output;
}

pub trait FnMut<Args> : FnOnce<Args> {
    extern "rust-call" fn call_mut(&mut self, args: Args) -> Self::Output;
}

pub trait FnOnce<Args> {
    type Output;

    extern "rust-call" fn call_once(self, args: Args) -> Self::Output;
}
\end{rustc}

You'll notice a few differences between these traits, but a big one is \code{self}: \code{Fn} takes \code{\&self}, \code{FnMut} takes 
\code{\&mut self}, and \code{FnOnce} takes \code{self}. This covers all three kinds of self via the usual method call syntax. But we've 
split them up into three traits, rather than having a single one. This gives us a large amount of control over what kind of closures we 
can take.

\blank

The \code{|| \{\}} syntax for closures is sugar for these three traits. Rust will generate a \struct\ for the environment, \code{impl} 
the appropriate trait, and then use it.

\subsubsection*{Taking closures as arguments}

Now that we know that closures are traits, we already know how to accept and return closures: the same as any other trait!

\blank

This also means that we can choose static vs dynamic dispatch as well. First, let's write a function which takes something callable, 
calls it, and returns the result:

\begin{rustc}
fn call_with_one<F>(some_closure: F) -> i32
    where F : Fn(i32) -> i32 {

    some_closure(1)
}

let answer = call_with_one(|x| x + 2);

assert_eq!(3, answer);
\end{rustc}

We pass our closure, \code{|x| x + 2}, to \code{call\_with\_one}. It does what it suggests: it calls the closure, giving it \code{1} as 
an argument.

\blank

Let's examine the signature of \code{call\_with\_one} in more depth:

\begin{rustc}
fn call_with_one<F>(some_closure: F) -> i32
\end{rustc}

We take one parameter, and it has the type \code{F}. We also return a \itt. This part isn't interesting. The next part is:

\begin{rustc}
    where F : Fn(i32) -> i32 {
\end{rustc}

Because \code{Fn} is a trait, we can bound our generic with it. In this case, our closure takes a \itt\ as an argument and returns an 
\itt, and so the generic bound we use is \code{Fn(i32) -> i32}.

\blank

There's one other key point here: because we're bounding a generic with a trait, this will get monomorphized, and therefore, we'll 
be doing static dispatch into the closure. That's pretty neat. In many languages, closures are inherently heap allocated, and will 
always involve dynamic dispatch. In Rust, we can stack allocate our closure environment, and statically dispatch the call. This happens 
quite often with iterators and their adapters, which often take closures as arguments.

\blank

Of course, if we want dynamic dispatch, we can get that too. A trait object handles this case, as usual:

\begin{rustc}
fn call_with_one(some_closure: &Fn(i32) -> i32) -> i32 {
    some_closure(1)
}

let answer = call_with_one(&|x| x + 2);

assert_eq!(3, answer);
\end{rustc}

Now we take a trait object, a \code{\&Fn}. And we have to make a reference to our closure when we pass it to \code{call\_with\_one}, 
so we use \code{\&||}.

\subsubsection*{Function pointers and closures}

A function pointer is kind of like a closure that has no environment. As such, you can pass a function pointer to any function 
expecting a closure argument, and it will work:

\begin{rustc}
fn call_with_one(some_closure: &Fn(i32) -> i32) -> i32 {
    some_closure(1)
}

fn add_one(i: i32) -> i32 {
    i + 1
}

let f = add_one;

let answer = call_with_one(&f);

assert_eq!(2, answer);
\end{rustc}

In this example, we don't strictly need the intermediate variable \code{f}, the name of the function works just fine too:

\begin{rustc}
let answer = call_with_one(&add_one);
\end{rustc}

\subsubsection*{Returning closures}

It's very common for functional-style code to return closures in various situations. If you try to return a closure, you may run 
into an error. At first, it may seem strange, but we'll figure it out. Here's how you'd probably try to return a closure from a 
function:

\begin{rustc}
fn factory() -> (Fn(i32) -> i32) {
    let num = 5;

    |x| x + num
}

let f = factory();

let answer = f(1);
assert_eq!(6, answer);
\end{rustc}

This gives us these long, related errors:

\begin{verbatim}
error: the trait `core::marker::Sized` is not implemented for the type
`core::ops::Fn(i32) -> i32` [E0277]
fn factory() -> (Fn(i32) -> i32) {
                ^~~~~~~~~~~~~~~~
note: `core::ops::Fn(i32) -> i32` does not have a constant size known at compile-time
fn factory() -> (Fn(i32) -> i32) {
                ^~~~~~~~~~~~~~~~
error: the trait `core::marker::Sized` is not implemented for the type `core::ops::Fn(i32) -> i32` [E0277]
let f = factory();
    ^
note: `core::ops::Fn(i32) -> i32` does not have a constant size known at compile-time
let f = factory();
    ^
\end{verbatim}

In order to return something from a function, Rust needs to know what size the return type is. But since \code{Fn} is a trait, it 
could be various things of various sizes: many different types can implement \code{Fn}. An easy way to give something a size is to 
take a reference to it, as references have a known size. So we'd write this:

\begin{rustc}
fn factory() -> &(Fn(i32) -> i32) {
    let num = 5;

    |x| x + num
}

let f = factory();

let answer = f(1);
assert_eq!(6, answer);
\end{rustc}

But we get another error:

\begin{verbatim}
error: missing lifetime specifier [E0106]
fn factory() -> &(Fn(i32) -> i32) {
                ^~~~~~~~~~~~~~~~~
\end{verbatim}

Right. Because we have a reference, we need to give it a lifetime. But our \code{factory()} function takes no arguments, so 
elision (see \nameref{paragraph:lifetime_elision}) doesn't kick in here. Then what choices do we have? Try \code{'static}:

\begin{rustc}
fn factory() -> &'static (Fn(i32) -> i32) {
    let num = 5;

    |x| x + num
}

let f = factory();

let answer = f(1);
assert_eq!(6, answer);
\end{rustc}

But we get another error:

\begin{verbatim}
error: mismatched types:
 expected `&'static core::ops::Fn(i32) -> i32`,
    found `[closure@<anon>:7:9: 7:20]`
(expected &-ptr,
    found closure) [E0308]
         |x| x + num
         ^~~~~~~~~~~
\end{verbatim}

This error is letting us know that we don't have a \code{\&'static Fn(i32) -> i32}, we have a \code{[closure@<anon>:7:9: 7:20]}. Wait, what?

\blank

Because each closure generates its own environment \struct\ and implementation of \code{Fn} and friends, these types are anonymous. 
They exist solely for this closure. So Rust shows them as \code{closure@<anon>}, rather than some autogenerated name.

\blank

The error also points out that the return type is expected to be a reference, but what we are trying to return is not. Further, we 
cannot directly assign a \code{'static} lifetime to an object. So we'll take a different approach and return a 'trait object' by 
\code{Box}ing up the \code{Fn}. This almost works:

\begin{rustc}
fn factory() -> Box<Fn(i32) -> i32> {
    let num = 5;

    Box::new(|x| x + num)
}
let f = factory();

let answer = f(1);
assert_eq!(6, answer);
\end{rustc}

There's just one last problem:

\begin{verbatim}
error: closure may outlive the current function, but it borrows `num`,
which is owned by the current function [E0373]
Box::new(|x| x + num)
         ^~~~~~~~~~~
\end{verbatim}

Well, as we discussed before, closures borrow their environment. And in this case, our environment is based on a stack-allocated \code{5}, 
the \code{num} variable binding. So the borrow has a lifetime of the stack frame. So if we returned this closure, the function call 
would be over, the stack frame would go away, and our closure is capturing an environment of garbage memory! With one last fix, we can 
make this work:

\begin{rustc}
fn factory() -> Box<Fn(i32) -> i32> {
    let num = 5;

    Box::new(move |x| x + num)
}
let f = factory();

let answer = f(1);
assert_eq!(6, answer);
\end{rustc}

By making the inner closure a \code{move Fn}, we create a new stack frame for our closure. By \code{Box}ing it up, we've given it a known 
size, and allowing it to escape our stack frame.
