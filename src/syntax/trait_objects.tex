When code involves polymorphism, there needs to be a mechanism to determine which specific version is actually run. This is called 
'dispatch'. There are two major forms of dispatch: static dispatch and dynamic dispatch. While Rust favors static dispatch, it also 
supports dynamic dispatch through a mechanism called 'trait objects'.

\subsection*{Background}

For the rest of this chapter, we'll need a trait and some implementations. Let's make a simple one, \code{Foo}. It has one method that 
is expected to return a \String.

\begin{rustc}
trait Foo {
    fn method(&self) -> String;
}
\end{rustc}

We'll also implement this trait for \code{u8} and \String:

\begin{rustc}
impl Foo for u8 {
    fn method(&self) -> String { format!("u8: {}", *self) }
}

impl Foo for String {
    fn method(&self) -> String { format!("string: {}", *self) }
}
\end{rustc}

\subsection*{Static dispatch}

We can use this trait to perform static dispatch with trait bounds:

\begin{rustc}
fn do_something<T: Foo>(x: T) {
    x.method();
}

fn main() {
    let x = 5u8;
    let y = "Hello".to_string();

    do_something(x);
    do_something(y);
}
\end{rustc}

Rust uses 'monomorphization' to perform static dispatch here. This means that Rust will create a special version of \code{do\_something()} for 
both \code{u8} and \String, and then replace the call sites with calls to these specialized functions. In other words, Rust generates something 
like this:

\begin{rustc}
fn do_something_u8(x: u8) {
    x.method();
}

fn do_something_string(x: String) {
    x.method();
}

fn main() {
    let x = 5u8;
    let y = "Hello".to_string();

    do_something_u8(x);
    do_something_string(y);
}
\end{rustc}

This has a great upside: static dispatch allows function calls to be inlined because the callee is known at compile time, and inlining is the 
key to good optimization. Static dispatch is fast, but it comes at a tradeoff: 'code bloat', due to many copies of the same function existing 
in the binary, one for each type.

\blank

Furthermore, compilers aren't perfect and may \enquote{optimize} code to become slower. For example, functions inlined too eagerly will bloat 
the instruction cache (cache rules everything around us). This is part of the reason that \code{\#[inline]} and \code{\#[inline(always)]} 
should be used carefully, and one reason why using a dynamic dispatch is sometimes more efficient.

\blank

However, the common case is that it is more efficient to use static dispatch, and one can always have a thin statically-dispatched wrapper 
function that does a dynamic dispatch, but not vice versa, meaning static calls are more flexible. The standard library tries to be 
statically dispatched where possible for this reason.

\subsection*{Dynamic dispatch}

Rust provides dynamic dispatch through a feature called 'trait objects'. Trait objects, like \code{\&Foo} or \code{Box<Foo>}, are normal 
values that store a value of \emph{any type} that implements the given trait, where the precise type can only be known at runtime.

\blank

A trait object can be obtained from a pointer to a concrete type that implements the trait by casting it (e.g. \code{\&x} as \code{\&Foo}) 
or coercing it (e.g. using \code{\&x} as an argument to a function that takes \code{\&Foo}).

\blank

These trait object coercions and casts also work for pointers like \code{\&mut T} to \code{\&mut Foo} and \code{Box<T>} to \code{Box<Foo>}, 
but that's all at the moment. Coercions and casts are identical.

\blank

This operation can be seen as 'erasing' the compiler's knowledge about the specific type of the pointer, and hence trait objects are 
sometimes referred to as 'type erasure'.

\blank

Coming back to the example above, we can use the same trait to perform dynamic dispatch with trait objects by casting:

\begin{rustc}
fn do_something(x: &Foo) {
    x.method();
}

fn main() {
    let x = 5u8;
    do_something(&x as &Foo);
}
\end{rustc}

or by coercing:

\begin{rustc}
fn do_something(x: &Foo) {
    x.method();
}

fn main() {
    let x = "Hello".to_string();
    do_something(&x);
}
\end{rustc}

A function that takes a trait object is not specialized to each of the types that implements \code{Foo}: only one copy is generated, often 
(but not always) resulting in less code bloat. However, this comes at the cost of requiring slower virtual function calls, and effectively
inhibiting any chance of inlining and related optimizations from occurring.

\subsection*{Why pointers?}

Rust does not put things behind a pointer by default, unlike many managed languages, so types can have different sizes. Knowing the size of 
the value at compile time is important for things like passing it as an argument to a function, moving it about on the stack and allocating 
(and deallocating) space on the heap to store it.

\blank

For \code{Foo}, we would need to have a value that could be at least either a \String\ (24 bytes) or a \code{u8} (1 byte), as well as any 
other type for which dependent crates may implement \code{Foo} (any number of bytes at all). There's no way to guarantee that this last 
point can work if the values are stored without a pointer, because those other types can be arbitrarily large.

\blank

Putting the value behind a pointer means the size of the value is not relevant when we are tossing a trait object around, only the size 
of the pointer itself.

\subsection*{Representation}

The methods of the trait can be called on a trait object via a special record of function pointers traditionally called a 'vtable' 
(created and managed by the compiler).

\blank

Trait objects are both simple and complicated: their core representation and layout is quite straight-forward, but there are some curly 
error messages and surprising behaviors to discover.

\blank

Let's start simple, with the runtime representation of a trait object. The \code{std::raw} module contains structs with layouts that are 
the same as the complicated built-in types, \href{https://doc.rust-lang.org/std/raw/struct.TraitObject.html}{including trait objects}:

\begin{rustc}
pub struct TraitObject {
    pub data: *mut (),
    pub vtable: *mut (),
}
\end{rustc}

That is, a trait object like \code{\&Foo} consists of a 'data' pointer and a 'vtable' pointer.

\blank

The data pointer addresses the data (of some unknown type \code{T}) that the trait object is storing, and the vtable pointer points to the 
vtable ('virtual method table') corresponding to the implementation of \code{Foo} for \code{T}.

\blank

A vtable is essentially a struct of function pointers, pointing to the concrete piece of machine code for each method in the implementation. 
A method call like \code{trait\_object.method()} will retrieve the correct pointer out of the vtable and then do a dynamic call of it. 
For example:

\begin{rustc}
struct FooVtable {
    destructor: fn(*mut ()),
    size: usize,
    align: usize,
    method: fn(*const ()) -> String,
}

// u8:

fn call_method_on_u8(x: *const ()) -> String {
    // the compiler guarantees that this function is only called
    // with `x` pointing to a u8
    let byte: &u8 = unsafe { &*(x as *const u8) };

    byte.method()
}

static Foo_for_u8_vtable: FooVtable = FooVtable {
    destructor: /* compiler magic */,
    size: 1,
    align: 1,

    // cast to a function pointer
    method: call_method_on_u8 as fn(*const ()) -> String,
};


// String:

fn call_method_on_String(x: *const ()) -> String {
    // the compiler guarantees that this function is only called
    // with `x` pointing to a String
    let string: &String = unsafe { &*(x as *const String) };

    string.method()
}

static Foo_for_String_vtable: FooVtable = FooVtable {
    destructor: /* compiler magic */,
    // values for a 64-bit computer, halve them for 32-bit ones
    size: 24,
    align: 8,

    method: call_method_on_String as fn(*const ()) -> String,
};
\end{rustc}

The \code{destructor} field in each vtable points to a function that will clean up any resources of the vtable's type: for \code{u8} it is 
trivial, but for \String\ it will free the memory. This is necessary for owning trait objects like \code{Box<Foo>}, which need to clean-up 
both the \code{Box} allocation as well as the internal type when they go out of scope. The \code{size} and \code{align} fields store the 
size of the erased type, and its alignment requirements; these are essentially unused at the moment since the information is embedded in 
the destructor, but will be used in the future, as trait objects are progressively made more flexible.

\blank

Suppose we've got some values that implement \code{Foo}. The explicit form of construction and use of \code{Foo} trait objects might look 
a bit like (ignoring the type mismatches: they're all pointers anyway):

\begin{rustc}
let a: String = "foo".to_string();
let x: u8 = 1;

// let b: &Foo = &a;
let b = TraitObject {
    // store the data
    data: &a,
    // store the methods
    vtable: &Foo_for_String_vtable
};

// let y: &Foo = x;
let y = TraitObject {
    // store the data
    data: &x,
    // store the methods
    vtable: &Foo_for_u8_vtable
};

// b.method();
(b.vtable.method)(b.data);

// y.method();
(y.vtable.method)(y.data);
\end{rustc}

\subsection*{Object Safety}

Not every trait can be used to make a trait object. For example, vectors implement \code{Clone}, but if we try to make a trait object:

\begin{rustc}
let v = vec![1, 2, 3];
let o = &v as &Clone;
\end{rustc}

We get an error:

\begin{verbatim}
error: cannot convert to a trait object because trait `core::clone::Clone` is not object-safe [E0038]
let o = &v as &Clone;
        ^~
note: the trait cannot require that `Self : Sized`
let o = &v as &Clone;
        ^~
\end{verbatim}

The error says that \code{Clone} is not 'object-safe'. Only traits that are object-safe can be made into trait objects. A trait is 
object-safe if both of these are true:

\begin{itemize}
  \item{the trait does not require that \code{Self: Sized}}
  \item{all of its methods are object-safe}
\end{itemize}

So what makes a method object-safe? Each method must require that \code{Self: Sized} or all of the following:

\begin{itemize}
  \item{must not have any type parameters}
  \item{must not use \code{Self}}
\end{itemize}

Whew! As we can see, almost all of these rules talk about \code{Self}. A good intuition is \enquote{except in special circumstances, if your 
trait's method uses \code{Self}, it is not object-safe.}
