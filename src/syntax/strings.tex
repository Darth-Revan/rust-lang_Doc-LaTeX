Strings are an important concept for any programmer to master. Rust's string handling system is a bit different from other languages, due to 
its systems focus. Any time you have a data structure of variable size, things can get tricky, and strings are a re-sizable data structure. 
That being said, Rust's strings also work differently than in some other systems languages, such as C.

\blank

Let's dig into the details. A 'string' is a sequence of Unicode scalar values encoded as a stream of UTF-8 bytes. All strings are guaranteed 
to be a valid encoding of UTF-8 sequences. Additionally, unlike some systems languages, strings are not null-terminated and can contain null 
bytes.

\blank

Rust has two main types of strings: \code{\&str} and \String. Let's talk about \code{\&str} first. These are called 'string slices'. A string 
slice has a fixed size, and cannot be mutated. It is a reference to a sequence of UTF-8 bytes.

\begin{rustc}
let greeting = "Hello there."; // greeting: &'static str
\end{rustc}

\code{"Hello there."} is a string literal and its type is \code{\&'static str}. A string literal is a string slice that is statically allocated,
meaning that it's saved inside our compiled program, and exists for the entire duration it runs. The \code{greeting} binding is a reference to 
this statically allocated string. Any function expecting a string slice will also accept a string literal.

\blank

String literals can span multiple lines. There are two forms. The first will include the newline and the leading spaces:

\begin{rustc}
let s = "foo
    bar";

assert_eq!("foo\n        bar", s);
\end{rustc}

The second, with a \code{\\}, trims the spaces and the newline:

\begin{rustc}
let s = "foo\
    bar"; 

assert_eq!("foobar", s);
\end{rustc}

Rust has more than only \code{\&str}s though. A \String, is a heap-allocated string. This string is growable, and is also guaranteed to be 
UTF-8. \String s are commonly created by converting from a string slice using the \code{to\_string} method.

\begin{rustc}
let mut s = "Hello".to_string(); // mut s: String
println!("{}", s);

s.push_str(", world.");
println!("{}", s);
\end{rustc}

\String s will coerce into \code{\&str} with an \&:

\begin{rustc}
fn takes_slice(slice: &str) {
    println!("Got: {}", slice);
}

fn main() {
    let s = "Hello".to_string();
    takes_slice(&s);
}
\end{rustc}

This coercion does not happen for functions that accept one of \code{\&str}'s traits instead of \code{\&str}. For example, 
\href{https://doc.rust-lang.org/std/net/struct.TcpStream.html#method.connect}{TcpStream::connect} has a parameter of type \code{ToSocketAddrs}. 
A \code{\&str} is okay but a \String\ must be explicitly converted using \code{\&*}.

\begin{rustc}
use std::net::TcpStream;

TcpStream::connect("192.168.0.1:3000"); // &str parameter

let addr_string = "192.168.0.1:3000".to_string();
TcpStream::connect(&*addr_string); // convert addr_string to &str
\end{rustc}

Viewing a \String\ as a \code{\&str} is cheap, but converting the \code{\&str} to a \String\ involves allocating memory. No reason to do 
that unless you have to!

\subsection*{Indexing}

Because strings are valid UTF-8, strings do not support indexing:

\begin{rustc}
let s = "hello";

println!("The first letter of s is {}", s[0]); // ERROR!!!
\end{rustc}

Usually, access to a vector with \code{[]} is very fast. But, because each character in a UTF-8 encoded string can be multiple bytes, 
you have to walk over the string to find the $n_{th}$ letter of a string. This is a significantly more expensive operation, and we don't 
want to be misleading. Furthermore, 'letter' isn't something defined in Unicode, exactly. We can choose to look at a string as individual 
bytes, or as codepoints:

\begin{rustc}
let hachiko = "忠犬ハチ公";

for b in hachiko.as_bytes() {
    print!("{}, ", b);
}

println!("");

for c in hachiko.chars() {
    print!("{}, ", c);
}

println!("");
\end{rustc}

\begin{rustc}
let hachiko = "忠犬ハチ公";

for b in hachiko.as_bytes() {
    print!("{}, ", b);
}

println!("");

for c in hachiko.chars() {
    print!("{}, ", c);
}

println!("");
\end{rustc}

This prints:

\begin{verbatim}
229, 191, 160, 231, 138, 172, 227, 131, 143, 227, 131, 129, 229, 133, 172,
忠, 犬, ハ, チ, 公,
\end{verbatim}

As you can see, there are more bytes than \varchar s.

\blank

You can get something similar to an index like this:

\begin{rustc}
let dog = hachiko.chars().nth(1); // kinda like hachiko[1]
\end{rustc}

This emphasizes that we have to walk from the beginning of the list of \varchar s.

\subsection*{Slicing}

You can get a slice of a string with slicing syntax:

\begin{rustc}
let dog = "hachiko";
let hachi = &dog[0..5];
\end{rustc}

But note that these are \emph{byte offsets}, not \emph{character offsets}. So this will fail at runtime:

\begin{rustc}
let dog = "忠犬ハチ公";
let hachi = &dog[0..2];
\end{rustc}

with this error:

\begin{verbatim}
thread '<main>' panicked at 'index 0 and/or 2 in `忠犬ハチ公` do not lie on
character boundary'
\end{verbatim}

\subsection*{Concatenation}

If you have a \String, you can concatenate a \code{\&str} to the end of it:

\begin{rustc}
let hello = "Hello ".to_string();
let world = "world!";

let hello_world = hello + world;
\end{rustc}

But if you have two \String s, you need an \code{\&}:

\begin{rustc}
let hello = "Hello ".to_string();
let world = "world!".to_string();

let hello_world = hello + &world;
\end{rustc}

% TODO make deref coercion a hyperref
This is because \code{\&String} can automatically coerce to a \code{\&str}. This is a feature called 'Deref coercions'.
