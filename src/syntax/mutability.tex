Mutability, the ability to change something, works a bit differently in Rust than in other languages. The first aspect of 
mutability is its non-default status:

\begin{rustc}
let x = 5;
x = 6; // error!
\end{rustc}

We can introduce mutability with the \mut\ keyword:

\begin{rustc}
let mut x = 5;

x = 6; // no problem!
\end{rustc}

This is a mutable \nameref{sec:syntax_variableBindings}. When a binding is mutable, it means you're allowed to change what the 
binding points to. So in the above example, it's not so much that the value at \x\ is changing, but that the binding changed
from one \itt\ to another.

\blank

If you want to change what the binding points to, you'll need a mutable reference (see \nameref{sec:syntax_referencesBorrowing}):

\begin{rustc}
let mut x = 5;
let y = &mut x;
\end{rustc}

\y\ is an immutable binding to a mutable reference, which means that you can't bind \y\ to something else (
\code{y = \&mut z}), but you can mutate the thing that's bound to \y\ (\code{*y = 5}). A subtle distinction.

\blank

Of course, if you need both:

\begin{rustc}
let mut x = 5;
let mut y = &mut x;
\end{rustc}

Now \y\ can be bound to another value, and the value it's referencing can be changed.

\blank

% TODO make pattern a hyperref
It's important to note that \mut\ is part of a pattern, so you can do things like this:

\begin{rustc}
let (mut x, y) = (5, 6);

fn foo(mut x: i32) {
\end{rustc}

\subsubsection*{Interior vs. Exterior Mutability}

However, when we say something is 'immutable' in Rust, that doesn't mean that it's not able to be changed: we mean something has 
'exterior mutability'. Consider, for example, \href{https://doc.rust-lang.org/std/sync/struct.Arc.html}{Arc<T>}:

\begin{rustc}
use std::sync::Arc;

let x = Arc::new(5);
let y = x.clone();
\end{rustc}

When we call \code{clone()}, the \code{Arc<T>} needs to update the reference count. Yet we've not used any muts here, \x\ is 
an immutable binding, and we didn't take \code{\&mut 5} or anything. So what gives?

\blank

To understand this, we have to go back to the core of Rust's guiding philosophy, memory safety, and the mechanism by which Rust 
guarantees it, the ownership system (see \nameref{sec:syntax_ownership}), and more specifically, borrowing (see 
\nameref{sec:syntax_referencesBorrowing}):

\begin{myquote}
You may have one or the other of these two kinds of borrows, but not both at the same time:
\begin{itemize}
  \item{one or more references (\code{\&T}) to a resource,}
  \item{exactly one mutable reference (\code{\&mut T}).}
\end{itemize}
\end{myquote}

So, that's the real definition of 'immutability': is this safe to have two pointers to? In \code{Arc<T>}'s case, yes: the mutation 
is entirely contained inside the structure itself. It's not user facing. For this reason, it hands out \code{\&T} with \code{clone()}. 
If it handed out \code{\&mut T}s, though, that would be a problem.

\blank

Other types, like the ones in the \href{https://doc.rust-lang.org/std/cell/}{std::cell} module, have the opposite: interior 
mutability. For example:

\begin{rustc}
use std::cell::RefCell;

let x = RefCell::new(42);

let y = x.borrow_mut();
\end{rustc}

RefCell hands out \code{\&mut} references to what's inside of it with the \code{borrow\_mut()} method. Isn't that dangerous? What if 
we do:

\begin{rustc}
use std::cell::RefCell;

let x = RefCell::new(42);

let y = x.borrow_mut();
let z = x.borrow_mut();
\end{rustc}

This will in fact panic, at runtime. This is what \code{RefCell} does: it enforces Rust's borrowing rules at runtime, and 
\panic s if they're violated. This allows us to get around another aspect of Rust's mutability rules. Let's talk about it 
first.

\myparagraph{Field-level mutability}

% TODO make struct a hyperref
Mutability is a property of either a borrow (\code{\&mut}) or a binding (\code{let mut}). This means that, for example, you cannot 
have a struct with some fields mutable and some immutable:

\begin{rustc}
struct Point {
    x: i32,
    mut y: i32, // nope
}
\end{rustc}

The mutability of a struct is in its binding:

\begin{rustc}
struct Point {
    x: i32,
    y: i32,
}

let mut a = Point { x: 5, y: 6 };

a.x = 10;

let b = Point { x: 5, y: 6};

b.x = 10; // error: cannot assign to immutable field `b.x`
\end{rustc}

However, by using \href{https://doc.rust-lang.org/std/cell/struct.Cell.html}{Cell<T>}, you can emulate field-level mutability:

\begin{rustc}
use std::cell::Cell;

struct Point {
    x: i32,
    y: Cell<i32>,
}

let point = Point { x: 5, y: Cell::new(6) };

point.y.set(7);

println!("y: {:?}", point.y);
\end{rustc}

This will print \code{y: Cell { value: 7 }}. We've successfully updated \y.
