\code{if let} allows you to combine \keyif\ and \keylet\ together to reduce the overhead of certain kinds of pattern matches.

\blank

For example, let's say we have some sort of \code{Option<T>}. We want to call a function on it if it's \code{Some<T>}, but do nothing if it's 
\code{None}. That looks like this:

\begin{rustc}
match option {
    Some(x) => { foo(x) },
    None => {},
}
\end{rustc}

We don't have to use \match\ here, for example, we could use \keyif:

\begin{rustc}
if option.is_some() {
    let x = option.unwrap();
    foo(x);
}
\end{rustc}

Neither of these options is particularly appealing. We can use \code{if let} to do the same thing in a nicer way:

\begin{rustc}
if let Some(x) = option {
    foo(x);
}
\end{rustc}

If a pattern (see \nameref{sec:syntax_patterns}) matches successfully, it binds any appropriate parts of the value to the identifiers in 
the pattern, then evaluates the expression. If the pattern doesn't match, nothing happens.

\blank

If you want to do something else when the pattern does not match, you can use \code{else}:

\begin{rustc}
if let Some(x) = option {
    foo(x);
} else {
    bar();
}
\end{rustc}

\subsection*{\code{while let}}

In a similar fashion, \code{while let} can be used when you want to conditionally loop as long as a value matches a certain pattern. It 
turns code like this:

\begin{rustc}
let mut v = vec![1, 3, 5, 7, 11];
loop {
    match v.pop() {
        Some(x) =>  println!("{}", x),
        None => break,
    }
}
\end{rustc}

Into code like this:

\begin{rustc}
let mut v = vec![1, 3, 5, 7, 11];
while let Some(x) = v.pop() {
    println!("{}", x);
}
\end{rustc}
