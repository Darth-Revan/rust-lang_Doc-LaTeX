Rust's main draw is its powerful static guarantees about behavior. But safety checks are conservative by nature: there are some 
programs that are actually safe, but the compiler is not able to verify this is true. To write these kinds of programs, we need to 
tell the compiler to relax its restrictions a bit. For this, Rust has a keyword, \code{unsafe}. Code using \code{unsafe} has less 
restrictions than normal code does.

\blank

Let's go over the syntax, and then we'll talk semantics. \code{unsafe} is used in four contexts. The first one is to mark a function 
as unsafe:

\begin{rustc}
unsafe fn danger_will_robinson() {
    // scary stuff
}
\end{rustc}

All functions called from FFI must be marked as \code{unsafe}, for example (see \nameref{sec:effective_FFI}). The second use 
of \code{unsafe} is an unsafe block:

\begin{rustc}
unsafe {
    // scary stuff
}
\end{rustc}

The third is for unsafe traits:

\begin{rustc}
unsafe trait Scary { }
\end{rustc}

And the fourth is for \code{impl}ementing one of those traits:

\begin{rustc}
unsafe impl Scary for i32 {}
\end{rustc}

It's important to be able to explicitly delineate code that may have bugs that cause big problems. If a Rust program segfaults, you 
can be sure the cause is related to something marked \code{unsafe}.

\subsection*{What does 'safe' mean?}

Safe, in the context of Rust, means 'doesn't do anything unsafe'. It's also important to know that there are certain behaviors that 
are probably not desirable in your code, but are expressly \emph{not} unsafe:

\begin{itemize}
  \item{Deadlocks}
  \item{Leaks of memory or other resources}
  \item{Exiting without calling destructors}
  \item{Integer overflow}
\end{itemize}

Rust cannot prevent all kinds of software problems. Buggy code can and will be written in Rust. These things aren't great, but they 
don't qualify as \code{unsafe} specifically.

\blank

In addition, the following are all undefined behaviors in Rust, and must be avoided, even when writing \code{unsafe} code:

\begin{itemize}
  \item{Data races}
  \item{Dereferencing a null/dangling raw pointer}
  \item{Reads of \href{http://llvm.org/docs/LangRef.html\#undefined-values}{undef} (uninitialized) memory}
  \item{Breaking the \href{http://llvm.org/docs/LangRef.html\#pointer-aliasing-rules}{pointer aliasing rules} with raw pointers.}
  \item{\code{\&mut T} and \code{\&T} follow LLVM's scoped \href{http://llvm.org/docs/LangRef.html\#noalias}{noalias} model, except if 
      the \code{\&T} contains an \code{UnsafeCell<U>}. Unsafe code must not violate these aliasing guarantees.}
  \item{Mutating an immutable value/reference without \code{UnsafeCell<U>}}
  \item{Invoking undefined behavior via compiler intrinsics:}
  \begin{itemize}
    \item{Indexing outside of the bounds of an object with \code{std::ptr::offset} (\code{offset} intrinsic), with the exception of one 
        byte past the end which is permitted.}
    \item{Using \code{std::ptr::copy\_nonoverlapping\_memory} (\code{memcpy32/memcpy64} intrinsics) on overlapping buffers}
  \end{itemize}
  \item{Invalid values in primitive types, even in private fields/locals:}
  \begin{itemize}
    \item{Null/dangling references or boxes}
    \item{A value other than \code{false} (0) or \code{true} (1) in a \code{bool}}
    \item{A discriminant in an \enum\ not included in its type definition}
    \item{A value in a \varchar\ which is a surrogate or above \code{char::MAX}}
    \item{Non-UTF-8 byte sequences in a \code{str}}
  \end{itemize}
  \item{Unwinding into Rust from foreign code or unwinding from Rust into foreign code.}
\end{itemize}

\subsection*{Unsafe Superpowers}

In both unsafe functions and unsafe blocks, Rust will let you do three things that you normally can not do. Just three. Here they are:

\begin{itemize}
  \item{Access or update a static mutable variable (see \nameref{paragraph:static}.}
  \item{Dereference a raw pointer.}
  \item{Call unsafe functions. This is the most powerful ability.}
\end{itemize}

That's it. It's important that \code{unsafe} does not, for example, 'turn off the borrow checker'. Adding \code{unsafe} to some 
random Rust code doesn't change its semantics, it won't start accepting anything. But it will let you write things that do break 
some of the rules.

\blank

You will also encounter the \code{unsafe} keyword when writing bindings to foreign (non-Rust) interfaces. You're encouraged to write 
a safe, native Rust interface around the methods provided by the library.

\blank

Let's go over the basic three abilities listed, in order.

\myparagraph{Access or update a \code{static mut}}

Rust has a feature called '\code{static mut}' which allows for mutable global state. Doing so can cause a data race, and as such 
is inherently not safe. For more details, see the static section of the book (see \nameref{paragraph:static}).

\myparagraph{Dereference a raw pointer}

Raw pointers let you do arbitrary pointer arithmetic, and can cause a number of different memory safety and security issues. In 
some senses, the ability to dereference an arbitrary pointer is one of the most dangerous things you can do. For more on raw pointers, 
see their section of the book (\nameref{sec:syntax_rawPointers}).

\myparagraph{Call unsafe functions}

This last ability works with both aspects of \code{unsafe}: you can only call functions marked \code{unsafe} from inside an unsafe block.

\blank

This ability is powerful and varied. Rust exposes some compiler intrinsics as unsafe functions (see \nameref{sec:nightly_intrinsics}), 
and some unsafe functions bypass safety checks, trading safety for speed.

\blank

I'll repeat again: even though you \emph{can do} arbitrary things in unsafe blocks and functions doesn't mean you should. The compiler 
will act as though you're upholding its invariants, so be careful!
