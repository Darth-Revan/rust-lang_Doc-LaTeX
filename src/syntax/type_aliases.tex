The \code{type} keyword lets you declare an alias of another type:

\begin{rustc}
type Name = String;
\end{rustc}

You can then use this type as if it were a real type:

\begin{rustc}
type Name = String;

let x: Name = "Hello".to_string();
\end{rustc}

Note, however, that this is an \emph{alias}, not a new type entirely. In other words, because Rust is strongly typed, you'd expect 
a comparison between two different types to fail:

\begin{rustc}
let x: i32 = 5;
let y: i64 = 5;

if x == y {
   // ...
}
\end{rustc}

this gives

\begin{verbatim}
error: mismatched types:
 expected `i32`,
    found `i64`
(expected i32,
    found i64) [E0308]
     if x == y {
             ^
\end{verbatim}

But, if we had an alias:

\begin{rustc}
type Num = i32;

let x: i32 = 5;
let y: Num = 5;

if x == y {
   // ...
}
\end{rustc}

This compiles without error. Values of a \code{Num} type are the same as a value of type \itt, in every way. You can use 
tuple struct (see \nameref{paragraph:tuple_structs}) to really get a new type.

\blank

You can also use type aliases with generics:

\begin{rustc}
use std::result;

enum ConcreteError {
    Foo,
    Bar,
}

type Result<T> = result::Result<T, ConcreteError>;
\end{rustc}

This creates a specialized version of the \code{Result} type, which always has a \code{ConcreteError} for the \code{E} part of 
\code{Result<T, E>}. This is commonly used in the standard library to create custom errors for each subsection. For example, 
\href{https://doc.rust-lang.org/std/io/type.Result.html}{io::Result}.
