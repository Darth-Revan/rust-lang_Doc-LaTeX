A trait is a language feature that tells the Rust compiler about functionality a type must provide.

\blank

Recall the \code{impl} keyword, used to call a function with method syntax:

\begin{rustc}
struct Circle {
    x: f64,
    y: f64,
    radius: f64,
}

impl Circle {
    fn area(&self) -> f64 {
        std::f64::consts::PI * (self.radius * self.radius)
    }
}
\end{rustc}

Traits are similar, except that we first define a trait with a method signature, then implement the trait for a type. In this example, we 
implement the trait \code{HasArea} for \code{Circle}:

\begin{rustc}
struct Circle {
    x: f64,
    y: f64,
    radius: f64,
}

trait HasArea {
    fn area(&self) -> f64;
}

impl HasArea for Circle {
    fn area(&self) -> f64 {
        std::f64::consts::PI * (self.radius * self.radius)
    }
}
\end{rustc}

As you can see, the \code{trait} block looks very similar to the \code{impl} block, but we don't define a body, only a type signature. When 
we \code{impl} a trait, we use \code{impl Trait for Item}, rather than only \code{impl Item}.

\subsubsection*{Trait bounds on generic functions}

Traits are useful because they allow a type to make certain promises about its behavior. Generic functions can exploit this to constrain, 
or \nameref{sec:gloss_bounds}, the types they accept. Consider this function, which does not compile:

\begin{rustc}
fn print_area<T>(shape: T) {
    println!("This shape has an area of {}", shape.area());
}
\end{rustc}

Rust complains:

\begin{verbatim}
error: no method named `area` found for type `T` in the current scope
\end{verbatim}

Because \code{T} can be any type, we can't be sure that it implements the \code{area} method. But we can add a trait bound to our generic 
\code{T}, ensuring that it does:

\begin{rustc}
fn print_area<T: HasArea>(shape: T) {
    println!("This shape has an area of {}", shape.area());
}
\end{rustc}

The syntax \code{<T: HasArea>} means \enquote{any type that implements the \code{HasArea} trait.} Because traits define function type signatures, 
we can be sure that any type which implements \code{HasArea} will have an \code{.area()} method.

\blank

Here's an extended example of how this works:

\begin{rustc}
trait HasArea {
    fn area(&self) -> f64;
}

struct Circle {
    x: f64,
    y: f64,
    radius: f64,
}

impl HasArea for Circle {
    fn area(&self) -> f64 {
        std::f64::consts::PI * (self.radius * self.radius)
    }
}

struct Square {
    x: f64,
    y: f64,
    side: f64,
}

impl HasArea for Square {
    fn area(&self) -> f64 {
        self.side * self.side
    }
}

fn print_area<T: HasArea>(shape: T) {
    println!("This shape has an area of {}", shape.area());
}

fn main() {
    let c = Circle {
        x: 0.0f64,
        y: 0.0f64,
        radius: 1.0f64,
    };

    let s = Square {
        x: 0.0f64,
        y: 0.0f64,
        side: 1.0f64,
    };

    print_area(c);
    print_area(s);
}
\end{rustc}

This program outputs:

\begin{verbatim}
This shape has an area of 3.141593
This shape has an area of 1
\end{verbatim}

As you can see, \code{print\_area} is now generic, but also ensures that we have passed in the correct types. If we pass in an incorrect type:

\begin{rustc}
print_area(5);
\end{rustc}

We get a compile-time error:

\begin{verbatim}
error: the trait `HasArea` is not implemented for the type `_` [E0277]
\end{verbatim}

\subsubsection*{Trait bounds on generic structs}

Your generic structs can also benefit from trait bounds. All you need to do is append the bound when you declare type parameters. Here is a 
new type \code{Rectangle<T>} and its operation \code{is\_square()}:

\begin{rustc}
struct Rectangle<T> {
    x: T,
    y: T,
    width: T,
    height: T,
}

impl<T: PartialEq> Rectangle<T> {
    fn is_square(&self) -> bool {
        self.width == self.height
    }
}

fn main() {
    let mut r = Rectangle {
        x: 0,
        y: 0,
        width: 47,
        height: 47,
    };

    assert!(r.is_square());

    r.height = 42;
    assert!(!r.is_square());
}
\end{rustc}

\code{is\_square()} needs to check that the sides are equal, so the sides must be of a type that implements the 
\href{https://doc.rust-lang.org/core/cmp/trait.PartialEq.html}{core::cmp::PartialEq} trait:

\begin{rustc}
impl<T: PartialEq> Rectangle<T> { ... }
\end{rustc}

Now, a rectangle can be defined in terms of any type that can be compared for equality.

\blank

% TODO make operators traits a hyperref
Here we defined a new struct \code{Rectangle} that accepts numbers of any precision—really, objects of pretty much any type—as long as they 
can be compared for equality. Could we do the same for our \code{HasArea} structs, \code{Square} and \code{Circle}? Yes, but they need
multiplication, and to work with that we need to know more about operator traits.

\subsubsection*{Rules for implementing traits}

So far, we've only added trait implementations to structs, but you can implement a trait for any type. So technically, we \emph{could} 
implement \code{HasArea} for \itt:

\begin{rustc}
trait HasArea {
    fn area(&self) -> f64;
}

impl HasArea for i32 {
    fn area(&self) -> f64 {
        println!("this is silly");

        *self as f64
    }
}

5.area();
\end{rustc}

It is considered poor style to implement methods on such primitive types, even though it is possible.

\blank

This may seem like the Wild West, but there are two restrictions around implementing traits that prevent this from getting out of hand. 
The first is that if the trait isn't defined in your scope, it doesn't apply. Here's an example: the standard library provides a 
\href{https://doc.rust-lang.org/std/io/trait.Write.html}{Write} trait which adds extra functionality to \code{File}s, for doing file I/O. 
By default, a \code{File} won't have its methods:

\begin{rustc}
let mut f = std::fs::File::open("foo.txt").expect("Couldn't open foo.txt");
let buf = b"whatever"; // byte string literal. buf: &[u8; 8]
let result = f.write(buf);
\end{rustc}

Here's the error:

\begin{verbatim}
error: type `std::fs::File` does not implement any method in scope named `write`
let result = f.write(buf);
               ^~~~~~~~~~
\end{verbatim}

We need to \code{use} the \code{Write} trait first:

\begin{rustc}
use std::io::Write;

let mut f = std::fs::File::open("foo.txt").expect("Couldn't open foo.txt");
let buf = b"whatever";
let result = f.write(buf);
\end{rustc}

This will compile without error.

\blank

This means that even if someone does something bad like add methods to \itt, it won't affect you, unless you \code{use} that trait.

\blank

There's one more restriction on implementing traits: either the trait, or the type you're writing the \code{impl} for, must be defined 
by you. So, we could implement the \code{HasArea} type for \itt, because \code{HasArea} is in our code. But if we tried to implement 
\code{ToString}, a trait provided by Rust, for \itt, we could not, because neither the trait nor the type are in our code.

\blank

% TODO make trait objects a hyperref
One last thing about traits: generic functions with a trait bound use 'monomorphization' (mono: one, morph: form), so they are 
statically dispatched. What's that mean? Check out the chapter on trait objects for more details.

\subsubsection*{Multiple trait bounds}

You've seen that you can bound a generic type parameter with a trait:

\begin{rustc}
fn foo<T: Clone>(x: T) {
    x.clone();
}
\end{rustc}

If you need more than one bound, you can use \code{+}:

\begin{rustc}
use std::fmt::Debug;

fn foo<T: Clone + Debug>(x: T) {
    x.clone();
    println!("{:?}", x);
}
\end{rustc}

\code{T} now needs to be both \code{Clone} as well as \code{Debug}.

\subsubsection*{Where clause}

Writing functions with only a few generic types and a small number of trait bounds isn't too bad, but as the number increases, the 
syntax gets increasingly awkward:

\begin{rustc}
use std::fmt::Debug;

fn foo<T: Clone, K: Clone + Debug>(x: T, y: K) {
    x.clone();
    y.clone();
    println!("{:?}", y);
}
\end{rustc}

The name of the function is on the far left, and the parameter list is on the far right. The bounds are getting in the way.

\blank

Rust has a solution, and it's called a '\code{where} clause':

\begin{rustc}
use std::fmt::Debug;

fn foo<T: Clone, K: Clone + Debug>(x: T, y: K) {
    x.clone();
    y.clone();
    println!("{:?}", y);
}

fn bar<T, K>(x: T, y: K) where T: Clone, K: Clone + Debug {
    x.clone();
    y.clone();
    println!("{:?}", y);
}

fn main() {
    foo("Hello", "world");
    bar("Hello", "world");
}
\end{rustc}

\code{foo()} uses the syntax we showed earlier, and \code{bar()} uses a \code{where} clause. All you need to do is leave off the bounds 
when defining your type parameters, and then add \code{where} after the parameter list. For longer lists, whitespace can be added:

\begin{rustc}
use std::fmt::Debug;

fn bar<T, K>(x: T, y: K)
    where T: Clone,
          K: Clone + Debug {

    x.clone();
    y.clone();
    println!("{:?}", y);
}
\end{rustc}

This flexibility can add clarity in complex situations.

\blank

\code{where} is also more powerful than the simpler syntax. For example:

\begin{rustc}
trait ConvertTo<Output> {
    fn convert(&self) -> Output;
}

impl ConvertTo<i64> for i32 {
    fn convert(&self) -> i64 { *self as i64 }
}

// can be called with T == i32
fn normal<T: ConvertTo<i64>>(x: &T) -> i64 {
    x.convert()
}

// can be called with T == i64
fn inverse<T>() -> T
        // this is using ConvertTo as if it were "ConvertTo<i64>"
        where i32: ConvertTo<T> {
    42.convert()
}
\end{rustc}

This shows off the additional feature of \code{where} clauses: they allow bounds on the left-hand side not only of type parameters \code{T}, 
but also of types (\itt\ in this case). In this example, \itt\ must implement \code{ConvertTo<T>}. Rather than defining what \itt\ is (since 
that's obvious), the \code{where} clause here constrains \code{T}.

\subsubsection*{Default methods}

A default method can be added to a trait definition if it is already known how a typical implementor will define a method. For example, 
\code{is\_invalid()} is defined as the opposite of \code{is\_valid()}:

\begin{rustc}
trait Foo {
    fn is_valid(&self) -> bool;

    fn is_invalid(&self) -> bool { !self.is_valid() }
}
\end{rustc}

Implementors of the \code{Foo} trait need to implement \code{is\_valid()} but not \code{is\_invalid()} due to the added default behavior. 
This default behavior can still be overridden as in:

\begin{rustc}
struct UseDefault;

impl Foo for UseDefault {
    fn is_valid(&self) -> bool {
        println!("Called UseDefault.is_valid.");
        true
    }
}

struct OverrideDefault;

impl Foo for OverrideDefault {
    fn is_valid(&self) -> bool {
        println!("Called OverrideDefault.is_valid.");
        true
    }

    fn is_invalid(&self) -> bool {
        println!("Called OverrideDefault.is_invalid!");
        true // overrides the expected value of is_invalid()
    }
}

let default = UseDefault;
assert!(!default.is_invalid()); // prints "Called UseDefault.is_valid."

let over = OverrideDefault;
assert!(over.is_invalid()); // prints "Called OverrideDefault.is_invalid!"
\end{rustc}

\subsubsection*{Inheritance}

Sometimes, implementing a trait requires implementing another trait:

\begin{rustc}
trait Foo {
    fn foo(&self);
}

trait FooBar : Foo {
    fn foobar(&self);
}
\end{rustc}

Implementors of \code{FooBar} must also implement \code{Foo}, like this:

\begin{rustc}
struct Baz;

impl Foo for Baz {
    fn foo(&self) { println!("foo"); }
}

impl FooBar for Baz {
    fn foobar(&self) { println!("foobar"); }
}
\end{rustc}

If we forget to implement \code{Foo}, Rust will tell us:

\begin{verbatim}
error: the trait `main::Foo` is not implemented for the type `main::Baz` [E0277]
\end{verbatim}

\subsubsection*{Deriving}

% TODO make atttribute a hyperref
Implementing traits like \code{Debug} and \code{Default} repeatedly can become quite tedious. For that reason, Rust provides an attribute 
that allows you to let Rust automatically implement traits for you:

\begin{rustc}
#[derive(Debug)]
struct Foo;

fn main() {
    println!("{:?}", Foo);
}
\end{rustc}

However, deriving is limited to a certain set of traits:

\begin{itemize}
  \item{\href{https://doc.rust-lang.org/core/clone/trait.Clone.html}{Clone}}
  \item{\href{https://doc.rust-lang.org/core/marker/trait.Copy.html}{Copy}}
  \item{\href{https://doc.rust-lang.org/core/fmt/trait.Debug.html}{Debug}}
  \item{\href{https://doc.rust-lang.org/core/default/trait.Default.html}{Default}}
  \item{\href{https://doc.rust-lang.org/core/cmp/trait.Eq.html}{Eq}}
  \item{\href{https://doc.rust-lang.org/core/hash/trait.Hash.html}{Hash}}
  \item{\href{https://doc.rust-lang.org/core/cmp/trait.Ord.html}{Ord}}
  \item{\href{https://doc.rust-lang.org/core/cmp/trait.PartialEq.html}{PartialEq}}
  \item{\href{https://doc.rust-lang.org/core/cmp/trait.PartialOrd.html}{PartialOrd}}
\end{itemize}
