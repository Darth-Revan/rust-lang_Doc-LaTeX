\struct s are a way of creating more complex data types. For example, if we were doing calculations involving coordinates in 
2D space, we would need both an \x\ and a \y\ value:

\begin{rustc}
let origin_x = 0;
let origin_y = 0;
\end{rustc}

A \struct\ lets us combine these two into a single, unified datatype with \x\ and \y\ as field labels:

\begin{rustc}
struct Point {
    x: i32,
    y: i32,
}

fn main() {
    let origin = Point { x: 0, y: 0 }; // origin: Point

    println!("The origin is at ({}, {})", origin.x, origin.y);
}
\end{rustc}

There's a lot going on here, so let's break it down. We declare a \struct\ with the \struct\ keyword, and then with a name. By 
convention, \struct s begin with a capital letter and are camel cased: \code{PointInSpace}, not \code{Point\_In\_Space}.

\blank

We can create an instance of our \struct\ via \keylet, as usual, but we use a \code{key: value} style syntax to set each field. The 
order doesn't need to be the same as in the original declaration.

\blank

Finally, because fields have names, we can access them through dot notation: \code{origin.x}.

\blank

The values in \struct s are immutable by default, like other bindings in Rust. Use \mut\ to make them mutable:

\begin{rustc}
struct Point {
    x: i32,
    y: i32,
}

fn main() {
    let mut point = Point { x: 0, y: 0 };

    point.x = 5;

    println!("The point is at ({}, {})", point.x, point.y);
}
\end{rustc}

This will print \code{The point is at (5, 0)}.

\blank

Rust does not support field mutability at the language level, so you cannot write something like this:

\begin{rustc}
struct Point {
    mut x: i32,
    y: i32,
}
\end{rustc}

Mutability is a property of the binding, not of the structure itself. If you're used to field-level mutability, this may seem 
strange at first, but it significantly simplifies things. It even lets you make things mutable on a temporary basis:

\begin{rustc}
struct Point {
    x: i32,
    y: i32,
}

fn main() {
    let mut point = Point { x: 0, y: 0 };

    point.x = 5;

    let point = point; // now immutable

    point.y = 6; // this causes an error
}
\end{rustc}

Your structure can still contain \code{\&mut} pointers, which will let you do some kinds of mutation:

\begin{rustc}
struct Point {
    x: i32,
    y: i32,
}

struct PointRef<'a> {
    x: &'a mut i32,
    y: &'a mut i32,
}

fn main() {
    let mut point = Point { x: 0, y: 0 };

    {
        let r = PointRef { x: &mut point.x, y: &mut point.y };

        *r.x = 5;
        *r.y = 6;
    }

    assert_eq!(5, point.x);
    assert_eq!(6, point.y);
}
\end{rustc}

\subsection*{Update syntax}

A \struct\ can include \code{..} to indicate that you want to use a copy of some other \struct\ for some of the values. For example:

\begin{rustc}
struct Point3d {
    x: i32,
    y: i32,
    z: i32,
}

let mut point = Point3d { x: 0, y: 0, z: 0 };
point = Point3d { y: 1, .. point };
\end{rustc}

This gives \code{point} a new \y, but keeps the old \x\ and \z\ values. It doesn't have to be the same \struct\ either, you can use 
this syntax when making new ones, and it will copy the values you don't specify:

\begin{rustc}
let origin = Point3d { x: 0, y: 0, z: 0 };
let point = Point3d { z: 1, x: 2, .. origin };
\end{rustc}

\subsection*{Tuple structs}
\label{paragraph:tuple_structs}

Rust has another data type that's like a hybrid between a tuple and a \struct, called a 'tuple struct'. Tuple structs have a name, 
but their fields don't. They are declared with the \struct\ keyword, and then with a name followed by a tuple:

\begin{rustc}
struct Color(i32, i32, i32);
struct Point(i32, i32, i32);

let black = Color(0, 0, 0);
let origin = Point(0, 0, 0);
\end{rustc}

Here, \code{black} and \code{origin} are not equal, even though they contain the same values.

\blank

It is almost always better to use a \struct\ than a tuple struct. We would write \code{Color} and \code{Point} like this instead:

\begin{rustc}
struct Color {
    red: i32,
    blue: i32,
    green: i32,
}

struct Point {
    x: i32,
    y: i32,
    z: i32,
}
\end{rustc}

Good names are important, and while values in a tuple struct can be referenced with dot notation as well, a \struct\ gives us 
actual names, rather than positions.

\blank

There is one case when a tuple struct is very useful, though, and that is when it has only one element. We call this the 'newtype' 
pattern, because it allows you to create a new type that is distinct from its contained value and also expresses its own semantic 
meaning:

\begin{rustc}
struct Inches(i32);

let length = Inches(10);

let Inches(integer_length) = length;
println!("length is {} inches", integer_length);
\end{rustc}

As you can see here, you can extract the inner integer type through a destructuring \keylet, as with regular tuples. In this case, the 
\code{let Inches(integer\_length)} assigns \code{10} to \code{integer\_length}.

\subsection*{Unit-like structs}

You can define a \struct\ with no members at all:

\begin{rustc}
struct Electron;

let x = Electron;
\end{rustc}

Such a \struct\ is called 'unit-like' because it resembles the empty tuple, \code{()}, sometimes called 'unit'. Like a tuple struct, 
it defines a new type.

\blank

This is rarely useful on its own (although sometimes it can serve as a marker type), but in combination with other features, it can 
become useful. For instance, a library may ask you to create a structure that implements a certain trait to handle events 
(see \nameref{sec:syntax_traits}). If you don't have any data you need to store in the structure, you can create a unit-like \struct.
