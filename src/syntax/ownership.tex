This guide is one of three presenting Rust's ownership system. This is one of Rust's most unique and compelling features, with 
which Rust developers should become quite acquainted. Ownership is how Rust achieves its largest goal, memory safety. There are 
a few distinct concepts, each with its own chapter:

\begin{itemize}
  \item{ownership, which you're reading now}
  \item{\nameref{sec:syntax_referencesBorrowing}, and their associated feature 'references'}
  \item{\nameref{sec:syntax_lifetimes}, an advanced concept of borrowing}
\end{itemize}

These three chapters are related, and in order. You'll need all three to fully understand the ownership system.

\subsection*{Meta}

Before we get to the details, two important notes about the ownership system.

\blank

Rust has a focus on safety and speed. It accomplishes these goals through many 'zero-cost abstractions', which means that in 
Rust, abstractions cost as little as possible in order to make them work. The ownership system is a prime example of a zero-cost
abstraction. All of the analysis we'll talk about in this guide is \emph{done at compile time}. You do not pay any run-time cost 
for any of these features.

\blank

However, this system does have a certain cost: learning curve. Many new users to Rust experience something we like to call 
'fighting with the borrow checker', where the Rust compiler refuses to compile a program that the author thinks is valid. 
This often happens because the programmer's mental model of how ownership should work doesn't match the actual rules that Rust 
implements. You probably will experience similar things at first. There is good news, however: more experienced Rust developers 
report that once they work with the rules of the ownership system for a period of time, they fight the borrow checker less and less.

\blank

With that in mind, let's learn about ownership.

\subsection*{Ownership}

\nameref{sec:syntax_variableBindings} have a property in Rust: they 'have ownership' of what they're bound to. This means that when 
a binding goes out of scope, Rust will free the bound resources. For example:

\begin{rustc}
fn foo() {
    let v = vec![1, 2, 3];
}
\end{rustc}

When \code{v} comes into scope, a new [vector] is created, and it allocates space on the heap for each of its elements 
(see \nameref{sec:effective_stackAndHeap}). When \code{v} goes out of scope at the end of \code{foo()}, Rust will clean up 
everything related to the vector, even the heap-allocated memory. This happens deterministically, at the end of the scope.

\blank

We'll cover vectors in detail later in this chapter (see \nameref{sec:syntax_vectors}); we only use them here as an example of a 
type that allocates space on the heap at runtime. They behave like arrays, except their size may change by \code{push()}ing more 
elements onto them.

\blank

Vectors have a generic type \code{Vec<T>} , so in this example \code{v} will have type 
\code{Vec<i32>}. We'll cover generics in detail later in this chapter (see \nameref{sec:syntax_generics}).

\subsection*{Move semantics}

There's some more subtlety here, though: Rust ensures that there is \emph{exactly one} binding to any given resource. For 
example, if we have a vector, we can assign it to another binding:

\begin{rustc}
let v = vec![1, 2, 3];

let v2 = v;
\end{rustc}

But, if we try to use \code{v} afterwards, we get an error:

\begin{rustc}
let v = vec![1, 2, 3];

let v2 = v;

println!("v[0] is: {}", v[0]);
\end{rustc}

It looks like this:

\begin{verbatim}
error: use of moved value: `v`
println!("v[0] is: {}", v[0]);
                        ^
\end{verbatim}

A similar thing happens if we define a function which takes ownership, and try to use something after we've passed it as an 
argument:

\begin{rustc}
fn take(v: Vec<i32>) {
    // what happens here isn't important.
}

let v = vec![1, 2, 3];

take(v);

println!("v[0] is: {}", v[0]);
\end{rustc}

Same error: 'use of moved value'. When we transfer ownership to something else, we say that we've 'moved' the thing we refer to. 
You don't need some sort of special annotation here, it's the default thing that Rust does.

\myparagraph{The details}

The reason that we cannot use a binding after we've moved it is subtle, but important. When we write code like this:

\begin{rustc}
let v = vec![1, 2, 3];

let v2 = v;
\end{rustc}

The first line allocates memory for the vector object, \code{v}, and for the data it contains. The vector object is stored on the 
stack and contains a pointer to the content (\code{[1, 2, 3]}) stored on the heap. When we move \code{v} to \code{v2}, it creates 
a copy of that pointer, for \code{v2}. Which means that there would be two pointers to the content of the vector on the heap. It 
would violate Rust's safety guarantees by introducing a data race. Therefore, Rust forbids using \code{v} after we've done the move.

\blank

It's also important to note that optimizations may remove the actual copy of the bytes on the stack, depending on circumstances. 
So it may not be as inefficient as it initially seems.

\myparagraph{\code{Copy} types}

We've established that when ownership is transferred to another binding, you cannot use the original binding. However, there's a 
trait that changes this behavior, and it's called \code{Copy}. We haven't discussed traits yet, but for now, you can think of them 
as an annotation to a particular type that adds extra behavior (see \nameref{sec:syntax_generics}). For example:

\begin{rustc}
let v = 1;

let v2 = v;

println!("v is: {}", v);
\end{rustc}

In this case, \code{v} is an \itt, which implements the \code{Copy} trait. This means that, just like a move, when we assign 
\code{v} to \code{v2}, a copy of the data is made. But, unlike a move, we can still use \code{v} afterward. This is because an 
\itt\ has no pointers to data somewhere else, copying it is a full copy.

\blank

All primitive types implement the \code{Copy} trait and their ownership is therefore not moved like one would assume, following the
'ownership rules'. To give an example, the two following snippets of code only compile because the \itt\ and \code{bool} types
implement the \code{Copy} trait.

\begin{rustc}
fn main() {
    let a = 5;

    let _y = double(a);
    println!("{}", a);
}

fn double(x: i32) -> i32 {
    x * 2
}


fn main() {
    let a = true;

    let _y = change_truth(a);
    println!("{}", a);
}

fn change_truth(x: bool) -> bool {
    !x
}
\end{rustc}

If we had used types that do not implement the \code{Copy} trait, we would have gotten a compile error because we tried to use a 
moved value.

\begin{verbatim}
error: use of moved value: `a`
println!("{}", a);
               ^
\end{verbatim}

We will discuss how to make your own types \code{Copy} in the traits section (see \nameref{sec:syntax_traits}).

\subsection*{More than ownership}

Of course, if we had to hand ownership back with every function we wrote:

\begin{rustc}
fn foo(v: Vec<i32>) -> Vec<i32> {
    // do stuff with v

    // hand back ownership
    v
}
\end{rustc}

This would get very tedious. It gets worse the more things we want to take ownership of:

\begin{rustc}
fn foo(v1: Vec<i32>, v2: Vec<i32>) -> (Vec<i32>, Vec<i32>, i32) {
    // do stuff with v1 and v2

    // hand back ownership, and the result of our function
    (v1, v2, 42)
}

let v1 = vec![1, 2, 3];
let v2 = vec![1, 2, 3];

let (v1, v2, answer) = foo(v1, v2);
\end{rustc}

Ugh! The return type, return line, and calling the function gets way more complicated.

\blank

Luckily, Rust offers a feature, borrowing, which helps us solve this problem. It's the topic of the next section!
