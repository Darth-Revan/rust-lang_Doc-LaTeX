Associated types are a powerful part of Rust's type system. They're related to the idea of a 'type family', in other words, grouping 
multiple types together. That description is a bit abstract, so let's dive right into an example. If you want to write a \code{Graph} 
trait, you have two types to be generic over: the node type and the edge type. So you might write a trait, \code{Graph<N, E>}, that
looks like this:

\begin{rustc}
trait Graph<N, E> {
    fn has_edge(&self, &N, &N) -> bool;
    fn edges(&self, &N) -> Vec<E>;
    // etc
}
\end{rustc}

While this sort of works, it ends up being awkward. For example, any function that wants to take a \code{Graph} as a parameter now also 
needs to be generic over the \code{N}ode and \code{E}dge types too:

\begin{rustc}
fn distance<N, E, G: Graph<N, E>>(graph: &G, start: &N, end: &N) -> u32 { ... }
\end{rustc}

Our distance calculation works regardless of our \code{Edge} type, so the \code{E} stuff in this signature is a distraction.

\blank

What we really want to say is that a certain \code{E}dge and \code{N}ode type come together to form each kind of \code{Graph}. We can 
do that with associated types:

\begin{rustc}
trait Graph {
    type N;
    type E;

    fn has_edge(&self, &Self::N, &Self::N) -> bool;
    fn edges(&self, &Self::N) -> Vec<Self::E>;
    // etc
}
\end{rustc}

Now, our clients can be abstract over a given \code{Graph}:

\begin{rustc}
fn distance<G: Graph>(graph: &G, start: &G::N, end: &G::N) -> u32 { ... }
\end{rustc}

No need to deal with the \code{E}dge type here!

\blank

Let's go over all this in more detail.

\subsection*{Defining associated types}

Let's build that \code{Graph} trait. Here's the definition:

\begin{rustc}
trait Graph {
    type N;
    type E;

    fn has_edge(&self, &Self::N, &Self::N) -> bool;
    fn edges(&self, &Self::N) -> Vec<Self::E>;
}
\end{rustc}

Simple enough. Associated types use the \code{type} keyword, and go inside the body of the trait, with the functions.

\blank

These \code{type} declarations can have all the same thing as functions do. For example, if we wanted our \code{N} type to implement 
\code{Display}, so we can print the nodes out, we could do this:

\begin{rustc}
use std::fmt;

trait Graph {
    type N: fmt::Display;
    type E;

    fn has_edge(&self, &Self::N, &Self::N) -> bool;
    fn edges(&self, &Self::N) -> Vec<Self::E>;
}
\end{rustc}

\subsection*{Implementing associated types}

Just like any trait, traits that use associated types use the \code{impl} keyword to provide implementations. Here's a simple 
implementation of Graph:

\begin{rustc}
struct Node;

struct Edge;

struct MyGraph;

impl Graph for MyGraph {
    type N = Node;
    type E = Edge;

    fn has_edge(&self, n1: &Node, n2: &Node) -> bool {
        true
    }

    fn edges(&self, n: &Node) -> Vec<Edge> {
        Vec::new()
    }
}
\end{rustc}

This silly implementation always returns \code{true} and an empty \code{Vec<Edge>}, but it gives you an idea of how to implement 
this kind of thing. We first need three \struct s, one for the graph, one for the node, and one for the edge. If it made more sense 
to use a different type, that would work as well, we're going to use \struct s for all three here.

\blank

Next is the \code{impl} line, which is an implementation like any other trait.

\blank

From here, we use \code{=} to define our associated types. The name the trait uses goes on the left of the \code{=}, and the concrete 
type we're \code{impl}ementing this for goes on the right. Finally, we use the concrete types in our function declarations.

\subsection*{Trait objects with associated types}

There's one more bit of syntax we should talk about: trait objects. If you try to create a trait object from an associated type, like this:

\begin{rustc}
let graph = MyGraph;
let obj = Box::new(graph) as Box<Graph>;
\end{rustc}

You'll get two errors:

\begin{verbatim}
error: the value of the associated type `E` (from the trait `main::Graph`) must
be specified [E0191]
let obj = Box::new(graph) as Box<Graph>;
          ^~~~~~~~~~~~~~~~~~~~~~~~~~~~~
24:44 error: the value of the associated type `N` (from the trait
`main::Graph`) must be specified [E0191]
let obj = Box::new(graph) as Box<Graph>;
          ^~~~~~~~~~~~~~~~~~~~~~~~~~~~~
\end{verbatim}

We can't create a trait object like this, because we don't know the associated types. Instead, we can write this:

\begin{rustc}
let graph = MyGraph;
let obj = Box::new(graph) as Box<Graph<N=Node, E=Edge>>;
\end{rustc}

The \code{N=Node} syntax allows us to provide a concrete type, \code{Node}, for the \code{N} type parameter. Same with \code{E=Edge}. 
If we didn't provide this constraint, we couldn't be sure which \code{impl} to match this trait object to.
