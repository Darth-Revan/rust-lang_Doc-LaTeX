The Rust language has a number of types that are considered 'primitive'. This means that they're built-in to the language. Rust is
structured in such a way that the standard library also provides a number of useful types built on top of these ones, as well, but 
these are the most primitive.

\subsection*{Booleans}

Rust has a built in boolean type, named \code{bool}. It has two values, \code{true} and \code{false}:

\begin{rustc}
let x = true;

let y: bool = false;
\end{rustc}

A common use of booleans is in if conditionals (see \nameref{sec:syntax_if}).

\blank

You can find more documentation for \code{bool}s \href{https://doc.rust-lang.org/std/primitive.bool.html}{in the standard library
documentation}.

\subsection*{char}

The \varchar\ type represents a single Unicode scalar value. You can create \varchar s with a single tick: (\code{'})

\begin{rustc}
let x = 'x';
let two_hearts = '💕';
\end{rustc}

Unlike some other languages, this means that Rust's \varchar\ is not a single byte, but four.

\blank

You can find more documentation for \varchar s \href{https://doc.rust-lang.org/std/primitive.char.html}{in the standard library
documentation}.

\subsection*{Numeric types}

Rust has a variety of numeric types in a few categories: signed and unsigned, fixed and variable, floating-point and integer.

\blank

These types consist of two parts: the category, and the size. For example, \code{u16} is an unsigned type with sixteen bits of size. 
More bits lets you have bigger numbers.

\blank

If a number literal has nothing to cause its type to be inferred, it defaults:

\begin{rustc}
let x = 42; // x has type i32

let y = 1.0; // y has type f64
\end{rustc}

Here's a list of the different numeric types, with links to their documentation in the standard library:

\begin{itemize}
  \item{\href{https://doc.rust-lang.org/std/primitive.i8.html}{i8}}
  \item{\href{https://doc.rust-lang.org/std/primitive.i16.html}{i16}}
  \item{\href{https://doc.rust-lang.org/std/primitive.i32.html}{i32}}
  \item{\href{https://doc.rust-lang.org/std/primitive.i64.html}{i64}}
  \item{\href{https://doc.rust-lang.org/std/primitive.u8.html}{u8}}
  \item{\href{https://doc.rust-lang.org/std/primitive.u16.html}{u16}}
  \item{\href{https://doc.rust-lang.org/std/primitive.u32.html}{u32}}
  \item{\href{https://doc.rust-lang.org/std/primitive.u64.html}{u64}}
  \item{\href{https://doc.rust-lang.org/std/primitive.isize.html}{isize}}
  \item{\href{https://doc.rust-lang.org/std/primitive.usize.html}{usize}}
  \item{\href{https://doc.rust-lang.org/std/primitive.f32.html}{f32}}
  \item{\href{https://doc.rust-lang.org/std/primitive.f64.html}{f64}}
\end{itemize}

Let's go over them by category:

\myparagraph{Signed and Unsigned}

Integer types come in two varieties: signed and unsigned. To understand the difference, let's consider a number with four bits of size. 
A signed, four-bit number would let you store numbers from \code{-8} to \code{+7}. Signed numbers use “two's complement representation”. 
An unsigned four bit number, since it does not need to store negatives, can store values from \code{0} to \code{+15}.

\blank

Unsigned types use a \code{u} for their category, and signed types use \code{i}. The \code{i} is for 'integer'. So \code{u8} is an 
eight-bit unsigned number, and \code{i8} is an eight-bit signed number.

\myparagraph{Fixed size types}

Fixed size types have a specific number of bits in their representation. Valid bit sizes are \code{8}, \code{16}, \code{3}2, and \code{64}.
So, \code{u32} is an unsigned, 32-bit integer, and \code{i64} is a signed, 64-bit integer.

\myparagraph{Variable sized types}

Rust also provides types whose size depends on the size of a pointer of the underlying machine. These types have 'size' as the category,
and come in signed and unsigned varieties. This makes for two types: \code{isize} and \code{usize}.

\myparagraph{Floating-point types}

Rust also has two floating point types: \code{f32} and \code{f64}. These correspond to IEEE-754 single and double precision numbers.

\subsection*{Arrays}

Like many programming languages, Rust has list types to represent a sequence of things. The most basic is the \emph{array}, a 
fixed-size list of elements of the same type. By default, arrays are immutable.

\begin{rustc}
let a = [1, 2, 3]; // a: [i32; 3]
let mut m = [1, 2, 3]; // m: [i32; 3]
\end{rustc}

Arrays have type \code{[T; N]}. We'll talk about this \code{T} notation in the generics section (see \nameref{sec:syntax_generics}). 
The \code{N} is a compile-time constant, for the length of the array.

\blank

There's a shorthand for initializing each element of an array to the same value. In this example, each element of a will be initialized 
to \code{0}:

\begin{rustc}
let a = [0; 20]; // a: [i32; 20]
\end{rustc}

You can get the number of elements in an array \code{a} with \code{a.len()}:

\begin{rustc}
let a = [1, 2, 3];

println!("a has {} elements", a.len());
\end{rustc}

You can access a particular element of an array with \emph{subscript notation}:

\begin{rustc}
let names = ["Graydon", "Brian", "Niko"]; // names: [&str; 3]

println!("The second name is: {}", names[1]);
\end{rustc}

Subscripts start at zero, like in most programming languages, so the first name is \code{names[0]} and the second name is \code{names[1]}.
The above example prints \code{The second name is: Brian}. If you try to use a subscript that is not in the array, you will get an error:
array access is bounds-checked at run-time. Such errant access is the source of many bugs in other systems programming languages.

\blank

You can find more documentation for \code{array}s \href{https://doc.rust-lang.org/std/primitive.array.html}{in the standard library
documentation}.

\subsection*{Slices}

A 'slice' is a reference to (or “view” into) another data structure. They are useful for allowing safe, efficient access to a portion 
of an array without copying. For example, you might want to reference only one line of a file read into memory. By nature, a slice is 
not created directly, but from an existing variable binding. Slices have a defined length, can be mutable or immutable.

\myparagraph{Slicing syntax}

You can use a combo of \code{\&} and \code{[]} to create a slice from various things. The \code{\&} indicates that slices are similar 
to references (see \nameref{sec:syntax_referencesBorrowing}), which we will cover in detail later in this section. The \code{[]}s, with 
a range, let you define the length of the slice:

\begin{rustc}
let a = [0, 1, 2, 3, 4];
let complete = &a[..]; // A slice containing all of the elements in a
let middle = &a[1..4]; // A slice of a: only the elements 1, 2, and 3
\end{rustc}

Slices have type \code{\&[T]}. We'll talk about that \code{T} when we cover generics (see \nameref{sec:syntax_generics}).

\blank

You can find more documentation for slices \href{https://doc.rust-lang.org/std/primitive.slice.html}{in the standard library documentation}.

\subsection*{str}

Rust's \code{str} type is the most primitive string type. As an unsized type (see \nameref{sec:syntax_unsizedTypes}), it's not very 
useful by itself, but becomes useful when placed behind a reference, like \code{\&str}. We'll elaborate further when we cover 
\nameref{sec:syntax_strings} and references (see \nameref{sec:syntax_referencesBorrowing}).

\blank

You can find more documentation for \code{str} \href{https://doc.rust-lang.org/std/primitive.str.html}{in the standard library
documentation}.

\subsection*{Tuples}

A tuple is an ordered list of fixed size. Like this:

\begin{rustc}
let x = (1, "hello");
\end{rustc}

The parentheses and commas form this two-length tuple. Here's the same code, but with the type annotated:

\begin{rustc}
let x: (i32, &str) = (1, "hello");
\end{rustc}

As you can see, the type of a tuple looks like the tuple, but with each position having a type name rather than the value. Careful 
readers will also note that tuples are heterogeneous: we have an \itt\ and a \code{\&str} in this tuple. In systems programming
languages, strings are a bit more complex than in other languages. For now, read \code{\&str} as a \emph{string slice}, and we'll 
learn more soon.

\blank

You can assign one tuple into another, if they have the same contained types and \nameref{sec:gloss_arity}. Tuples have the same arity 
when they have the same length.

\begin{rustc}
let mut x = (1, 2); // x: (i32, i32)
let y = (2, 3); // y: (i32, i32)

x = y;
\end{rustc}

You can access the fields in a tuple through a \emph{destructuring let}. Here's an example:

\begin{rustc}
let (x, y, z) = (1, 2, 3);

println!("x is {}", x);
\end{rustc}

Remember before when I said the left-hand side of a \keylet\ statement was more powerful than assigning a binding? Here we are. 
We can put a pattern on the left-hand side of the \keylet, and if it matches up to the right-hand side, we can assign multiple 
bindings at once. In this case, \keylet\ “destructures” or “breaks up” the tuple, and assigns the bits to three bindings.

\blank

This pattern is very powerful, and we'll see it repeated more later.

\blank

You can disambiguate a single-element tuple from a value in parentheses with a comma:

\begin{rustc}
(0,); // single-element tuple
(0); // zero in parentheses
\end{rustc}

\myparagraph{Tuple Indexing}

You can also access fields of a tuple with indexing syntax:

\begin{rustc}
let tuple = (1, 2, 3);

let x = tuple.0;
let y = tuple.1;
let z = tuple.2;

println!("x is {}", x);
\end{rustc}

Like array indexing, it starts at zero, but unlike array indexing, it uses a \code{.}, rather than \code{[]}s.

\blank

You can find more documentation for tuples \href{https://doc.rust-lang.org/std/primitive.tuple.html}{in the standard library 
documentation}.

\subsection*{Functions}

Functions also have a type! They look like this:

\begin{rustc}
fn foo(x: i32) -> i32 { x }

let x: fn(i32) -> i32 = foo;
\end{rustc}

In this case, \x\ is a 'function pointer' to a function that takes an \itt\ and returns an \itt.
