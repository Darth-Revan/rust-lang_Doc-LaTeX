Patterns are quite common in Rust. We use them in \nameref{sec:syntax_variableBindings}, match statements (see \nameref{sec:syntax_match}),
and other places, too. Let's go on a whirlwind tour of all of the things patterns can do!

\blank

A quick refresher: you can match against literals directly, and \code{\_} acts as an 'any' case:

\begin{rustc}
let x = 1;

match x {
    1 => println!("one"),
    2 => println!("two"),
    3 => println!("three"),
    _ => println!("anything"),
}
\end{rustc}

This prints \code{one}.

There's one pitfall with patterns: like anything that introduces a new binding, they introduce shadowing. For example:

\begin{rustc}
let x = 1;
let c = 'c';

match c {
    x => println!("x: {} c: {}", x, c),
}

println!("x: {}", x)
\end{rustc}

This prints:

\begin{verbatim}
x: c c: c
x: 1
\end{verbatim}

In other words, \code{x =>} matches the pattern and introduces a new binding named \x. This new binding is in scope for the match arm 
and takes on the value of \code{c}. Notice that the value of \x\ outside the scope of the match has no bearing on the value of \x\ 
within it. Because we already have a binding named \x\, this new \x\ shadows it.

\subsection*{Multiple patterns}

You can match multiple patterns with \code{|}:

\begin{rustc}
let x = 1;

match x {
    1 | 2 => println!("one or two"),
    3 => println!("three"),
    _ => println!("anything"),
}
\end{rustc}

This prints \code{one or two}.

\subsection*{Destructuring}

If you have a compound data type, like a \struct\ (see \nameref{sec:syntax_structs}), you can destructure it inside of a pattern:

\begin{rustc}
struct Point {
    x: i32,
    y: i32,
}

let origin = Point { x: 0, y: 0 };

match origin {
    Point { x, y } => println!("({},{})", x, y),
}
\end{rustc}

We can use \code{:} to give a value a different name.

\begin{rustc}
struct Point {
    x: i32,
    y: i32,
}

let origin = Point { x: 0, y: 0 };

match origin {
    Point { x: x1, y: y1 } => println!("({},{})", x1, y1),
}
\end{rustc}

If we only care about some of the values, we don't have to give them all names:

\begin{rustc}
struct Point {
    x: i32,
    y: i32,
}

let origin = Point { x: 0, y: 0 };

match origin {
    Point { x, .. } => println!("x is {}", x),
}
\end{rustc}

This prints \code{x is 0}.

You can do this kind of match on any member, not only the first:

\begin{rustc}
struct Point {
    x: i32,
    y: i32,
}

let origin = Point { x: 0, y: 0 };

match origin {
    Point { y, .. } => println!("y is {}", y),
}
\end{rustc}

This prints \code{y is 0}.

\blank

This 'destructuring' behavior works on any compound data type, like tuples (see \nameref{sec:syntax_primitives}) or enums 
(see \nameref{sec:syntax_enums}).

\subsection*{Ignoring bindings}

You can use \code{\_} in a pattern to disregard the type and value. For example, here's a \match\ against a \code{Result<T, E>}:

\begin{rustc}
match some_value {
    Ok(value) => println!("got a value: {}", value),
    Err(_) => println!("an error occurred"),
}
\end{rustc}

In the first arm, we bind the value inside the \code{Ok} variant to value. But in the \code{Err} arm, we use \code{\_} to disregard 
the specific error, and print a general error message.

\blank

\code{\_} is valid in any pattern that creates a binding. This can be useful to ignore parts of a larger structure:

\begin{rustc}
fn coordinate() -> (i32, i32, i32) {
    // generate and return some sort of triple tuple
}

let (x, _, z) = coordinate();
\end{rustc}

Here, we bind the first and last element of the tuple to \x\ and \z, but ignore the middle element.

\blank

Similarly, you can use \code{..} in a pattern to disregard multiple values.

\begin{rustc}
enum OptionalTuple {
    Value(i32, i32, i32),
    Missing,
}

let x = OptionalTuple::Value(5, -2, 3);

match x {
    OptionalTuple::Value(..) => println!("Got a tuple!"),
    OptionalTuple::Missing => println!("No such luck."),
}
\end{rustc}

This prints \code{Got a tuple!}.

\subsection*{ref and ref mut}

If you want to get a reference (see \nameref{sec:syntax_referencesBorrowing}), use the \keyref\ keyword:

\begin{rustc}
let x = 5;

match x {
    ref r => println!("Got a reference to {}", r),
}
\end{rustc}

This prints \code{Got a reference to 5}.

\blank

Here, the \code{r} inside the \match\ has the type \code{\&i32}. In other words, the \keyref\ keyword creates a reference, for 
use in the pattern. If you need a mutable reference, \keyref\code{ mut} will work in the same way:

\begin{rustc}
let mut x = 5;

match x {
    ref mut mr => println!("Got a mutable reference to {}", mr),
}
\end{rustc}

\subsection*{Ranges}

You can match a range of values with \code{...}:

\begin{rustc}
let x = 1;

match x {
    1 ... 5 => println!("one through five"),
    _ => println!("anything"),
}
\end{rustc}

This prints \code{one through five}.

\blank

Ranges are mostly used with integers and \varchar s:

\begin{rustc}
let x = 'ä';

match x {
    'a' ... 'j' => println!("early letter"),
    'k' ... 'z' => println!("late letter"),
    _ => println!("something else"),
}
\end{rustc}

This prints \code{something else}.

\subsection*{Bindings}

You can bind values to names with \code{@}:

\begin{rustc}
let x = 1;

match x {
    e @ 1 ... 5 => println!("got a range element {}", e),
    _ => println!("anything"),
}
\end{rustc}

This prints \code{got a range element 1}. This is useful when you want to do a complicated match of part of a data structure:

\begin{rustc}
#[derive(Debug)]
struct Person {
    name: Option<String>,
}

let name = "Steve".to_string();
let mut x: Option<Person> = Some(Person { name: Some(name) });
match x {
    Some(Person { name: ref a @ Some(_), .. }) => println!("{:?}", a),
    _ => {}
}
\end{rustc}

This prints \code{Some("Steve")}: we've bound the inner \code{name} to \code{a}.

\blank

If you use \code{@} with \code{|}, you need to make sure the name is bound in each part of the pattern:

\begin{rustc}
let x = 5;

match x {
    e @ 1 ... 5 | e @ 8 ... 10 => println!("got a range element {}", e),
    _ => println!("anything"),
}
\end{rustc}

\subsection*{Guards}

You can introduce 'match guards' with \keyif:

\begin{rustc}
enum OptionalInt {
    Value(i32),
    Missing,
}

let x = OptionalInt::Value(5);

match x {
    OptionalInt::Value(i) if i > 5 => println!("Got an int bigger than five!"),
    OptionalInt::Value(..) => println!("Got an int!"),
    OptionalInt::Missing => println!("No such luck."),
}
\end{rustc}

This prints \code{Got an int!}.

\blank

If you're using \keyif\ with multiple patterns, the \keyif\ applies to both sides:

\begin{rustc}
let x = 4;
let y = false;

match x {
    4 | 5 if y => println!("yes"),
    _ => println!("no"),
}
\end{rustc}

This prints \code{no}, because the \keyif\ applies to the whole of \code{4 | 5}, and not to only the \code{5}. In other words, 
the precedence of \keyif\ behaves like this:

\begin{verbatim}
(4 | 5) if y => ...
\end{verbatim}

not this:

\begin{verbatim}
4 | (5 if y) => ...
\end{verbatim}

\subsection*{Mix and Match}

Whew! That's a lot of different ways to match things, and they can all be mixed and matched, depending on what you're doing:

\begin{rustc}
match x {
    Foo { x: Some(ref name), y: None } => ...
}
\end{rustc}

Patterns are very powerful. Make good use of them.
