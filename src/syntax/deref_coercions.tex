The standard library provides a special trait, \href{https://doc.rust-lang.org/std/ops/trait.Deref.html}{Deref}. It's normally used to 
overload \code{*}, the dereference operator:

\begin{rustc}
use std::ops::Deref;

struct DerefExample<T> {
    value: T,
}

impl<T> Deref for DerefExample<T> {
    type Target = T;

    fn deref(&self) -> &T {
        &self.value
    }
}

fn main() {
    let x = DerefExample { value: 'a' };
    assert_eq!('a', *x);
}
\end{rustc}

This is useful for writing custom pointer types. However, there's a language feature related to \code{Deref}: 'deref coercions'. Here's the 
rule: If you have a type \code{U}, and it implements \code{Deref<Target=T>}, values of \code{\&U} will automatically coerce to a \code{\&T}. 
Here's an example:

\begin{rustc}
fn foo(s: &str) {
    // borrow a string for a second
}

// String implements Deref<Target=str>
let owned = "Hello".to_string();

// therefore, this works:
foo(&owned);
\end{rustc}

Using an ampersand in front of a value takes a reference to it. So \code{owned} is a \String, \code{\&owned} is an \code{\&String}, and since 
\code{impl Deref<Target=str> for String}, \code{\&String} will deref to \code{\&str}, which \code{foo()} takes.

\blank

That's it. This rule is one of the only places in which Rust does an automatic conversion for you, but it adds a lot of flexibility. For 
example, the \code{Rc<T>} type implements \code{Deref<Target=T>}, so this works:

\begin{rustc}
use std::rc::Rc;

fn foo(s: &str) {
    // borrow a string for a second
}

// String implements Deref<Target=str>
let owned = "Hello".to_string();
let counted = Rc::new(owned);

// therefore, this works:
foo(&counted);
\end{rustc}

All we've done is wrap our \String\ in an \code{Rc<T>}. But we can now pass the \code{Rc<String>} around anywhere we'd have a \String. The 
signature of \code{foo} didn't change, but works just as well with either type. This example has two conversions: \code{Rc<String>} to 
\String\ and then \String\ to \code{\&str}. Rust will do this as many times as possible until the types match.

\blank

Another very common implementation provided by the standard library is:

\begin{rustc}
fn foo(s: &[i32]) {
    // borrow a slice for a second
}

// Vec<T> implements Deref<Target=[T]>
let owned = vec![1, 2, 3];

foo(&owned);
\end{rustc}

Vectors can \code{Deref} to a slice.

\subsection*{Deref and method calls}

\code{Deref} will also kick in when calling a method. Consider the following example.

\begin{rustc}
struct Foo;

impl Foo {
    fn foo(&self) { println!("Foo"); }
}

let f = &&Foo;

f.foo();
\end{rustc}

Even though \code{f} is a \code{\&\&Foo} and \code{foo} takes \code{\&self}, this works. That's because these things are the same:

\begin{rustc}
f.foo();
(&f).foo();
(&&f).foo();
(&&&&&&&&f).foo();
\end{rustc}

A value of type \code{\&\&\&\&\&\&\&\&\&\&\&\&\&\&\&\&Foo} can still have methods defined on \code{Foo} called, because the compiler 
will insert as many \code{*} operations as necessary to get it right. And since it's inserting \code{*}s, that uses \code{Deref}.
