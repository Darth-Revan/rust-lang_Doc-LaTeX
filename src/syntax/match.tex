Often, a simple \code{if/else} (see \nameref{sec:syntax_if}) isn't enough, because you have more than two possible options. Also, 
conditions can get quite complex. Rust has a keyword, \match, that allows you to replace complicated \code{if/else} groupings with 
something more powerful. Check it out:

\begin{rustc}
let x = 5;

match x {
    1 => println!("one"),
    2 => println!("two"),
    3 => println!("three"),
    4 => println!("four"),
    5 => println!("five"),
    _ => println!("something else"),
}
\end{rustc}

% TODO make sections a hyperref
\match\ takes an expression and then branches based on its value. Each 'arm' of the branch is of the form \code{val => expression}. When 
the value matches, that arm's expression will be evaluated. It's called \match\ because of the term 'pattern matching', which \match\ is 
an implementation of. There's a separate section on patterns that covers all the patterns that are possible here.

\blank

One of the many advantages of \match\ is it enforces 'exhaustiveness checking'. For example if we remove the last arm with the 
underscore \code{\_}, the compiler will give us an error:

\begin{verbatim}
error: non-exhaustive patterns: `_` not covered
\end{verbatim}

Rust is telling us that we forgot a value. The compiler infers from \x\ that it can have any positive 32bit value; for example 1 
to 2,147,483,647. The \code{\_} acts as a 'catch-all', and will catch all possible values that aren't specified in an arm of \match. 
As you can see with the previous example, we provide \match\ arms for integers 1-5, if \x\ is 6 or any other value, then it is caught 
by \code{\_}.

\blank

\match\ is also an expression, which means we can use it on the right-hand side of a \keylet\ binding or directly where an expression 
is used:

\begin{rustc}
let x = 5;

let number = match x {
    1 => "one",
    2 => "two",
    3 => "three",
    4 => "four",
    5 => "five",
    _ => "something else",
};
\end{rustc}

Sometimes it's a nice way of converting something from one type to another; in this example the integers are converted to \String.

\subsubsection*{Matching on enums}

Another important use of the \match\ keyword is to process the possible variants of an \enum:

\begin{rustc}
enum Message {
    Quit,
    ChangeColor(i32, i32, i32),
    Move { x: i32, y: i32 },
    Write(String),
}

fn quit() { /* ... */ }
fn change_color(r: i32, g: i32, b: i32) { /* ... */ }
fn move_cursor(x: i32, y: i32) { /* ... */ }

fn process_message(msg: Message) {
    match msg {
        Message::Quit => quit(),
        Message::ChangeColor(r, g, b) => change_color(r, g, b),
        Message::Move { x: x, y: y } => move_cursor(x, y),
        Message::Write(s) => println!("{}", s),
    };
}
\end{rustc}

Again, the Rust compiler checks exhaustiveness, so it demands that you have a match arm for every variant of the enum. If you leave 
one off, it will give you a compile-time error unless you use \code{\_} or provide all possible arms.

\blank

% TODO make if let a hyperref
Unlike the previous uses of \match, you can't use the normal \keyif\ statement to do this. You can use the \code{if let} (see) 
statement, which can be seen as an abbreviated form of \match.
