When a project starts getting large, it's considered good software engineering practice to split it up into a bunch of smaller 
pieces, and then fit them together. It is also important to have a well-defined interface, so that some of your functionality is 
private, and some is public. To facilitate these kinds of things, Rust has a module system.

\subsubsection*{Basic terminology: Crates and Modules}

Rust has two distinct terms that relate to the module system: 'crate' and 'module'. A crate is synonymous with a 'library' or 
'package' in other languages. Hence “Cargo” as the name of Rust's package management tool: you ship your crates to others with 
Cargo. Crates can produce an executable or a library, depending on the project.

\blank

Each crate has an implicit root module that contains the code for that crate. You can then define a tree of sub-modules under 
that root module. Modules allow you to partition your code within the crate itself.

\blank

As an example, let's make a phrases crate, which will give us various phrases in different languages. To keep things simple, we'll 
stick to 'greetings' and 'farewells' as two kinds of phrases, and use English and Japanese (日本語) as two languages for those phrases 
to be in. We'll use this module layout:

\begin{verbatim}
                                    +-----------+
                                +---| greetings |
                                |   +-----------+
                  +---------+   |
              +---| english |---+
              |   +---------+   |   +-----------+
              |                 +---| farewells |
+---------+   |                     +-----------+
| phrases |---+
+---------+   |                     +-----------+
              |                 +---| greetings |
              |   +----------+  |   +-----------+
              +---| japanese |--+
                  +----------+  |
                                |   +-----------+
                                +---| farewells |
                                    +-----------+
\end{verbatim}

In this example, \code{phrases} is the name of our crate. All of the rest are modules. You can see that they form a tree, branching out 
from the crate \emph{root}, which is the root of the tree: \code{phrases} itself.

\blank

Now that we have a plan, let's define these modules in code. To start, generate a new crate with Cargo:

\begin{verbatim}
$ cargo new phrases
$ cd phrases
\end{verbatim}

If you remember, this generates a simple project for us:

\begin{verbatim}
$ tree .
.
├── Cargo.toml
└── src
    └── lib.rs

1 directory, 2 files
\end{verbatim}

\code{src/lib.rs} is our crate root, corresponding to the \code{phrases} in our diagram above.

\subsubsection*{Defining Modules}

To define each of our modules, we use the \code{mod} keyword. Let's make our \code{src/lib.rs} look like this:

\begin{rustc}
mod english {
    mod greetings {
    }

    mod farewells {
    }
}

mod japanese {
    mod greetings {
    }

    mod farewells {
    }
}
\end{rustc}

After the \code{mod} keyword, you give the name of the module. Module names follow the conventions for other Rust identifiers: 
\code{lower\_snake\_case}. The contents of each module are within curly braces (\code{\{\}}).

\blank

Within a given mod, you can declare sub-\code{mod}s. We can refer to sub-modules with double-colon (\code{::}) notation: our four 
nested modules are \code{english::greetings}, \code{english::farewells}, \code{japanese::greetings}, and \code{japanese::farewells}. 
Because these sub-modules are namespaced under their parent module, the names don't conflict: \code{english::greetings} and 
\code{japanese::greetings} are distinct, even though their names are both \code{greetings}.

\blank

Because this crate does not have a \code{main()} function, and is called \code{lib.rs}, Cargo will build this crate as a library:

\begin{verbatim}
$ cargo build
   Compiling phrases v0.0.1 (file:///home/you/projects/phrases)
$ ls target/debug
build  deps  examples  libphrases-a7448e02a0468eaa.rlib  native
\end{verbatim}

\code{libphrases-hash.rlib} is the compiled crate. Before we see how to use this crate from another crate, let's break it up into 
multiple files.

\subsubsection*{Multiple file crates}

If each crate were just one file, these files would get very large. It's often easier to split up crates into multiple files, 
and Rust supports this in two ways.

\blank

Instead of declaring a module like this:

\begin{rustc}
mod english {
    // contents of our module go here
}
\end{rustc}

We can instead declare our module like this:

\begin{rustc}
mod english;
\end{rustc}

If we do that, Rust will expect to find either a \code{english.rs} file, or a \code{english/mod.rs} file with the contents of our module.

\blank

Note that in these files, you don't need to re-declare the module: that's already been done with the initial \code{mod} declaration.

\blank

Using these two techniques, we can break up our crate into two directories and seven files:

\begin{verbatim}
$ tree .
.
├── Cargo.lock
├── Cargo.toml
├── src
│   ├── english
│   │   ├── farewells.rs
│   │   ├── greetings.rs
│   │   └── mod.rs
│   ├── japanese
│   │   ├── farewells.rs
│   │   ├── greetings.rs
│   │   └── mod.rs
│   └── lib.rs
└── target
    └── debug
        ├── build
        ├── deps
        ├── examples
        ├── libphrases-a7448e02a0468eaa.rlib
        └── native
\end{verbatim}

\code{src/lib.rs} is our crate root, and looks like this:

\begin{rustc}
mod english;
mod japanese;
\end{rustc}

These two declarations tell Rust to look for either \code{src/english.rs} and \code{src/japanese.rs}, or \code{src/english/mod.rs} 
and \code{src/japanese/mod.rs}, depending on our preference. In this case, because our modules have sub-modules, we've chosen the 
second. Both \code{src/english/mod.rs} and \code{src/japanese/mod.rs} look like this:

\begin{rustc}
mod greetings;
mod farewells;
\end{rustc}

Again, these declarations tell Rust to look for either \code{src/english/greetings.rs} and \code{src/japanese/greetings.rs} or 
\code{src/english/farewells/mod.rs} and \code{src/japanese/farewells/mod.rs}. Because these sub-modules don't have their own sub-modules, 
we've chosen to make them \code{src/english/greetings.rs} and \code{src/japanese/farewells.rs}. Whew!

\blank

The contents of \code{src/english/greetings.rs} and \code{src/japanese/farewells.rs} are both empty at the moment. Let's add some functions.

\blank

Put this in \code{src/english/greetings.rs}:

\begin{rustc}
fn hello() -> String {
    "Hello!".to_string()
}
\end{rustc}

Put this in \code{src/english/farewells.rs}:

\begin{rustc}
fn goodbye() -> String {
    "Goodbye.".to_string()
}
\end{rustc}

Put this in \code{src/japanese/greetings.rs}:

\begin{rustc}
fn hello() -> String {
    "こんにちは".to_string()
}
\end{rustc}

Of course, you can copy and paste this from this web page, or type something else. It's not important that you actually put 
'konnichiwa' to learn about the module system.

\blank

Put this in \code{src/japanese/farewells.rs}:

\begin{rustc}
fn goodbye() -> String {
    "さようなら".to_string()
}
\end{rustc}

(This is 'Sayōnara', if you're curious.)

\blank

Now that we have some functionality in our crate, let's try to use it from another crate.

\subsubsection*{Importing External Crates}

We have a library crate. Let's make an executable crate that imports and uses our library.

\blank

Make a \code{src/main.rs} and put this in it (it won't quite compile yet):

\begin{rustc}
extern crate phrases;

fn main() {
    println!("Hello in English: {}", phrases::english::greetings::hello());
    println!("Goodbye in English: {}", phrases::english::farewells::goodbye());

    println!("Hello in Japanese: {}", phrases::japanese::greetings::hello());
    println!("Goodbye in Japanese: {}", phrases::japanese::farewells::goodbye());
}
\end{rustc}

The \code{extern crate} declaration tells Rust that we need to compile and link to the \code{phrases} crate. We can then use \code{phrases}' 
modules in this one. As we mentioned earlier, you can use double colons to refer to sub-modules and the functions inside of them.

\blank

(Note: when importing a crate that has dashes in its name \enquote{like-this}, which is not a valid Rust identifier, it will be 
converted by changing the dashes to underscores, so you would write \code{extern crate like\_this;}.)

\blank

Also, Cargo assumes that \code{src/main.rs} is the crate root of a binary crate, rather than a library crate. Our package now has two 
crates: \code{src/lib.rs} and \code{src/main.rs}. This pattern is quite common for executable crates: most functionality is in a library 
crate, and the executable crate uses that library. This way, other programs can also use the library crate, and it's also a nice separation 
of concerns.

\blank

This doesn't quite work yet, though. We get four errors that look similar to this:

\begin{verbatim}
$ cargo build
   Compiling phrases v0.0.1 (file:///home/you/projects/phrases)
src/main.rs:4:38: 4:72 error: function `hello` is private
src/main.rs:4     println!("Hello in English: {}", phrases::english::greetings::hello());
                                                   ^~~~~~~~~~~~~~~~~~~~~~~~~~~~~~~~~~
note: in expansion of format_args!
<std macros>:2:25: 2:58 note: expansion site
<std macros>:1:1: 2:62 note: in expansion of print!
<std macros>:3:1: 3:54 note: expansion site
<std macros>:1:1: 3:58 note: in expansion of println!
phrases/src/main.rs:4:5: 4:76 note: expansion site
\end{verbatim}

By default, everything is private in Rust. Let's talk about this in some more depth.

\subsubsection*{Exporting a Public Interface}

Rust allows you to precisely control which aspects of your interface are public, and so private is the default. To make things public, 
you use the \code{pub} keyword. Let's focus on the english module first, so let's reduce our \code{src/main.rs} to only this:

\begin{rustc}
extern crate phrases;

fn main() {
    println!("Hello in English: {}", phrases::english::greetings::hello());
    println!("Goodbye in English: {}", phrases::english::farewells::goodbye());
}
\end{rustc}

In our \code{src/lib.rs}, let's add \code{pub} to the \code{english} module declaration:

\begin{rustc}
pub mod english;
mod japanese;
\end{rustc}

And in our \code{src/english/mod.rs}, let's make both \code{pub}:

\begin{rustc}
pub mod greetings;
pub mod farewells;
\end{rustc}

In our \code{src/english/greetings.rs}, let's add \code{pub} to our \code{fn} declaration:

\begin{rustc}
pub fn hello() -> String {
    "Hello!".to_string()
}
\end{rustc}

And also in \code{src/english/farewells.rs}:

\begin{rustc}
pub fn goodbye() -> String {
    "Goodbye.".to_string()
}
\end{rustc}

Now, our crate compiles, albeit with warnings about not using the \code{japanese} functions:

\begin{verbatim}
$ cargo run
   Compiling phrases v0.0.1 (file:///home/you/projects/phrases)
src/japanese/greetings.rs:1:1: 3:2 warning: function is never used: `hello`, #[warn(dead_code)] on by default
src/japanese/greetings.rs:1 fn hello() -> String {
src/japanese/greetings.rs:2     "こんにちは".to_string()
src/japanese/greetings.rs:3 }
src/japanese/farewells.rs:1:1: 3:2 warning: function is never used: `goodbye`, #[warn(dead_code)] on by default
src/japanese/farewells.rs:1 fn goodbye() -> String {
src/japanese/farewells.rs:2     "さようなら".to_string()
src/japanese/farewells.rs:3 }
     Running `target/debug/phrases`
Hello in English: Hello!
Goodbye in English: Goodbye.
\end{verbatim}

\code{pub} also applies to \struct s and their member fields. In keeping with Rust's tendency toward safety, simply making a \struct\ 
public won't automatically make its members public: you must mark the fields individually with \code{pub}.

\blank

Now that our functions are public, we can use them. Great! However, typing out \code{phrases::english::greetings::hello()} is very long 
and repetitive. Rust has another keyword for importing names into the current scope, so that you can refer to them with shorter names. 
Let's talk about \code{use}.

\subsubsection*{Importing Modules with \code{use}}

Rust has a \code{use} keyword, which allows us to import names into our local scope. Let's change our \code{src/main.rs} to look like this:

\begin{rustc}
extern crate phrases;

use phrases::english::greetings;
use phrases::english::farewells;

fn main() {
    println!("Hello in English: {}", greetings::hello());
    println!("Goodbye in English: {}", farewells::goodbye());
}
\end{rustc}

The two \code{use} lines import each module into the local scope, so we can refer to the functions by a much shorter name. By convention, 
when importing functions, it's considered best practice to import the module, rather than the function directly. In other words, you can 
do this:

\begin{rustc}
extern crate phrases;

use phrases::english::greetings::hello;
use phrases::english::farewells::goodbye;

fn main() {
    println!("Hello in English: {}", hello());
    println!("Goodbye in English: {}", goodbye());
}
\end{rustc}

But it is not idiomatic. This is significantly more likely to introduce a naming conflict. In our short program, it's not a big deal, but 
as it grows, it becomes a problem. If we have conflicting names, Rust will give a compilation error. For example, if we made the 
\code{japanese} functions public, and tried to do this:

\begin{rustc}
extern crate phrases;

use phrases::english::greetings::hello;
use phrases::japanese::greetings::hello;

fn main() {
    println!("Hello in English: {}", hello());
    println!("Hello in Japanese: {}", hello());
}
\end{rustc}

Rust will give us a compile-time error:

\begin{verbatim}
   Compiling phrases v0.0.1 (file:///home/you/projects/phrases)
src/main.rs:4:5: 4:40 error: a value named `hello` has already been imported in this module [E0252]
src/main.rs:4 use phrases::japanese::greetings::hello;
                  ^~~~~~~~~~~~~~~~~~~~~~~~~~~~~~~~~~~
error: aborting due to previous error
Could not compile `phrases`.
\end{verbatim}

If we're importing multiple names from the same module, we don't have to type it out twice. Instead of this:

\begin{rustc}
use phrases::english::greetings;
use phrases::english::farewells;
\end{rustc}

We can use this shortcut:

\begin{rustc}
use phrases::english::{greetings, farewells};
\end{rustc}

\subsubsection*{Re-exporting with \code{pub use}}

You don't only use \code{use} to shorten identifiers. You can also use it inside of your crate to re-export a function inside another 
module. This allows you to present an external interface that may not directly map to your internal code organization.

\blank

Let's look at an example. Modify your \code{src/main.rs} to read like this:

\begin{rustc}
extern crate phrases;

use phrases::english::{greetings,farewells};
use phrases::japanese;

fn main() {
    println!("Hello in English: {}", greetings::hello());
    println!("Goodbye in English: {}", farewells::goodbye());

    println!("Hello in Japanese: {}", japanese::hello());
    println!("Goodbye in Japanese: {}", japanese::goodbye());
}
\end{rustc}

Then, modify your \code{src/lib.rs} to make the \code{japanese} mod public:

\begin{rustc}
pub mod english;
pub mod japanese;
\end{rustc}

Next, make the two functions public, first in \code{src/japanese/greetings.rs}:

\begin{rustc}
pub fn hello() -> String {
    "こんにちは".to_string()
}
\end{rustc}

And then in \code{src/japanese/farewells.rs}:

\begin{rustc}
pub fn goodbye() -> String {
    "さようなら".to_string()
}
\end{rustc}

Finally, modify your \code{src/japanese/mod.rs} to read like this:

\begin{rustc}
pub use self::greetings::hello;
pub use self::farewells::goodbye;

mod greetings;
mod farewells;
\end{rustc}

The \code{pub use} declaration brings the function into scope at this part of our module hierarchy. Because we've \code{pub use}d this 
inside of our \code{japanese} module, we now have a \code{phrases::japanese::hello()} function and a \code{phrases::japanese::goodbye()}
function, even though the code for them lives in \code{phrases::japanese::greetings::hello()} and \code{phrases::japanese::farewells::goodbye()}.
Our internal organization doesn't define our external interface.

\blank

Here we have a \code{pub use} for each function we want to bring into the \code{japanese} scope. We could alternatively use the wildcard 
syntax to include everything from \code{greetings} into the current scope: \code{pub use self::greetings::*}.

\blank

What about the \code{self}? Well, by default, use declarations are absolute paths, starting from your crate root. \code{self} makes that 
path relative to your current place in the hierarchy instead. There's one more special form of \code{use}: you can \code{use super::} to 
reach one level up the tree from your current location. Some people like to think of \code{self} as \code{.} and \code{super} as \code{..}, 
from many shells' display for the current directory and the parent directory.

\blank

Outside of \code{use}, paths are relative: \code{foo::bar()} refers to a function inside of \code{foo} relative to where we are. If 
that's prefixed with \code{::}, as in \code{::foo::bar()}, it refers to a different \code{foo}, an absolute path from your crate root.

\blank

This will build and run:

\begin{verbatim}
$ cargo run
   Compiling phrases v0.0.1 (file:///home/you/projects/phrases)
     Running `target/debug/phrases`
Hello in English: Hello!
Goodbye in English: Goodbye.
Hello in Japanese: こんにちは
Goodbye in Japanese: さようなら
\end{verbatim}

\subsubsection*{Complex imports}

Rust offers several advanced options that can add compactness and convenience to your \code{extern crate} and \code{use} statements. 
Here is an example:

\begin{rustc}
extern crate phrases as sayings;

use sayings::japanese::greetings as ja_greetings;
use sayings::japanese::farewells::*;
use sayings::english::{self, greetings as en_greetings, farewells as en_farewells};

fn main() {
    println!("Hello in English; {}", en_greetings::hello());
    println!("And in Japanese: {}", ja_greetings::hello());
    println!("Goodbye in English: {}", english::farewells::goodbye());
    println!("Again: {}", en_farewells::goodbye());
    println!("And in Japanese: {}", goodbye());
}
\end{rustc}

What's going on here?

\blank

First, both \code{extern crate} and \code{use} allow renaming the thing that is being imported. So the crate is still called \enquote{phrases}, 
but here we will refer to it as \enquote{sayings}. Similarly, the first \code{use} statement pulls in the \code{japanese::greetings} module 
from the crate, but makes it available as \code{ja\_greetings} as opposed to simply \code{greetings}. This can help to avoid ambiguity 
when importing similarly-named items from different places.

\blank

The second \code{use} statement uses a star glob to bring in \emph{all} symbols from the \code{sayings::japanese::farewells} module. As 
you can see we can later refer to the Japanese \code{goodbye} function with no module qualifiers. This kind of glob should be used sparingly.

\blank

The third \code{use} statement bears more explanation. It's using \enquote{brace expansion} globbing to compress three \code{use} statements 
into one (this sort of syntax may be familiar if you've written Linux shell scripts before). The uncompressed form of this statement would be:

\begin{rustc}
use sayings::english;
use sayings::english::greetings as en_greetings;
use sayings::english::farewells as en_farewells;
\end{rustc}

As you can see, the curly brackets compress \code{use} statements for several items under the same path, and in this context \code{self} 
refers back to that path. Note: The curly brackets cannot be nested or mixed with star globbing.
