An \enum\ in Rust is a type that represents data that is one of several possible variants. Each variant in the \enum\ can optionally 
have data associated with it:

\begin{rustc}
enum Message {
    Quit,
    ChangeColor(i32, i32, i32),
    Move { x: i32, y: i32 },
    Write(String),
}
\end{rustc}

The syntax for defining variants resembles the syntaxes used to define structs: you can have variants with no data (like unit-like 
structs), variants with named data, and variants with unnamed data (like tuple structs). Unlike separate struct definitions, however, 
an \enum\ is a single type. A value of the enum can match any of the variants. For this reason, an enum is sometimes called a 
'sum type': the set of possible values of the enum is the sum of the sets of possible values for each variant.

\blank

We use the \code{::} syntax to use the name of each variant: they're scoped by the name of the \enum\ itself. This allows both of 
these to work:

\begin{rustc}
let x: Message = Message::Move { x: 3, y: 4 };

enum BoardGameTurn {
    Move { squares: i32 },
    Pass,
}

let y: BoardGameTurn = BoardGameTurn::Move { squares: 1 };
\end{rustc}

Both variants are named \code{Move}, but since they're scoped to the name of the \enum\, they can both be used without conflict.

\blank

A value of an \enum\ type contains information about which variant it is, in addition to any data associated with that variant. This 
is sometimes referred to as a 'tagged union', since the data includes a 'tag' indicating what type it is. The compiler uses this 
information to enforce that you're accessing the data in the enum safely. For instance, you can't simply try to destructure a value 
as if it were one of the possible variants:

\begin{rustc}
fn process_color_change(msg: Message) {
    let Message::ChangeColor(r, g, b) = msg; // compile-time error
}
\end{rustc}

% TODO make traits and match hyperrefs
Not supporting these operations may seem rather limiting, but it's a limitation which we can overcome. There are two ways: by 
implementing equality ourselves, or by pattern matching variants with match expressions, which you'll learn in the next section. 
We don't know enough about Rust to implement equality yet, but we'll find out in the traits section.

\subsubsection*{Constructors as functions}

An \enum\ constructor can also be used like a function. For example:

\begin{rustc}
let m = Message::Write("Hello, world".to_string());
\end{rustc}

is the same as

\begin{rustc}
fn foo(x: String) -> Message {
    Message::Write(x)
}

let x = foo("Hello, world".to_string());
\end{rustc}

% TODO make closures and iterators hyperrefs
This is not immediately useful to us, but when we get to closures, we'll talk about passing functions as arguments to other functions. 
For example, with iterators, we can do this to convert a vector of \String s into a vector of \code{Message::Writes}:

\begin{rustc}
let v = vec!["Hello".to_string(), "World".to_string()];

let v1: Vec<Message> = v.into_iter().map(Message::Write).collect();
\end{rustc}
